\section{Dornbusch, Fischer and Samuelson(1997): Continuum of Goods\cite{dornbusch1977comparative}}

Consider a world economy with only two countries: Home and Foreign.
Use asterisks to denote foreign variables. 
\begin{note}
    \

    Ricardian models differ from other neoclassical trade models in that
    there is only \textbf{one factor of production} in the model.
\end{note}

We denote by:
\begin{itemize}
    \item $L$ and $L^*$ as the labor endowments in Home and Foreign respectively.
    \item $w$ and $w^*$ as the wages(in efficiency units) in Home and Foreign respectively.
\end{itemize}

There's a continuum of goods indexed by $z \in [0,1]$.
Since there are CRS, we can define the constant unit labot requirements in both countries: $a(z)$ and $a^*(z)$.

Without loss of generaliry, we order goods such that $A(z) \equiv \frac{a^*(z)}{a(z)}$ is decreasing.
\begin{itemize}
    \item Hence Home has a comparative advantage in goods with low-z goods.
    \item For simplicity, we'll assume strict monotonicity.
\end{itemize}

Previous supply-side assumptions are all we need to make qualitative
predictions about pattern of trade.

Let $p(z)$ denote the price of good $z$ in Home under free trade.

The profit-maximization condition for a firm producing good $z$ is:
\begin{align*}
    p(z) - w\,a(z) &\leq 0, \text{with equality if z produced at home.} \\
    p(z) - w^*\,a^*(z) &\leq 0, \text{with equality if z produced abroad.}  
\end{align*}
\begin{proposition}\label{prop:CA}
    \

    There exists $\tilde{z} \in [0,1]$, s.t. Home produces all goods with $z < \tilde{z}$ 
    and Foreign produces all goods with $z > \tilde{z}$.
\end{proposition}
\begin{proof}
    \

    By contradiction. Suppose that there exists $z^{\prime} < z$, s.t. $z$ produced at Home
    and $z^{\prime} $ is produced abroad. Then:
    \begin{align*}
        p(z) - w\,a(z) &= 0, \\
        p(z^{\prime}) - w\,a(z^{\prime}) &\leq 0, \\
        p(z^{\prime}) - w^*\,a^*(z^{\prime}) &= 0, \\
        p(z) - w^*\,a^*(z) &\leq 0.
    \end{align*}
    This implies that:
    \[w\,a(z)w^*\,a^*(z) = p(z)\,p(z^{\prime}) \leq w\,a(z^{\prime})\,w^*\,a^*(z),\]
    which can be rearranged as:
    \[\frac{a^*(z^{\prime})}{a(z^{\prime})} \leq \frac{a^*(z)}{a(z)}\]
    which contradicts that $A(z)$ is strictly decreasing.
\end{proof}

This proposition(\ref{prop:CA}) simply shows tha tHome should produce and specialize in
the goods in which it has a CA.

\begin{note}
    \

    \begin{itemize}
        \item Proposition(\ref{prop:CA}) doesn't rely on continuum of goods.
        \item Continuum of goods + continuity of $A(z)$ is important to derive the relative wage:
        \begin{equation}\label{eq:CA}
            A(\tilde{z}) = \frac{w}{w^*} \equiv \omega.
        \end{equation}
    \end{itemize}
    Equation(\ref{eq:CA}) is the key result of the Dornbusch, Fischer and Samuelson(1997) model.
    It gives us that: conditional on wages, goods should be produced in the country where it's cheaper to do so.
\end{note}

Then, we take a look at the demand side to pin down the relative wage $\omega$.
\subsection{Demand side}

Assume that consumers have identical Cobb-Douglas preferences around the world.

We denote by $b(z) \in (0,1)$ the share of expenditure spent on good $z$:
\[b(z) = \frac{p(z) c(z)}{wL} = \frac{p(z)c^*(z)}{w^* L^*}\]
where $c(z)$ and $c^*(z)$ are the consumption of good $z$ in Home and Foreign respectively.

By definition, share of expenditure satisfies: $\int_{0}^{1} b(z) dz = 1$.

Then, we denote by $\theta(\tilde{z}) = \int_0^{\tilde{z}} b(z) dz$ the fraction of income spent
on goods produced at Home(spent in both countries).

The trade balance requires: 
\[\theta(\tilde{z}) w^* L^* = [1-\theta (\tilde{z})]wL\]
where LHS od the Home exports an d RHE is the Home imports.

WE can then rearrange the equation(\ref{eq:CA}) to get:
\begin{equation}
    \omega  = \frac{\theta(\tilde{z})}{1 - \theta(\tilde{z})}\Bigl(\frac{L^*}{L}\Bigr) \equiv B(\tilde{z}).
\end{equation}

Note that $B^{\prime} > 0$ and: an increase in $\tilde{z}$ leads to a trade surplus at Home,
which must be compensated by an increase in Home's relative wage $\omega$.

\begin{tikzpicture}[scale=5]
    % Axes
    \draw[->] (0,0) -- (1.3,0) node[right] {$z$};
    \draw[->] (0,0) -- (0,1) node[above] {$\omega$};
    
    % Origin and points
    \node at (0,0) [below left] {$0$};
    \node at (1.2,0) [below] {$1$};
    
    % A(z) curve - decreasing blue curve
    \draw[blue, thick] (0.1,0.9) .. controls (0.3,0.55) and (0.55,0.4) .. (1.1,0.3) node[right, blue] {$A(z)$};
    
    % B(z) curve - increasing green curve
    \draw[green, thick] (0,0) .. controls (0.3,0.12) and (0.6,0.4) .. (1.1,0.75) node[below, green] {$B(z)$};
    
    % Equilibrium point
    \node at (0.6,0.4) [circle, fill, inner sep=1pt] {};
    
    % Equilibrium labels
    \draw[dashed] (0.6,0) -- (0.6,0.4);
    \draw[dashed] (0,0.4) -- (0.6,0.4);
    \node at (0.6,0) [below] {$\bar{z}$};
    \node at (0,0.4) [left] {$\bar{\omega}$};
    
    % H and F labels
    \draw[<->] (0,-0.1) -- (0.6,-0.1) node[midway, below] {H};
    \draw[<->] (0.6,-0.1) -- (1.2,-0.1) node[midway, below] {F};
    
    % Equation at top right
    \node at (1.5,0.8) {$\frac{\theta(\bar{z})}{1-\theta(\bar{z})}\left(\frac{L^*}{L}\right) = \bar{\omega} = A(\bar{z})$};
\end{tikzpicture}

\begin{itemize}
    \item Efficient international specialization, Equation (3), and trade balance, (4), jointly determine $(\bar{z}, \bar{\omega})$
\end{itemize}

Since Ricardian model is a neoclassical model, general results derived
in previous lecture hold. However, we can directly show the existence of gains from trade in this environment.

\begin{itemize}
    \item Set $w=1$ under autarky and free trade.
    \item Indirect utility\footnote{The maximum utility under budget constraint.} of Home representative houssehold depends on $p(\cdot).$
    \item For goods $z$ produced at Home under free trade: no change sompared to autarky.
    \item For goods $z$ produced Foreign under free trade:
    \[p(z) = w^* a^*(z) < a(z)\]
    \item Since all prices go down, indirect utility must go up.
\end{itemize}

\subsection{Consequences of (Relative) Country Growth}

Suppose that $L^*/L$ increases, then the relative wage $\omega$ must increase and $\tilde{z}$ goes down.
At initial wages, an increase in $L^*/L$ leads to a trade deficit in Foreign, which must be compensated by an increase in $\omega$.

An increase in $L^*/L$ raises indirect utility, i.e. real wage, of representative household at Home and lowers it in Foreign.

\begin{itemize}
    \item We set $w=1$ before and after the change in $L^*/L.$
    \item For goods $z$ whose production remains at Home, there's no change in $p(z)$.
    \item For goods $z$ whose production remains in Foreign, we have: $w \uparrow \Rightarrow w^* \downarrow \Rightarrow p(z) = w^* a^*(z) \downarrow.$
    \item For goods $z$ whose production changes from Home to Foreign, we have: $p(z) = w^* a^*(z) \leq a(z) \Rightarrow p(z) \downarrow.$
    \item So Home gains. Similar logic implies Foreign welfare loss.
\end{itemize}

\begin{note}
    \
    
    In spite of CRS at the industry level, everything is as if we had DRS at the country-level.

    As foreing's size increases, it specialized in sectors in which it is relatively less productive(compared to home),
    which worsens it's terms of trade, so lowers real GDP per capita.

    The flatter the A schedule, the smaller this effect.
\end{note}

If there's an improvement in Home technology, then unit of labor requirement in Home $a(z)$ will be reducing, hence $A(z)$ is increasing - shifting upward. 
As we have shown in \ref{eq:CA}, this shows \textbf{an increase in home relative wage} i.e. higher income, 
and \textbf{an increase in home's share of world production and trade.}\footnote{As shown previously, $\omega = \frac{\theta(\tilde{z})}{1-\theta(\tilde{z})} \Bigl(\frac{L^*}{L}\Bigr)$, 
if $\omega$ increases, and we fix $\frac{L^*}{L}$, then $\theta(\tilde{z})$ increases.} 

\section{The Eaton-Kortum model (2002): Many Countries\cite{eaton2002technology}}
The model of Dornbusch, Fischer, and Samuelson (1977)\cite{dornbusch1977comparative} is based on the Ricardian model
where trade and specialization patterns are determined by different productivities.

In the Dornbusch-Fischer-Samuelson model, there were no restrictions on the productivity distributions,
Eaton and Kortum were able to extent the model to a case with many countries by sacrificing this generality.

The Eaton-Kortum model is based on the following assumptions:
\begin{itemize}
    \item Perfect Competition
    \item $N$ countries, insteat of 2 in DFS
    \item Continuum of goods $(u \in [0,1])$
    \item CRS technology: labor only
    \item CES preferences with elasticity of substitution $\sigma > 0$
\end{itemize}

\subsection{Model Setup}

\subsubsection{The world}

In this model, there a finite number of countries $i \in S \equiv \{1, \cdots, N\}$; 
(unlike previous models, there are technical difficulties in extending the model to a continuum of countries).
There are a continuum of goods $\Omega$.

However, unlike Krugman (1980)\cite{krugman1980scale} and Melitz (2003)\cite{melitz2003impact}, models which follow(in Lecture 4 and 5),
\textit{every country is able to produce every good.}
Countries, however, varies(exogeneously) in their productivity of each good $z_i(\omega)$ - varying by country $i$ and good $\omega \in \Omega$.

The Eaton and Kortum (2002)\cite{eaton2002technology} has no concept of a firm. Instead, it is assumed that all
goods $\omega$ are are produced using the same bundle of inputs with a constant returns to
scale technology. Let the cost of a bundle of inputs in country $i$ be $c_i$,
so that the cost of producing one unit of $\omega \in \Omega$ in country $i$ is $\frac{c_i}{z_i(\omega)}$.

Firstly, we introduce the \textbf{trade costs}: transport costs and tariffs. This is not an innovation of Eaton
and Kortum, the original Dornbush-Fischer-Samuelson model also had trade costs, but for simplification, we left them out.
    
Trade costs are modelled as iceberg costs, a popular device invented by Samuelson\cite{samuelson1954transfer}.
That is, we assume that for delivering one unit of a good from country $i$ to country $j$, 
it's necessary to ship $d_{ij} \geq 1$ units of good: the rest “melts away” during transit. 

\subsubsection{Supply}

Each good is assumed to be sold in perfectly competitive markets, so that the price a
consumer in country $j$ would pay for good $\omega$ produced in country $i$ is given by:
\begin{gather*}
    p_{ij}(\omega) = \frac{c_i}{z_i(\omega)} d_{ij}. \label{eq:EKprice}
\end{gather*}

However, consumers in country $j \in S$ are assumed to only purchase good $\omega \in \Omega$
from the country who can provide it at the lowest price, so the price a consumer in
$j \in N$ actually pays for good $\omega$ is:
\begin{gather*}
    p_j(\omega) = \min_{i \in S} p_{ij}(\omega) = \min _{i \in S} \Bigl\{ \frac{c_i}{z_i(\omega)} d_{ij} \Bigr\} \label{eq:EKprice2}
\end{gather*}

Here, the basic idea of Eaton and Kortum is already present in equation above:
a country $j \in S$ is more likely to purchase a good $\omega \in \Omega$ from country $i \in S$ if
\begin{enumerate}
    \item[(1)] it has a lower cost $c_i$;
    \item[(2)] it has a higher good productivity $z_i(\omega)$;
    \item[(3)] it has a lower trade cost $d_{ij}$. 
\end{enumerate}

\textbf{The main innovation of Eaton and Kortum}, however, is to assume that the productivities $z_i(\omega)$ are drawn from a Fréchet distribution\footnote{To find more properties of the Fréchet distribution, see the appendix \ref{sec:frechet}.}.
This means that for each $i \in S$ and $\forall \omega \in \Omega$, the cumulative distribution function $F_i$ is:
\[
F_i(z) = \operatorname{Pr}\Bigl\{z_i(\omega) \leq z \Bigr\} = \exp \Bigl\{-T_i z^{-\theta} \Bigr\}
\]
where $T_i > 0$ is a measure of the aggregate productivity of country $i$(country $i$'s state of technology)\footnote{A larger value of $T_i$ decreases $F_i$ fro any $z\geq 0$. It increases the probability of larger values
of $z$ and $\theta>1$, which is assumed to be constant across countries, governs the distribution of productivities across goods within countries. As $\theta$ increases, the heterogeneity of productivity across goods declines.}.

\begin{note}[Why Fréchet?]
    \

    Why make this particular distributional assumption for productivities?

    Kortum (1997)\cite{kortum1997research} showed that if the technology of producing goods is determined by
    the best ``idea'' of how to produce, then the limiting distribution is indeed Fréchet, where
    $T_i$ reflects the country`s stock of ideas. More generally, consider the random variable:
    \[
    M_n = \max \{X_1, \cdots, X_n \}
    \]
    where $X_n$ are i.i.d.

    The Fisher-Tippett-Gnedenko theorem states that the only (normalized) distribution of
    $M_n$ as $n \to \infty $ is an extreme value distribution, of which Fréchet is one of three types (Type II).
    Note that a conditional logit model assumes that the error term is Gumbel (Type I) extreme value distributed.

    If random variable $x$ is Gumbel distributed, then $\ln{x}$ is Fréchet distributed. Hence, loosely speaking,
    the Fréchet distribution works better for models that are log linear (like the gravity equation), whereas the Gumbel
    distribution works better for models that are linear.
\end{note}

\subsubsection{Demand}

As in previous models, consumers have CES preferences so that the representative agent
in country $j$ has utility:
\[
U_j = \Bigl( \int_{\Omega} q_j(\omega)^{\frac{\sigma -1}{\sigma}} d \omega \Bigr)^{\frac{\sigma}{\sigma-1}},
\]
where $q_j(\omega)$ is the quantity that country $j$ consumes of good $\omega$.

Note that unlike the Krugman (1980)\cite{krugman1980scale} model, not every good produced in every country will be sold to country $j$.

Indeed, good $\omega \in \Omega$ will be produced by all countries but country $j$ will only purchase
it from one country. The CES preferences will yield a Dixit-Stiglitz price index:
\[
P_j \equiv \Bigl( \int_{\Omega} p_j(\omega)^{1-\sigma} \Bigr)^{\frac{1}{1-\sigma}}.
\]

\subsection{Equilibrium}

We now consider the equilibrium of the model. Instead of relying on the CES demand
equation as in the previous models, we use a probabilistic formulation in order to solve
the model.

\subsubsection{Prices}

In perfect competition only the lowest cost producer of a good will supply that particular
good. Thus, we want to derive the distribution of the minimum price over a set of prices
offered by producers in different countries:
\[
p_j = \min \{p_{ij}, \cdots, p_{Nj} \}.
\]
In order to solve the model, we take advantage of the properties of the Fréchet distribution.

First, we consider the probability that country $i \in S$ is able to offer country $j$ good $\omega$
for a price less than $p$. Because the technology is i.i.d. across goods, we can define:
\begin{gather*}
    G_{ij}(p) \equiv \operatorname{Pr} \Bigl\{ p_{ij}(\omega) \leq p \Bigr\}
\end{gather*}
Using the perfect competition price equation(\ref{eq:EKprice}), we can write:
\begin{align*}
    G_{ij}(p) &= \operatorname{Pr} \Bigl\{ \frac{c_i}{z_i(\omega)} d_{ij} \leq p \Bigr\} \\
    &= 1 - \operatorname{Pr} \Bigl\{ z_i(\omega) \geq \frac{c_i d_{ij}}{p} \Bigr\} \\
    &= 1 - F_i \Bigl( \frac{c_i d_{ij}}{p} \Bigr) \\
    &= 1 - \exp \Bigl\{ -T_i \Bigl( \frac{c_i d_{ij}}{p} \Bigr)^{-\theta} \Bigr\}
\end{align*}
Consider now the probability that country $j$ pays a price less than $p$ for good $\omega $.

Again, because the technology is i.i.d. across goods, the probability will be tha same fo all goods $\omega \in \Omega$.
Define:
\begin{gather}
    G_j(p) \equiv \operatorname{Pr} \Bigl\{ p_j(\omega) \leq p \Bigr\}
\end{gather}
Because country $j$ buys from the least cost provider, using equation \ref{eq:EKprice2}, we can write:
\begin{align*}
    G_j(p) &= \operatorname{Pr} \Bigl\{ \min_{i \in S} p_{ij}(\omega) \leq p \Bigr\} \\
    &= 1 - \operatorname{Pr} \Bigl\{ \min_{i \in S} p_{ij}(\omega) \geq p \Bigr\} \\
    &= 1 - \operatorname{Pr} \Bigl\{ \cap_{i \in S} \Bigl( p_{ij}(\omega) \geq p \Bigr) \Bigr\} \\
    &= 1 - \prod_{i \in S} \operatorname{Pr} \Bigl\{ p_{ij}(\omega) \geq p \Bigr\} \\
    &= 1 - \prod_{i \in S} \Bigl( 1 - G_{ij}(p) \Bigr)
\end{align*}
Substituting $G_{ij}$ into this equation, we have:
\begin{align*}
    G_j(p) &= 1 - \prod_{i \in S} \Bigl( 1 - \Bigl( 1 - \exp \Bigl\{ -T_i \Bigl( \frac{c_i d_{ij}}{p} \Bigr)^{-\theta} \Bigr\} \Bigr) \Bigr) \\
    &= 1 - \prod_{i \in S} \exp \Bigl\{ -T_i \Bigl( \frac{c_i d_{ij}}{p} \Bigr)^{-\theta} \Bigr\} \\
    &= 1 - \exp \Bigl\{ -p^{\theta} \sum_{i \in S} T_i \Bigl( c_i d_{ij} \Bigr)^{-\theta} \Bigr\} \\
    &= 1 - \exp \Bigl\{-p^{\theta} \Phi_j \Bigr\}
\end{align*}
where $\Phi_j = \sum_{i \in S} T_i (c_i d_{ij})^{-\theta}$.