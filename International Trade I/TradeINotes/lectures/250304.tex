\subsection{Continuum of Goods: Dornbusch, Fischer and Samuelson(1997)}

Consider a world economy with only two countries: Home and Foreign.
Use asterisks to denote foreign variables. 
\begin{note}
    \

    Ricardian models differ from other neoclassical trade models in that
    there is only \textbf{one factor of production} in the model.
\end{note}

We denote by:
\begin{itemize}
    \item $L$ and $L^*$ as the labor endowments in Home and Foreign respectively.
    \item $w$ and $w^*$ as the wages(in efficiency units) in Home and Foreign respectively.
\end{itemize}

There's a continuum of goods indexed by $z \in [0,1]$.
Since there are CRS, we can define the constant unit labot requirements in both countries: $a(z)$ and $a^*(z)$.

Without loss of generaliry, we order goods such that $A(z) \equiv \frac{a^*(z)}{a(z)}$ is decreasing.
\begin{itemize}
    \item Hence Home has a comparative advantage in goods with low-z goods.
    \item For simplicity, we'll assume strict monotonicity.
\end{itemize}

Previous supply-side assumptions are all we need to make qualitative
predictions about pattern of trade.

Let $p(z)$ denote the price of good $z$ in Home under free trade.

The profit-maximization condition for a firm producing good $z$ is:
\begin{align*}
    p(z) - w\,a(z) &\leq 0, \text{with equality if z produced at home.} \\
    p(z) - w^*\,a^*(z) &\leq 0, \text{with equality if z produced abroad.}  
\end{align*}
\begin{proposition}\label{prop:CA}
    \

    There exists $\tilde{z} \in [0,1]$, s.t. Home produces all goods with $z < \tilde{z}$ 
    and Foreign produces all goods with $z > \tilde{z}$.
\end{proposition}
\begin{proof}
    \

    By contradiction. Suppose that there exists $z^{\prime} < z$, s.t. $z$ produced at Home
    and $z^{\prime} $ is produced abroad. Then:
    \begin{align*}
        p(z) - w\,a(z) &= 0, \\
        p(z^{\prime}) - w\,a(z^{\prime}) &\leq 0, \\
        p(z^{\prime}) - w^*\,a^*(z^{\prime}) &= 0, \\
        p(z) - w^*\,a^*(z) &\leq 0.
    \end{align*}
    This implies that:
    \[w\,a(z)w^*\,a^*(z) = p(z)\,p(z^{\prime}) \leq w\,a(z^{\prime})\,w^*\,a^*(z),\]
    which can be rearranged as:
    \[\frac{a^*(z^{\prime})}{a(z^{\prime})} \leq \frac{a^*(z)}{a(z)}\]
    which contradicts that $A(z)$ is strictly decreasing.
\end{proof}

This proposition(\ref{prop:CA}) simply shows tha tHome should produce and specialize in
the goods in which it has a CA.

\begin{note}
    \begin{itemize}
        \item Proposition(\ref{prop:CA}) doesn't rely on continuum of goods.
        \item Continuum of goods + continuity of $A(z)$ is important to derive the relative wage:
        \begin{equation}\label{eq:CA}
            A(\tilde{z}) = \frac{w}{w^*} \equiv \omega.
        \end{equation}
    \end{itemize}
    Equation(\ref{eq:CA}) is the key result of the Dornbusch, Fischer and Samuelson(1997) model.
    It gives us that: conditional on wages, goods should be produced in the country where it's cheaper to do so.
\end{note}

Then, we take a look at the demand side to pin down the relative wage $\omega$.
\subsubsection{Demand side}

Assume that consumers have identical Cobb-Douglas preferences around the world.

We denote by $b(z) \in (0,1)$ the share of expenditure spent on good $z$:
\[b(z) = \frac{p(z) c(z)}{wL} = \frac{p(z)c^*(z)}{w^* L^*}\]
where $c(z)$ and $c^*(z)$ are the consumption of good $z$ in Home and Foreign respectively.

By definition, share of expenditure satisfies: $\int_{0}^{1} b(z) dz = 1$.

Then, we denote by $\theta(\tilde{z}) = \int_0^{\tilde{z}} b(z) dz$ the fraction of income spent
on goods produced at Home(spent in both countries).

The trade balance requires: 
\[\theta(\tilde{z}) w^* L^* = [1-\theta (\tilde{z})]wL\]
where LHS od the Home exports an d RHE is the Home imports.

WE can then rearrange the equation(\ref{eq:CA}) to get:
\begin{equation}
    \omega  = \frac{\theta(\tilde{z})}{1 - \theta(\tilde{z})}\Bigl(\frac{L^*}{L}\Bigr) \equiv B(\tilde{z}).
\end{equation}

Note that $B^{\prime} > 0$ and: an increase in $\tilde{z}$ leads to a trade surplus at Home,
which must be compensated by an increase in Home's relative wage $\omega$.

\begin{tikzpicture}[scale=5]
    % Axes
    \draw[->] (0,0) -- (1.3,0) node[right] {$z$};
    \draw[->] (0,0) -- (0,1) node[above] {$\omega$};
    
    % Origin and points
    \node at (0,0) [below left] {$0$};
    \node at (1.2,0) [below] {$1$};
    
    % A(z) curve - decreasing blue curve
    \draw[blue, thick] (0.1,0.9) .. controls (0.3,0.55) and (0.55,0.4) .. (1.1,0.3) node[right, blue] {$A(z)$};
    
    % B(z) curve - increasing green curve
    \draw[green, thick] (0,0) .. controls (0.3,0.12) and (0.6,0.4) .. (1.1,0.75) node[below, green] {$B(z)$};
    
    % Equilibrium point
    \node at (0.6,0.4) [circle, fill, inner sep=1pt] {};
    
    % Equilibrium labels
    \draw[dashed] (0.6,0) -- (0.6,0.4);
    \draw[dashed] (0,0.4) -- (0.6,0.4);
    \node at (0.6,0) [below] {$\bar{z}$};
    \node at (0,0.4) [left] {$\bar{\omega}$};
    
    % H and F labels
    \draw[<->] (0,-0.1) -- (0.6,-0.1) node[midway, below] {H};
    \draw[<->] (0.6,-0.1) -- (1.2,-0.1) node[midway, below] {F};
    
    % Equation at top right
    \node at (1.5,0.8) {$\frac{\theta(\bar{z})}{1-\theta(\bar{z})}\left(\frac{L^*}{L}\right) = \bar{\omega} = A(\bar{z})$};
\end{tikzpicture}

\begin{itemize}
    \item Efficient international specialization, Equation (3), and trade balance, (4), jointly determine $(\bar{z}, \bar{\omega})$
\end{itemize}

Since Ricardian model is a neoclassical model, general results derived
in previous lecture hold. However, we can directly show the existence of gains from trade in this environment.

\begin{itemize}
    \item Set $w=1$ under autarky and free trade.
    \item Indirect utility\footnote{The maximum utility under budget constraint.} of Home representative houssehold depends on $p(\cdot).$
    \item For goods $z$ produced at Home under free trade: no change sompared to autarky.
    \item For goods $z$ produced Abroad under free trade:
    \[p(z) = w^* a^*(z) < a(z)\]
    \item Since all prices go down, indirect utility must go up.
\end{itemize}

\subsubsection{Consequences of (Relative) Country Growth}

Suppose that $L^*/L$ increases, then the relative wage $\omega$ must increase and $\tilde{z}$ goes down.
At initial wages, an increase in $L^*/L$ leads to a trade deficit in Abroad, which must be compensated by an increase in $\omega$.