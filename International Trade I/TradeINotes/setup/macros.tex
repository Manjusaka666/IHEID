%% From SeniorMars lecture template

%From M275 "Topology" at SJSU
\newcommand{\id}{\mathrm{id}}
\newcommand{\taking}[1]{\xrightarrow{#1}}
\newcommand{\inv}{^{-1}}

%From M170 "Introduction to Graph Theory" at SJSU
\DeclareMathOperator{\diam}{diam}
\DeclareMathOperator{\ord}{ord}
\newcommand{\defeq}{\overset{\mathrm{def}}{=}}

%From the USAMO .tex files
\newcommand{\ts}{\textsuperscript}
\newcommand{\dg}{^\circ}
\newcommand{\ii}{\item}

% \newenvironment{subproof}[1][Proof]{%
% \begin{proof}[#1] \renewcommand{\qedsymbol}{$\blacksquare$}}%
% {\end{proof}}

\newcommand{\liff}{\leftrightarrow}
\newcommand{\lthen}{\rightarrow}
\newcommand{\opname}{\operatorname}
\newcommand{\surjto}{\twoheadrightarrow}
\newcommand{\injto}{\hookrightarrow}
\newcommand{\On}{\mathrm{On}} % ordinals
\DeclareMathOperator{\img}{im} % Image
\DeclareMathOperator{\Img}{Im} % Image
% \DeclareMathOperator{\coker}{coker} % Cokernel
% \DeclareMathOperator{\Coker}{Coker} % Cokernel
% \DeclareMathOperator{\Ker}{Ker} % Kernel
\DeclareMathOperator{\rank}{rank}
\DeclareMathOperator{\Spec}{Spec} % spectrum
\DeclareMathOperator{\Tr}{Tr} % trace
\DeclareMathOperator{\pr}{pr} % projection
\DeclareMathOperator{\ext}{ext} % extension
\DeclareMathOperator{\pred}{pred} % predecessor
\DeclareMathOperator{\dom}{dom} % domain
\DeclareMathOperator{\ran}{ran} % range
% \DeclareMathOperator{\Hom}{Hom} % homomorphism
% \DeclareMathOperator{\Mor}{Mor} % morphisms
% \DeclareMathOperator{\End}{End} % endomorphism
\DeclareMathOperator*{\plim}{plim} % probability limit

\newcommand{\eps}{\epsilon}
\newcommand{\veps}{\varepsilon}
\newcommand{\ol}{\overline}
\newcommand{\ul}{\underline}
\newcommand{\wt}{\widetilde}
\newcommand{\wh}{\widehat}
\newcommand{\vocab}[1]{\textbf{\color{blue} #1}}
\providecommand{\half}{\frac{1}{2}}
\newcommand{\dang}{\measuredangle} %% Directed angle
\newcommand{\ray}[1]{\overrightarrow{#1}}
\newcommand{\seg}[1]{\overline{#1}}
\newcommand{\arc}[1]{\wideparen{#1}}
\DeclareMathOperator{\cis}{cis}
\DeclareMathOperator*{\lcm}{lcm}
\DeclareMathOperator*{\argmin}{arg min}
\DeclareMathOperator*{\argmax}{arg max}
\newcommand{\cycsum}{\sum_{\mathrm{cyc}}}
\newcommand{\symsum}{\sum_{\mathrm{sym}}}
\newcommand{\cycprod}{\prod_{\mathrm{cyc}}}
\newcommand{\symprod}{\prod_{\mathrm{sym}}}
\newcommand{\Qed}{\begin{flushright}\qed\end{flushright}}
\newcommand{\parinn}{\setlength{\parindent}{1cm}}
\newcommand{\parinf}{\setlength{\parindent}{0cm}}
% \newcommand{\norm}{\|\cdot\|}
\newcommand{\inorm}{\norm_{\infty}}
\newcommand{\opensets}{\{V_{\alpha}\}_{\alpha\in I}}
\newcommand{\oset}{V_{\alpha}}
\newcommand{\opset}[1]{V_{\alpha_{#1}}}
\newcommand{\lub}{\text{lub}}

% OD - Ordinary derivates
\newcommand{\od}[2]{\frac{\mathrm d #1}{\mathrm d #2}}
\newcommand{\oD}[3]{\frac{\mathrm d^{#1} #2}{\mathrm d {#3}^{#1}}}
% Ordinary derivates displaystyle with dfrac
\newcommand{\odd}[2]{\dfrac{\mathrm d #1}{\mathrm d #2}}
\newcommand{\oDd}[3]{\dfrac{\mathrm d^{#1} #2}{\mathrm d {#3}^{#1}}}

% PD - Partial derivates
\newcommand{\del}{\partial}
\newcommand{\pd}[2]{\frac{\partial #1}{\partial #2}}
\newcommand{\pD}[3]{\frac{\partial^{#1} #2}{\partial {#3}^{#1}}}
% Partial derivates displaystyle with dfrac
\newcommand{\pdd}[2]{\dfrac{\partial #1}{\partial #2}}
\newcommand{\pDd}[3]{\dfrac{\partial^{#1} #2}{\partial {#3}^{#1}}}

%%%
\newcommand{\lm}{\lambda}
\newcommand{\uin}{\mathbin{\rotatebox[origin=c]{90}{$\in$}}}
\newcommand{\usubset}{\mathbin{\rotatebox[origin=c]{90}{$\subset$}}}
\newcommand{\lt}{\left}
\newcommand{\rt}{\right}
\newcommand{\bs}[1]{\boldsymbol{#1}}
\newcommand{\exs}{\exists}
\newcommand{\st}{\strut}
\newcommand{\dps}[1]{\displaystyle{#1}}

\newcommand{\sol}{\setlength{\parindent}{0cm}\textbf{\textit{Solution:}}\setlength{\parindent}{1cm} }
\newcommand{\solve}[1]{\setlength{\parindent}{0cm}\textbf{\textit{Solution: }}\setlength{\parindent}{1cm}#1 \Qed}

%%%%%%%%%%%%%%%%%%%%%%%%%%%%%%

%%% Preliminary declarations:
%%%% These are some commands where we declare new commands for the notes

% We define the macro for the name of the professor
% \newcommand{\professor}[1]{ \newcommand{\professorloc}{#1} }
% We define the macro for the name of the course
\newcommand{\course}[1]{ \newcommand{\courseloc}{#1} }
% We define the macro for the name of the institution
\newcommand{\institute}[1]{ \newcommand{\instituteloc}{#1} }
% We define the macro for the name of the class roll
\newcommand{\roll}[1]{ \newcommand{\rollloc}{#1} }
% We define the macro for the name of the class
\newcommand{\class}[1]{ \newcommand{\classloc}{#1} }
% We define the macro for the name of the session
\newcommand{\session}[1]{ \newcommand{\sessionloc}{#1} }

% We define the macro for my (student/author) name and email
\newcommand*{\meloc}{}
\newcommand*{\mynameloc}{}
\newcommand*{\myemailloc}{}

\newcommand*{\me}[2][]{%
   \renewcommand*{\mynameloc}{#2}%
   \if\relax\detokenize{#1}\relax
      \renewcommand*{\meloc}{\textbf{#2}}%
      \renewcommand*{\myemailloc}{}%
   \else
      \renewcommand*{\meloc}{\textbf{\href{mailto:#1}{#2}}}%
      \renewcommand*{\myemailloc}{\texttt{\href{mailto:#1}{#1}}}%
   \fi
}

% We define the macro for the professor's name and email
\newcommand*{\profloc}{}
\newcommand*{\profnameloc}{}
\newcommand*{\profemailloc}{}

\newcommand*{\professor}[2][]{%
   \renewcommand*{\profnameloc}{#2}%
   \if\relax\detokenize{#1}\relax
      \renewcommand*{\profloc}{\textbf{#2}}%
      \renewcommand*{\profemailloc}{}%
   \else
      \renewcommand*{\profloc}{\textbf{\href{#1}{#2}}}%
      \renewcommand*
      {\profemailloc}{\texttt{\href{#1}{#1}}}%
   \fi
}

%these are auxiliary definitions used in the title section
\newcommand{\CourseLang}{Course}
\newcommand{\DateLang}{Submission date}
\newcommand{\StudentLang}{Name}
\newcommand{\ProfessorLang}{Professor}
\newcommand{\RollLang}{Roll}          
\newcommand{\ClassLang}{Class}                 
\newcommand{\SessionLang}{Session}     
\newcommand{\InstituteLang}{Institute}   
\newcommand{\EmailLang}{Email}   



% Customize author command
% \newcommand{\myname}[1]{\gdef\printmyname{#1}}
\author{\huge \mynameloc}

% Customize title command
\newcommand{\mytitle}[1]{\gdef\printmytitle{#1}}
\title{\Huge Lecture Notes: \\ \courseloc}

% Custom command for equation numbering inside list (itemize...)
\newcommand{\itemnumber}{\hfill\refstepcounter{equation}(\theequation)} 

% Custom commands to indicate the parts where correction, details, clarifications and diagrams are needed to add  
\newcommand{\eqdetails}{\textcolor{red}{[need details...] }}
\newcommand{\details}{\textcolor{red}{[need to add details here...] }}
\newcommand{\diagram}{\textcolor{red}{[need to add diagram \& details here...] }}