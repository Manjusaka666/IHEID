% basics
\usepackage[utf8]{inputenc}
\usepackage[T1]{fontenc}
\usepackage{textcomp}
% \usepackage[dutch]{babel}
\usepackage{url}
\usepackage{hyperref}
\hypersetup{
    colorlinks,
    linkcolor={black},
    citecolor={black},
    urlcolor={blue!80!black},
    % backref=true, 
    % pagebackref=true
}
% Page Margins
\usepackage[
    margin=2.5cm, 
    % top=2.8cm, bottom=2.8cm,
    % left=1in, right=1in, 
    % headheight=14.5pt
    ]{geometry}

\usepackage{setspace}
\setstretch{1.2}

\usepackage{graphicx}
\usepackage{float}
\usepackage{booktabs}
\usepackage{pdfpages}
\usepackage[shortlabels]{enumitem}
% \usepackage{parskip}
\usepackage{emptypage}
\usepackage{subcaption}
\usepackage{multicol}
\usepackage[usenames,dvipsnames]{xcolor}

% \usepackage{cmbright}

\usepackage{amsmath, amsfonts, mathtools, amsthm, amssymb}
\usepackage{mathrsfs}
% \usepackage{dutchcal} % \mathcal: also small letters. \mathbcal = bold ones. % https://tex.stackexchange.com/a/641540/114006
\usepackage{braket} % https://tex.stackexchange.com/a/253232/114006
\usepackage{cancel}
\usepackage{bm}
\usepackage[ruled,vlined,linesnumbered]{algorithm2e} % SeniorMars template
% \usepackage{lipsum}
\usepackage{pgfplots}
\usepgfplotslibrary{fillbetween}

\newcommand\N{\ensuremath{\mathbb{N}}}
\newcommand\R{\ensuremath{\mathbb{R}}}
\newcommand\Z{\ensuremath{\mathbb{Z}}}
\renewcommand\O{\ensuremath{\emptyset}}
\newcommand\Q{\ensuremath{\mathbb{Q}}}
\newcommand\C{\ensuremath{\mathbb{C}}}
\DeclareMathOperator{\sgn}{sgn}
\usepackage{systeme}
\let\svlim\lim\def\lim{\svlim\limits}
\let\implies\Rightarrow
\let\impliedby\Leftarrow
\let\iff\Leftrightarrow
\let\epsilon\varepsilon
\usepackage{stmaryrd} % for \lightning
\newcommand\contra{\scalebox{1.1}{$\lightning$}}
% \let\phi\varphi





% correct
\definecolor{correct}{HTML}{009900}
\newcommand\correct[2]{\ensuremath{\:}{\color{red}{#1}}\ensuremath{\to }{\color{correct}{#2}}\ensuremath{\:}}
\newcommand\green[1]{{\color{correct}{#1}}}



% horizontal rule
\newcommand\hr{
    \noindent\rule[0.5ex]{\linewidth}{0.5pt}
}


% hide parts
\newcommand\hide[1]{}



% si unitx
\usepackage{siunitx}
\sisetup{locale = FR}
% \renewcommand\vec[1]{\mathbf{#1}}
\newcommand\mat[1]{\mathbf{#1}}


% tikz
\usepackage{tikz}
\usepackage{tikz-cd}
\usepackage{tikzsymbols} % for symbols such as smiley
\usetikzlibrary{intersections, angles, quotes, calc, positioning}
\usetikzlibrary{arrows.meta}
\usepackage{pgfplots}
\pgfplotsset{compat=1.13}


\tikzset{
    force/.style={thick, {Circle[length=2pt]}-stealth, shorten <=-1pt}
}


% theorems
\usepackage{thmtools}
\usepackage[framemethod=TikZ]{mdframed}
% \mdfsetup{skipabove=1em,skipbelow=0em, innertopmargin=5pt, innerbottommargin=6pt}


\theoremstyle{definition}

\makeatletter


\@ifclasswith{report}{nocolor}{
    \declaretheoremstyle[headfont=\bfseries\sffamily, bodyfont=\normalfont, mdframed={ nobreak } ]{thmgreenbox}
    \declaretheoremstyle[headfont=\bfseries\sffamily, bodyfont=\normalfont, mdframed={ nobreak } ]{thmredbox}
    \declaretheoremstyle[headfont=\bfseries\sffamily, bodyfont=\normalfont]{thmbluebox}
    \declaretheoremstyle[headfont=\bfseries\sffamily, bodyfont=\normalfont]{thmblueline}
    \declaretheoremstyle[headfont=\bfseries\sffamily, bodyfont=\normalfont, numbered=no, mdframed={ rightline=false, topline=false, bottomline=false, }, qed=\qedsymbol ]{thmproofbox}
    \declaretheoremstyle[headfont=\bfseries\sffamily, bodyfont=\normalfont, numbered=no, mdframed={ nobreak, rightline=false, topline=false, bottomline=false } ]{thmexplanationbox}
    \AtEndEnvironment{eg}{\null\hfill$\diamond$}%
}{
    \declaretheoremstyle[
      headfont=\bfseries\sffamily\color{ForestGreen!70!black}, bodyfont=\normalfont,
      mdframed={
          linewidth=2pt,
          rightline=false, topline=false, bottomline=false,
          linecolor=ForestGreen, backgroundcolor=ForestGreen!5,
          nobreak=false
        }
    ]{thmgreenbox}
    
    \declaretheoremstyle[
      headfont=\bfseries\sffamily\color{ForestGreen!70!black}, bodyfont=\normalfont,
      mdframed={
          linewidth=2pt,
          rightline=false, topline=false, bottomline=false,
          linecolor=ForestGreen, backgroundcolor=ForestGreen!8,
          nobreak=false
        }
    ]{thmgreen2box}
    
    \declaretheoremstyle[
      headfont=\bfseries\sffamily\color{NavyBlue!70!black}, bodyfont=\normalfont,
      mdframed={
          linewidth=2pt,
          rightline=false, topline=false, bottomline=false,
          linecolor=NavyBlue, backgroundcolor=NavyBlue!5,
          nobreak=false
        }
    ]{thmbluebox}
    
    \declaretheoremstyle[
      headfont=\bfseries\sffamily\color{TealBlue!70!black}, bodyfont=\normalfont,
      mdframed={
          linewidth=2pt,
          rightline=false, topline=false, bottomline=false,
          linecolor=TealBlue,
          nobreak=false
        }
    ]{thmblueline}
    
    \declaretheoremstyle[
      headfont=\bfseries\sffamily\color{RawSienna!70!black}, bodyfont=\normalfont,
      mdframed={
          linewidth=2pt,
          rightline=false, topline=false, bottomline=false,
          linecolor=RawSienna, backgroundcolor=RawSienna!5,
          nobreak=false
        }
    ]{thmredbox}
    
    \declaretheoremstyle[
      headfont=\bfseries\sffamily\color{RawSienna!70!black}, bodyfont=\normalfont,
      mdframed={
          linewidth=2pt,
          rightline=false, topline=false, bottomline=false,
          linecolor=RawSienna, backgroundcolor=RawSienna!8,
          nobreak=false
        }
    ]{thmred2box}
    
    \declaretheoremstyle[
      headfont=\bfseries\sffamily\color{SeaGreen!70!black}, bodyfont=\normalfont,
      mdframed={
          linewidth=2pt,
          rightline=false, topline=false, bottomline=false,
          linecolor=SeaGreen, backgroundcolor=SeaGreen!2,
          nobreak=false
        }
    ]{thmgreen3box}
    
    \declaretheoremstyle[
      headfont=\bfseries\sffamily\color{WildStrawberry!70!black}, bodyfont=\normalfont,
      mdframed={
          linewidth=2pt,
          rightline=false, topline=false, bottomline=false,
          linecolor=WildStrawberry, backgroundcolor=WildStrawberry!5,
          nobreak=false
        }
    ]{thmpinkbox}
    
    \declaretheoremstyle[
      headfont=\bfseries\sffamily\color{MidnightBlue!70!black}, bodyfont=\normalfont,
      mdframed={
          linewidth=2pt,
          rightline=false, topline=false, bottomline=false,
          linecolor=MidnightBlue, backgroundcolor=MidnightBlue!5,
          nobreak=false
        }
    ]{thmblue2box}
    
    \declaretheoremstyle[
      headfont=\bfseries\sffamily\color{Gray!70!black}, bodyfont=\normalfont,
      mdframed={
          linewidth=2pt,
          rightline=false, topline=false, bottomline=false,
          linecolor=Gray, backgroundcolor=Gray!5,
          nobreak=false
        }
    ]{notgraybox}
    
    \declaretheoremstyle[
      headfont=\bfseries\sffamily\color{Gray!70!black}, bodyfont=\normalfont,
      mdframed={
          linewidth=2pt,
          rightline=false, topline=false, bottomline=false,
          linecolor=Gray,
          nobreak=false
        }
    ]{notgrayline}
    
    \declaretheoremstyle[
      headfont=\bfseries\sffamily\color{RawSienna!70!black}, bodyfont=\normalfont,
      numbered=no,
      mdframed={
          linewidth=2pt,
          rightline=false, topline=false, bottomline=false,
          linecolor=RawSienna, backgroundcolor=RawSienna!1,
        },
      qed=\qedsymbol
    ]{thmproofbox}
    
    \declaretheoremstyle[
      headfont=\bfseries\sffamily\color{NavyBlue!70!black}, bodyfont=\normalfont,
      numbered=no,
      mdframed={
          linewidth=2pt,
          rightline=false, topline=false, bottomline=false,
          linecolor=NavyBlue, backgroundcolor=NavyBlue!1,
          nobreak=false
        }
    ]{thmexplanationbox}
    
    \declaretheoremstyle[
      headfont=\bfseries\sffamily\color{WildStrawberry!70!black}, bodyfont=\normalfont,
      numbered=no,
      mdframed={
          linewidth=2pt,
          rightline=false, topline=false, bottomline=false,
          linecolor=WildStrawberry, backgroundcolor=WildStrawberry!1,
          nobreak=false
        }
    ]{thmanswerbox}
    
    \declaretheoremstyle[
      headfont=\bfseries\sffamily\color{Violet!70!black}, bodyfont=\normalfont,
      mdframed={
          linewidth=2pt,
          rightline=false, topline=false, bottomline=false,
          linecolor=Violet, backgroundcolor=Violet!1,
          nobreak=false
        }
    ]{conjpurplebox}
}





\declaretheorem[style=thmbluebox, numbered=yes, name=Example]{eg}
\declaretheorem[style=thmbluebox, numbered=yes, name=Exercise]{ex}
\declaretheorem[style=thmgreenbox, name=Definition, numberwithin=section]{definition}
\declaretheorem[style=thmgreen2box, name=Definition, numbered=no]{definition*}
\declaretheorem[style=thmredbox, name=Theorem, numberwithin=section]{theorem}
\declaretheorem[style=thmred2box, name=Theorem, numbered=no]{theorem*}
\declaretheorem[style=thmredbox, name=Lemma, numberwithin=section]{lemma}
\declaretheorem[style=thmredbox, name=Assumption, numberwithin=section]{assumption}
\declaretheorem[style=thmredbox, name=Proposition, numberwithin=section]{proposition}
\declaretheorem[style=thmredbox, name=Corollary, numberwithin=section]{corollary}
\declaretheorem[style=thmpinkbox, name=Problem, numberwithin=section]{problem}
\declaretheorem[style=thmpinkbox, name=Problem, numbered=no]{problem*}
\declaretheorem[style=thmpinkbox, name=Question, numbered=no]{question}
\declaretheorem[style=thmblue2box, name=Claim, numbered=no]{claim}
\declaretheorem[style=conjpurplebox, name=Conjecture, numberwithin=section]{conjecture}

\@ifclasswith{report}{nocolor}{
    \declaretheorem[style=thmproofbox, name=Proof]{replacementproof}
    \declaretheorem[style=thmexplanationbox, name=Proof]{explanation}
    \renewenvironment{proof}[1][\proofname]{\begin{replacementproof}}{\end{replacementproof}}
}{
    \declaretheorem[style=thmproofbox, name=Proof]{replacementproof}
    \renewenvironment{proof}[1][\proofname]{\vspace{-10pt}\begin{replacementproof}}{\end{replacementproof}}

    \declaretheorem[style=thmexplanationbox, name=Proof]{tmpexplanation}
    \newenvironment{explanation}[1][]{\vspace{-10pt}\begin{tmpexplanation}}{\end{tmpexplanation}}
}

\makeatother

\declaretheorem[style=thmblueline, numbered=no, name=Remark]{remark}
\declaretheorem[style=thmblueline, numbered=no, name=Note]{note}
\declaretheorem[style=thmpinkbox, numbered=no, name=Exercise]{exercise}
\declaretheorem[style=notgrayline, numbered=no, name=As previously seen]{prev}
\declaretheorem[style=thmgreen3box, numbered=no, name=Intuition]{intuition}
\declaretheorem[style=notgraybox, numbered=no, name=Notation]{notation}
\declaretheorem[style=thmanswerbox, numbered=no, name=Answer]{tmpanswer}
\newenvironment{answer}[1][]{\vspace{-10pt}\pushQED{\(\circledast\)}\begin{tmpanswer}}{\null\hfill\popQED\end{tmpanswer}}

\newtheorem*{uovt}{UOVT}
\newtheorem*{previouslyseen}{As previously seen}
\newtheorem*{solution}{Solution}
% \newtheorem*{question}{Question}
\newtheorem*{observe}{Observe}
\newtheorem*{property}{Property}


\usepackage{etoolbox}
\AtEndEnvironment{vb}{\null\hfill$\diamond$}%
\AtEndEnvironment{intermezzo}{\null\hfill$\diamond$}%
% \AtEndEnvironment{opmerking}{\null\hfill$\diamond$}%

% http://tex.stackexchange.com/questions/22119/how-can-i-change-the-spacing-before-theorems-with-amsthm
% \def\thm@space@setup{%
%   \thm@preskip=\parskip \thm@postskip=0pt
% }

\newcommand{\oefening}[1]{%
    \def\@oefening{#1}%
    \subsection*{Oefening #1}
}

\newcommand{\suboefening}[1]{%
    \subsubsection*{Oefening \@oefening.#1}
}

% \newcommand{\exercise}[1]{%
%     \def\@exercise{#1}%
%     \subsection*{Exercise #1}
% }

% \newcommand{\subexercise}[1]{%
%     \subsubsection*{Exercise \@exercise.#1}
% }


\usepackage{xifthen}

% \def\testdateparts#1{\dateparts#1\relax}
% \def\dateparts#1 #2 #3 #4 #5\relax{
%     \marginpar{\small\textsf{\mbox{#1 #2 #3 #5}}}
% }

% \def\@lesson{}%
% \newcommand{\lesson}[3]{
%     \ifthenelse{\isempty{#3}}{%
%         \def\@lesson{Lecture #1}%
%     }{%
%         \def\@lesson{Lecture #1: #3}%
%     }%
%     \subsection*{\@lesson}
%     \testdateparts{#2}
% }


% Chapter title
\usepackage{titlesec}
\titleformat{\chapter}[frame]
  {\normalfont}
  {\filright
   \footnotesize
   \enspace Lecture \arabic{chapter}.\enspace}
  {8pt}
  {\Large\bfseries\filcenter}
\usepackage[dotinlabels]{titletoc}
\titlecontents{chapter}[1.5em]{}{\contentslabel{2.3em}}{\hspace*{-2.3em}}{\hfill\contentspage}
\titlespacing*{\chapter} {0pt}{0pt}{40pt}     % this alters "before" spacing (the second length argument) to 0

% \renewcommand\date[1]{\marginpar{#1}}


% % fancy headers
% \usepackage{fancyhdr}
% \pagestyle{fancy}

% % \fancypagestyle{header}{%
% % \fancyhead[LE,RO]{Mubtasim Fuad}
% \fancyhead[R]{Lecture \thechapter}
% \fancyhead[L]{}
% \fancyfoot[C]{\thepage}
% % \fancyfoot[C]{\leftmark}
% % }

% Style for page headers and footers
\usepackage{fancyhdr}
\fancypagestyle{head}{
  \fancyhf{}     % clear all header and footer fields
  \lhead{\courseloc}
  \rhead{Lecture \thechapter}
  \cfoot{\thepage}    % Use \pageref{LastPage} instead if you want to add the link
  \renewcommand{\headrulewidth}{0.5pt}
  \renewcommand{\footrulewidth}{0.5pt}
}

\fancypagestyle{plain}{
  \fancyhf{}     % clear all header and footer fields
  \rhead{\courseloc}
  \cfoot{\thepage}    % Use \pageref{LastPage} instead if you want to add the link
  \renewcommand{\headrulewidth}{0.5pt}
  \renewcommand{\footrulewidth}{0.5pt}   
}

\makeatother


% notes
% \usepackage{todonotes} % slowdown compilation speed
\usepackage{marginnote}
\let\marginpar\marginnote

% sidenotes with arrows in equations
\usepackage{witharrows}

% Some custom colorbox and sidenote environments
\usepackage{tcolorbox}

\tcbuselibrary{breakable}
\newenvironment{verbetering}{\begin{tcolorbox}[
    arc=0mm,
    colback=white,
    colframe=green!60!black,
    title=Opmerking,
    fonttitle=\sffamily,
    breakable
]}{\end{tcolorbox}}

% \newenvironment{noot}[1]{\begin{tcolorbox}[
%     arc=0mm,
%     colback=white,
%     colframe=white!60!black,
%     title=#1,
%     fonttitle=\sffamily,
%     breakable
% ]}{\end{tcolorbox}}

% Side note with custom color
\newtcolorbox{noot}[3][]
{
  arc=0mm,
  colback  = white,
  colframe = #2!50,
  coltitle = #2!20!black,   
  title    = {Side-note: #3},
  fonttitle=\sffamily,
  breakable,
  #1,
}

% https://tex.stackexchange.com/a/172480/114006
\newtcolorbox{mybox}[3][]
{
  colback  = #2!10,
  colframe = #2!25,
  coltitle = #2!20!black,  
  title    = {#3},
  #1,
}

% Side note environment with gray color and title
\newenvironment{sidenote}[1]{\begin{noot}{gray}{#1}}{\end{noot}}
% Side note environment with user-defined color and title
\newenvironment{sidenotex}[2]{\begin{noot}{#1}{#2}}{\end{noot}}

\newenvironment{myminipage}
    {
    \begin{center}
    \begin{minipage}{0.85\textwidth}
    %  \begin{mdframed}
    }
    { 
    %  \end{mdframed}
    \end{minipage}   
    \end{center}
    }


% figure support
\usepackage{import}
\usepackage{xifthen}
\pdfminorversion=7
\usepackage{pdfpages}
\usepackage{transparent}
\newcommand{\incfig}[2]{%
    \def\svgwidth{#1\columnwidth}
    \import{./figures/}{#2.pdf_tex}
}
\graphicspath{{./figures/}}
%% http://tex.stackexchange.com/questions/76273/multiple-pdfs-with-page-group-included-in-a-single-page-warning
\pdfsuppresswarningpagegroup=1

\newcommand{\nchapter}[2]{%
    \setcounter{chapter}{#1}%
    \addtocounter{chapter}{-1}%
    \chapter{#2}
}

\newcommand{\nsection}[3]{%
    \setcounter{chapter}{#1}%
    \setcounter{section}{#2}%
    \addtocounter{section}{-1}%
    \section{#3}
}%

\usepackage{listings}
\lstset
{
    language=[LaTeX]TeX,
    breaklines=true,
    basicstyle=\tt\scriptsize,
    keywordstyle=\color{blue},
    identifierstyle=\color{black},
}

% Change paragraph spacing
\setlength{\parskip}{4pt}
% Change paragraph indent
\setlength{\parindent}{0cm}

% for creating \NewDocumentCommand
% \usepackage{xparse}


%% for setting arrows beneath equations 

% new command notate - https://tex.stackexchange.com/a/263491/114006
\usepackage[usestackEOL]{stackengine}
\usepackage{scalerel}

% \parskip \baselineskip  % it's creating problem with my theorem environment box designs
\def\svmybf#1{\rotatebox{90}{\stretchto{\{}{#1}}}
\def\svnobf#1{}
\def\rlwd{.5pt}
\newcommand\notate[4][B]{%
  \if B#1\let\myupbracefill\svmybf\else\let\myupbracefill\svnobf\fi%
  \def\useanchorwidth{T}%
  \setbox0=\hbox{$\displaystyle#2$}%
  \def\stackalignment{c}\stackunder[-6pt]{%
    \def\stackalignment{c}\stackunder[-1.5pt]{%
      \stackunder[2pt]{\strut $\displaystyle#2$}{\myupbracefill{\wd0}}}{%
    \rule{\rlwd}{#3\baselineskip}}}{%
  \strut\kern9pt$\rightarrow$\smash{\rlap{$~\displaystyle#4$}}}%
}

% new command indicate - https://tex.stackexchange.com/a/15742/114006

% trees
\usepackage[linguistics]{forest}

% some packages from my other templates 
\usepackage{array} % for tables
\usepackage[export]{adjustbox}
% \usepackage{wrapfig} % not properly fitting here
\usepackage{multirow}
\usepackage{tabularx}
\usepackage{extarrows}  % https://tex.stackexchange.com/a/407628/114006

% bibliography
\usepackage[
    backend=biber, 
    backref=true, 
    style=authoryear, 
    sortcites=true, 
    sorting=none, 
    defernumbers=true
]{biblatex}       % bibliography  % https://www.overleaf.com/learn/latex/bibliography_management_with_biblatex
\usepackage{xurl}                % handling the urls in bib file and it should be loaded after loading biblatex 

% Allow page breaks inside equation environments of amsmath package
\allowdisplaybreaks

% To cancel terms in equations with diagonal arrows
\usepackage{cancel} % https://tex.stackexchange.com/a/75530/114006
\usepackage{bookmark}
% To write tensors with mixed indices
\usepackage{tensor} % https://tex.stackexchange.com/a/25976/114006
\usepackage{derivative} % https://tex.stackexchange.com/a/14822/114006 % https://tex.stackexchange.com/a/501336/114006
\usepackage{annotate-equations} % https://github.com/st--/annotate-equations

% To create dashed box in equations - https://tex.stackexchange.com/a/285489/114006
\usepackage{dashbox}
\newcommand\dboxed[1]{\dbox{\ensuremath{#1}}}

% To add Left-aligned texts in equations (doesn't work in equation environment) - https://tex.stackexchange.com/a/60168/114006 
\makeatletter
\newif\if@gather@prefix 
\preto\place@tag@gather{% 
  \if@gather@prefix\iftagsleft@ 
    \kern-\gdisplaywidth@ 
    \rlap{\gather@prefix}% 
    \kern\gdisplaywidth@ 
  \fi\fi 
} 
\appto\place@tag@gather{% 
  \if@gather@prefix\iftagsleft@\else 
    \kern-\displaywidth 
    \rlap{\gather@prefix}% 
    \kern\displaywidth 
  \fi\fi 
  \global\@gather@prefixfalse 
} 
\preto\place@tag{% 
  \if@gather@prefix\iftagsleft@ 
    \kern-\gdisplaywidth@ 
    \rlap{\gather@prefix}% 
    \kern\displaywidth@ 
  \fi\fi 
} 
\appto\place@tag{% 
  \if@gather@prefix\iftagsleft@\else 
    \kern-\displaywidth 
    \rlap{\gather@prefix}% 
    \kern\displaywidth 
  \fi\fi 
  \global\@gather@prefixfalse 
} 
\def\math@cr@@@align{%
  \ifst@rred\nonumber\fi
  \if@eqnsw \global\tag@true \fi
  \global\advance\row@\@ne
  \add@amps\maxfields@
  \omit
  \kern-\alignsep@
  \if@gather@prefix\tag@true\fi
  \iftag@
    \setboxz@h{\@lign\strut@{\make@display@tag}}%
    \place@tag
  \fi
  \ifst@rred\else\global\@eqnswtrue\fi
  \global\lineht@\z@
  \cr
}
\newcommand*{\lefttext}[1]{% 
  \ifmeasuring@\else
  \gdef\gather@prefix{#1}% 
  \global\@gather@prefixtrue 
  \fi
} 
\makeatother

% \usepackage[none]{hyphenat}
\usepackage{nicematrix} 

% Command for circle around text [used chatgpt]
\newcommand{\mycir}[1]{%
    \mathchoice%
        {\mycirAux{\displaystyle}{#1}}%
        {\mycirAux{\textstyle}{#1}}%
        {\mycirAux{\scriptstyle}{#1}}%
        {\mycirAux{\scriptscriptstyle}{#1}}%
}
\newcommand{\mycirAux}[2]{%
        \tikz[baseline=(char.base)]{%
            \node[draw, circle, inner sep=1pt, font={\fontsize{8}{8}\selectfont}] (char) 
            {\ensuremath{#1{#2}}};
        }
}

% Custom enviroment for table-like matrix / cagedmatrix % https://tex.stackexchange.com/a/686197/114006

\ExplSyntaxOn
\NewDocumentEnvironment{cagedmatrix}{b}
 {
  % split the input at \\
  \seq_set_split:Nnn \l_tmpa_seq { \\ } { #1 }
  % check for a missing trailing \\
  \seq_pop_right:NN \l_tmpa_seq \l_tmpa_tl
  \tl_if_empty:NF \l_tmpa_tl { \seq_put_right:NV \l_tmpa_seq \l_tmpa_tl }
  % build the array, inserting \\ \hline between rows
  \array{|*{\value{MaxMatrixCols}}{c|}}\hline
  \seq_use:Nn \l_tmpa_seq { \\ \hline }
  % finish up
  \\ \hline
  \endarray
}{}
\ExplSyntaxOff

% Another method: using ytableau but it has issues in case of horizontal-aligning with other terms in equations.  
\usepackage{ytableau}

% Another custom commnad for tabularmatrix % https://tex.stackexchange.com/a/686232/114006
\usepackage{tabularray}
\NewDocumentEnvironment{tabularmatrix}{+b}{
    \begin{tblr}{
       hlines, vlines, columns={c},
       rowsep=0.1pt, colsep=5pt,
       }
    #1
    \end{tblr}
    }{}

% Define cagedbox command
\newcommand{\cagedbox}[1]{%
  \begin{tabularmatrix}%
    #1%
  \end{tabularmatrix}%
}

%% Change formatting of back references % https://tex.stackexchange.com/a/606518/114006
\DefineBibliographyStrings{english}{
   backrefpage={p.},
  % backrefpage={},
   backrefpages={pp.}
  % backrefpages={
}
\renewcommand*{\finentrypunct}{}
\usepackage{xpatch}

% \DeclareFieldFormat{backrefparens}{\mkbibparens{#1\addperiod}}
\DeclareFieldFormat{backrefparens}{\raisebox{-4pt}{\scriptsize{\mkbibparens{#1}}}}
\xpatchbibmacro{pageref}{parens}{backrefparens}{}{}