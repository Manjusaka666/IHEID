% basics
\usepackage[utf8]{inputenc}
\usepackage[T1]{fontenc}
\usepackage{textcomp}
\usepackage[hyphens]{url}
\usepackage[style=alphabetic,maxcitenames=1]{biblatex}
\usepackage[colorlinks=true,linkcolor=cyan,urlcolor=magenta,citecolor=violet]{hyperref}
\usepackage{graphicx}
\usepackage{float}
\usepackage{booktabs}
\usepackage{emptypage}
\usepackage{subcaption}
\usepackage[usenames,dvipsnames]{xcolor}

\usepackage{amsmath, amsfonts, mathtools, amsthm, amssymb, mathrsfs}
\usepackage{geometry}
\usepackage{cancel}
\usepackage{systeme}

\usepackage[inline, shortlabels]{enumitem}
\usepackage{multicol}
\setlength\multicolsep{0pt}

\usepackage{caption}
\captionsetup{belowskip=0pt}
\geometry{a4paper,left=2.54cm,right=2.54cm,top=1.73cm,bottom=1.73cm}

% for the big braces
\usepackage{bigdelim}

% correct
\definecolor{correct}{HTML}{009900}
\newcommand\correct[2]{\ensuremath{\:}{\color{red}{#1}}\ensuremath{\to }{\color{correct}{#2}}\ensuremath{\:}}
\newcommand\green[1]{{\color{correct}{#1}}}

% hide parts
\newcommand\hide[1]{}

% si unitx
\usepackage{siunitx}
\sisetup{locale = FR}
\newcommand\mat[1]{\mathbf{#1}}

% tikz
\usepackage{tikz}
\usetikzlibrary{intersections, angles, quotes, positioning}
\usetikzlibrary{arrows.meta}
\usepackage{pgfplots}
\pgfplotsset{compat=1.13}

\tikzset{
    force/.style={thick, {Circle[length=2pt]}-stealth, shorten <=-1pt}
}

% Algorithm Env
\usepackage[linesnumbered,lined,vlined,ruled,commentsnumbered,resetcount,algochapter]{algorithm2e}
\SetKwComment{Comment}{// }{}
\SetArgSty{textsl}
\def\algocflineautorefname{Algorithm}
\counterwithin{algocfline}{chapter}

% Clear out the item autoref name
\makeatletter
\def\itemautorefname{\@gobble}
\makeatother

% theorems
\usepackage{thmtools}
\usepackage[framemethod=TikZ]{mdframed}

\mdfsetup{skipabove=1em,skipbelow=0em}

\theoremstyle{definition}

% ... [All theorem styles from header.tex]
\declaretheoremstyle[
	headfont=\bfseries\sffamily\color{ForestGreen!70!black}, bodyfont=\normalfont,
	mdframed={
			linewidth=2pt,
			rightline=false, topline=false, bottomline=false,
			linecolor=ForestGreen, backgroundcolor=ForestGreen!5,
			nobreak=false
		}
]{thmgreenbox}

\declaretheoremstyle[
	headfont=\bfseries\sffamily\color{ForestGreen!70!black}, bodyfont=\normalfont,
	mdframed={
			linewidth=2pt,
			rightline=false, topline=false, bottomline=false,
			linecolor=ForestGreen, backgroundcolor=ForestGreen!8,
			nobreak=false
		}
]{thmgreen2box}

\declaretheoremstyle[
	headfont=\bfseries\sffamily\color{NavyBlue!70!black}, bodyfont=\normalfont,
	mdframed={
			linewidth=2pt,
			rightline=false, topline=false, bottomline=false,
			linecolor=NavyBlue, backgroundcolor=NavyBlue!5,
			nobreak=false
		}
]{thmbluebox}

\declaretheoremstyle[
	headfont=\bfseries\sffamily\color{TealBlue!70!black}, bodyfont=\normalfont,
	mdframed={
			linewidth=2pt,
			rightline=false, topline=false, bottomline=false,
			linecolor=TealBlue,
			nobreak=false
		}
]{thmblueline}

\declaretheoremstyle[
	headfont=\bfseries\sffamily\color{RawSienna!70!black}, bodyfont=\normalfont,
	mdframed={
			linewidth=2pt,
			rightline=false, topline=false, bottomline=false,
			linecolor=RawSienna, backgroundcolor=RawSienna!5,
			nobreak=false
		}
]{thmredbox}

\declaretheoremstyle[
	headfont=\bfseries\sffamily\color{RawSienna!70!black}, bodyfont=\normalfont,
	mdframed={
			linewidth=2pt,
			rightline=false, topline=false, bottomline=false,
			linecolor=RawSienna, backgroundcolor=RawSienna!8,
			nobreak=false
		}
]{thmred2box}

\declaretheoremstyle[
	headfont=\bfseries\sffamily\color{SeaGreen!70!black}, bodyfont=\normalfont,
	mdframed={
			linewidth=2pt,
			rightline=false, topline=false, bottomline=false,
			linecolor=SeaGreen, backgroundcolor=SeaGreen!2,
			nobreak=false
		}
]{thmgreen3box}

\declaretheoremstyle[
	headfont=\bfseries\sffamily\color{WildStrawberry!70!black}, bodyfont=\normalfont,
	mdframed={
			linewidth=2pt,
			rightline=false, topline=false, bottomline=false,
			linecolor=WildStrawberry, backgroundcolor=WildStrawberry!5,
			nobreak=false
		}
]{thmpinkbox}

\declaretheoremstyle[
	headfont=\bfseries\sffamily\color{MidnightBlue!70!black}, bodyfont=\normalfont,
	mdframed={
			linewidth=2pt,
			rightline=false, topline=false, bottomline=false,
			linecolor=MidnightBlue, backgroundcolor=MidnightBlue!5,
			nobreak=false
		}
]{thmblue2box}

\declaretheoremstyle[
	headfont=\bfseries\sffamily\color{Gray!70!black}, bodyfont=\normalfont,
	mdframed={
			linewidth=2pt,
			rightline=false, topline=false, bottomline=false,
			linecolor=Gray, backgroundcolor=Gray!5,
			nobreak=false
		}
]{notgraybox}

\declaretheoremstyle[
	headfont=\bfseries\sffamily\color{Gray!70!black}, bodyfont=\normalfont,
	mdframed={
			linewidth=2pt,
			rightline=false, topline=false, bottomline=false,
			linecolor=Gray,
			nobreak=false
		}
]{notgrayline}

\declaretheoremstyle[
	headfont=\bfseries\sffamily\color{RawSienna!70!black}, bodyfont=\normalfont,
	numbered=no,
	mdframed={
			linewidth=2pt,
			rightline=false, topline=false, bottomline=false,
			linecolor=RawSienna, backgroundcolor=RawSienna!1,
		},
	qed=\qedsymbol
]{thmproofbox}

\declaretheoremstyle[
	headfont=\bfseries\sffamily\color{NavyBlue!70!black}, bodyfont=\normalfont,
	numbered=no,
	mdframed={
			linewidth=2pt,
			rightline=false, topline=false, bottomline=false,
			linecolor=NavyBlue, backgroundcolor=NavyBlue!1,
			nobreak=false
		}
]{thmexplanationbox}

\declaretheoremstyle[
	headfont=\bfseries\sffamily\color{WildStrawberry!70!black}, bodyfont=\normalfont,
	numbered=no,
	mdframed={
			linewidth=2pt,
			rightline=false, topline=false, bottomline=false,
			linecolor=WildStrawberry, backgroundcolor=WildStrawberry!1,
			nobreak=false
		}
]{thmanswerbox}

\declaretheoremstyle[
	headfont=\bfseries\sffamily\color{Violet!70!black}, bodyfont=\normalfont,
	mdframed={
			linewidth=2pt,
			rightline=false, topline=false, bottomline=false,
			linecolor=Violet, backgroundcolor=Violet!1,
			nobreak=false
		}
]{conjpurplebox}

\declaretheorem[style=thmgreenbox, name=Definition, numberwithin=section]{definition}
\declaretheorem[style=thmgreen2box, name=Definition, numbered=no]{definition*}
\declaretheorem[style=thmredbox, name=Theorem, numberwithin=section]{theorem}
\declaretheorem[style=thmred2box, name=Theorem, numbered=no]{theorem*}
\declaretheorem[style=thmredbox, name=Lemma, numberwithin=section]{lemma}
\declaretheorem[style=thmredbox, name=Proposition, numberwithin=section]{proposition}
\declaretheorem[style=thmredbox, name=Corollary, numberwithin=section]{corollary}
\declaretheorem[style=thmpinkbox, name=Problem, numberwithin=section]{problem}
\declaretheorem[style=thmpinkbox, name=Problem, numbered=no]{problem*}
\declaretheorem[style=thmblue2box, name=Claim, numbered=no]{claim}
\declaretheorem[style=conjpurplebox, name=Conjecture, numberwithin=section]{conjecture}

% Redefine proof environment to get a full control.
\makeatletter
\renewenvironment{proof}[1][\proofname]{\par
	\pushQED{\qed}%
	\normalfont \topsep-2\p@\@plus6\p@\relax
	\trivlist
	\item[\hskip\labelsep
	            \color{RawSienna!70!black}\sffamily\bfseries
	            #1\@addpunct{.}]\ignorespaces
	\begin{mdframed}[linewidth=2pt,rightline=false, topline=false, bottomline=false,linecolor=RawSienna, backgroundcolor=RawSienna!1]
		}{%
		\popQED\endtrivlist\@endpefalse
	\end{mdframed}
}
\makeatother

\declaretheorem[style=thmbluebox, numbered=no, name=Example]{eg}
\declaretheorem[style=thmexplanationbox, numbered=no, name=Proof]{tmpexplanation}
\newenvironment{explanation}[1][]{\vspace{-10pt}\pushQED{\(\circledast\)}\begin{tmpexplanation}}{\null\hfill\popQED\end{tmpexplanation}}

\declaretheorem[style=thmblueline, numbered=no, name=Remark]{remark}
\declaretheorem[style=thmblueline, numbered=no, name=Note]{note}
\declaretheorem[style=thmpinkbox, numbered=no, name=Exercise]{exercise}
\declaretheorem[style=notgrayline, numbered=no, name=As previously seen]{prev}
\declaretheorem[style=thmgreen3box, numbered=no, name=Intuition]{intuition}
\declaretheorem[style=notgraybox, numbered=no, name=Notation]{notation}
\declaretheorem[style=thmanswerbox, numbered=no, name=Answer]{tmpanswer}
\newenvironment{answer}[1][]{\vspace{-10pt}\pushQED{\(\circledast\)}\begin{tmpanswer}}{\null\hfill\popQED\end{tmpanswer}}


\usepackage{etoolbox}
\renewcommand{\qed}{\null\hfill\(\blacksquare\)}

\makeatletter

\def\testdateparts#1{\dateparts#1\relax}
\def\dateparts#1 #2 #3 #4 #5\relax{
	\marginpar{\small\textsf{\mbox{#1 #2 #3 #5}}}
}

\def\@lecture{}%
\newcommand{\lecture}[3]{
	\ifthenelse{\isempty{#3}}{%
		\global\def\@lecture{Lecture #1}%
	}{%
		\global\def\@lecture{Lecture #1: #3}%
	}%
	\section*{\@lecture}
	\marginpar{\small\textsf{\mbox{#2}}}
}

\usepackage{pgffor}%
\newcommand{\lec}[2]{%
	\foreach \c in {#1,...,#2}{%
			\IfFileExists{Lectures/lec_\c.tex} {%
				\input{Lectures/lec_\c.tex}%
			}{}%
		}%
}

% fancy headers
\usepackage{fancyhdr}
\pagestyle{fancy}
\fancyhead[L]{}
\fancyhead[R]{\@lecture}
\fancyfoot[L]{}
\fancyfoot[R]{\thepage}
\fancyfoot[C]{\leftmark}

\makeatother

% notes
\usepackage[color=pink]{todonotes}
\usepackage{marginnote}
\let\marginpar\marginnote

% Fix some stuff
% %http://tex.stackexchange.com/questions/76273/multiple-pdfs-with-page-group-included-in-a-single-page-warning
% \pdfsuppresswarningpagegroup=1

% Appendix environment
\usepackage{appendix}
\def\chapterautorefname{Section}
\def\sectionautorefname{Section}
\def\appendixautorefname{Appendix}
\renewcommand\appendixname{Appendix}
\renewcommand\appendixtocname{Appendix}
\renewcommand\appendixpagename{Appendix}
% begin appendix autoref patch [\autoref subsections in appendix](https://tex.stackexchange.com/questions/149807/autoref-subsections-in-appendix)
\makeatletter
\patchcmd{\hyper@makecurrent}{%
	\ifx\Hy@param\Hy@chapterstring
		\let\Hy@param\Hy@chapapp
	\fi
}{%
	\iftoggle{inappendix}{%true-branch
		% list the names of all sectioning counters here
		\@checkappendixparam{chapter}%
		\@checkappendixparam{section}%
		\@checkappendixparam{subsection}%
		\@checkappendixparam{subsubsection}%
		\@checkappendixparam{paragraph}%
		\@checkappendixparam{subparagraph}%
	}{}%
}{}{\errmessage{failed to patch}}

\newcommand*{\@checkappendixparam}[1]{%
	\def\@checkappendixparamtmp{#1}%
	\ifx\Hy@param\@checkappendixparamtmp
		\let\Hy@param\Hy@appendixstring
	\fi
}
\makeatletter

\newtoggle{inappendix}
\togglefalse{inappendix}

\apptocmd{\appendix}{\toggletrue{inappendix}}{}{\errmessage{failed to patch}}
\apptocmd{\subappendices}{\toggletrue{inappendix}}{}{\errmessage{failed to patch}}
% end appendix autoref patch

\setcounter{tocdepth}{1}

% ListStyle.tex
\usepackage{listings}
\usepackage[utf8]{inputenc}
\usepackage[english]{babel}
\definecolor{maroon}{cmyk}{0, 0.87, 0.68, 0.32}
\definecolor{halfgray}{gray}{0.55}
\definecolor{ipython_frame}{RGB}{207, 207, 207}
\definecolor{ipython_bg}{RGB}{247, 247, 247}
\definecolor{ipython_red}{RGB}{186, 33, 33}
\definecolor{ipython_green}{RGB}{0, 128, 0}
\definecolor{ipython_cyan}{RGB}{64, 128, 128}
\definecolor{ipython_purple}{RGB}{170, 34, 255}

\colorlet{mygray}{black!30}
\colorlet{mygreen}{green!60!blue}
\colorlet{mymauve}{red!60!blue}
\lstdefinelanguage{commandline}{
	basicstyle=\ttfamily,
	columns=fullflexible,
	breakatwhitespace=false,
	breaklines=true,
	captionpos=b,
	commentstyle=\color{mygreen},
	extendedchars=true,
	frame=single,
	keepspaces=true,
	keywordstyle=\color{blue},
	language=c++,
	numbers=none,
	numbersep=5pt,
	numberstyle=\tiny\color{blue},
	rulecolor=\color{mygray},
	showspaces=false,
	showtabs=false,
	stepnumber=5,
	stringstyle=\color{mymauve},
	tabsize=3,
	title=\lstname
}
\lstset{
breaklines=true,
%
extendedchars=true,
literate=
	{á}{{\'a}}1 {é}{{\'e}}1 {í}{{\'i}}1 {ó}{{\'o}}1 {ú}{{\'u}}1
{Á}{{\'A}}1 {É}{{\'E}}1 {Í}{{\'I}}1 {Ó}{{\'O}}1 {Ú}{{\'U}}1
{à}{{\`a}}1 {è}{{\`e}}1 {ì}{{\`i}}1 {ò}{{\`o}}1 {ù}{{\`u}}1
{À}{{\`A}}1 {È}{{\'E}}1 {Ì}{{\`I}}1 {Ò}{{\`O}}1 {Ù}{{\`U}}1
{ä}{{\"a}}1 {ë}{{\"e}}1 {ï}{{\"i}}1 {ö}{{\"o}}1 {ü}{{\"u}}1
{Ä}{{\"A}}1 {Ë}{{\"E}}1 {Ï}{{\"I}}1 {Ö}{{\"O}}1 {Ü}{{\"U}}1
{â}{{\^a}}1 {ê}{{\^e}}1 {î}{{\^i}}1 {ô}{{\^o}}1 {û}{{\^u}}1
{Â}{{\^A}}1 {Ê}{{\^E}}1 {Î}{{\^I}}1 {Ô}{{\^O}}1 {Û}{{\^U}}1
{œ}{{\oe}}1 {Œ}{{\OE}}1 {æ}{{\ae}}1 {Æ}{{\AE}}1 {ß}{{\ss}}1
{ç}{{\c c}}1 {Ç}{{\c C}}1 {ø}{{\o}}1 {å}{{\r a}}1 {Å}{{\r A}}1
{€}{{\EUR}}1 {£}{{\pounds}}1
}

%%
%% Python definition (c) 1998 Michael Weber
%% Additional definitions (2013) Alexis Dimitriadis
%% modified by me (should not have empty lines)
%%
\lstdefinelanguage{iPython}{
morekeywords={access,and,break,class,continue,def,del,elif,else,except,exec,finally,for,from,global,if,import,in,is,lambda,not,or,pass,print,raise,return,try,while},%
%
% Built-ins
morekeywords=[2]{abs,all,any,basestring,bin,bool,bytearray,callable,chr,classmethod,cmp,compile,complex,delattr,dict,dir,divmod,enumerate,eval,execfile,file,filter,float,format,frozenset,getattr,globals,hasattr,hash,help,hex,id,input,int,isinstance,issubclass,iter,len,list,locals,long,map,max,memoryview,min,next,object,oct,open,ord,pow,property,range,raw_input,reduce,reload,repr,reversed,round,set,setattr,slice,sorted,staticmethod,str,sum,super,tuple,type,unichr,unicode,vars,xrange,zip,apply,buffer,coerce,intern},%
%
sensitive=true,%
morecomment=[l]\#,%
morestring=[b]',%
morestring=[b]",%
%
morestring=[s]{'''}{'''},% used for documentation text (mulitiline strings)
morestring=[s]{"""}{"""},% added by Philipp Matthias Hahn
%
morestring=[s]{r'}{'},% `raw' strings
morestring=[s]{r"}{"},%
morestring=[s]{r'''}{'''},%
morestring=[s]{r"""}{"""},%
morestring=[s]{u'}{'},% unicode strings
morestring=[s]{u"}{"},%
morestring=[s]{u'''}{'''},%
morestring=[s]{u"""}{"""},%
%
% {replace}{replacement}{lenght of replace}
% *{-}{-}{1} will not replace in comments and so on
literate=
	{á}{{\'a}}1 {é}{{\'e}}1 {í}{{\'i}}1 {ó}{{\'o}}1 {ú}{{\'u}}1
{Á}{{\'A}}1 {É}{{\'E}}1 {Í}{{\'I}}1 {Ó}{{\'O}}1 {Ú}{{\'U}}1
{à}{{\`a}}1 {è}{{\`e}}1 {ì}{{\`i}}1 {ò}{{\`o}}1 {ù}{{\`u}}1
{À}{{\`A}}1 {È}{{\'E}}1 {Ì}{{\`I}}1 {Ò}{{\`O}}1 {Ù}{{\`U}}1
{ä}{{\"a}}1 {ë}{{\"e}}1 {ï}{{\"i}}1 {ö}{{\"o}}1 {ü}{{\"u}}1
{Ä}{{\"A}}1 {Ë}{{\"E}}1 {Ï}{{\"I}}1 {Ö}{{\"O}}1 {Ü}{{\"U}}1
{â}{{\^a}}1 {ê}{{\^e}}1 {î}{{\^i}}1 {ô}{{\^o}}1 {û}{{\^u}}1
{Â}{{\^A}}1 {Ê}{{\^E}}1 {Î}{{\^I}}1 {Ô}{{\^O}}1 {Û}{{\^U}}1
{œ}{{\oe}}1 {Œ}{{\OE}}1 {æ}{{\ae}}1 {Æ}{{\AE}}1 {ß}{{\ss}}1
{ç}{{\c c}}1 {Ç}{{\c C}}1 {ø}{{\o}}1 {å}{{\r a}}1 {Å}{{\r A}}1
{€}{{\EUR}}1 {£}{{\pounds}}1
%
{^}{{{\color{ipython_purple}\^{}}}}1
{=}{{{\color{ipython_purple}=}}}1
%
{+}{{{\color{ipython_purple}+}}}1
{*}{{{\color{ipython_purple}$^\ast$}}}1
{/}{{{\color{ipython_purple}/}}}1
%
{+=}{{{+=}}}1
{-=}{{{-=}}}1
{*=}{{{$^\ast$=}}}1
{/=}{{{/=}}}1,
literate=
	*{-}{{{\color{ipython_purple}-}}}1
{?}{{{\color{ipython_purple}?}}}1,
%
identifierstyle=\color{black}\ttfamily,
commentstyle=\color{ipython_cyan}\ttfamily,
stringstyle=\color{ipython_red}\ttfamily,
keepspaces=true,
showspaces=false,
showstringspaces=false,
%
rulecolor=\color{ipython_frame},
frame=single,
frameround={t}{t}{t}{t},
framexleftmargin=6mm,
numbers=left,
numberstyle=\tiny\color{halfgray},
%
%
backgroundcolor=\color{ipython_bg},
%   extendedchars=true,
basicstyle=\scriptsize,
keywordstyle=\color{ipython_green}\ttfamily,
}
\lstdefinestyle{DOS}
{
	backgroundcolor=\color{black},
	basicstyle=\scriptsize\color{white}\ttfamily
}

% Additional packages and settings from Math.tex
% quiver style
\usepackage{tikz-cd}
% `calc` is necessary to draw curved arrows.
\usetikzlibrary{calc}
% `pathmorphing` is necessary to draw squiggly arrows.
\usetikzlibrary{decorations.pathmorphing}

% A TikZ style for curved arrows of a fixed height, due to AndréC.
\tikzset{curve/.style={settings={#1},to path={(\tikztostart)
					.. controls ($(\tikztostart)!\pv{pos}!(\tikztotarget)!\pv{height}!270:(\tikztotarget)$)
					and ($(\tikztostart)!1-\pv{pos}!(\tikztotarget)!\pv{height}!270:(\tikztotarget)$)
					.. (\tikztotarget)\tikztonodes}},
	settings/.code={\tikzset{quiver/.cd,#1}
			\def\pv##1{\pgfkeysvalueof{/tikz/quiver/##1}}},
	quiver/.cd,pos/.initial=0.35,height/.initial=0}

% TikZ arrowhead/tail styles.
\tikzset{tail reversed/.code={\pgfsetarrowsstart{tikzcd to}}}
\tikzset{2tail/.code={\pgfsetarrowsstart{Implies[reversed]}}}
\tikzset{2tail reversed/.code={\pgfsetarrowsstart{Implies}}}
% TikZ arrow styles.
\tikzset{no body/.style={/tikz/dash pattern=on 0 off 1mm}}

% useful macro for class
\newcommand{\at}[3]{\left.#1\right\vert_{#2}^{#3}}
\newcommand\quotient[2]{
	\mathchoice
	{% \displaystyle
		\text{\raise1ex\hbox{$#1$}\Big/\lower1ex\hbox{$#2$}}%
	}
	{% \textstyle
		#1\,/\,#2
	}
	{% \scriptstyle
		#1\,/\,#2
	}
	{% \scriptscriptstyle
		#1\,/\,#2
	}
}

\newcommand{\True}{\textsf{True}}
\newcommand{\T}{\textsf{T}}
\newcommand{\False}{\textsf{False}}
\newcommand{\F}{\textsf{F}}
\newcommand{\act}{\rotatebox[origin=c]{-180}{\(\,\circlearrowright\,\)}}

\usepackage{physics}
\usepackage{complexity}

\DeclareMathOperator{\id}{id}
\DeclareMathOperator{\Span}{span}
\DeclareMathOperator{\supp}{supp}
\DeclareMathOperator{\vol}{vol}
\DeclareMathOperator{\dist}{dist}
\DeclareMathOperator{\diam}{diam}
\DeclareMathOperator{\im}{Im}
\DeclareMathOperator{\sgn}{sgn}
\DeclareMathOperator{\Int}{Int}
\DeclareMathOperator{\diag}{diag}
\DeclareMathOperator{\dom}{dom}
\DeclareMathOperator*{\argmax}{arg\,max}
\DeclareMathOperator*{\argmin}{arg\,min}
\DeclareMathOperator{\Var}{Var}
\DeclareMathOperator{\Cov}{Cov}
\DeclareMathOperator{\Corr}{Corr}

\let\implies\Rightarrow
\let\impliedby\Leftarrow
\let\iff\Leftrightarrow

\usepackage{stmaryrd} % for \lightning
\newcommand\conta{\scalebox{1.1}{\(\lightning\)}}

\usepackage{bm}
\usepackage{bbm}

% % figure support
% \usepackage{import}
% \usepackage{xifthen}
% \pdfminorversion=7
% \usepackage{pdfpages}
% \usepackage{transparent}
% \newcommand{\incfig}[1]{%
% 	\def\svgwidth{\columnwidth}
% 	\import{./Figures/}{#1.pdf_tex}
% }
\AtBeginDocument{\RenewCommandCopy\qty\SI}

\endinput
