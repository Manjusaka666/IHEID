\documentclass[a4paper,12pt]{article} % This defines the style of your paper

\usepackage[top = 2.5cm, bottom = 2.5cm, left = 2.5cm, right = 2.5cm]{geometry} 

% Unfortunately, LaTeX has a hard time interpreting German Umlaute. The following two lines and packages should help. If it doesn't work for you please let me know.
\usepackage[T1]{fontenc}
\usepackage[utf8]{inputenc}
\usepackage{pifont}
% \usepackage{ctex}
\usepackage{amsthm, amsmath, amssymb, mathrsfs,mathtools}
% % ---------------------------------------------------------------------
% Program: listings-stata.tex
% Author:  github.com/mcaceresb
% Purpose: Stata language definition for LaTeX listings package
% Usage:   Add \input{listings-stata.tex} to your preamble

% Syntax from
% - https://github.com/isagalaev/highlight.js/blob/master/src/languages/stata.js
% - https://github.com/jpitblado/vim-stata/blob/master/syntax/stata.vim
% - http://fmwww.bc.edu/RePEc/bocode/s/synlightlist.ado

\RequirePackage{listings}
\RequirePackage{color}
\RequirePackage[svgnames]{xcolor}
\definecolor{spRed}{HTML}{BE646C}

% ---------------------------------------------------------------------
% Stata language definition

\lstdefinelanguage{stata}{
  sensitive=true,
  %
  % Macros, global and local
  alsoletter={\{\}0123456789},
  keywordsprefix=\$,
  morecomment=[n][keywordstyle9]{`}{'},
  morekeywords={},
  %
  % Comments
  morecomment=[f][\color{Green}\slshape][0]*,
  morecomment=[l]{//},
  morecomment=[s]{/*}{*/},
  %
  % Strings
  morecomment=[n][\color{Maroon}]{`"}{"'},
  morestring=[b]",
  %
  % Add-ons and system Commands
  morekeywords=[2]{
    if ,else ,in ,foreach ,for ,forv ,forva ,forval ,forvalu ,forvalue
    ,forvalues ,by ,bys ,bysort ,xi ,quietly ,qui ,capture ,about
    ,ac ,ac_7 ,acprplot ,acprplot_7 adjust ,ado ,adopath ,adoupdate
    ,alpha ,ameans ,an ,ano ,anov ,anova ,anova_estat ,anova_terms
    ,anovadef ,aorder ,ap ,app ,appe ,appen ,append ,arch ,arch_dr
    ,arch_estat ,arch_p ,archlm ,areg ,areg_p ,args ,arima ,arima_dr
    ,arima_estat ,arima_p ,as ,asmprobit ,asmprobit_estat ,asmprobit_lf
    ,asmprobit_mfx__dlg ,asmprobit_p ,ass ,asse ,asser ,assert ,avplot
    ,avplot_7 ,avplots ,avplots_7 bcskew0 ,bgodfrey ,binreg ,bip0_lf
    ,biplot ,bipp_lf ,bipr_lf ,bipr_p ,biprobit ,bitest ,bitesti
    ,bitowt ,blogit ,bmemsize ,boot ,bootsamp ,bootstrap ,bootstrap_8
    ,boxco_l ,boxco_p ,boxcox ,boxcox_6 ,boxcox_p ,bprobit ,br ,break
    ,brier ,bro ,brow ,brows ,browse ,brr ,brrstat ,bs ,bs_7 ,bsampl_w
    ,bsample ,bsample_7 ,bsqreg ,bstat ,bstat_7 ,bstat_8 ,bstrap
    ,bstrap_7 ,ca ,ca_estat ,ca_p ,cabiplot ,camat ,canon ,canon_8
    ,canon_8_p ,canon_estat ,canon_p ,cap ,caprojection ,capt ,captu
    ,captur ,capture ,cat ,cc ,cchart ,cchart_7 ,cci ,cd ,censobs_table
    ,centile ,cf ,char ,chdir ,checkdlgfiles ,checkestimationsample
    ,checkhlpfiles ,checksum ,chelp ,ci ,cii ,cl ,class ,classutil
    ,clear ,cli ,clis ,clist ,clo ,clog ,clog_lf ,clog_p ,clogi
    ,clogi_sw ,clogit ,clogit_lf ,clogit_p ,clogitp ,clogl_sw ,cloglog
    ,clonevar ,clslistarray ,cluster ,cluster_measures ,cluster_stop
    ,cluster_tree ,cluster_tree_8 ,clustermat ,cmdlog ,cnr ,cnre
    ,cnreg ,cnreg_p ,cnreg_sw ,cnsreg ,codebook ,collaps4 ,collapse
    ,colormult_nb ,colormult_nw ,compare ,compress ,conf ,confi
    ,confir ,confirm ,conren ,cons ,const ,constr ,constra ,constrai
    ,constrain ,constraint ,continue ,contract ,copy ,copyright
    ,copysource ,cor ,corc ,corr ,corr2data ,corr_anti ,corr_kmo
    ,corr_smc ,corre ,correl ,correla ,correlat ,correlate ,corrgram
    ,cou ,coun ,count ,cox ,cox_p ,cox_sw ,coxbase ,coxhaz ,coxvar
    ,cprplot ,cprplot_7 ,crc ,cret ,cretu ,cretur ,creturn ,cross ,cs
    ,cscript ,cscript_log ,csi ,ct ,ct_is ,ctset ,ctst_5 ,ctst_st
    ,cttost ,cumsp ,cumsp_7 ,cumul ,cusum ,cusum_7 ,cutil ,d ,datasig
    ,datasign ,datasigna ,datasignat ,datasignatu ,datasignatur
    ,datasignature ,datetof ,db ,dbeta ,de ,dec ,deco ,decod ,decode
    ,deff ,des ,desc ,descr ,descri ,describ ,describe ,destring
    ,dfbeta ,dfgls ,dfuller ,di ,di_g ,dir ,dirstats ,dis ,discard
    ,disp ,disp_res ,disp_s ,displ ,displa ,display ,distinct ,do
    ,doe ,doed ,doedi ,doedit ,dotplot ,dotplot_7 ,dprobit ,drawnorm
    ,drop ,ds ,ds_util ,dstdize ,duplicates ,durbina ,dwstat ,dydx ,e
    ,ed ,edi ,edit ,egen ,eivreg ,emdef ,en ,enc ,enco ,encod ,encode
    ,eq ,erase ,ereg ,ereg_lf ,ereg_p ,ereg_sw ,ereghet ,ereghet_glf
    ,ereghet_glf_sh ,ereghet_gp ,ereghet_ilf ,ereghet_ilf_sh ,ereghet_ip
    ,eret ,eretu ,eretur ,ereturn ,err ,erro ,error ,est ,est_cfexist
    ,est_cfname ,est_clickable ,est_expand ,est_hold ,est_table
    ,est_unhold ,est_unholdok ,estat ,estat_default ,estat_summ
    ,estat_vce_only ,esti ,estimates ,etodow ,etof ,etomdy ,ex ,exi
    ,exit ,expand ,expandcl ,fac ,fact ,facto ,factor ,factor_estat
    ,factor_p ,factor_pca_rotated ,factor_rotate ,factormat ,fcast
    ,fcast_compute ,fcast_graph ,fdades ,fdadesc ,fdadescr ,fdadescri
    ,fdadescrib ,fdadescribe ,fdasav ,fdasave ,fdause ,fh_st ,file
    ,open ,file ,read ,file ,close ,file ,filefilter ,fillin
    ,find_hlp_file ,findfile ,findit ,findit_7 ,fit ,fl ,fli ,flis
    ,flist ,for5_0 ,form ,forma ,format ,fpredict ,frac_154 ,frac_adj
    ,frac_chk ,frac_cox ,frac_ddp ,frac_dis ,frac_dv ,frac_in ,frac_mun
    ,frac_pp ,frac_pq ,frac_pv ,frac_wgt ,frac_xo ,fracgen ,fracplot
    ,fracplot_7 ,fracpoly ,fracpred ,fron_ex ,fron_hn ,fron_p ,fron_tn
    ,fron_tn2 ,frontier ,ftodate ,ftoe ,ftomdy ,ftowdate ,g ,gamhet_glf
    ,gamhet_gp ,gamhet_ilf ,gamhet_ip ,gamma ,gamma_d2 ,gamma_p
    ,gamma_sw ,gammahet ,gdi_hexagon ,gdi_spokes ,ge ,gen ,gene ,gener
    ,genera ,generat ,generate ,genrank ,genstd ,genvmean ,gettoken
    ,gl ,gladder ,gladder_7 ,glim_l01 ,glim_l02 glim_l03 ,glim_l04
    ,glim_l05 ,glim_l06 ,glim_l07 ,glim_l08 ,glim_l09 ,glim_l10 glim_l11
    ,glim_l12 ,glim_lf ,glim_mu ,glim_nw1 ,glim_nw2 ,glim_nw3 ,glim_p
    ,glim_v1 ,glim_v2 ,glim_v3 ,glim_v4 ,glim_v5 ,glim_v6 ,glim_v7 ,glm
    ,glm_6 glm_p ,glm_sw ,glmpred ,glo ,glob ,globa ,global ,glogit
    ,glogit_8 ,glogit_p ,gmeans ,gnbre_lf ,gnbreg ,gnbreg_5 ,gnbreg_p
    ,gomp_lf ,gompe_sw ,gomper_p ,gompertz ,gompertzhet ,gomphet_glf
    ,gomphet_glf_sh ,gomphet_gp ,gomphet_ilf ,gomphet_ilf_sh ,gomphet_ip
    ,gphdot ,gphpen ,gphprint ,gprefs ,gprobi_p ,gprobit ,gprobit_8
    ,gr ,gr7 ,gr_copy ,gr_current ,gr_db ,gr_describe ,gr_dir ,gr_draw
    ,gr_draw_replay ,gr_drop ,gr_edit ,gr_editviewopts ,gr_example
    ,gr_example2 gr_export ,gr_print ,gr_qscheme ,gr_query ,gr_read
    ,gr_rename ,gr_replay ,gr_save ,gr_set ,gr_setscheme ,gr_table
    ,gr_undo ,gr_use ,graph ,graph7 grebar ,greigen ,greigen_7
    ,greigen_8 ,grmeanby ,grmeanby_7 ,gs_fileinfo ,gs_filetype
    ,gs_graphinfo ,gs_stat ,gsort ,gwood ,h ,hadimvo ,hareg ,hausman
    ,haver ,he ,heck_d2 ,heckma_p ,heckman ,heckp_lf ,heckpr_p ,heckprob
    ,hel ,help ,hereg ,hetpr_lf ,hetpr_p ,hetprob ,hettest ,hexdump
    ,hilite ,hist ,hist_7 histogram ,hlogit ,hlu ,hmeans ,hotel
    ,hotelling ,hprobit ,hreg ,hsearch ,icd9 ,icd9_ff ,icd9p ,iis
    ,impute ,imtest ,inbase ,include ,inf ,infi ,infil ,infile ,infix
    ,inp ,inpu ,input ,ins ,insheet ,insp ,inspe ,inspec ,inspect ,integ
    ,inten ,intreg ,intreg_7 ,intreg_p ,intrg2_ll ,intrg_ll ,intrg_ll2
    ,ipolate ,iqreg ,ir ,irf ,irf_create ,irfm ,iri ,is_svy ,is_svysum
    ,isid ,istdize ,ivprob_1_lf ,ivprob_lf ,ivprobit ,ivprobit_p ,ivreg
    ,ivreg_footnote ,ivtob_1_lf ,ivtob_lf ,ivtobit ,ivtobit_p ,jackknife
    ,jacknife ,jknife ,jknife_6 ,jknife_8 ,jkstat ,joinby ,kalarma1
    ,kap ,kap_3 ,kapmeier ,kappa ,kapwgt ,kdensity ,kdensity_7 keep
    ,ksm ,ksmirnov ,ktau ,kwallis ,l ,la ,lab ,labe ,label ,labelbook
    ,ladder ,levels ,levelsof ,leverage ,lfit ,lfit_p ,li ,lincom ,line
    ,linktest ,lis ,list ,lloghet_glf ,lloghet_glf_sh ,lloghet_gp
    ,lloghet_ilf ,lloghet_ilf_sh ,lloghet_ip ,llogi_sw ,llogis_p
    ,llogist ,llogistic ,llogistichet ,lnorm_lf ,lnorm_sw ,lnorma_p
    ,lnormal ,lnormalhet ,lnormhet_glf ,lnormhet_glf_sh ,lnormhet_gp
    ,lnormhet_ilf ,lnormhet_ilf_sh ,lnormhet_ip ,lnskew0 ,loadingplot
    ,loc ,loca ,local ,log ,logi ,logis_lf ,logistic ,logistic_p
    ,logit ,logit_estat ,logit_p ,loglogs ,logrank ,loneway ,lookfor
    ,lookup ,lowess ,lowess_7 ,lpredict ,lrecomp ,lroc ,lroc_7 ,lrtest
    ,ls ,lsens ,lsens_7 ,lsens_x ,lstat ,ltable ,ltable_7 ,ltriang
    ,lv ,lvr2plot ,lvr2plot_7 ,m ,ma ,mac ,macr ,macro ,makecns ,man
    ,manova ,manova_estat ,manova_p ,manovatest ,mantel ,mark ,markin
    ,markout ,marksample ,mat ,mat_capp ,mat_order ,mat_put_rr ,mat_rapp
    ,mata ,mata_clear ,mata_describe ,mata_drop ,mata_matdescribe
    ,mata_matsave ,mata_matuse ,mata_memory ,mata_mlib ,mata_mosave
    ,mata_rename ,mata_which ,matalabel ,matcproc ,matlist ,matname
    ,matr ,matri ,matrix ,matrix_input__dlg ,matstrik ,mcc ,mcci ,md0_
    ,md1_ ,md1debug_ ,md2_ ,md2debug_ ,mds ,mds_estat ,mds_p ,mdsconfig
    ,mdslong ,mdsmat ,mdsshepard ,mdytoe ,mdytof ,me_derd ,mean ,means
    ,median ,memory ,memsize ,meqparse ,mer ,merg ,merge ,mfp ,mfx
    ,mhelp ,mhodds ,minbound ,mixed_ll ,mixed_ll_reparm ,mkassert
    ,mkdir ,mkmat ,mkspline ,ml ,ml_5 ml_adjs ,ml_bhhhs ,ml_c_d
    ,ml_check ,ml_clear ,ml_cnt ,ml_debug ,ml_defd ,ml_e0 ml_e0_bfgs
    ,ml_e0_cycle ,ml_e0_dfp ,ml_e0i ,ml_e1 ,ml_e1_bfgs ,ml_e1_bhhh
    ,ml_e1_cycle ,ml_e1_dfp ,ml_e2 ,ml_e2_cycle ,ml_ebfg0 ,ml_ebfr0
    ,ml_ebfr1 ml_ebh0q ,ml_ebhh0 ,ml_ebhr0 ,ml_ebr0i ,ml_ecr0i ,ml_edfp0
    ,ml_edfr0 ,ml_edfr1 ml_edr0i ,ml_eds ,ml_eer0i ,ml_egr0i ,ml_elf
    ,ml_elf_bfgs ,ml_elf_bhhh ,ml_elf_cycle ,ml_elf_dfp ,ml_elfi
    ,ml_elfs ,ml_enr0i ,ml_enrr0 ,ml_erdu0 ml_erdu0_bfgs ,ml_erdu0_bhhh
    ,ml_erdu0_bhhhq ,ml_erdu0_cycle ,ml_erdu0_dfp ,ml_erdu0_nrbfgs
    ,ml_exde ,ml_footnote ,ml_geqnr ,ml_grad0 ,ml_graph ,ml_hbhhh
    ,ml_hd0 ,ml_hold ,ml_init ,ml_inv ,ml_log ,ml_max ,ml_mlout
    ,ml_mlout_8 ,ml_model ,ml_nb0 ,ml_opt ,ml_p ,ml_plot ,ml_query
    ,ml_rdgrd ,ml_repor ,ml_s_e ,ml_score ,ml_searc ,ml_technique
    ,ml_unhold ,mleval ,mlf_ ,mlmatbysum ,mlmatsum ,mlog ,mlogi ,mlogit
    ,mlogit_footnote ,mlogit_p ,mlopts ,mlsum ,mlvecsum ,mnl0_ ,mor
    ,more ,mov ,move ,mprobit ,mprobit_lf ,mprobit_p ,mrdu0_ ,mrdu1_
    ,mvdecode ,mvencode ,mvreg ,mvreg_estat ,n ,nbreg ,nbreg_al
    ,nbreg_lf ,nbreg_p ,nbreg_sw ,nestreg ,net ,newey ,newey_7 ,newey_p
    ,news ,nl ,nl_7 ,nl_9 ,nl_9_p ,nl_p ,nl_p_7 nlcom ,nlcom_p ,nlexp2
    ,nlexp2_7 ,nlexp2a ,nlexp2a_7 ,nlexp3 ,nlexp3_7 ,nlgom3 nlgom3_7
    ,nlgom4 ,nlgom4_7 ,nlinit ,nllog3 ,nllog3_7 ,nllog4 ,nllog4_7
    ,nlog_rd ,nlogit ,nlogit_p ,nlogitgen ,nlogittree ,nlpred ,no
    ,nobreak ,noi ,nois ,noisi ,noisil ,noisily ,note ,notes ,notes_dlg
    ,nptrend ,numlabel ,numlist ,odbc ,old_ver ,olo ,olog ,ologi
    ,ologi_sw ,ologit ,ologit_p ,ologitp ,on ,one ,onew ,onewa ,oneway
    ,op_colnm ,op_comp ,op_diff ,op_inv ,op_str ,opr ,opro ,oprob
    ,oprob_sw ,oprobi ,oprobi_p ,oprobit ,oprobitp ,opts_exclusive
    ,order ,orthog ,orthpoly ,ou ,out ,outf ,outfi ,outfil ,outfile
    ,outs ,outsh ,outshe ,outshee ,outsheet ,ovtest ,pac ,pac_7 ,palette
    ,parse ,parse_dissim ,pause ,pca ,pca_8 pca_display ,pca_estat
    ,pca_p ,pca_rotate ,pcamat ,pchart ,pchart_7 ,pchi ,pchi_7 ,pcorr
    ,pctile ,pentium ,pergram ,pergram_7 ,permute ,permute_8 ,personal
    ,peto_st ,pkcollapse ,pkcross ,pkequiv ,pkexamine ,pkexamine_7
    ,pkshape ,pksumm ,pksumm_7 ,pl ,plo ,plot ,plugin ,pnorm ,pnorm_7
    ,poisgof ,poiss_lf ,poiss_sw ,poisso_p ,poisson ,poisson_estat
    ,post ,postclose ,postfile ,postutil ,pperron ,pr ,prais ,prais_e
    ,prais_e2 ,prais_p ,predict ,predictnl ,preserve ,print ,pro ,prob
    ,probi ,probit ,probit_estat ,probit_p ,proc_time ,procoverlay
    ,procrustes ,procrustes_estat ,procrustes_p ,profiler ,prog ,progr
    ,progra ,program ,prop ,proportion ,prtest ,prtesti ,pwcorr ,pwd
    ,q ,s ,qby ,qbys ,qchi ,qchi_7 ,qladder ,qladder_7 ,qnorm ,qnorm_7
    ,qqplot ,qqplot_7 ,qreg ,qreg_c ,qreg_p ,qreg_sw ,qu ,quadchk
    ,quantile ,quantile_7 ,que ,quer ,query ,range ,ranksum ,ratio
    ,rchart ,rchart_7 ,rcof ,recast ,reclink ,recode ,reg ,reg3
    ,reg3_p ,regdw ,regr ,regre ,regre_p2 ,regres ,regres_p ,regress
    ,regress_estat ,regriv_p ,remap ,ren ,rena ,renam ,rename ,renpfix
    ,repeat ,replace ,report ,reshape ,restore ,ret ,retu ,retur ,return
    ,rm ,rmdir ,robvar ,roccomp ,roccomp_7 ,roccomp_8 ,rocf_lf ,rocfit
    ,rocfit_8 ,rocgold ,rocplot ,rocplot_7 ,roctab ,roctab_7 ,rolling
    ,rologit ,rologit_p ,rot ,rota ,rotat ,rotate ,rotatemat ,rreg
    ,rreg_p ,ru ,run ,runtest ,rvfplot ,rvfplot_7 ,rvpplot ,rvpplot_7
    ,sa ,safesum ,sample ,sampsi ,sav ,save ,savedresults ,saveold ,sc
    ,sca ,scal ,scala ,scalar ,scatter ,scm_mine ,sco ,scob_lf ,scob_p
    ,scobi_sw ,scobit ,scor ,score ,scoreplot ,scoreplot_help ,scree
    ,screeplot ,screeplot_help ,sdtest ,sdtesti ,se ,search ,separate
    ,seperate ,serrbar ,serrbar_7 ,serset ,set ,set_defaults ,sfrancia
    ,sh ,she ,shel ,shell ,shewhart ,shewhart_7 ,signestimationsample
    ,signrank ,signtest ,simul ,simul_7 simulate ,simulate_8 ,sktest
    ,sleep ,slogit ,slogit_d2 ,slogit_p ,smooth ,snapspan ,so ,sor
    ,sort ,spearman ,spikeplot ,spikeplot_7 ,spikeplt ,spline_x ,split
    ,sqreg ,sqreg_p ,sret ,sretu ,sretur ,sreturn ,ssc ,st ,st_ct ,st_hc
    ,st_hcd ,st_hcd_sh ,st_is ,st_issys ,st_note ,st_promo ,st_set
    ,st_show ,st_smpl ,st_subid ,stack ,statsby ,statsby_8 ,stbase
    ,stci ,stci_7 ,stcox ,stcox_estat ,stcox_fr ,stcox_fr_ll ,stcox_p
    ,stcox_sw ,stcoxkm ,stcoxkm_7 ,stcstat ,stcurv ,stcurve ,stcurve_7
    ,stdes ,stem ,stepwise ,stereg ,stfill ,stgen ,stir ,stjoin ,stmc
    ,stmh ,stphplot ,stphplot_7 ,stphtest ,stphtest_7 ,stptime ,strate
    ,strate_7 ,streg ,streg_sw ,streset ,sts ,sts_7 ,stset ,stsplit
    ,stsum ,sttocc ,sttoct ,stvary ,stweib ,su ,suest ,suest_8 ,sum
    ,summ ,summa ,summar ,summari ,summariz ,summarize ,sunflower
    ,sureg ,survcurv ,survsum ,svar ,svar_p ,svmat ,svy ,svy_disp
    ,svy_dreg ,svy_est ,svy_est_7 ,svy_estat ,svy_get ,svy_gnbreg_p
    ,svy_head ,svy_header ,svy_heckman_p ,svy_heckprob_p ,svy_intreg_p
    ,svy_ivreg_p ,svy_logistic_p ,svy_logit_p ,svy_mlogit_p ,svy_nbreg_p
    ,svy_ologit_p ,svy_oprobit_p ,svy_poisson_p ,svy_probit_p
    ,svy_regress_p ,svy_sub ,svy_sub_7 ,svy_x ,svy_x_7 ,svy_x_p ,svydes
    ,svydes_8 ,svygen ,svygnbreg ,svyheckman ,svyheckprob ,svyintreg
    ,svyintreg_7 ,svyintrg ,svyivreg ,svylc ,svylog_p ,svylogit
    ,svymarkout ,svymarkout_8 ,svymean ,svymlog ,svymlogit ,svynbreg
    ,svyolog ,svyologit ,svyoprob ,svyoprobit ,svyopts ,svypois
    ,svypois_7 svypoisson ,svyprobit ,svyprobt ,svyprop ,svyprop_7
    ,svyratio ,svyreg ,svyreg_p ,svyregress ,svyset ,svyset_7 ,svyset_8
    ,svytab ,svytab_7 ,svytest ,svytotal ,sw ,sw_8 ,swcnreg ,swcox
    ,swereg ,swilk ,swlogis ,swlogit ,swologit ,swoprbt ,swpois
    ,swprobit ,swqreg ,swtobit ,swweib ,symmetry ,symmi ,symplot
    ,symplot_7 syntax ,sysdescribe ,sysdir ,sysuse ,szroeter ,ta ,tab
    ,tab1 ,tab2 ,tab_or ,tabd ,tabdi ,tabdis ,tabdisp ,tabi ,table
    ,tabodds ,tabodds_7 ,tabstat ,tabu ,tabul ,tabula ,tabulat ,tabulate
    ,te ,tempfile ,tempname ,tempvar ,tes ,test ,testnl ,testparm
    ,teststd ,tetrachoric ,time_it ,timer ,tis ,tob ,tobi ,tobit
    ,tobit_p ,tobit_sw ,token ,tokeni ,tokeniz ,tokenize ,tostring
    ,total ,translate ,translator ,transmap ,treat_ll ,treatr_p
    ,treatreg ,trim ,trnb_cons ,trnb_mean ,trpoiss_d2 ,trunc_ll
    ,truncr_p ,truncreg ,tsappend ,tset ,tsfill ,tsline ,tsline_ex
    ,tsreport ,tsrevar ,tsrline ,tsset ,tssmooth ,tsunab ,ttest
    ,ttesti ,tut_chk ,tut_wait ,tutorial ,tw ,tware_st ,two ,twoway
    ,twoway__fpfit_serset ,twoway__function_gen ,twoway__histogram_gen
    ,twoway__ipoint_serset ,twoway__ipoints_serset ,twoway__kdensity_gen
    ,twoway__lfit_serset ,twoway__normgen_gen ,twoway__pci_serset
    ,twoway__qfit_serset ,twoway__scatteri_serset ,twoway__sunflower_gen
    ,twoway_ksm_serset ,ty ,typ ,type ,typeof ,u ,unab ,unabbrev
    ,unabcmd ,update ,us ,use ,uselabel ,var ,var_mkcompanion
    ,var_p ,varbasic ,varfcast ,vargranger ,varirf ,varirf_add
    ,varirf_cgraph ,varirf_create ,varirf_ctable ,varirf_describe
    ,varirf_dir ,varirf_drop ,varirf_erase ,varirf_graph ,varirf_ograph
    ,varirf_rename ,varirf_set ,varirf_table ,varlist ,varlmar
    ,varnorm ,varsoc ,varstable ,varstable_w ,varstable_w2 ,varwle
    ,vce ,vec ,vec_fevd ,vec_mkphi ,vec_p ,vec_p_w ,vecirf_create
    ,veclmar ,veclmar_w ,vecnorm ,vecnorm_w ,vecrank ,vecstable
    ,verinst ,vers ,versi ,versio ,version ,view ,viewsource ,vif
    ,vwls ,wdatetof ,webdescribe ,webseek ,webuse ,weib1_lf ,weib2_lf
    ,weib_lf ,weib_lf0 weibhet_glf ,weibhet_glf_sh ,weibhet_glfa
    ,weibhet_glfa_sh ,weibhet_gp ,weibhet_ilf ,weibhet_ilf_sh
    ,weibhet_ilfa ,weibhet_ilfa_sh ,weibhet_ip ,weibu_sw ,weibul_p
    ,weibull ,weibull_c ,weibull_s ,weibullhet ,wh ,whelp ,whi ,which
    ,whil ,while ,wilc_st ,wilcoxon ,win ,wind ,windo ,window ,winexec
    ,wntestb ,wntestb_7 ,wntestq ,xchart ,xchart_7 ,xcorr ,xcorr_7 ,xi
    ,xi_6 ,xmlsav ,xmlsave ,xmluse ,xpose ,xsh ,xshe ,xshel ,xshell
    ,xt_iis ,xt_tis ,xtab_p ,xtabond ,xtbin_p ,xtclog ,xtcloglog
    ,xtcloglog_8 ,xtcloglog_d2 ,xtcloglog_pa_p ,xtcloglog_re_p ,xtcnt_p
    ,xtcorr ,xtdata ,xtdes ,xtfront_p ,xtfrontier ,xtgee ,xtgee_elink
    ,xtgee_estat ,xtgee_makeivar ,xtgee_p ,xtgee_plink ,xtgls ,xtgls_p
    ,xthaus ,xthausman ,xtht_p ,xthtaylor ,xtile ,xtint_p ,xtintreg
    ,xtintreg_8 ,xtintreg_d2 xtintreg_p ,xtivp_1 ,xtivp_2 ,xtivreg
    ,xtline ,xtline_ex ,xtlogit ,xtlogit_8 xtlogit_d2 ,xtlogit_fe_p
    ,xtlogit_pa_p ,xtlogit_re_p ,xtmixed ,xtmixed_estat ,xtmixed_p
    ,xtnb_fe ,xtnb_lf ,xtnbreg ,xtnbreg_pa_p ,xtnbreg_refe_p ,xtpcse
    ,xtpcse_p ,xtpois ,xtpoisson ,xtpoisson_d2 ,xtpoisson_pa_p
    ,xtpoisson_refe_p ,xtpred ,xtprobit ,xtprobit_8 ,xtprobit_d2
    ,xtprobit_re_p ,xtps_fe ,xtps_lf ,xtps_ren ,xtps_ren_8 ,xtrar_p
    ,xtrc ,xtrc_p ,xtrchh ,xtrefe_p ,xtreg ,xtreg_be ,xtreg_fe
    ,xtreg_ml ,xtreg_pa_p ,xtreg_re ,xtregar ,xtrere_p ,xtset
    ,xtsf_ll ,xtsf_llti ,xtsum ,xttab ,xttest0 ,xttobit ,xttobit_8
    ,xttobit_p ,xttrans ,yx ,yxview__barlike_draw ,yxview_area_draw
    ,yxview_bar_draw ,yxview_dot_draw ,yxview_dropline_draw
    ,yxview_function_draw ,yxview_iarrow_draw ,yxview_ilabels_draw
    ,yxview_normal_draw ,yxview_pcarrow_draw ,yxview_pcbarrow_draw
    ,yxview_pccapsym_draw ,yxview_pcscatter_draw ,yxview_pcspike_draw
    ,yxview_rarea_draw ,yxview_rbar_draw ,yxview_rbarm_draw
    ,yxview_rcap_draw ,yxview_rcapsym_draw ,yxview_rconnected_draw
    ,yxview_rline_draw ,yxview_rscatter_draw ,yxview_rspike_draw
    ,yxview_spike_draw ,yxview_sunflower_draw ,zap_s ,zinb ,zinb_llf
    ,zinb_plf ,zip ,zip_llf ,zip_p ,zip_plf ,zt_ct_5 ,zt_hc_5 ,zt_hcd_5
    ,zt_is_5 ,zt_iss_5 ,zt_sho_5 zt_smp_5 ,ztbase_5 ,ztcox_5 ,ztdes_5
    ,ztereg_5 ,ztfill_5 ,ztgen_5 ,ztir_5 ztjoin_5 ,ztnb ,ztnb_p ,ztp
    ,ztp_p ,zts_5 ,ztset_5 ,ztspli_5 ,ztsum_5 ,zttoct_5 ztvary_5
    ,ztweib_5
  },
  %
  % Built-in functions
  morekeywords=[3]{
    Cdhms ,Chms ,Clock ,Cmdyhms ,Cofc ,Cofd ,F ,Fden ,Ftail ,I ,J
    ,_caller ,abbrev ,abs ,acos ,acosh ,asin ,asinh ,atan ,atan2
    ,atanh ,autocode ,betaden ,binomial ,binomialp ,binomialtail
    ,binormal ,bofd ,byteorder ,c ,ceil ,char ,chi2 ,chi2den ,chi2tail
    ,cholesky ,chop ,clip ,clock ,cloglog ,cofC ,cofd ,colnumb ,colsof
    ,comb ,cond ,corr ,cos ,cosh ,d ,daily ,date ,day ,det ,dgammapda
    ,dgammapdada ,dgammapdadx ,dgammapdx ,dgammapdxdx ,dhms ,diag
    ,diag0cnt ,digamma ,dofC ,dofb ,dofc ,dofh ,dofm ,dofq ,dofw ,dofy
    ,dow ,doy ,dunnettprob ,e ,el ,epsdouble ,epsfloat ,exp ,fileexists
    ,fileread ,filereaderror ,filewrite ,float ,floor ,fmtwidth
    ,gammaden ,gammap ,gammaptail ,get ,group ,h ,hadamard ,halfyear
    ,halfyearly ,has_eprop ,hh ,hhC ,hms ,hofd ,hours ,hypergeometric
    ,hypergeometricp ,ibeta ,ibetatail ,index ,indexnot ,inlist
    ,inrange ,int ,inv ,invF ,invFtail ,invbinomial ,invbinomialtail
    ,invchi2 ,invchi2tail ,invcloglog ,invdunnettprob ,invgammap
    ,invgammaptail ,invibeta ,invibetatail ,invlogit ,invnFtail
    ,invnbinomial ,invnbinomialtail ,invnchi2 ,invnchi2tail ,invnibeta
    ,invnorm ,invnormal ,invnttail ,invpoisson ,invpoissontail ,invsym
    ,invt ,invttail ,invtukeyprob ,irecode ,issym ,issymmetric ,itrim
    ,length ,ln ,lnfact ,lnfactorial ,lngamma ,lnnormal ,lnnormalden
    ,log ,log10 ,logit ,lower ,ltrim ,m ,match ,matmissing ,matrix
    ,matuniform ,max ,maxbyte ,maxdouble ,maxfloat ,maxint ,maxlong ,mdy
    ,mdyhms ,mi ,mi ,min ,minbyte ,mindouble ,minfloat ,minint ,minlong
    ,minutes ,missing ,mm ,mmC ,mod ,mofd ,month ,monthly ,mreldif
    ,msofhours ,msofminutes ,msofseconds ,nF ,nFden ,nFtail ,nbetaden
    ,nbinomial ,nbinomialp ,nbinomialtail ,nchi2 ,nchi2den ,nchi2tail
    ,nibeta ,norm ,normal ,normalden ,normd ,npnF ,npnchi2 ,npnt ,nt
    ,ntden ,nttail ,nullmat ,plural ,poisson ,poissonp ,poissontail
    ,proper ,q ,qofd ,quarter ,quarterly ,r ,rbeta ,rbinomial ,rchi2
    real ,recode ,regexm ,regexr ,regexs ,reldif ,replay ,return
    ,reverse ,rgamma ,rhypergeometric ,rnbinomial ,rnormal ,round
    ,rownumb ,rowsof ,rpoisson ,rt ,rtrim ,runiform ,s ,scalar ,seconds
    ,sign ,sin ,sinh ,smallestdouble ,soundex ,soundex_nara ,sqrt ,ss
    ,ssC ,strcat ,strdup ,string ,strlen ,strlower ,strltrim ,strmatch
    ,strofreal ,strpos ,strproper ,strreverse ,strrtrim ,strtoname
    ,strtrim ,strupper ,subinstr ,subinword ,substr ,sum ,sweep ,syminv
    ,t ,tC ,tan ,tanh ,tc ,td ,tden ,th ,tin ,tm ,tq ,trace ,trigamma
    ,trim ,trunc ,ttail ,tukeyprob ,tw ,twithin ,uniform ,upper ,vec
    ,vecdiag ,w ,week ,weekly ,wofd ,word ,wordcount ,year ,yearly
    ,yh ,ym ,yofd ,yq ,yw
  },
  %
  % Numbers
  morekeywords=[4]{
    0 ,1 ,2 ,3 ,4 ,5 ,6 ,7 ,8 ,9
  },
}

% ---------------------------------------------------------------------
% Stata editor style

\providecommand{\textcolordummy}[2]{#2}
\lstalias{Stata}{stata}
\lstdefinestyle{stata-editor}{
    language=stata,
    %
    % Global variables
    keywordstyle={\bfseries\color{spRed}},
    %
    % Add-ons system commands
    keywordstyle=[2]{\bfseries\color{NavyBlue}},
    %
    % Built-in functions
    keywordstyle=[3]{\color{blue}},
    %
    % Numbers
    keywordstyle=[4]{\color{blue}},
    %
    % User macros (variables)
    keywordstyle = [9]{\bfseries\color{LightSteelBlue}\let\textcolor\textcolordummy},
    %
    % Strings and comments
    stringstyle  = \color{Maroon},
    commentstyle = \color{Green}\slshape,
}

% ---------------------------------------------------------------------
% Suggested settings

% \lstset{
%   basicstyle        = \setmonofont{DejaVu Sans Mono}\footnotesize\ttfamily,
%   tabsize           = 4,      % Tab size
%   showstringspaces  = false,  % Don't underline spaces in strings
%   showspaces        = false,  % Don't underline spaces
%   breaklines        = true,   % Automatic line breaking
%   breakatwhitespace = true,   % Breaks only at white space.
%   lineskip          = 1.5pt,  % Sparing between lines of code
%   commentstyle      = \color{black!50}\itshape \let\textcolor\textcolordummy,
% }

% Defining a new theorem style without italics
\newtheoremstyle{nonitalic}% name
  {\topsep}% Space above
  {\topsep}% Space below
  {\upshape}% Body font
  {}% Indent amount
  {\bfseries}% Theorem head font
  {.}% Punctuation after theorem head
  {.5em}% Space after theorem head
  {}% Theorem head spec (can be left empty, meaning ‘normal`)
  
\theoremstyle{nonitalic}
% Define new 'solution' environment
\newtheorem{innercustomsol}{Solution}

% the outer wrapper that takes an argument
\newenvironment{solution}[1]
  {\renewcommand\theinnercustomsol{#1}%
   \innercustomsol}
  {\endinnercustomsol}

% now solutionctr will reset whenever section increments
\newcounter{solutionctr}[section]
\renewcommand{\thesolutionctr}{(\alph{solutionctr})}

% and the autosolution environment steps that counter
\newenvironment{autosolution}
  {\refstepcounter{solutionctr}%
   \begin{solution}{\thesolutionctr}}
  {\end{solution}}
\makeatother


\newtheorem{problem}{Problem}
\usepackage{color}

% The following two packages - multirow and booktabs - are needed to create nice looking tables.
\usepackage{multirow} % Multirow is for tables with multiple rows within one cell.
\usepackage{booktabs} % For even nicer tables.

% As we usually want to include some plots (.pdf files) we need a package for that.
\usepackage{graphicx} 
\usepackage{subfigure}


% The default setting of LaTeX is to indent new paragraphs. This is useful for articles. But not really nice for homework problem sets. The following command sets the indent to 0.
\usepackage{setspace}
\setlength{\parindent}{0in}
\usepackage{longtable}

% Package to place figures where you want them.
\usepackage{float}

% The fancyhdr package let's us create nice headers.
\usepackage{fancyhdr}

\usepackage{fancyvrb}

%Code environment 
\usepackage{listings} % Required for insertion of code
\usepackage{xcolor} % Required for custom colors

% Define colors for code listing
\definecolor{codegreen}{rgb}{0,0.6,0}
\definecolor{codegray}{rgb}{0.5,0.5,0.5}
\definecolor{codepurple}{rgb}{0.58,0,0.82}
\definecolor{backcolour}{rgb}{0.95,0.95,0.92}

% Code listing style named "mystyle"
\lstdefinestyle{mystyle}{
    backgroundcolor=\color{backcolour},   
    commentstyle=\color{codegreen},
    keywordstyle=\color{magenta},
    numberstyle=\tiny\color{codegray},
    stringstyle=\color{codepurple},
    basicstyle=\ttfamily\footnotesize, % Change to serif font
    breakatwhitespace=false,         
    breaklines=true,                 
    captionpos=b,                    
    keepspaces=true,                 
    numbers=left,                    
    numbersep=5pt,                  
    showspaces=false,                
    showstringspaces=false,
    showtabs=false,                  
    tabsize=2
}

\lstset{style=mystyle}

%%%%%%%%%%%%%%%%%%%%%%%%%%%%%%%%%%%%%%%%%%%%%%%%
% 3. Header (and Footer)
%%%%%%%%%%%%%%%%%%%%%%%%%%%%%%%%%%%%%%%%%%%%%%%%

% To make our document nice we want a header and number the pages in the footer.

\pagestyle{fancy} % With this command we can customize the header style.

\fancyhf{} % This makes sure we do not have other information in our header or footer.

\lhead{\footnotesize EI035 Econometrics II}% \lhead puts text in the top left corner. \footnotesize sets our font to a smaller size.

%\rhead works just like \lhead (you can also use \chead)
\rhead{\footnotesize Jingle Fu} %<---- Fill in your lastnames.

% Similar commands work for the footer (\lfoot, \cfoot and \rfoot).
% We want to put our page number in the center.
\cfoot{\footnotesize \thepage}
\IfFileExists{upquote.sty}{\usepackage{upquote}}{}

\begin{document}

\thispagestyle{empty} % This command disables the header on the first page. 

\begin{tabular}{p{15.5cm}} % This is a simple tabular environment to align your text nicely 
{\large \bf EI035 Econometrics II} \\
The Graduate Institute, Spring 2025, Marko Mlikota\\
\hline % \hline produces horizontal lines.
\\
\end{tabular} % Our tabular environment ends here.

\vspace*{0.3cm} % Now we want to add some vertical space in between the line and our title.

\begin{center} % Everything within the center environment is centered.
	{\Large \bf PS5 Solutions} % <---- Don't forget to put in the right number
	\vspace{2mm}
	
        % YOUR NAMES GO HERE
	{\bf Jingle Fu} % <---- Fill in your names here!
		
\end{center}  

\vspace{0.4cm}
\setstretch{1.2}

We use the following notation:
\[
\Phi_1 =
\begin{bmatrix}
\phi_{11} & \phi_{12}\\
\phi_{21} & \phi_{22}
\end{bmatrix},\quad
\Phi_{\varepsilon} =
\begin{bmatrix}
b_{11} & b_{12}\\
b_{21} & b_{22}
\end{bmatrix},\quad
\Sigma_u := \mathbb{V}[u_t]=\mathbb{E}[u_t u_t^{\prime} ] = \Phi_{\varepsilon}\Phi_{\varepsilon}'.
\]

\begin{autosolution}
    \

    The VAR(1) is weakly stationary iff all eigenvalues of $\Phi_1$ lie strictly inside the unit circle,
    \[
    \rho(\Phi_1)=\max_i\bigl|\lambda_i(\Phi_1)\bigr|<1.
    \]
\end{autosolution}

\begin{autosolution}
    \

    Assuming $y_t$ is stationary, $\mathbb{E}[y_t] = \mathbb{E}[\Phi_1 y_{t-1} + \Phi_{\varepsilon} \varepsilon_t] = \Phi_1 \mathbb{E}[y_{t-1}]$,
    as $\mathbb{E}[y_t] = \mu, \forall t$, $\mu =0$.
    \begin{align*}
        \Gamma_{yy}(0) = \mathbb{E}[y_t y_t'] &= \mathbb{E}\left[(\Phi_1 y_{t-1} + u_t)(\Phi_1 y_{t-1} + u_t)' \right] \\
        &= \mathbb{E}\left[ \Phi_1 y_{t-1} y_{t-1}' \Phi_1' + \Phi_1 y_{t-1} u_t' + u_t y_{t-1}' \Phi_1' + u_t u_t' \right] \\
        &= \Phi_1 \mathbb{E}[y_{t-1} y_{t-1}'] \Phi_1' + \Phi_1 \mathbb{E}[y_{t-1} u_t'] + \mathbb{E}[u_t y_{t-1}'] \Phi_1' + \mathbb{E}[u_t u_t'] \\
        &= \Phi_1 \Gamma_{yy}(0) \Phi_1' + \mathbb{E}[u_t u_t'] \\
        &= \Phi_1 \Gamma_{yy}(0) \Phi_1' + \Sigma_u
    \end{align*}
    Since $u_t$ contains contemporaneous shocks $\varepsilon_t$ which are independent of past $y_{t-1}$,
    $\mathbb{E}[y_{t-s} u_t']=0$ for $s \ge 0$, $\mathbb{E}[y_{t-1} u_t'] = 0$ and $\mathbb{E}[u_t y_{t-1}'] = 0$.
    As $\Sigma_u^{\prime} = \left( \Phi_{\varepsilon} \Phi_{\varepsilon}^{\prime} \right)^{\prime} = \Phi_{\varepsilon} \Phi_{\varepsilon}^{\prime} = \Sigma_u$ is Hermitian,
    this is a discrete Lyapunov equation, which can be solved for $\Gamma_{yy}(0)$ using the vectorization operator:
    \[ \operatorname{vec}(\Gamma_{yy}(0)) = (I_{k^2} - \Phi_1 \otimes \Phi_1)^{-1} \operatorname{vec}(\Sigma_u) \]
    where $k=2$ is the dimension of $y_t$ (so $k^2=4$), and $\otimes$ is the Kronecker product.
    \begin{align*}
        \Gamma_{yy}(1) = \mathbb{E}[y_t y_{t-1}'] &= \mathbb{E}[(\Phi_1 y_{t-1} + u_t)y_{t-1}'] \\
        &= \Phi_1 \mathbb{E}[y_{t-1} y_{t-1}'] + \mathbb{E}[u_t y_{t-1}']
    \end{align*}
    Again, $\mathbb{E}[u_t y_{t-1}'] = 0$.
    So, $\Gamma_{yy}(1) = \Phi_1 \Gamma_{yy}(0)$.
\end{autosolution}

\begin{autosolution}
    \

    Starting from $y_t = \Phi_1 y_{t-1} + \Phi_{\varepsilon} \varepsilon_t$.
    By repeated substitution, we can write $y_t$ in its MA($\infty$) representation (assuming stationarity):
    \[ y_t = \sum_{j=0}^{\infty} \Phi_1^j \Phi_{\varepsilon} \varepsilon_{t-j} + \lim_{k \to \infty} \Phi_1^k y_{t-k}. \]
    (Here $\Phi_1^0 = I_k$, where $k=2$).
    As we assume that $y_t$ is stationary, we know that $\Phi_1$ has all the eigenvalues in the unit circle, and the last term vanishes as $k \to \infty$.
    Define $e_1=(1,0)'$, $e_2=(0,1)'$.  A one-unit structural shock at $t$ affects $y_{t+h}$ by
    \[
    \Psi(h):=\frac{\partial y_{t+h}}{\partial\varepsilon_t}
            =\Phi_1^{h}\Phi_{\varepsilon},\quad h=0,1,\dots
    \]

    Impact of a labour-supply shock three periods ago on log wages
    \[
    \frac{\partial w_t}{\partial\varepsilon_{b,t-3}}
    = e_1'\Psi(3)\,e_2
    = e_1' \Phi_1^{3}\Phi_{\varepsilon} e_2
    = \bigl(\Phi_1^{3}\bigr)_{1\bullet}\,b_{12}.
    \]
    where $\bigl(\Phi_1^{3}\bigr)_{1\bullet}$ denotes the first row of $\Phi_1^{3}$.
    % We want to know how log wages ($w_t$, the first element of $y_t$) react to a labor supply shock ($\varepsilon_{b,t}$, the second element of $\varepsilon_t$) that occurred 3 periods before. This corresponds to the element $(1,2)$ of the matrix $\Psi(3)$.
    % Let $\Phi_{\varepsilon} = \begin{bmatrix} b_{11} & b_{12} \\ b_{21} & b_{22} \end{bmatrix}$.
    % The vector of responses of $y_{t+3} = (w_{t+3}, h_{t+3})'$ to $\varepsilon_t$ is $\Psi(3) \varepsilon_t = \Phi_1^3 \Phi_{\varepsilon} \varepsilon_t$.
    % The response of $w_{t+3}$ to $\varepsilon_{b,t}$ is the $(1,2)$ element of $\Psi(3) = \Phi_1^3 \Phi_{\varepsilon}$.
    % Specifically, if $\Phi_1^3 = \begin{bmatrix} \phi_{11}^{(3)} & \phi_{12}^{(3)} \\ \phi_{21}^{(3)} & \phi_{22}^{(3)} \end{bmatrix}$, then
    % \[ \Phi_1^3 \Phi_{\varepsilon} = \begin{bmatrix} \phi_{11}^{(3)} & \phi_{12}^{(3)} \\ \phi_{21}^{(3)} & \phi_{22}^{(3)} \end{bmatrix} \begin{bmatrix} b_{11} & b_{12} \\ b_{21} & b_{22} \end{bmatrix} = \begin{bmatrix} \phi_{11}^{(3)}b_{11}+\phi_{12}^{(3)}b_{21} & \phi_{11}^{(3)}b_{12}+\phi_{12}^{(3)}b_{22} \\ \phi_{21}^{(3)}b_{11}+\phi_{22}^{(3)}b_{21} & \phi_{21}^{(3)}b_{12}+\phi_{22}^{(3)}b_{22} \end{bmatrix} \]
    % The required response is $\Psi(3)_{12} = \phi_{11}^{(3)}b_{12}+\phi_{12}^{(3)}b_{22}$.
\end{autosolution}

\begin{autosolution}
    \

    From the reduced-form VAR, we can consistently estimate $\Phi_1$ and $\Sigma_u = \Phi_{\varepsilon} \Phi_{\varepsilon}^{\prime}$.
    \[
    \Sigma_u=
    \begin{bmatrix}
    b_{11}^2+b_{12}^2 &
    b_{11}b_{21}+b_{12}b_{22}\\
    \cdot &
    b_{21}^2+b_{22}^2
    \end{bmatrix},
    \]
    provides three distinct equations for the four unknowns in $\Phi_{\varepsilon}$.
    If $\Phi_{\varepsilon}$ is a solution, then for any $k \times k$ (here we have $k=2$) orthogonal matrix $P$ (such that $PP' = I_k$), $\Phi_{\varepsilon}^* = \Phi_{\varepsilon} P$ is also a solution because $\Phi_{\varepsilon}^* (\Phi_{\varepsilon}^*)' = (\Phi_{\varepsilon} P)(\Phi_{\varepsilon} P)' = \Phi_{\varepsilon} P P' \Phi_{\varepsilon}' = \Phi_{\varepsilon} I_k \Phi_{\varepsilon}' = \Phi_{\varepsilon} \Phi_{\varepsilon}' = \Sigma_u$.
    The identification problem is to find restrictions to pin down $P$.
\end{autosolution}

\begin{autosolution}
    \

    As $u_t = (u_{w,t}, u_{h,t})' = \Phi_{\varepsilon} \varepsilon_t$, we know:
    \begin{align*}
        u_{w,t} &= b_{11} \varepsilon_{a,t} + b_{12} \varepsilon_{b,t} \\
        u_{h,t} &= b_{21} \varepsilon_{a,t} + b_{22} \varepsilon_{b,t}
    \end{align*}
    and the asusmption gives that $u_{h,t}$ is only affected by $\varepsilon_{b,t}$, so $b_{21} = 0.$
    % Then $b_{11},b_{12},b_{22}$ are exactly determined by the three moments in $\Sigma_u$,

    Since we need 
    % $k(k-1)/2 = 1$ 
    only 1 restriction for $2 \times 2$ matrix, this is exactly enough for identification (up to sign normalizations).
    With $b_{21} = 0$, $\Phi_{\varepsilon}$ becomes upper triangular:
    \[ \Phi_{\varepsilon} = \begin{bmatrix}
        b_{11} & b_{12} \\
        0 & b_{22}
    \end{bmatrix} \]
    Then $\Sigma_u = \Phi_{\varepsilon} \Phi_{\varepsilon}' = \begin{bmatrix}
        b_{11} & b_{12} \\
        0 & b_{22}
    \end{bmatrix}
    \begin{bmatrix}
        b_{11} & 0 \\
        b_{12} & b_{22}
    \end{bmatrix}
    = \begin{bmatrix}
        b_{11}^2 + b_{12}^2 & b_{12}b_{22}
        \\ b_{12}b_{22} & b_{22}^2
    \end{bmatrix}$.
    
    Let $\Sigma_u = \begin{bmatrix}
        \sigma_{11} & \sigma_{12} \\
        \sigma_{21} & \sigma_{22}
    \end{bmatrix}$ (where $\sigma_{12}=\sigma_{21}$).
    \begin{enumerate}
        \item From $\sigma_{22} = b_{22}^2$, we get $b_{22} = \sqrt{\sigma_{22}}$ (by convention, positive).
        \item From $\sigma_{12} = b_{12}b_{22}$, we get $b_{12} = \sigma_{12} / b_{22}$ (assuming $b_{22} \neq 0$).
        \item From $\sigma_{11} = b_{11}^2 + b_{12}^2$, we get $b_{11} = \sqrt{\sigma_{11} - b_{12}^2}$ (by convention, positive, and assuming $\sigma_{11} - b_{12}^2 \ge 0$).
    \end{enumerate}
    This uniquely identifies $\Phi_{\varepsilon}$ (given sign normalizations for diagonal elements). This procedure is equivalent to finding an upper Cholesky factor of $\Sigma_u$.
    % so $\Phi_{\varepsilon}$ is \emph{point-identified} (up to column sign changes).
\end{autosolution}

\begin{autosolution}
    \

    % \begin{itemize}
    %     \item Supply shock $\varepsilon_b$: wages and hours move \emph{opposite} on impact $\;\Rightarrow\; b_{12}b_{22}<0$.
    %     \item Demand/technology shock $\varepsilon_a$: wages and hours move \emph{together} $\;\Rightarrow\; b_{11}b_{21}>0$.
    % \end{itemize}
    \begin{enumerate}
        \item Labor supply shock ($\varepsilon_{b,t}$) moves wages ($w_t$) and hours ($h_t$) in opposite directions upon impact:
        $\frac{\partial w_t}{\partial \varepsilon_{b,t}} = b_{12}$ and $\frac{\partial h_t}{\partial \varepsilon_{b,t}} = b_{22}$.
        So, $b_{12} \cdot b_{22} < 0$.
        \item Demand shock ($\varepsilon_{a,t}$) moves wages ($w_t$) and hours ($h_t$) in the same direction upon impact:
        $\frac{\partial w_t}{\partial \varepsilon_{a,t}} = b_{11}$ and $\frac{\partial h_t}{\partial \varepsilon_{a,t}} = b_{21}$.
        So, $b_{11} \cdot b_{21} > 0$.
    \end{enumerate}
    These are inequality restrictions. They do not typically lead to point identification.
    Let $\Phi_{\varepsilon,0}$ be any matrix such that $\Phi_{\varepsilon,0} \Phi_{\varepsilon,0}' = \Sigma_u$ (e.g., from a Cholesky decomposition of $\Sigma_u$).
    Then any other valid matrix is $\Phi_{\varepsilon} = \Phi_{\varepsilon,0} P$, where $P$ is an orthogonal matrix.
    
    For $k=2$, $P$ can be a rotation matrix 
    $P(\theta) = \begin{bmatrix}
        \cos\theta & -\sin\theta \\
        \sin\theta & \cos\theta
    \end{bmatrix}$
    (or $\begin{bmatrix}
        \cos\theta & \sin\theta \\
        -\sin\theta & \cos\theta
    \end{bmatrix}$, depending on convention).
    The sign restrictions define a set of admissible rotation angles $\theta$.
    If this set is not a singleton (or two points corresponding to $P$ and $-P$ after sign normalizations),
    then $\Phi_{\varepsilon}$ is not uniquely identified. Generally, sign restrictions lead to set identification,
    meaning there is a range of $\theta$ values (and thus a set of $\Phi_{\varepsilon}$ matrices) consistent with the restrictions.
    So, this is not enough to uniquely identify $\Phi_{\varepsilon}$, the model is only \emph{set-identified}.
\end{autosolution}

\begin{autosolution}
    \

    % Linearise labour demand $h^{D}$ and supply $h^{S}$ around the steady state $(w_0,h_0)$:
    % \[
    % \begin{aligned}
    % h^{D} &= \eta_D\,(w-w_0)+\delta_a\,\varepsilon_a,\\
    % h^{S} &= \eta_S\,(w-w_0)+\gamma_a\,\varepsilon_a + \gamma_b\,\varepsilon_b,
    % \end{aligned}\qquad
    % \eta_D<0,\;\eta_S>0.
    % \]
    % Equating $h^{D}=h^{S}$ leaves one structural parameter (the slope gap) unidentified; the system is \emph{under-identified}.
\end{autosolution}

\begin{autosolution}
    \

    % The assumption is: "Labor demand is only affected by the technology shock ($\varepsilon_{a,t}$), not the preference shock ($\varepsilon_{b,t}$), whereas labor supply is affected by both shocks."
    % This is a restriction on the underlying structural economic model. Consider a linear structural model for the innovations:
    % \begin{align*}
    %     \text{Demand:} \quad & a_{11} u_{w,t} + a_{12} u_{h,t} = \gamma_{1a} \varepsilon_{a,t} \quad (\text{no } \varepsilon_{b,t} \text{ term, so } \gamma_{1b}=0) \\
    %     \text{Supply:} \quad & a_{21} u_{w,t} + a_{22} u_{h,t} = \gamma_{2a} \varepsilon_{a,t} + \gamma_{2b} \varepsilon_{b,t}
    % \end{align*}
    % In matrix form, $A_0 u_t = \Gamma \varepsilon_t$, where $A_0 = \begin{bmatrix} a_{11} & a_{12} \\ a_{21} & a_{22} \end{bmatrix}$ and $\Gamma = \begin{bmatrix} \gamma_{1a} & 0 \\ \gamma_{2a} & \gamma_{2b} \end{bmatrix}$.
    % The reduced form innovations are $u_t = A_0^{-1} \Gamma \varepsilon_t$. So $\Phi_{\varepsilon} = A_0^{-1} \Gamma$.
    % Let $A_0^{-1} = B = \begin{bmatrix} b_{11} & b_{12} \\ b_{21} & b_{22} \end{bmatrix}$.
    % Then $\Phi_{\varepsilon} = \begin{bmatrix} b_{11} & b_{12} \\ b_{21} & b_{22} \end{bmatrix} \begin{bmatrix} \gamma_{1a} & 0 \\ \gamma_{2a} & \gamma_{2b} \end{bmatrix} = \begin{bmatrix} b_{11}\gamma_{1a} + b_{12}\gamma_{2a} & b_{12}\gamma_{2b} \\ b_{21}\gamma_{1a} + b_{22}\gamma_{2a} & b_{22}\gamma_{2b} \end{bmatrix}$.
    % So, $b_{12} = b_{12}\gamma_{2b}$ and $b_{22} = b_{22}\gamma_{2b}$.
    % This implies $b_{12} / b_{22} = b_{12} / b_{22}$ (if $\gamma_{2b} \neq 0$ and $b_{22} \neq 0$).
    % The ratio $b_{12} / b_{22}$ depends on the elements of $A_0$, which are structural parameters (related to slopes of demand/supply curves). For instance, if the demand equation (Equation 1) is $u_{w,t} + \alpha_D u_{h,t} = \dots$ (so $a_{11}=1, a_{12}=\alpha_D$) and the supply equation is $u_{w,t} + \alpha_S u_{h,t} = \dots$ (so $a_{21}=1, a_{22}=\alpha_S$), then $A_0 = \begin{bmatrix} 1 & \alpha_D \\ 1 & \alpha_S \end{bmatrix}$.
    % Then $A_0^{-1} = \frac{1}{\alpha_S-\alpha_D} \begin{bmatrix} \alpha_S & -\alpha_D \\ -1 & 1 \end{bmatrix}$. So $b_{12} = \frac{-\alpha_D}{\alpha_S-\alpha_D}$ and $b_{22} = \frac{1}{\alpha_S-\alpha_D}$.
    % Thus $b_{12}/b_{22} = -\alpha_D$.
    % The restriction becomes $b_{12} = (-\alpha_D) b_{22}$. This is one restriction on the elements of $\Phi_{\varepsilon}$, but it involves an unknown structural parameter $\alpha_D$. Without knowing $\alpha_D$, this restriction is not sufficient to identify $\Phi_{\varepsilon}$ from $\Sigma_u$. Thus, this is not enough to uniquely identify $\Phi_{\varepsilon}$.
    % The hint "remember that demand must equal supply at all times" is implicitly used by solving for equilibrium $w_t, h_t$ which give rise to $u_t$ and thus $\Phi_\varepsilon$.
    From the structural model, $\frac{\partial y_t}{\partial \varepsilon_t} = \Phi_{\varepsilon}$.
    This gives us:
    \begin{gather*}
        \frac{\partial w_t}{\partial \varepsilon_{a,t}} = b_{11}, \quad
        \frac{\partial w_t}{\partial \varepsilon_{b,t}} = b_{12} \\
        \frac{\partial h_t}{\partial \varepsilon_{a,t}} = b_{21}, \quad
        \frac{\partial h_t}{\partial \varepsilon_{b,t}} = b_{22}
    \end{gather*}
    The assumption that labor demand depends only on technology shocks means: $\frac{\partial \varphi^D}{\partial \varepsilon_{b,t}} = 0$.
    However, this doesn't mean that hours don't respond to preference shocks at all.
    Since hours are determined in equilibrium where demand equals supply, a preference shock will affect wages,
    which will then affect labor demand.

    To formalize this, we can use the total derivative of the labor demand function with respect to the preference shock:
    \begin{gather*}
        \frac{\partial h_t}{\partial \varepsilon_{b,t}} = \frac{\partial \varphi^D}{\partial \varepsilon_{b,t}} + \frac{\partial \varphi^D}{\partial w_t} \frac{\partial w_t}{\partial \varepsilon_{b,t}}
    \end{gather*}
    Since $\frac{\partial \varphi^D}{\partial \varepsilon_{b,t}} = 0$, we have:
    \begin{gather*}
        \frac{\partial h_t}{\partial \varepsilon_{b,t}} = \frac{\partial \varphi^D}{\partial w_t} \frac{\partial w_t}{\partial \varepsilon_{b,t}}
        \Rightarrow b_{22} = \frac{\partial \varphi^D}{\partial w_t} b_{12}
    \end{gather*}
    From $\Sigma_u = \Phi_{\varepsilon} \Phi_{\varepsilon}'$, we have three restrictions,
    and our labor market restriction gives us a relationship between $b_{12}$ and $b_{22}$,
    but with an unknown parameter $\frac{\partial \varphi^D}{\partial w_t}$.

    Therefore, the given restriction alone is not sufficient to uniquely identify $\Phi_{\varepsilon}$.
    This is why in the following question, we give a certain $\alpha$ as this parameter to help identify $\Phi_{\varepsilon}$.
\end{autosolution}

\begin{autosolution}
    \

    % Impose
    % \[
    % \boxed{\displaystyle
    % \frac{\partial w_t}{\partial\varepsilon_{b,t}}
    % = (\alpha-1)\,\frac{\partial h_t}{\partial\varepsilon_{b,t}}},
    % \qquad \alpha\neq1.
    % \]
    % Hence $b_{12}=(\alpha-1)b_{22}$ and, together with $\Sigma_u$,
    % \[
    % \begin{cases}
    % \sigma_{11}=b_{11}^2+(\alpha-1)^2 b_{22}^2,\\
    % \sigma_{22}=b_{21}^2+b_{22}^2,\\
    % \sigma_{12}=b_{11}b_{21}+(\alpha-1)b_{22}^2,\\
    % b_{12}=(\alpha-1)b_{22}.
    % \end{cases}
    % \]
    % Eliminating $b_{11},b_{21}$ and squaring the third line yields
    % \[
    % \bigl[\sigma_{12}-(\alpha-1)b_{22}^2\bigr]^2
    % =\bigl[\sigma_{11}-(\alpha-1)^2b_{22}^2\bigr]\,
    %     \bigl[\sigma_{22}-b_{22}^2\bigr],
    % \]
    % a quartic in $b_{22}^2$ that generically has a unique positive root in
    % $0<b_{22}^2<\min\{\sigma_{11}/(\alpha-1)^2,\sigma_{22}\}$.
    % Back-substitution gives $b_{21},b_{11}$ and hence a \emph{point-identified} $\Phi_{\varepsilon}$ 
    % (again up to a simultaneous sign flip of an entire column).
    Let $L$ be the lower Cholesky factor of $\Sigma_u$, such that $L L' = \Sigma_u$.
    $L = \begin{bmatrix}
        l_{11} & 0 \\
        l_{21} & l_{22}
    \end{bmatrix}$, where 
    \begin{gather*}
        \sigma_{11} = l_{11}^2, \quad
        \sigma_{12} = l_{11} l_{21}, \quad
        \sigma_{22} = l_{21}^2 + l_{22}^2
    \end{gather*}
    Any $\Phi_{\varepsilon}$ such that $\Phi_{\varepsilon} \Phi_{\varepsilon}' = \Sigma_u$ can be written as $\Phi_{\varepsilon} = L P(\theta)$ for some orthogonal matrix $P(\theta)$.
    For $k=2$, a common rotation matrix is $P(\theta) = \begin{bmatrix} \cos\theta & -\sin\theta \\ \sin\theta & \cos\theta \end{bmatrix}$.
    \[ \Phi_{\varepsilon} = \begin{bmatrix}
        l_{11} & 0 \\
        l_{21} & l_{22}
    \end{bmatrix}
    \begin{bmatrix}
        \cos\theta & -\sin\theta \\
        \sin\theta & \cos\theta
    \end{bmatrix}
    = \begin{bmatrix}
        l_{11}\cos\theta & -l_{11}\sin\theta \\
        l_{21}\cos\theta+l_{22}\sin\theta & -l_{21}\sin\theta+l_{22}\cos\theta
    \end{bmatrix} \]
    So, $b_{12} = -l_{11}\sin\theta$ and $b_{22} = -l_{21}\sin\theta+l_{22}\cos\theta$.
    
    The restriction $b_{12} = (\alpha-1)b_{22}$ becomes:
    \[ -l_{11}\sin\theta = (\alpha-1)[-l_{21}\sin\theta+l_{22}\cos\theta] \]
    \[ [(\alpha-1)l_{21} - l_{11}]\sin\theta = (\alpha-1)l_{22}\cos\theta \]
    If $(\alpha-1)l_{22} \neq 0$ and the coefficient of $\sin\theta$ is not zero (and $\cos\theta \neq 0$ to avoid division by zero for $\tan\theta$):
    \[ \tan\theta = \frac{(\alpha-1)l_{22}}{(\alpha-1)l_{21} - l_{11}} \]
    As Cholesky decomposition is unique (up to sign normalizations), $l_{11}, l_{21}, l_{22}$ are uniquely determined,
    hence $\tan\theta$ is uniquely determined.
    Hence this equation determines $\theta$ up to a multiple of $\pi$.
    For example, if $\theta_0$ is a solution, then $\theta_0+\pi$ is also a solution.
    Adding $\pi$ to $\theta$ changes $P(\theta)$ to $-P(\theta)$, which flips the sign of all elements in $\Phi_{\varepsilon}$.
    This means $\Phi_{\varepsilon}$ is identified up to an overall sign change.
    % the elements of $\Phi_{\varepsilon}$ are uniquely identified (up to sign normalizations).
    % Specific column sign normalizations (e.g., requiring diagonal elements of $\Phi_{\varepsilon}$ to be positive)
    % would then typically fix $\theta$ (and thus $\Phi_{\varepsilon}$) uniquely.
    % For instance, one might choose $\theta$ in $(-\pi/2, \pi/2]$ or $[0, \pi)$ and then adjust signs of columns of $\Phi_{\varepsilon}$ if necessary.

    % Final answer for (i): Yes, if $\alpha$ is given, it is possible to uniquely identify the elements of $\Phi_{\varepsilon}$ (up to conventional sign normalizations).
    % The steps above show how $\Phi_{\varepsilon}$ can be solved for based on $\alpha$ and the elements of $\Sigma_u$ (which are estimated from the reduced-form VAR).
\end{autosolution}

% \section{Gemini Solution}

% The bivariate VAR(1) model is given by:
% \begin{align*}
% y_t &= \Phi_1 y_{t-1} + u_t \\
% u_t &= \Phi_{\varepsilon} \varepsilon_t
% \end{align*}
% where $y_t = (w_t, h_t)'$, $\varepsilon_t = (\varepsilon_{a,t}, \varepsilon_{b,t})'$, and $\varepsilon_t \sim \text{i.i.d. } N(0, I)$.
% The covariance matrix of the reduced-form errors $u_t$ is $\Sigma_u = \mathbb{E}[u_t u_t'] = \mathbb{E}[\Phi_{\varepsilon} \varepsilon_t \varepsilon_t' \Phi_{\varepsilon}'] = \Phi_{\varepsilon} \mathbb{E}[\varepsilon_t \varepsilon_t'] \Phi_{\varepsilon}' = \Phi_{\varepsilon} I \Phi_{\varepsilon}' = \Phi_{\varepsilon} \Phi_{\varepsilon}'$.

% \section*{(a) Stationarity condition for $y_t$}
% For $y_t$ to be stationary, all eigenvalues of the matrix $\Phi_1$ must lie inside the unit circle (i.e., have modulus less than 1). Equivalently, the roots of the characteristic polynomial $\det(I - \Phi_1 z) = 0$ must lie outside the unit circle.



% \section*{(d) Identification Problem}
% From the reduced-form VAR, we can consistently estimate $\Phi_1$ and $\Sigma_u = \mathbb{E}[u_t u_t']$. The structural relationship is $u_t = \Phi_{\varepsilon} \varepsilon_t$, which implies $\Sigma_u = \Phi_{\varepsilon} \Phi_{\varepsilon}'$.
% $\Sigma_u$ is a $k \times k$ symmetric matrix (here $k=2$). It has $k(k+1)/2$ distinct elements. For $k=2$, this is $2(3)/2 = 3$ distinct elements.
% The matrix $\Phi_{\varepsilon}$ is $k \times k$, containing $k^2$ unknown parameters. For $k=2$, this is $2^2 = 4$ unknowns.
% We have 3 equations from $\Sigma_u = \Phi_{\varepsilon} \Phi_{\varepsilon}'$ to determine 4 unknowns in $\Phi_{\varepsilon}$. The system is underidentified.
% We need $k^2 - k(k+1)/2 = k(k-1)/2$ additional restrictions. For $k=2$, we need $2(1)/2 = 1$ restriction.

% \section*{(e) Contemporaneous restriction on hours}
% % The assumption is: "Contemporaneously, hours worked ($h_t$) are only affected by preferences ($\varepsilon_{b,t}$), not technology ($\varepsilon_{a,t}$)."

% The restriction means that the coefficient of $\varepsilon_{a,t}$ in the equation for $u_{h,t}$ (the innovation in hours) is zero.
% So, $b_{21} = 0$.
% This is one restriction. Since we need $k(k-1)/2 = 1$ restriction for $k=2$, this is exactly enough for identification (up to sign normalizations).
% With $b_{21} = 0$, $\Phi_{\varepsilon}$ becomes upper triangular:
% \[ \Phi_{\varepsilon} = \begin{bmatrix} b_{11} & b_{12} \\ 0 & b_{22} \end{bmatrix} \]
% Then $\Sigma_u = \Phi_{\varepsilon} \Phi_{\varepsilon}' = \begin{bmatrix} b_{11} & b_{12} \\ 0 & b_{22} \end{bmatrix} \begin{bmatrix} b_{11} & 0 \\ b_{12} & b_{22} \end{bmatrix} = \begin{bmatrix} b_{11}^2 + b_{12}^2 & b_{12}b_{22} \\ b_{12}b_{22} & b_{22}^2 \end{bmatrix}$.
% Let $\Sigma_u = \begin{bmatrix} \sigma_{11} & \sigma_{12} \\ \sigma_{21} & \sigma_{22} \end{bmatrix}$ (where $\sigma_{12}=\sigma_{21}$).
% \begin{enumerate}
%     \item From $\sigma_{22} = b_{22}^2$, we get $b_{22} = \sqrt{\sigma_{22}}$ (by convention, positive).
%     \item From $\sigma_{12} = b_{12}b_{22}$, we get $b_{12} = \sigma_{12} / b_{22}$ (assuming $b_{22} \neq 0$).
%     \item From $\sigma_{11} = b_{11}^2 + b_{12}^2$, we get $b_{11} = \sqrt{\sigma_{11} - b_{12}^2}$ (by convention, positive, and assuming $\sigma_{11} - b_{12}^2 \ge 0$).
% \end{enumerate}
% This uniquely identifies $\Phi_{\varepsilon}$ (given sign normalizations for diagonal elements). This procedure is equivalent to finding an upper Cholesky factor of $\Sigma_u$.

% \section*{(f) Sign restrictions}
% \begin{enumerate}
%     \item Labor supply shock ($\varepsilon_{b,t}$) moves wages ($w_t$) and hours ($h_t$) in opposite directions upon impact:
%     $\frac{\partial w_t}{\partial \varepsilon_{b,t}} = b_{12}$ and $\frac{\partial h_t}{\partial \varepsilon_{b,t}} = b_{22}$.
%     So, $b_{12} \cdot b_{22} < 0$.
%     \item Demand shock ($\varepsilon_{a,t}$) moves wages ($w_t$) and hours ($h_t$) in the same direction upon impact:
%     $\frac{\partial w_t}{\partial \varepsilon_{a,t}} = b_{11}$ and $\frac{\partial h_t}{\partial \varepsilon_{a,t}} = b_{21}$.
%     So, $b_{11} \cdot b_{21} > 0$.
% \end{enumerate}
% These are inequality restrictions. They do not typically lead to point identification.
% Let $\Phi_{\varepsilon,0}$ be any matrix such that $\Phi_{\varepsilon,0} \Phi_{\varepsilon,0}' = \Sigma_u$ (e.g., from a Cholesky decomposition of $\Sigma_u$). Then any other valid matrix is $\Phi_{\varepsilon} = \Phi_{\varepsilon,0} P$, where $P$ is an orthogonal matrix. For $k=2$, $P$ can be a rotation matrix $P(\theta) = \begin{bmatrix} \cos\theta & -\sin\theta \\ \sin\theta & \cos\theta \end{bmatrix}$ (or $P(\theta) = \begin{bmatrix} \cos\theta & \sin\theta \\ -\sin\theta & \cos\theta \end{bmatrix}$, depending on convention).
% The sign restrictions define a set of admissible rotation angles $\theta$. If this set is not a singleton (or two points corresponding to $P$ and $-P$ after sign normalizations), then $\Phi_{\varepsilon}$ is not uniquely identified. Generally, sign restrictions lead to set identification, meaning there is a range of $\theta$ values (and thus a set of $\Phi_{\varepsilon}$ matrices) consistent with the restrictions. So, this is not enough to uniquely identify $\Phi_{\varepsilon}$.

% \section*{(h) Restrictions on structural demand/supply functions}
% The assumption is: "Labor demand is only affected by the technology shock ($\varepsilon_{a,t}$), not the preference shock ($\varepsilon_{b,t}$), whereas labor supply is affected by both shocks."
% This is a restriction on the underlying structural economic model. Consider a linear structural model for the innovations:
% \begin{align*}
% \text{Demand:} \quad & a_{11} u_{w,t} + a_{12} u_{h,t} = \gamma_{1a} \varepsilon_{a,t} \quad (\text{no } \varepsilon_{b,t} \text{ term, so } \gamma_{1b}=0) \\
% \text{Supply:} \quad & a_{21} u_{w,t} + a_{22} u_{h,t} = \gamma_{2a} \varepsilon_{a,t} + \gamma_{2b} \varepsilon_{b,t}
% \end{align*}
% In matrix form, $A_0 u_t = \Gamma \varepsilon_t$, where $A_0 = \begin{bmatrix} a_{11} & a_{12} \\ a_{21} & a_{22} \end{bmatrix}$ and $\Gamma = \begin{bmatrix} \gamma_{1a} & 0 \\ \gamma_{2a} & \gamma_{2b} \end{bmatrix}$.
% The reduced form innovations are $u_t = A_0^{-1} \Gamma \varepsilon_t$. So $\Phi_{\varepsilon} = A_0^{-1} \Gamma$.
% Let $A_0^{-1} = B = \begin{bmatrix} b_{11} & b_{12} \\ b_{21} & b_{22} \end{bmatrix}$.
% Then $\Phi_{\varepsilon} = \begin{bmatrix} b_{11} & b_{12} \\ b_{21} & b_{22} \end{bmatrix} \begin{bmatrix} \gamma_{1a} & 0 \\ \gamma_{2a} & \gamma_{2b} \end{bmatrix} = \begin{bmatrix} b_{11}\gamma_{1a} + b_{12}\gamma_{2a} & b_{12}\gamma_{2b} \\ b_{21}\gamma_{1a} + b_{22}\gamma_{2a} & b_{22}\gamma_{2b} \end{bmatrix}$.
% So, $b_{12} = b_{12}\gamma_{2b}$ and $b_{22} = b_{22}\gamma_{2b}$.
% This implies $b_{12} / b_{22} = b_{12} / b_{22}$ (if $\gamma_{2b} \neq 0$ and $b_{22} \neq 0$).
% The ratio $b_{12} / b_{22}$ depends on the elements of $A_0$, which are structural parameters (related to slopes of demand/supply curves). For instance, if the demand equation (Equation 1) is $u_{w,t} + \alpha_D u_{h,t} = \dots$ (so $a_{11}=1, a_{12}=\alpha_D$) and the supply equation is $u_{w,t} + \alpha_S u_{h,t} = \dots$ (so $a_{21}=1, a_{22}=\alpha_S$), then $A_0 = \begin{bmatrix} 1 & \alpha_D \\ 1 & \alpha_S \end{bmatrix}$.
% Then $A_0^{-1} = \frac{1}{\alpha_S-\alpha_D} \begin{bmatrix} \alpha_S & -\alpha_D \\ -1 & 1 \end{bmatrix}$. So $b_{12} = \frac{-\alpha_D}{\alpha_S-\alpha_D}$ and $b_{22} = \frac{1}{\alpha_S-\alpha_D}$.
% Thus $b_{12}/b_{22} = -\alpha_D$.
% The restriction becomes $b_{12} = (-\alpha_D) b_{22}$. This is one restriction on the elements of $\Phi_{\varepsilon}$, but it involves an unknown structural parameter $\alpha_D$. Without knowing $\alpha_D$, this restriction is not sufficient to identify $\Phi_{\varepsilon}$ from $\Sigma_u$. Thus, this is not enough to uniquely identify $\Phi_{\varepsilon}$.
% The hint "remember that demand must equal supply at all times" is implicitly used by solving for equilibrium $w_t, h_t$ which give rise to $u_t$ and thus $\Phi_\varepsilon$.

% \section*{(i) Adding a quantitative restriction}
% The assumption from (h) implies a relationship of the form $b_{12} = \kappa b_{22}$, where $\kappa$ is a function of structural parameters.
% The new assumption is $\frac{\partial w_t}{\partial \varepsilon_{b,t}} = (\alpha-1) \frac{\partial h_t}{\partial \varepsilon_{b,t}}$ for a \textit{given} $\alpha$.
% This translates directly to a restriction on the elements of $\Phi_{\varepsilon}$ governing the impact responses to $\varepsilon_{b,t}$:
% \[ b_{12} = (\alpha-1) b_{22} \]
% This is one linear restriction on the elements of $\Phi_{\varepsilon}$, and $(\alpha-1)$ is a known constant. Since we need $k(k-1)/2 = 1$ restriction for $k=2$, and this provides one, it is possible to uniquely identify $\Phi_{\varepsilon}$ (up to conventional sign normalizations).

% To solve for $\Phi_{\varepsilon} = \begin{bmatrix} b_{11} & b_{12} \\ b_{21} & b_{22} \end{bmatrix} = \begin{bmatrix} a & b \\ c & d \end{bmatrix}$:
% The restriction is $b = (\alpha-1)d$.
% The equations from $\Sigma_u = \Phi_{\varepsilon} \Phi_{\varepsilon}'$ are:
% \begin{align}
% \sigma_{11} &= a^2 + b^2 \label{eq:i1} \\
% \sigma_{22} &= c^2 + d^2 \label{eq:i2} \\
% \sigma_{12} &= ac + bd \label{eq:i3}
% \end{align}
% Substitute $b = (\alpha-1)d$ into these equations:
% \begin{align}
% \sigma_{11} &= a^2 + ((\alpha-1)d)^2 = a^2 + (\alpha-1)^2 d^2 \label{eq:i4} \\
% \sigma_{22} &= c^2 + d^2 \label{eq:i5} \\
% \sigma_{12} &= ac + (\alpha-1)d^2 \label{eq:i6}
% \end{align}
% We have 3 equations for 3 unknowns $a, c, d$. (Then $b$ is found from $d$).
% From \eqref{eq:i4}, $a^2 = \sigma_{11} - (\alpha-1)^2 d^2$. So $a = \pm\sqrt{\sigma_{11} - (\alpha-1)^2 d^2}$.
% From \eqref{eq:i5}, $c^2 = \sigma_{22} - d^2$. So $c = \pm\sqrt{\sigma_{22} - d^2}$.
% Substitute $ac = \sigma_{12} - (\alpha-1)d^2$ from \eqref{eq:i6} and square it:
% \[ (ac)^2 = (\sigma_{12} - (\alpha-1)d^2)^2 \]
% Also, $(ac)^2 = a^2 c^2 = (\sigma_{11} - (\alpha-1)^2 d^2)(\sigma_{22} - d^2)$.
% Let $X = d^2$.
% \[ (\sigma_{12} - (\alpha-1)X)^2 = (\sigma_{11} - (\alpha-1)^2 X)(\sigma_{22} - X) \]
% Expanding this:
% \begin{align*}
% \sigma_{12}^2 - 2(\alpha-1)\sigma_{12}X + (\alpha-1)^2 X^2 &= \sigma_{11}\sigma_{22} - \sigma_{11}X - (\alpha-1)^2 \sigma_{22}X + (\alpha-1)^2 X^2 \\
% \sigma_{12}^2 - 2(\alpha-1)\sigma_{12}X &= \sigma_{11}\sigma_{22} - (\sigma_{11} + (\alpha-1)^2 \sigma_{22})X \\
% [\sigma_{11} + (\alpha-1)^2 \sigma_{22} - 2(\alpha-1)\sigma_{12}] X &= \sigma_{11}\sigma_{22} - \sigma_{12}^2
% \end{align*}
% The RHS is $\det(\Sigma_u)$.
% The term in square brackets is $\begin{bmatrix} 1 & -(\alpha-1) \end{bmatrix} \Sigma_u \begin{bmatrix} 1 \\ -(\alpha-1) \end{bmatrix}$.
% So, $X = d^2 = \frac{\det(\Sigma_u)}{\sigma_{11} - 2(\alpha-1)\sigma_{12} + (\alpha-1)^2 \sigma_{22}}$.
% For a unique solution for $X=d^2$, the denominator must be non-zero. For $d$ to be real, $X$ must be non-negative.
% \begin{enumerate}
%     \item $d = \pm\sqrt{X}$. Conventionally, one might choose $d > 0$ (normalizing the sign of $\varepsilon_{b,t}$'s impact on $h_t$).
%     \item $b = (\alpha-1)d$.
%     \item $a = \pm\sqrt{\sigma_{11} - b^2}$. Conventionally, one might choose $a > 0$ (normalizing the sign of $\varepsilon_{a,t}$'s impact on $w_t$).
%     \item $c = (\sigma_{12} - bd) / a$. (If $a=0$, then $\sigma_{12} - bd$ must be zero for consistency. In this case, $c$ is determined from $c^2 = \sigma_{22} - d^2$, with its sign chosen based on other considerations or normalizations if needed).
% \end{enumerate}
% The elements of $\Phi_{\varepsilon}$ are thus uniquely identified once $\alpha$ and $\Sigma_u$ are known, up to the sign normalizations for $d$ and $a$. These sign normalizations define what constitutes a "positive" technology shock $\varepsilon_{a,t}$ and a "positive" preference shock $\varepsilon_{b,t}$.

% \textbf{Alternative method using rotation:}
% Let $L$ be the lower Cholesky factor of $\Sigma_u$, such that $L L' = \Sigma_u$.
% $L = \begin{bmatrix} l_{11} & 0 \\ l_{21} & l_{22} \end{bmatrix}$, where $l_{11}=\sqrt{\sigma_{11}}$, $l_{21}=\sigma_{12}/l_{11}$ (if $l_{11} \neq 0$), $l_{22}=\sqrt{\sigma_{22}-l_{21}^2}$.
% Any $\Phi_{\varepsilon}$ such that $\Phi_{\varepsilon} \Phi_{\varepsilon}' = \Sigma_u$ can be written as $\Phi_{\varepsilon} = L P(\theta)$ for some orthogonal matrix $P(\theta)$. For $k=2$, a common rotation matrix is $P(\theta) = \begin{bmatrix} \cos\theta & -\sin\theta \\ \sin\theta & \cos\theta \end{bmatrix}$.
% \[ \Phi_{\varepsilon} = \begin{bmatrix} l_{11} & 0 \\ l_{21} & l_{22} \end{bmatrix} \begin{bmatrix} \cos\theta & -\sin\theta \\ \sin\theta & \cos\theta \end{bmatrix} = \begin{bmatrix} l_{11}\cos\theta & -l_{11}\sin\theta \\ l_{21}\cos\theta+l_{22}\sin\theta & -l_{21}\sin\theta+l_{22}\cos\theta \end{bmatrix} \]
% So, $b_{12} = -l_{11}\sin\theta$ and $b_{22} = -l_{21}\sin\theta+l_{22}\cos\theta$.
% The restriction $b_{12} = (\alpha-1)b_{22}$ becomes:
% \[ -l_{11}\sin\theta = (\alpha-1)[-l_{21}\sin\theta+l_{22}\cos\theta] \]
% \[ [(\alpha-1)l_{21} - l_{11}]\sin\theta = (\alpha-1)l_{22}\cos\theta \]
% If $(\alpha-1)l_{22} \neq 0$ and the coefficient of $\sin\theta$ is not zero (and $\cos\theta \neq 0$ to avoid division by zero for $\tan\theta$):
% \[ \tan\theta = \frac{(\alpha-1)l_{22}}{(\alpha-1)l_{21} - l_{11}} \]
% This equation determines $\theta$ up to a multiple of $\pi$. For example, if $\theta_0$ is a solution, then $\theta_0+\pi$ is also a solution. Adding $\pi$ to $\theta$ changes $P(\theta)$ to $-P(\theta)$, which flips the sign of all elements in $\Phi_{\varepsilon}$. This means $\Phi_{\varepsilon}$ is identified up to an overall sign change. Specific column sign normalizations (e.g., requiring diagonal elements of $\Phi_{\varepsilon}$ to be positive) would then typically fix $\theta$ (and thus $\Phi_{\varepsilon}$) uniquely. For instance, one might choose $\theta$ in $(-\pi/2, \pi/2]$ or $[0, \pi)$ and then adjust signs of columns of $\Phi_{\varepsilon}$ if necessary.

% Final answer for (i): Yes, if $\alpha$ is given, it is possible to uniquely identify the elements of $\Phi_{\varepsilon}$ (up to conventional sign normalizations). The steps above show how $\Phi_{\varepsilon}$ can be solved for based on $\alpha$ and the elements of $\Sigma_u$ (which are estimated from the reduced-form VAR).
\end{document}