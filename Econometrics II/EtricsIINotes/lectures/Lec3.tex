Rubin (1975\cite{rubin1975bayesian}) and Holland (1986\cite{holland1986statistics}) made up the aphorism\cite{ding2023causalinference}:
\begin{quote}
  \textit{``No causation without manipulation''}
\end{quote}
Not everybody agrees with this point of view.

In our lectuere, we'll define causal effects using the potential outcomes framework
(Neyman, 1923\cite{neyman1923experiment}; Rubin, 1974\cite{rubin1974estimating}).


\section{Potential Outcomes Framework}

In this framework, an experiment, or at least a thought experiment, 
has a treatment, and we are interested in its effect on an outcome 
or multiple outcomes. Sometimes, the treatment is also called an
intervention or a manipulation.

Firstly, we consider an experiment with $n$ units indexed by $i=1, 2, \cdots, n$.
We focus on a treatment with two levels:
\begin{gather*}
  d_i = \left\{\begin{matrix}
    0 & \text{control}\\
    1 & \text{treatment}
  \end{matrix} \right.
\end{gather*}

We seek to identify the causal effect of treatment $d_i$ on some outcome $y_i$.
For each $i$, the outcome od interest $y_i$ has two versions:
\begin{gather*}
  y_i = \left\{\begin{matrix}
    y_{0i} & d_i=0\\
    y_{1i} & d_i=1
  \end{matrix} \right.
\end{gather*}
This notation emphasizes that $y_{di}$ is the realization of the outcome $y_i$ that would materialize if unit $i$
received treatment $d_i = d$.

Neyman (1923\cite{neyman1923experiment}) first used this notation. It seems intuitive but has some hidden
assumptions. Rubin (1980\cite{rubin1980comment}) made the following clarifications on the hidden assumptions.
\begin{assumption}[No interference]\label{assumption:no_interference}
  \

  Unit $i$'s potential outcomes do not depend on other units' treatments. 
  This is sometimes called the no-interference assumption.
\end{assumption}
\begin{assumption}[Consistency]\label{assumption:consistency}
  \

  There are no other versions of the treatment. 
  Equivalently, we require that the treatment levels be well-defined, 
  or have no ambiguity at least for the outcome of interest. 
  This is sometimes called the consistency assumption.
\end{assumption}

The causal effect of the treatment on the $i$-th unit is then defined as:
\begin{gather*}
  \Delta_i = y_{1i} - y_{0i} 
\end{gather*}
These potential outcomes are constants at the level of unit $i$.

\begin{remark}[Problem of causal inference]
  \

  The fundamental problem in causal inference is that only one treatment can be assigned to a given individual, 
  and so only one of $y_{0i}$ and $y_{1i}$ can be observed. Thus $\Delta_i$ can never be observed.
\end{remark}

\begin{definition}[Stable Unit Treatment Value Assumption (SUTVA)]
\label{def:sutva}
  \

  Rubin (1980\cite{rubin1980comment}) called the Assumptions \ref{assumption:no_interference} and \ref{assumption:consistency} above together 
  the \textit{Stable Unit Treatment Value Assumption (SUTVA).}
\end{definition}
The observed outcome of unit $i$ is a function of the potential
outcomes and the treatment indicator, we can write:
\begin{gather*}
  y_i = d_i y_{1i} + (1 - d_i) y_{0i}
\end{gather*}
In principle, by virtue of being (discrete) RVs, both $d_i$ and $y_i$ each have a distribution function,
which, together with their possible realizations, defines various moments.
However, their unconditional probabilities and moments at the level of unit $i$ is not of interest.
Only the conditional probabilities of $y_i$ given $d_i$ is of interest.

\begin{remark}[Rubin (2005\cite{rubin2005causal})]
  \

  Under SUTVA, Rubin (2005) called the $n \times 2$ matrix of potential outcomes the Science Table:
  $$\begin{array}{ccc}
  \hline
  i & y_{1i}  & y_{0i}  \\
  \hline
  1 & y_{11}  & y_{01}  \\
  2 & y_{12}  & y_{02}  \\
  \vdots & \vdots & \vdots \\
  n & y_{1n}  & y_{0n} \\
  \hline
  \end{array}$$
  Due to the fundamental contributions of Neyman and Rubin to statistical causal inference, the potential outcomes framework is sometimes referred to as the Neyman Model, 
  the Neyman-Rubin Model, or the Rubin Causal Model.
  Causal effects are functions of the Science Table. Inferring individual causal effects
  $$\tau_i = y_{1i}  - y_{0i} , \quad (i=1,\ldots,n)$$
  is fundamentally challenging because we can only observe either $y_{1i} $ or $y_{0i}$,
  for each unit $i$, that is, we can observe only half of the Science Table.
\end{remark}

SUTVA(\ref{def:sutva}) ensures that the individual treatment effect is well defined.

Now, although $\Delta_i$ itself is unobservable, we can (perhaps remarkably) 
use randomized experiments to learn certain properties of it. The expectations
$\mathbb{E}[y_{0i}]$ and $\mathbb{E}[y_{1i}]$ denote the average potential outcomes across unit $i$ in population.

In particular, large randomized experiments let us recover the \textbf{Average
Treatment Effect (ATE)}:
\begin{gather*}
  \text{ATE} = \mathbb{E}[y_{1i} - y_{0i}] = \mathbb{E}[y_{1i}] - \mathbb{E}[y_{0i}]
\end{gather*}

For a population, we can define the treatment conditional expectations:
\[\mathbb{E}[y_i | d_i=1], \mathbb{E}[y_{0i} | d_i=1 ], \mathbb{E}[y_{1i} | d_i=1 ] = \mathbb{E}[y_i | d_i=1]\]
that denote the averages of the outcome $y_i$.

Analogously, we can define the control conditional expectations:
\[\mathbb{E}[y_i | d_i=0], \mathbb{E}[y_{0i} | d_i=0 ] = \mathbb{E}[y_i | d_i=0], \mathbb{E}[y_{1i} | d_i=0 ]\]
for the non-treated subpopulation.

Similar to ATE, we candefine the Average Treatment Effect for the Treatment-Group (ATT) and the Average
Treatment Effect for the Control-Group (ATC) as distinct objects:
\begin{align*}
  \text{ATT} &= \mathbb{E}[y_{1i}- y_{0i} | d_i=1]\\
  \text{ATC} &= \mathbb{E}[y_{1i}- y_{0i} | d_i=0]
\end{align*}
\[\mathbb{E}[z] = \mathbb{E}[z|d=1] \mathbb{P}[d=1] + \mathbb{E}[z|d=0] \mathbb{P}[d=0] = \mathbb{E}[\mathbb{E}[z|d]].\]

For sample, $\{d_i, y_i\}_{i=1}^n = \{d_i, y_{d_{i}, i}\}_{i=1}^n$, because
$y_i = y_{1i} d_i + y_{0i} (1 - d_i)$.

$N = \{i=1,2,\cdots, n\}$, $N_1 = \{i \in N: d_i = 1\} \leftarrow n_1 = \vert N_1 \vert $, $N_0 = \{i: d_i = 0\} \leftarrow n_0 = \vert N_0 \vert $.
\begin{align*}
  \frac{1}{n_1} \sum_{i \in N_1} y_i &= \frac{1}{n_1} \sum_{i \in N_1} y_{1i} \overset{p}{\rightarrow} \mathbb{E}[y_{1i} | d_i=1] = \mathbb{E}[y_i | d_i=1]\\
  \frac{1}{n_0} \sum_{i \in N_0} y_i &= \frac{1}{n_0} \sum_{i \in N_0} y_{0i} \overset{p}{\rightarrow} \mathbb{E}[y_{0i} | d_i=0] = \mathbb{E}[y_i | d_i=0]
\end{align*}
\[\frac{1}{n_1} \sum_{i \in N_1} y_i - \frac{1}{n_0} \sum_{i \in N_0} y_i \overset{p}{\rightarrow} \mathbb{E}[y_{1i} | d_i = 1] - \mathbb{E}[y_{0i} | d_i = 0] = \text{ATE} = \text{ATT} = \text{ATC}.\]

We define the difference of treated and non-treated as: \textit{Naive Difference}.
\begin{align*}
  \text{ND} &= \mathbb{E}[y_{1i} | d_i = 1] - \mathbb{E}[y_{0i} | d_i = 0]\\
  &= \mathbb{E}[y_{1i} | d_i = 1] - \mathbb{E}[y_{0i} | d_i = 1] + \mathbb{E}[y_{0i} | d_i =1] - \mathbb{E}[y_{0i} | d_i = 0] \\
  &= ATT + \mathbb{E}[y_{0i} | d_i = 1] - \mathbb{E}[y_{0i} | d_i = 0]
\end{align*}

For LRM, $y_i = \beta_0 + \beta_1 d_i + u_i$,
\begin{align*}
  \text{ND} &= \mathbb{E}[y_i | d_i = 1] - \mathbb{E}[y_i | d_i = 0]\\
  &= \mathbb{E}[\beta_0 + \beta_1 + u_i | d_i = 1] - \mathbb{E}[\beta_0 + u_i | d_i = 0]\\
  &= \beta_1 + \mathbb{E}[u_i | d_i = 1] - \mathbb{E}[u_i | d_i = 0]  
\end{align*}



\[
\{Y_d\} \perp\!\!\!\perp D \mid X \implies \{Y_d\} \perp\!\!\!\perp D \mid \pi(X), \quad D \perp\!\!\!\perp X \mid \pi(X)
\]