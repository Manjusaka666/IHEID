\section{Fundamentals of Time Series Analysis}
\label{sec:fundamentals-of-time-series-analysis}

A \textbf{time series} $Y_t \in \mathbb{R}^m$ is a process which is sequentially ordered over time.
The time series is univariate if $m = 1$ and multivariate if $m > 1$.

Most economic time series are recorded at discrete intervals such as annual, quarterly, monthly,
weekly, or daily. The number of observed periods $s$ per year is called the \textbf{frequency}.
In most cases we will denote the observed sample by the periods $t = 1, \cdots, n$. 

% Suppose we have a sample $\{w_i\}_{i=1}^n$, with $w_i = (y_i, x_i^{\prime})^{\prime}$, $\{w_{it}\}_{i=1:n, t=1:T}$.

% Now, we look at $\{w_t\}_{i=1}^T$, usually written as $y_t$, is univariate time series data.
Recall that cross-sectional observations are conventionally treated as random draws from an under-
lying population. This is not an appropriate model for time series processes due to serial dependence.
Instead, we treat the observed sample $\{y_1, \cdots, y_n\}$ as a realization of a dependent stochastic process. It is
often useful to view $\{y_1, \cdots, y_n\}$ as a subset of an underlying doubly-infinite sequence $\{\cdots, y_{t-1}, y_t, y_{t+1}, \cdots \}$.

A random vector $Y_t$ can be characterized by its distribution. A set such as $\{y_t, y_{t+1}, \cdots, y_{t+l} \}$ can be
characterized by its joint distribution. Important features of these distributions are their \textbf{means, vari-
ances, and covariances}.

\begin{remark}
    \

    Time series theory is a mixture of probabilistic and statistical concepts. 
    The probabilistic part is to study and characterize probability distributions 
    of sets of variables $y_t$ that will typically be dependent.
    The statistical problem is to determine the probability distribution of the time series
    given observations $y_1, \cdots, y_n$ at times $1,2, \cdots, n.$
    The resulting stochastic model can be used in two ways:
    \begin{itemize}
        \item understanding the stochastic system;
        \item predicting the ``future'', i.e. $y_{n+1}, y_{n+2}, \cdots$
    \end{itemize}
\end{remark}
% In the cross-sectional context, we average over $i$ to get 
% \[\mathbb{E}[y_i] = \int y_i f_y(y_i) d y_i.\]


As mentioned above, under time series data, we care about the joint distribution of random variable $y_t$.

We give the following definitions of the mean, variance, and covariance of a random variable $y_t$.
\begin{definition}[Mean function]\label{def:mean-function}
    \

    The mean function of a random variable $y_t$ is defined as
    \begin{gather*}
        \mu_t = \mathbb{E}[y_t] = \int y_t f_{t}(y_t) d y_t,
    \end{gather*}
    where $f_{t}(y_t)$ is the probability density function (PDF) of $y_t$.
\end{definition}

\begin{definition}[Autocovariance function]\label{def:autocovariance-function}
    \

    The autocovariance function of a random variable $y_t$ is defined as
    \begin{gather*}
        \gamma_{y}(r,s) = \Cov(y_r, y_s) = \mathbb{E}\Bigl[ (y_r - \mu_r)(y_s - \mu_s) \Bigr].
    \end{gather*}    
\end{definition}

\begin{definition}[Autocorrelation function]\label{def:autocorrelation-function}
    \

    The autocorrelation function (ACF) of a random variable $y_t$ is defined as
    \begin{gather*}
        \rho_{y}(r,s) = \frac{\Cov(y_r, y_s)}{\sqrt{\operatorname{Var}(y_r) \operatorname{Var}(y_s)}} = \frac{\gamma_{y}(r,s)}{\sqrt{\gamma_{y}(r,r) \gamma_{y}(s,s)}}.
    \end{gather*}
    
\end{definition}

\subsection{Stationarity and Strict Stationarity}

\begin{definition}[Stationarity]\label{def:weak-stationarity}
    \

    The time serie $\{y_t, t \in \mathbb{Z}\}$, is said to be stationary if:
    \begin{enumerate}
        \item[(i)] $\mathbb{E}[\vert y_t \vert ^2] < \infty$, for all $t$,
        \item[(ii)] $\mathbb{E}[y_t] = \mu $, for all $t$,
        \item[(iii)] $\gamma_y(r,s) = \gamma_y(r+t, s+t)$, for all $r,s,t \in \mathbb{Z}$. 
    \end{enumerate}
\end{definition}

\begin{remark}[About Stationarity]
    \
    
    Stationarity as just defined is frequently referred to in the literature as weak stationarity,
    covariance stationarity, stationarity in the wide sense or second-order stationarity.
    For us however the term stationarity, without further qualification, 
    will always refer to the properties specified by Definition \ref{def:weak-stationarity},
    that is, when we say stationary, we mean weak stationary.
\end{remark}

If $\{y_t, t \in \mathbb{Z}\}$ is \textbf{stationary}, then $\gamma_y(r,s) = \gamma_y(r-s, 0)$ for all $r,s \in \mathbb{Z}$.
It is therefore convenient to redefine the autocovariance function of a stationary
process as the function of just one variable,
\begin{gather*}
    \gamma_y(h) \equiv \gamma_y(h, 0) = \Cov(y_{t+h}, y_t), \quad \forall t, h \in \mathbb{Z}.
\end{gather*}
The function $\gamma_y(\cdot)$ will be referred to as the autocovariance function of $\{y_t\}$
and $\gamma_y(h)$ as its value at lag $h$. The autocorrelation function (ACF) of $\{y_t\}$
is defined analogously as the function whose value at lag $h$ is given by
\begin{gather*}
    \rho_y(h) = \frac{\gamma_y(t+h, t)}{\sqrt{\gamma_y(t+h, t) \gamma_y(t,t)} } = \frac{\gamma_y(h)}{\gamma_y(0)} = \operatorname{Corr}(y_{t+h}, y_t).
\end{gather*}
The auto-covariance and auto-correlation are functions $\gamma_y: \mathbb{Z} \to \mathbb{R}$ and $\rho_y: \mathbb{Z} \to [-1,1].$
Together with the mean $\mu = \mathbb{E}[y_t]$, they determine
the first and second moments of the stationary time series. 

% We also think of $y_t$ as a random variable (RV). Without the i.i.d. assumption,
% we generally have $T$ realizations of different and mutually dependent variables.
% \begin{gather*}
%     \mathbb{E}[y_t] = \int y_t f_{t}(y_t) d y_t = \mu_t, \\
%     \mathbb{V}[y_t] = \mathbb{E}\Bigl[(y_t - \mu_t)^2 \Bigr] = \gamma_{0,t}, \\
%     \Cov(y_t, y_{t-h}) = \mathbb{E}\Bigl[ (y_t - \mu_t)(y_{t-h} - \mu_{t-h}  ) \Bigr] = \gamma_{h,t}.
% \end{gather*}

% \begin{definition}[Weak Stationarity]\label{def:weak-stationarity}
%     \

%     $y_t$ is a weakly stationary process if
%     \begin{enumerate}
%         \item $\mu_t = \mu$ for all $t$,
%         \item $\gamma_{0,t} = \gamma_0$ for all $t$,
%         \item $\gamma_{h,t} = \gamma_h$ for all $t$.
%     \end{enumerate}
%     autocovariance function (ACF): $\{\gamma_0, \gamma_1, \cdots\}$
%     autocorrelation function: $\{\rho_0, \rho_1, \cdots \}$, where $\rho_h = \frac{\gamma_h}{\gamma_0}$.
% \end{definition}


\begin{definition}[Strict Stationarity]\label{def:strict-stationarity}
    \

    The time series $\{y_t, t \in \mathbb{Z}\}$ is said to be strictly stationary if the joint distribution of $(y_{t_1}, y_{t_2}, \cdots, y_{t_k})$
    and $(y_{t_1+h}, y_{t_2+h}, \cdots, y_{t_k+h})$ are the same for all $h$ and $k$ and $t_1, \cdots, t_k$.
\end{definition}
% Strict stationarity means intuitively that the graphs over two equal-length
% time intervals of a realization of the time series should exhibit similar statistical
% characteristics. For example, the proportion of ordinates not exceeding a
% given level $y$ should be roughly the same for both intervals.
This is the natural generalization of the cross-section definition of identical distributions.
Strict stationarity implies that the (marginal) distribution of $y_t$ doesn't vary over time.
It also implies that the bivariate distributions of $(y_t, y_{t+1})$ and multivariate distributions of $(y_t, \cdots, y_{t+l})$
are stable over time.

\subsection*{The Relation Between Stationarity and Strict Stationarity}

If $\{y_t\}$ is strict stationary, it immediately follows, on taking $k=1$ in Definition \ref{def:strict-stationarity}, 
that $y_t$ has the same distribution for all $t$. If $\mathbb{E}[\vert y_t \vert ^2] < \infty $,
this implies in particular that $\mathbb{E}[y_t]$ and $\mathbb{V}[y_t]$ are constant.
Moreover, taking $k=2$, we find that $y_{t+h}$ and $y_t$ have the same joint distribution,
and hence the same covariance for all $h$. 
Thus a strictly stationary process with finite second moments is stationary.

The converse of the previous statement is not true. For example if $\{y_t\}$ is a sequence
of independent RVs s.t. $y_t$ is exponentially distributed with mean 1 when $t$ is odd
and normally distributed with mean 1 and variance 1 when $t$ is even, then $\{y_t\}$ is stationary
with $\gamma_y(0)=1$ and $\gamma_y(h) = 0$ for $h \neq 0$. However since $y_1$ and $y_2$ have different distributions,
$\{y_t\}$ cannot be strictly stationary.

\begin{note}
    \

    There is one important case however in which stationarity does imply strict stationarity.
\end{note}

\begin{definition}[Gaussian Time Series]\label{def:gaussian-time-series}
    \

    The process $\{y_t\}$ is a Gaussian time series if and only if the distribution functions of $\{y_t\}$ are all multivariate normal.
\end{definition}
If $\{y_t\}$ is a stationary Gaussian time series, then for all $k \in \{1, 2, \cdots\}$ and for all $h, t_1, t_2, \cdots$,
the random vectors $(y_{t_1}, \cdots, y_{t_k})^{\prime}$ and $(y_{t_1 + h}, \cdots, y_{t_k + h})^{\prime} $
have the same mean and covariance matrix, and hence the same distribution. $\{y_t\}$ is strictly stationary.

A straightforward but essential relationship is that an i.i.d. process is strictly stationary.
\begin{theorem}[IID Strict Stationary]\label{thm:iid-stationary}
    \

    If $\{y_t\}$ is an i.i.d. process, then $\{y_t\}$ is strictly stationary.
\end{theorem}

\subsection{Transformations of Stationary Processes}
One of the important properties of strict stationarity is that it is preserved by transformation. That is,
transformations of strictly stationary processes are also strictly stationary. This includes transformations
which include the full history of $y_t$.
\begin{theorem}[transformation Invariance]\label{thm:transformation-invariance}
    \

    If $\{y_t\}$ is strict stationary, and $X_t = \phi (y_t, y_{t-1}, \cdots) \in \mathbb{R}^q$ is a random vector
    then $X_t$ is also strict stationary.   
\end{theorem}

A transformation which includes the full past history is an infinite-order moving average.
For scalar $y$ and coefficients $a_j$ define the vector process
\begin{gather*}
    X_t = \sum_{j=0}^{\infty} a_j y_{t-j}.
\end{gather*}
Many time-series models involve representations and transformations of this form.

This infinite series exists if ti is convergent, meaning that the sequence $\sum_{j=0}^{N}a_j y_{t-j}$
has a finite limit as $N \to \infty$.  Since the inputs $y_t$ are random we define this as a probability limit.

\begin{definition}[Convergence]\label{def:convergence}
    \

    The infinite series $\{X_t\}$ converges almost surely if $\sum_{j=0}^{N} a_j y_{t-j}$
    has a finite limit as $N \to \infty$ with probability 1.
    In this case we describe $X_t$ as convergent.
    
\end{definition}
\begin{theorem}[Convergence-Stationary]\label{thm:convergence-stationary}
    \

    If $\{y_t\}$ is a strict stationary process, $\mathbb{E}[y] < \infty$ 
    and $\sum_{j=0}^{\infty} \vert a_j \vert < \infty$,
    then $X_t$ converges almost surely,
    and the process $\{X_t\}$ is strict stationary.  
\end{theorem}

\subsection{Ergodicity}
\label{sec:ergodicity}

Stationarity alone is not sufficient for the weak law of large numbers as there are strictly stationary
processes with no time series variation.

\begin{eg}
    \
    
    If the stationary process is $y_t = Z$ for some RV $Z$.
    This is random but constant over all time. 
    An implication is that the sample mean of $y_t = Z$ will be inconsistent for the population expectation.
\end{eg}

We now motivate the concept of \textbf{ergodicity}\footnotemark. Conceptionally, this is more difficult to understand
than the mean and variance. But it is a very helpful tool when analysing estimators. It allows one
to simply replace the sample mean by its expectation without the need to evaluating a variance,
which is extremely useful in some situations.

\footnotetext{In the late 1800s, the physicist Ludwig Boltzmann needed a word to express the idea that if you took
an isolated system at constant energy and let it run, any one trajectory, continued long enough, would
be representative of the system as a whole. Being a highly-educated nineteenth century German-speaker,
Boltzmann knew far too much ancient Greek, so he called this the ``ergodic property'', from \textit{ergon} ``energy,
work'' and hodos ``way, path''. The name stuck.}

It can be difficult to evaluate the mean and variance of an estimator. Therefore, we may want
an alternative form of convergence (instead of the mean squared error). To see whether this is
possible we recall that for i.i.d random variables we have the very useful law of large numbers
\begin{gather*}
    \frac{1}{n} \sum_{t=1}^{n} y_t \xrightarrow{a.s.} \mathbb{E}[y_t] = \mu.
\end{gather*}
and in general $\frac{1}{n} \sum_{t=1}^{n} g(y_t) \xrightarrow{a.s.} \mathbb{E}[g(y)]$ if $\mathbb{E}[g(y)] < \infty$.
Does such a result exists in time series? 
It does, but we require the slightly stronger condition that a time series is ergodic 
(which is a slightly stronger condition than the strictly stationary).

% What is a minimal assumption beyond stationarity so that the law of large numbers applies? This
% topic is called \textbf{ergodicity}.
% It is sufficiently important that it is treated as a separate area of study. We
% mention only a few highlights here. For a rigorous treatment see a standard textbook such as Walters
% (1982).

A useful intuition is that if $y_t$ is ergodic then its sample paths will pass through all parts of the sample
space never getting ``stuck'' in a subregion. 
% Or, in a more plain language, it ensures that observations become independent when we consider large enough displacements.

% \begin{definition}[Ergodicity]\label{def:ergodicity}
%     \

%     A process is said to be ergodic if time averages converge to ensemble averages.
%     \begin{gather*}
%         \lim_{n \to \infty} \left| \mathbb{E}[f(y_{t_1}, \cdots, y_{t_k}) g(y_{t_1 + n}, \cdots, y_{t_k + n})] \right| - \left| \mathbb{E}[f(y_{t_1}, \cdots, y_{t_k})] \right| \left| \mathbb{E}[g(y_{t_1}, \cdots, y_{t_k})] \right| = 0, \\
%         \forall t_1, \cdots, t_k, \text{ taking } k>l \text{ w.l.o.g}. 
%     \end{gather*}
% \end{definition}

\begin{definition}[Ergodicity: Formal Definition*]\label{def:ergodicity}
    \

    Let $(\Omega, \mathcal{F}, P)$ be a probability space.
    A transformation $T: \Omega \to \Omega$ is said to be a measure preserving
    if for every set $A \in \mathcal{F}$, $P(T^{-1}(A)) = P(A)$.
    Moreover, it is said to be an ergodic transformation if $T^{-1} A = A$ implies $P(A)=0$ or $1$.

    It is not obvious what this has to do with stochastic processes, but we attempt to make a link.
    Let us suppose that $y = \{y_t\}$ is a strictly stationary process defined on the probability space $(\Omega, \mathcal{F}, P)$.
    By strict stationarity the transformation (shifting a sequence by one)
    \begin{gather*}
        T(y_1, y_2, \cdots) = (y_2, y_3, \cdots)
    \end{gather*}
    is a measure preserving transformation. 
    To understand ergodicity we define the set $A$, where
    \begin{gather*}
        A = \{\omega: \left( y_1(\omega), y_0(\omega), \cdots \right) \in H \} = \{ \omega: \left( y_{-1}(\omega), y_{-2}(\omega), \cdots \right) \in H \}.
    \end{gather*}
    The stochastic process is said to be ergodic, 
    if the only sets which satisfies the above are such that $P(A) = 0$ or $1$.
    Roughly, this means there cannot be too many outcomes $\omega$ which generate sequences which `repeat' itself (are periodic in some sense).

    An equivalent definition is given later. From this definition it can be seen why `repeats' are a bad idea. 
    If a sequence repeats, the time average is unlikely to converge to the mean.
\end{definition}

The definition of ergodicity, given above, is quite complex and is rarely used in time series analysis.
However, one consequence of ergodicity is the ergodic theorem, which is extremely useful in time
series.

\begin{theorem}[Ergodic Theorem]\label{thm:ergodic-theorem}
    \

    If $\{y_t\}$ is a strict stationary process, and $\mathbb{E}[y] = \mu < \infty$, then as $n \to \infty$:
    \begin{gather*}
        \mathbb{E}\left[ \frac{1}{n} \sum_{t=1}^{n} y_t - \mu \right] \to 0
    \end{gather*}
    and
    \begin{gather*}
        \frac{1}{n} \sum_{t=1}^{n} y_t \xrightarrow{a.s.} \mu.
    \end{gather*} 
\end{theorem}

\begin{proposition}[CLT for SS \& Ergodic Processes]\label{prop:clt-ss-ergodic}
    \
    
    If the process $\{y_t\}$ is strict stationary and ergodic, and $\mathbb{E}[y] = \mu < \infty$,
    $\mathbb{V}[y_t] = \sigma^2 < \infty$, and $\bar{\sigma}_T^2 = \mathbb{V}[\frac{1}{\sqrt{T}}\sum_{t=1}^{T} y_t] \xrightarrow{p} \bar{\sigma}^2 < \infty$,
    then
    \begin{gather*}
        \frac{1}{\sqrt{T}} \sum_{t=1}^{T} y_t \xrightarrow{d} \mathcal{N}(\mu, \bar{\sigma}^2).
    \end{gather*}
\end{proposition}

\begin{definition}[Ergodicity: Plain Definition]\label{def:ergodicity-plain}
    \

    In general, for any shift $\tau_1, \cdots, \tau_k$, and function $g: \mathbb{R}^{k+1} \to \mathbb{R}$,
    we have:
    \begin{gather*}
        \frac{1}{n} \sum_{t=1}^{n} g(y_t, y_{t+\tau_1}, \cdots, y_{t+\tau_k}) \xrightarrow{a.s.} \mathbb{E}[g(y, y_{t+\tau_1}, \cdots, y_{t+\tau_k})].
    \end{gather*}
\end{definition}
This is mostly used as the definition of ergodicity,
as it is an iff with the ergodic definition.
This result generalises the strong law of large numbers
(which shows almost sure convergence for i.i.d. random variables)
to dependent random variables.

Definition \ref{def:ergodicity-plain} also gives us an idea of what constitutes an ergodic process.
Suppose that $\{\epsilon_t\}$ is an ergodic process (a classical example are i.i.d. random variables).
Then any reasonable (meaning measurable) function of $\{y_t\}$ is also ergodic
if $y_t$ is defined as:
\begin{gather*}
    y_t = h(\cdots, \epsilon_{t-1}, \epsilon_t, \epsilon_{t+1}, \cdots)
\end{gather*}
where $\{\epsilon_t\}$ are i.i.d. RVs and $h$ is a measurable function.

\begin{remark}
    \

    Definition \ref{def:ergodicity-plain} is roughly equivalent to:
    \begin{gather*}
        \lim_{n \to \infty} \frac{1}{n} \sum_{l=1}^{n} \mathbb{P}[A_l \cap B] = \mathbb{P}[A] \mathbb{P}[B],
    \end{gather*}
    which is a kind of asymptotic independence-on-average.
\end{remark}

Many standard time series processes can be shown to be ergodic.
A useful starting point is the observation that an i.i.d. sequence is ergodic. 

\begin{theorem}[IID Ergodic]\label{thm:iid-ergodic}
    \

    If $\{y_t\}$ is an i.i.d. process, then $\{y_t\}$ is strict stationary and ergodic.    
\end{theorem}

Second, ergodicity, like stationarity, is preserved by transformation.
\begin{theorem}[Ergodicity transformation Invariance]\label{thm:ergodicity-transformation-invariance}
    \

    If $\{y_t\} \in \mathbb{R}^m$ is strict stationary and ergodic,
    and $X_t = \phi (y_t, y_{t-1}, \cdots) \in \mathbb{R}^q$ is a random vector,
    then $X_t$ is also strict stationary and ergodic.
\end{theorem}

As an example, the infinite-order moving average transformation is ergodic
if the input is ergodic and the coefficients are absolutely convergent.

\begin{theorem}
    \

    If $y_t$ is strict stationary and ergodic, $\mathbb{E}[y] < \infty$,
    and $\sum_{j=0}^{\infty} \vert a_j \vert < \infty$,
    then $X_t = \sum_{j=0}^{\infty} a_j y_{t-j}$ is strict stationary and ergodic.    
\end{theorem}

We now present a useful property. It is that the Cesàro sum of the autocovariances limits to zero.

\begin{theorem}[Cesàro Sum of Autocovariances]\label{thm:cesaro-sum-autocovariances}
    \

    If $\{y_t\} \in \mathbb{R}$ is strict stationary and ergodic, and $\mathbb{E}[y^2] < \infty$,
    then the Cesàro sum of the autocovariances converges to zero:
    \begin{gather*}
        \lim_{n \to \infty} \frac{1}{n} \sum_{l=1}^{n} \Cov[y_t, y_{t+l}] = 0.
    \end{gather*}    
\end{theorem}

\subsection{Conditioning on Information Sets}

In the past few sections we have introduced the concept of the infinite histories. We now consider
conditional expectations given infinite histories.

Recall from probability theory that an \textbf{outcome} is an element of a sample space.
An \textbf{event} is a set of outcomes.

Now we wish to define a conditional expectation given an infinite past history. Specifically, we wish
to define
\begin{gather*}
    \mathbb{E}_{t-1}[y_t] = \mathbb{E}\left[y_t | y_{t-1}, y_{t-2}, \cdots \right].
\end{gather*}
the expected value of $y_t$ given the infinite past history $\tilde{y}_{t-1} = (y_{t-1}, y_{t-2}, \cdots)$ up to time $t$.
Intuitively, $\mathbb{E}_{t-1}[y_t]$ is the mean of the conditional distribution, the latter reflecting the information in the history.

Formally, in time series literature, we should follow the measure-theoretic approach to define $\mathbb{E}_{t-1}[y_t]$ as 
the conditional expectation given a $\sigma$-field:
\begin{gather*}
    \mathbb{E}_{t-1}[y_t] = \mathbb{E}\left[y_t | \mathcal{F}_{t-1} \right]
\end{gather*}
where $\mathcal{F}_{t-1}$ is the $\sigma$-field generated by the infinite past history $\tilde{y}_{t-1}$.
\footnote{A $\sigma$-field is a collection of subsets of a sample space which is closed under complementation and countable unions.}

The $\sigma$-field generated by a random variable $Y$ is the collection of measurable events involving $Y$.
Similarly, the $\sigma$-field generated by an infinite history is the collection of measurable events involving this history.

Intuitively, $\mathcal{F}_{t-1}$ contains all the information available in the history $\tilde{y}_{t-1}$.
Consequently, economists typically call $\mathcal{F}_{t-1}$ an \textbf{information set} rather than a $\sigma$-field.

We now describe some properties about information sets $\mathcal{F}_t$:
\begin{enumerate}
    \item[(i)] $\mathcal{F}_{t-1} \subset \mathcal{F}_{t}$. This means that information accumulates over time. Information
    is not lost. 
    \item[(ii)] It is important to be precise about which variables are contained in the information set.
    For example, the information sets $\mathcal{F}_{1t} = \sigma (y_t, y_{t-1}, \cdots)$ and $\mathcal{F}_{2t} = \sigma (y_t, x_t, y_{t-1}, x_{t-1}, \cdots)$ are distinct even though they are both dated at time $t$.
    \item[(iii)] the conditional expectations (14.10) follow the law of iterated expectations and the conditioning theorem, thus:
    \begin{gather*}
        \mathbb{E}\left[ \mathbb{E}[y_t | \mathcal{F}_{t-1}] | \mathcal{F}_{t-2} \right] = \mathbb{E}[y_t | \mathcal{F}_{t-2}], \\
        \mathbb{E}\left[ \mathbb{E}[y_t | \mathcal{F}_{t-1}] \right] = \mathbb{E}[y_t].
    \end{gather*}
    and
    \begin{gather*}
        \mathbb{E}\left[y_{t-1} y_t | \mathcal{F}_{t-1} \right] = y_{t-1} \mathbb{E}[y_t | \mathcal{F}_{t-1}].
    \end{gather*}
\end{enumerate}

\subsection{Martingale Difference Sequences}
An important concept in economics is unforecastability, meaning that the conditional expectation is
the unconditional expectation. This is similar to the properties of a regression error. An unforecastable
process is called a \textbf{martingale difference sequence (MDS)}.

A MDS $y_t$ is defined with respect to a specific sequence of information sets $\mathcal{F}_t$.
Most commonly, the latter are the natural filtration $\mathcal{F}_t = \sigma(y_t, y_{t-1}, \cdots)$
(the past history of $y_t$) but it could be a larger information set.
The only requirement is that $y_t$ is adapted to $\mathcal{F}_t$, meaning that $\mathbb{E}[y_t | \mathcal{F}_t] = y_t$.

\begin{definition}[Martingale Difference Sequence]\label{def:martingale-difference-sequence}
    \

    A process $\{y_t, \mathcal{F}_t\}$ is a martingale difference sequence (MDS) if $y_t$ is adapted to $\mathcal{F}_t$, $\mathbb{E}[y_t] < \infty$:
    \begin{gather*}
        \mathbb{E}[y_t | \mathcal{F}_{t-1}] = 0, \quad \forall t.
    \end{gather*}
\end{definition}

In words, a MDS $y_t$ is unforecastable in the mean. If we apply the iterated expectations, $\mathbb{E}[y_t] = \mathbb{E}\left[ \mathbb{E}[y_t | \mathcal{F}_{t-1}] \right] = 0$,
thus a MDS is mean zero.

\begin{note}
   \
   
   The term ``martingale difference sequence'' refers to the fact that the summed process $S_t = \sum_{j=1}^{t} y_j $, is a martingale and $y_t$ is its first-difference. 
   \begin{definition}[Martingale]\label{def:martingale}
       \

       A process $\{S_t, \mathcal{F}_t\}$ is a martingale if $S_t$ is adapted to $\mathcal{F}_t$, $\mathbb{E}[S_t] < \infty$:
       \begin{gather*}
           \mathbb{E}[S_t | \mathcal{F}_{t-1}] = S_{t-1}, \quad \forall t.
       \end{gather*}
   \end{definition}
\end{note}

\begin{proposition}[I.I.D. and MDS]\label{prop:mds-property-1}
    \

    If $y_t$ is i.i.d. and mean zero, it is a MDS but the reverse is not the case. 
\end{proposition}
\begin{proof}
    \

    Suppose that $y_t$ is i.i.d. and mean zero. It is then independent from $\mathcal{F}_{t-1} = \sigma(y_{t-1}, y_{t-2}, \cdots)$,
    so $\mathbb{E}[y_t | \mathcal{F}_{t-1}] = \mathbb{E}[y_t] = 0$.
    Thus an i.i.d. shock is a MDS as claimed.

    For the reverse, we let $u_t \sim \mathcal{N} (0,1)$ is i.i.d. and set $y_t = u_t u_{t-1}.$
    By the conditioning theorem,
    \begin{gather*}
        \mathbb{E}[y_t | \mathcal{F}_{t-1}] = \mathbb{E}[u_t u_{t-1} | \mathcal{F}_{t-1}] = u_{t-1} \mathbb{E}[u_t | \mathcal{F}_{t-1}] = 0.
    \end{gather*}
    So $y_t$ is a MDS. However, $y_t$ is not i.i.d., which can be shown by calculating the first autocovariance of $y_t^2$:
    \begin{align*}
        \Cov[y_t^2, y_{t-1}^2] &= \mathbb{E}[y_t^2 y_{t-1}^2] - \mathbb{E}[y_t^2] \mathbb{E}[y_{t-1}^2] \\
        &= \mathbb{E}[u_t^2]\mathbb{E}[u_{t-1}^4]\mathbb{E}[u_{t-2}^2] - 1 \\
        &= 2 \neq 0.
    \end{align*}
    Since the covariance is non-zero, $y_t$ is not an independent sequence.
    Thus $y_t$ is a MDS but not i.i.d.
\end{proof}

\begin{theorem}[Serial Uncorrelation and MDS]\label{thm:mds-serial-uncorrelated}
    \

    If $\{y_t, \mathcal{F}_t\}$ is a MDS, and $\mathbb{E}[y_t^2] < \infty$, then
    $y_t$ is serially uncorrelated.
\end{theorem}
\begin{proof}
    \

    Take the process $y_t = u_t + u_{t-1} u_{t-2}$ with $u_t \sim \mathcal{N}(0,1)$, i.i.d.
    The process is not MDS because $\mathbb{E}[y_t | \mathcal{F}_{t-1}] = u_{t-1} u_{t-2} \neq 0$,
    however,
    \begin{align*}
        \Cov[y_t, y_{t-1}] &= \mathbb{E}[y_t y_{t-1}] \\
        &= \mathbb{E}\left[ (u_t + u_{t-1} u_{t-2}) (u_{t-1} + u_{t-2} u_{t-3}) \right] \\
        &= \mathbb{E}\left[ u_t u_{t-1} + u_t u_{t-2} u_{t-3} + u_{t-1}^2 u_{t-2} + u_{t-1} u_{t-2}^2 u_{t-3} \right] \\
        &= \mathbb{E}[u_t] \mathbb{E}[u_{t-1}] + \mathbb{E}[u_t] \mathbb{E}[u_{t-2}] \mathbb{E}[u_{t-3}] + \mathbb{E}[u_{t-1}^2] \mathbb{E}[u_{t-2}] + \mathbb{E}[u_{t-1}] \mathbb{E}[u_{t-2}^2] \mathbb{E}[u_{t-3}] \\
        &= 0
    \end{align*}
    Similarly, $\Cov[y_t, y_{t+k}] = 0$ for all $k \neq 0$. Thus the process is serially uncorrelated.
\end{proof}

\begin{definition}[Homoskedastic Martingale Difference Sequence]\label{homoskedastic-mds}
    \

    A MDS $\{y_t, \mathcal{F}_t\}$ is said to be homoskedastic if
    \begin{gather*}
        \mathbb{E}[y_t^2 | \mathcal{F}_{t-1}] = \sigma^2 < \infty, \quad \forall t.
    \end{gather*}
    where $\sigma^2$ is a constant.
\end{definition}

\begin{theorem}[MDS CLT]\label{thm:mds-clt}
    \

    If $\{y_t, \mathcal{F}_t\}$ is a martingale difference sequence, $\mathbb{E}[\vert y_t \vert ^{2r}] < \infty$,
    and $\bar{\sigma}_T^2 =  \mathbb{V}\left[\frac{1}{\sqrt{T}} \sum_{t=1}^{T} y_t \right] \to \bar{\sigma}^2 > 0$, then
    \begin{gather*}
        \frac{1}{\sqrt{T}} \sum_{t=1}^{T} y_t \xrightarrow{d} \mathcal{N}(0, \bar{\sigma}^2).
    \end{gather*}
\end{theorem}