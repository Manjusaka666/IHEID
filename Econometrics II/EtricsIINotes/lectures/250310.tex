\section{Random Effects}
By defining $v_{it} = u_{it} + \alpha_i - \beta_0$, we can transform the random effect model to the following:
\begin{align*}
    y_{it} &= \alpha_1 + x_{it}^{\prime} \beta + u_{it} \\
    &= \underset{\tilde{x}_{it}^{\prime}}{\underbrace{\beta_0 + x_{it}^{\prime}\beta}} + \underset{\equiv v_{it}}{\underbrace{u_{it} + \alpha_i - \beta_0}}\\
\end{align*}

Defining again $\tilde{x}_{it} = (1, x_{it}^{\prime})^{\prime}$, $\tilde{\beta} = (\beta_0, \beta^{\prime})^{\prime}$, we can rewrite the model as:
\begin{align*}
    y_{it} &= \tilde{x}_{it}^{\prime} \beta + v_{it} \Leftrightarrow y_i = \tilde{X}_i^{\prime} \tilde{\beta} + v_i \\
    \rightarrow \hat{\tilde{\beta}} &= \left(\sum_i \tilde{X}_i^{\prime} \tilde{X}_i \right)^{-1} \sum_i \tilde{X}_i^{\prime} y_i
\end{align*}

With this intercept $beta_0$, $\mathbb{E}[v_i] = 0$ is guaranteed to hold.
Define $\tilde{\alpha}_i = \alpha_i - \beta_0$ as the mean-zero unit-specific heterogenety so that $v_i = u_i + \tilde{\alpha}_i.$


\begin{note}[POLS]
    \

    Homogenous spec: $y_{it} = \alpha  + x_{it}^{\prime} \beta + u_{it} = \tilde{x}_{it}^{\prime} \tilde{\beta} + u_{it}.$
    $\hat{\tilde{\beta}}$ is consistent if $\mathbb{E}[v_{it} x_{it}]=0, \forall t.$
\end{note}
Using pooled OLS to estimate $\hat{\tilde{\beta}}$,
\begin{align*}
    \hat{\tilde{\beta}}_{RE-OLS/POLS} &= \left(\frac{1}{n} \sum_i \tilde{X}_i^{\prime} \tilde{X}_i \right)^{-1} \frac{1}{n} \sum_i \tilde{X}_i^{\prime} y_i \\
    &= \tilde{\beta} + \left(\frac{1}{n} \sum_i \tilde{X}_i^{\prime} \tilde{X}_i \right)^{-1} \frac{1}{n} \sum_i \tilde{X}_i^{\prime} v_i \\
    &\overset{p}{\rightarrow} \tilde{\beta} + \mathbb{E}[\tilde{X}_i^{\prime} \tilde{X}_i]^{-1} \mathbb{E}[\tilde{X}_i^{\prime} v_i] \\
    \text{where} \quad \mathbb{E}[\tilde{X}_i^{\prime} v_i] &= \mathbb{E}\left[\sum_t \tilde{x}_{it}^{\prime} v_{it} \right] \\
    &= \sum_t \mathbb{E}\left[\tilde{x}_{it}^{\prime} v_{it}  \right] \\
    &= \sum_t \mathbb{E}\left[\tilde{x}_{it}(u_{it} + \alpha_i - \beta_0)\right]
\end{align*}
Here, the error term $v_i$ is not equal to the original error term $u_{it}$.
\begin{note}
    \

    Under the random effect, you have to use the heteroskedasticity-robust methods.
    Because even if we assume $u_{it}$ to be homoskedastic, $v_{it}$ is not,
    as it includes also the unit-specific heterogeneity $\alpha_i$.
\end{note}
So, to obtain consistency, we need to assume that:
\begin{itemize}
    \item $\mathbb{E}[u_{it} | \tilde{x}_{it}, \tilde{\alpha}_i] = 0, \forall t$.
    \item $\mathbb{E}[\tilde{\alpha_i} | \tilde{x}_{it}] = 0, \forall t$.
\end{itemize}
And, we are also obliged to use HAC-robust standard error because:
\[\Omega \equiv \mathbb{E}[v_i v_i^{\prime} | \tilde{X}_i] = \mathbb{E}[(\alpha_i \mathbf{1}_i + u_i)(\tilde{\alpha}_i \mathbf{1}_i + u_i)^{\prime} | \tilde{X}_i] = \mathbb{E}[\tilde{\alpha}_i^2 \mathbf{1}_i \mathbf{1}_i^{\prime} | \tilde{X}_i] + \mathbb{E}[u_i u_i'| \tilde{X}_i] \]
is not diagonal.

Given the error structure the natural estimator for $\beta$ is GLS. The GLS eimator for $\beta$ is:
\begin{align*}
    \hat{\tilde{\beta}}_{RE-GLS} &= \left( \sum_i \tilde{X}_i^{\prime} \Omega^{-1} \tilde{X}_i \right)^{-1} \sum_i \tilde{X}_i^{\prime} \Omega^{-1} y_i \\
    \Omega ^{-\frac{1}{2}} y_i &= \Omega ^{-\frac{1}{2}} \tilde{X}_i^{\prime} \tilde{\beta} + \Omega^{-\frac{1}{2}}v_i \\
    \Omega &= \mathbb{E}[v_i v_i^{\prime} | \tilde{X}_i] = \mathbb{E}\left[ \begin{bmatrix}
        v_{i1} \\
        v_{i2} \\
        \vdots \\
        v_{iT}
    \end{bmatrix} \begin{bmatrix}
        v_{i1} & v_{i2} & \cdots & v_{iT}
    \end{bmatrix} | \tilde{X}_i \right]\\
    &= \mathbb{E}\begin{bmatrix}
        \mathbb{E}[v_{i1}^2 | \tilde{X}_i] & \mathbb{E}[v_{i1}v_{i2} | \tilde{X}_i] & \cdots & \mathbb{E}[v_{i1}v_{iT} | \tilde{X}_i] \\
        \mathbb{E}[v_{i2}v_{i1} | \tilde{X}_i] & \mathbb{E}[v_{i2}^2 | \tilde{X}_i] & \cdots & \mathbb{E}[v_{i2}v_{iT} | \tilde{X}_i] \\
        \vdots & \vdots & \ddots & \vdots \\
        \mathbb{E}[v_{iT}v_{i1} | \tilde{X}_i] & \mathbb{E}[v_{iT}v_{i2} | \tilde{X}_i] & \cdots & \mathbb{E}[v_{iT}^2 | \tilde{X}_i]
    \end{bmatrix} \\
    &= \begin{bmatrix}
        \mathbb{E}\left[\alpha_i^2 | \tilde{X}_i \right] + \mathbb{E}[u_{i1}^2 | \tilde{X}_i] & \mathbb{E}\left[\alpha_i^2 | \tilde{X}_i \right] + \mathbb{E}[u_{i1}u_{i2} | \tilde{X}_i] & \cdots & \mathbb{E}\left[\alpha_i^2 | \tilde{X}_i \right] + \mathbb{E}[u_{i1}u_{iT} | \tilde{X}_i] \\
        \mathbb{E}\left[\alpha_i^2 | \tilde{X}_i \right] + \mathbb{E}[u_{i2}u_{i1} | \tilde{X}_i] & \mathbb{E}\left[\alpha_i^2 | \tilde{X}_i \right] + \mathbb{E}[u_{i2}^2 | \tilde{X}_i] & \cdots & \mathbb{E}\left[\alpha_i^2 | \tilde{X}_i \right] + \mathbb{E}[u_{i2}u_{iT} | \tilde{X}_i] \\
        \vdots & \vdots & \ddots & \vdots \\
        \mathbb{E}\left[\alpha_i^2 | \tilde{X}_i \right] + \mathbb{E}[u_{iT}u_{i1} | \tilde{X}_i] & \mathbb{E}\left[\alpha_i^2 | \tilde{X}_i \right] + \mathbb{E}[u_{iT}u_{i2} | \tilde{X}_i] & \cdots & \mathbb{E}\left[\alpha_i^2 | \tilde{X}_i \right] + \mathbb{E}[u_{iT}^2 | \tilde{X}_i]
    \end{bmatrix}\\
    &= \mathbb{E}\left[\mathbb{E}\left[\alpha_i^2 | \tilde{X}_i \right] \mathbf{1} \right] + \mathbb{E}[u_i u_i^{\prime} | \tilde{X}_i] \\
    &= \mathbb{E}[\alpha_i^2] \mathbf{1}_i \mathbf{1}_i^{\prime} + \begin{bmatrix}
        \mathbb{E}[u_{i1}^2 | \tilde{X}_i] & 0 & \cdots & 0 \\
        0 & \mathbb{E}[u_{i2}^2 | \tilde{X}_i] & \cdots & 0 \\
        \vdots & \vdots & \ddots & \vdots \\
        0 & 0 & \cdots & \mathbb{E}[u_{iT}^2 | \tilde{X}_i]
    \end{bmatrix} \\
    &= \sigma_{\alpha_i}^2 \mathbf{1}_i^{\prime} +_i^2 I \\
    &= \sigma_{\alpha}^2 \mathbf{1}_i \mathbf{1}_i^{\prime} + \sigma^2 I \\
    \text{beacuse } &\mathbb{V}[\tilde{\alpha}_i|\tilde{X}_i] = \sigma_{\alpha_i}^2 = \sigma_{\alpha}^2 \\
    &\mathbb{V}[u_{it} | \tilde{X}_i] = \sigma_i^2 = \sigma^2, \forall i.
\end{align*}
Under the assumption $\mathbb{E}[u_{it} x_{it}] = 0$, we now describe some statistical properties of $\hat{\tilde{\beta}}_{RE-GLS}.$
\begin{align*}
    \hat{\tilde{\beta}}_{RE-GLS} - \tilde{\beta} &= \Bigl( \sum_i \tilde{X}_i^{\prime} \Omega^{-1} \tilde{X}_i \Bigr)^{-1} \Bigl( \sum_i \tilde{X}_i^{\prime} \Omega^{-1} v_i \Bigr) \\
    & \rightarrow \mathbb{E}\Bigl[ \sum_i \tilde{X}_i^{\prime} \Omega^{-1} \tilde{X}_i \Bigr] \mathbb{E}\Bigl[ \sum_i \tilde{X}_i^{\prime} \Omega^{-1} v_i \Bigr] \\
    \text{where } \mathbb{E}\Bigl[ \sum_i \tilde{X}_i^{\prime} \Omega^{-1} v_i \Bigr] &= \sum_i \mathbb{E}\Bigl[ \tilde{X}_i^{\prime} \Omega^{-1} v_i \Bigr] \\
    &= \sum_i \tilde{X}_i^{\prime} \Omega^{-1} \mathbb{E}[v_i | \tilde{X}_i] \\
    &= \sum_i \tilde{X}_i^{\prime} \Omega^{-1} \mathbb{E}[u_i + \tilde{\alpha}_i | \tilde{X}_i] \\
    &=0
\end{align*}
Thus, $\hat{\tilde{\beta}}_{RE-GLS} $ is conditionally unbiased for $\tilde{\beta}$.
% The conditional variance of $\hat{\tilde{\beta}}_{RE-GLS}$ is:
% \begin{align*}
%     \mathbb{V}\Bigl[\hat{\tilde{\beta}}_{RE-GLS}\Bigr] = \Bigl( \sum_i \tilde{X}_i^{\prime} \Omega^{-1} \tilde{X}_i \Bigr)^{-1} \sigma_{\epsilon}^2
% \end{align*}
The asymptotic variance of $\hat{\tilde{\beta}}_{RE-GLS}$ is:
\begin{align*}
    &\sqrt{n} \left( \hat{\tilde{\beta}}_{RE-GLS} - \tilde{\beta} \right) \overset{d}{\rightarrow} \mathcal{N}\left(0, V \right) \\
    \text{where } V &= \mathbb{E}\left[\tilde{X}_i^{\prime} \Omega \tilde{X}_i \right]^{-1} \mathbb{E}\left[\tilde{X}_i^{\prime} \Omega v_i v_i^{\prime} \Omega \tilde{X}_i \right] \mathbb{E}\left[\tilde{X}_i^{\prime} \Omega \tilde{X}_i \right] \\
    &= \mathbb{E}\left[\tilde{X}_i^{\prime} \Omega \tilde{X}_i \underset{\equiv \Omega}{\underbrace{\mathbb{E}[v_i v_i^{\prime} | \tilde{X}_i]}}\right] \\
\end{align*}
Because we do not know $\Omega$, the RE-GLS estimator is infeasible. 

If indeed we have:
\begin{align*}
    \Omega &= \mathbb{E}[v_i v_i^{\prime}  | \tilde{X}_i] \\
    &= \mathbb{E}[(\alpha_i \mathbf{1}_i + u_i)(\alpha_i \mathbf{1}_i  + u_i)^{\prime}  | \tilde{X}_i] \\
    &= \mathbb{E}[\alpha_i^2] \mathbf{1}_i \mathbf{1}_i^{\prime} + \mathbb{E}[u_i u_i^{\prime} | \tilde{X}_i]
\end{align*}
which implies homoskedasticity.

A feasible version replaces $\Omega$ with an estimator $\hat{\Omega}_i$.
Assuming homoskedasticity of the original errors:
\begin{align*}
    \mathbb{E}[u_i u_i^{\prime}  | \tilde{X}_i, \tilde{\alpha}_i] &= \sigma_u^2 I_T \\
    \mathbb{E}[\tilde{\alpha}_i^2 | \tilde{x}_i] &= \sigma_{\alpha}^2
\end{align*}
We obtain: $\Omega = \sigma_{\alpha}^2 \mathbf{1}_i \mathbf{1}_i^{\prime} + \sigma_u^2 I_T$.

Hence, the motivation for using GLS is different than under a cross-sectional regression with heteroskedasticity.
We use GLS because of the autocorrelation in $v_{it}$ induced by the presence of time variant $\alpha_i$.



\section{Fixed Effects}
In the econometrics literature if the stochastic structure of $\alpha_i$ is treated as unknown
and possibly correlated with $x_{it}$, then $\alpha_i$ is called a \textbf{fixed effect}.

Correlation between $\alpha_i$ and $x_{it}$ will cause both pooled and random effect estimators ro be biased.

We transform equation to get rid of $\alpha_i$: $y_{it} = \alpha_i + x_{it}^{\prime} \beta + u_{it}.$
This is due to the classic problems of omitted variables bias and endogeneity. 

The presence of the unstructured individual effect $\alpha_i$ means that it is not possible to identify $\beta$ under a simple projection assumption such as $\mathbb{E}[u_{it} x_{it}] = 0$.
It turns out that a sufficient condition for identification is the following.

\begin{definition}[Strictly exogenous]\label{def:strictly_exogenous}
    \

    A regressor $x_{it}$ is said to be strictly exogenous if $\mathbb{E}[x_{it} u_{is}] = 0, \forall t, s = 1, \cdots, T$.
\end{definition}

\subsection{Within Transformation}
If we leave the relationship between $\alpha_i$ and $x_{it}$ fully unstructured,
then the only way to consistently estimate the coefficient $\beta$ is by an estimator
which is invariant to $\alpha_i$.

Define the mean of a variable for a given individual as
\begin{align*}
    \overline{y}_i &= \frac{1}{T} \sum_t y_{it} \\
    \overline{x}_i &= \frac{1}{T} \sum_t x_{it} \\
    \overline{u}_i &= \frac{1}{T} \sum_t u_{it}
\end{align*}
Then, 
\begin{align*}
    (y_{it} - \overline{y}_i) &= (x_{it} - \overline{x}_i)^{\prime} \beta + (u_{it} -\overline{u}_i) \\
    \ddot{y}_{it} &= \ddot{x}_{it}^{\prime} \beta + \ddot{u}_{it}
\end{align*}
Denote the time-averages method by $\hat{\beta}_{FE-W}$, the fixed effect estimator is consistent and asymptotically normal.
\begin{align*}
    \hat{\beta}_{FE-W} &= \left(\sum_i \sum_t \ddot{x}_{it} \ddot{x}_{it}^{\prime} \right)^{-1} \sum_i \sum_t \ddot{x}_{it} \ddot{y}_{it} \\
    &= \beta + \left(\sum_i \sum_t \ddot{x}_{it} \ddot{x}_{it}^{\prime} \right)^{-1} \sum_i \sum_t \ddot{x}_{it} \ddot{u}_{it} \\
    &\overset{p}{\rightarrow} \beta + \mathbb{E}\left[\sum_t \ddot{x}_{it} \ddot{x}_{it}^{\prime} \right]^{-1} \mathbb{E}\left[\sum_t \ddot{x}_{it} \ddot{u}_{it} \right] \\
    \text{where } \mathbb{E}\left[\sum_t \ddot{x}_{it} \ddot{u}_{it}\right] &= \sum_t \mathbb{E}\left[\ddot{x}_{it} \ddot{u}_{it} \right]\\
    \mathbb{E}\left[\ddot{x}_{it} \ddot{u}_{it} \right] &= \mathbb{E}\left[\left(x_{it} - \frac{1}{T}\sum_t x_{it} \right) \left(u_{it} - \frac{1}{T}\sum_t u_{it} \right)^{\prime} \right] \\
    &= 0 \quad \text{if } u_{it} \perp\!\!\!\perp x_{is}, \forall t, s = 1, \cdots, T.
\end{align*}

\subsection{First Difference Transformation}
\begin{align*}
    y_{it} - y_{i, t-1} &= (x_{it} - x_{i, t-1})^{\prime} \beta + (u_{it} - u_{i, t-1}) \\
    \Delta y_{it} &= \Delta x_{it}^{\prime} \beta + \Delta u_{it}, i=1 \cdots n, t=2 \cdots T.
\end{align*}
Denote the first difference method by $\hat{\beta}_{FE-FD}$, the fixed effect estimator is consistent and asymptotically normal.

\begin{align*}
    \hat{\beta}_{FE-FD} &= \left(\sum_i \sum_t \Delta x_{it} \Delta x_{it}^{\prime} \right)^{-1} \sum_i \sum_t \Delta x_{it} \Delta y_{it} \\
    &= \beta + \left(\frac{1}{n} \sum_i \sum_t \Delta x_{it} \Delta x_{it}^{\prime} \right)^{-1} \frac{1}{n} \sum_i \sum_t \Delta x_{it} \Delta u_{it} \\
    &\overset{p}{\rightarrow} \beta + \mathbb{E}\left[\sum_t \Delta x_{it} \Delta x_{it}^{\prime} \right]^{-1} \mathbb{E}\left[\sum_t \Delta x_{it} \Delta u_{it} \right] \\
    \text{where } \mathbb{E}\left[\sum_t \Delta x_{it} \Delta u_{it}\right] &= \sum_t \mathbb{E}\left[\Delta x_{it} \Delta u_{it} \right]\\
    \mathbb{E}\left[\Delta x_{it} \Delta u_{it} \right] &= \mathbb{E}\left[\left(x_{it} - x_{i, t-1} \right) \left(u_{it} - u_{i, t-1} \right)^{\prime} \right] \\
    &= 0 \quad \text{if } x_{it} \perp\!\!\!\perp (u_{it}, u_{i, t-1}), \forall t.
\end{align*}

\begin{note}
    \

    The FD method is not as strong as the within method, because it only requires that the variable is
    uncorrelated with the error term in the same period and the previous period.

    If there is a correlation between the error term in current period and two periods ago, there is a problem of feedback loop,
    which we will imply the correlated random effect model.
\end{note}
