Many time series arising in practice are best considered as components of
some vector-valued (multivariate) time series $y_t = (y_{1t}, y_{2t}, \cdots, y_{nt})^{\prime}$,
which is an $n \times 1$ vector process observed in sequence series $y_t$,
whose specification includes two parts:
\begin{enumerate}
    \item[(1)] The serial dependence of each component series $\{y_{it}\}$;
    \item[(2)] THe interdependence between different component series $\{y_{it}\}$ and $\{y_{jt}\}$.
\end{enumerate}
The most common multivariate time series models used by economists are vector autoregressions (VARs).

\section{Multivaraite Processes}
Stationarity and ergodicity of a multivariate process are defined analogously to the univariate case.
If all the finite dimensional joint distributions of the random variables $\{y_{it}\}$ were multivariate normal,
then the distributional properties of $\{y_{it}\}$ would be completely determined by the means,
\begin{gather*}
    \mu_{it} = \mathbb{E}[y_{it}]
\end{gather*}
and covariances,
\begin{gather*}
    \gamma_{ij}(t, t-h) = \Cov[y_{i,t}, y_{j,t-h}] = \mathbb{E}\left[ (y_{i,t} - \mu_{i,t})(y_{j,t-h} - \mu_{j,t-h}) \right].
\end{gather*}
It is more convenient when dealing with $n$ interrelated series to use vector notation. 
The second-order properties of the multivariate process are specified by the mean vectors:
\begin{gather*}
    \mu_t = \mathbb{E}[y_t]
\end{gather*}
and covariance matrices:
\begin{gather*}
    \Gamma_t(h) = \Cov[y_t, y_{t-h}] = \mathbb{E}\left[ (y_t - \mu_t)(y_{t-h} - \mu_{t-h})^{\prime} \right].
\end{gather*}
\begin{definition}[Stationary Multivariate Time Series]\label{def:stationary-multivariate}
    \

    A multivariate time series $\{y_t\}$ is said to be weakly stationary if:
    \begin{enumerate}
        \item[(1)] $\mu_t = \mu$ for all $t$;
        \item[(2)] $\Gamma_t(h) = \Gamma(h) = \mathbb{E}\left[(y_t - \mu)(y_{t-h} - \mu)^{\prime}\right]$ for all $t$ and $h$.
    \end{enumerate}
\end{definition}
We shall refer to $\mu$ as the mean of the series and $\Gamma(h)$ as the covariance function of the series at lag $h$. 
\begin{eg}
    \

    Take the bivariate case($n=2$) as an example, we have:
    \begin{gather*}
        \Gamma(h) = \begin{bmatrix}
            \Cov[y_{1,t}, y_{1,t-h}] & \Cov[y_{1,t}, y_{2,t-h}] \\
            \Cov[y_{1,t-h}, y_{2,t}] & \Cov[y_{2,t}, y_{2,t-h}] \\
        \end{bmatrix}
        = \begin{bmatrix}
            \gamma_{11}(h) & \gamma_{12}(h) \\
            \gamma_{21}(h) & \gamma_{22}(h) \\
        \end{bmatrix}
    \end{gather*}
    The diagonal elements are simply the univariate autocovariances of $y_{1t}$ and $y_{2t}$,
    while the off-diagonal elements are the cross-covariances between $y_{1t}$ and $y_{2t}$,
    with the former (latter) leading the latter (former) by a displacement of $h$ lags.
    We should notice that $\gamma_{12}(h) \neq \gamma_{21}(h)$,
    as the correlation of $y_{1t}$ with the past movement in $y_{2t}$ is not necessarily the same as the correlation of $y_{2t}$ with the past movement in $y_{1t}$.

    As a result, in contrast to the univariate case, in multivariate case, $\Gamma(h) \neq \Gamma(-h).$
    Instead, we have $\gamma_{12} (h) = \gamma_{21} (-h)$, because:
    \begin{gather*}
        \Gamma_t(-h)^{\prime} = \mathbb{E}\left[(y_t - \mu_t)(y_{t+h} - \mu_{t+h})^{\prime}\right]^{\prime} = \mathbb{E}\left[(y_{t+h} - \mu_{t+h})(y_t - \mu_t)^{\prime} \right] = \Gamma_{t+h}(h)
    \end{gather*}
    and under weakly stationarity, time doesn't matter: $\Gamma_{t+h}(h) = \Gamma_t(h) = \Gamma(h)$,
    and therefore, $\Gamma(h) = \Gamma(-h)^{\prime}$
\end{eg}
A multivariate White Noise (WN) process and the multivariate General Linear Process
(GLP) are also defined analogously to the univariate case.
\begin{definition}[Multivariate White Noise]\label{def:multivariate-wn}
    \

    The $n$-variate series $u_t$ is said to be a WN process with mean zero and covariance $\Sigma$,
    written $u_t \sim WN(0, \Sigma)$,
    if and only if $u_t$ is weakly stationary with mean vector $\mathbf{0}$ and covariance matrix function:
    \begin{gather*}
        \Gamma(h) = \Cov[u_t, u_{t-h}] = \mathbb{E}[u_t u_{t-h}^{\prime}] = \begin{cases}
            \Sigma, & h=0 \\
            0, & h \neq 0
        \end{cases}
    \end{gather*}
    Note that $\Sigma$ is not necessarily diagonal, i.e. the individual components of the vector valued process $u_t$ may be contemporaneously correlated.
    However, the defining feature of a WN process is that each of the components is uncorrelated with its own past (and future) movements as well as those of any of the other components.
\end{definition}
The multivariate GLP is defined as:
\begin{gather*}
    y_t = B(L) u_t = \sum_{l=0}^{\infty} B_l u_{t-l}, \quad u_t \sim WN(0, \Sigma)
\end{gather*}
where $B_0 = I$ and $\sum_{l=0}^{\infty} \lVert B_l \rVert^2 < \infty.$

\section{Vector Autoregressions (VARs)}\label{sec:VAR}
A VAR is a common way to approximate the multivariate GLP.
As in the previous lecture, we have built a VAR(1) model for solving the AR(2) process.
\begin{gather*}
    y_t = \Phi_0 + \Phi_1 y_{t-1} + u_t, \quad u_t \sim WN(0, \Sigma)
\end{gather*}
\begin{gather*}
    \begin{bmatrix}
        y_{1t} \\
        y_{2t}
    \end{bmatrix} = \begin{bmatrix}
         \Phi_{0,1} \\
         \Phi_{0,2} \\
    \end{bmatrix} + \begin{bmatrix}
        \Phi_{1,11} & \Phi_{1,12}  \\
        \Phi_{1,21} & \Phi_{1,22}  \\
    \end{bmatrix}
    \begin{bmatrix}
        y_{1,t-1} \\
        y_{2,t-1}
    \end{bmatrix} + \begin{bmatrix}
         u_{1t} \\
         u_{2t} \\
    \end{bmatrix}, \quad u_t \sim WN(0, \Sigma)
\end{gather*}
where $u_t \sim  WN\left( \begin{bmatrix}
     0 \\
     0 \\
\end{bmatrix}, \begin{bmatrix}
    \sigma_{11} & \sigma_{12} \\
    \sigma_{21} & \sigma_{22} \\
\end{bmatrix} \right)$.
There are two sources of interaction between $y_{1t}$ and $y_{2t}$:
\begin{enumerate}
    \item[(1)] The value of each series today depends on its own, but also on the other variable's value at time $t-1$;
    \item[(2)] The innovations of the two series are potentially correlated (contemporaneously).
\end{enumerate}
If the equations include $p$ lags of each variable we obtain the $p$-th order vector autoregressive (VAR) model:
\begin{gather}\label{eq:varp}
    y_t = \Phi_0 + \Phi_1 y_{t-1} + \cdots + \Phi_p y_{t-p} + u_t
\end{gather}
Our notation allows $y_t = [y_{1t}, y_{2t}, \cdots, y_{nt}]^{\prime}$ to be a vector of dimension $n$ in which case the matrices $\Phi_i$ are $n \times n$ and error $u_t$ is $n \times 1$.
We will denote the elements of $\Phi_i$ using the notation:
\begin{gather*}
    \Phi_i = \begin{bmatrix}
        \Phi_{i,11} & \Phi_{i,12} & \cdots & \Phi_{i,1n} \\
        \Phi_{i,21} & \Phi_{i,22} & \cdots & \Phi_{i,2n} \\
        \vdots & \vdots & \ddots & \vdots \\
        \Phi_{i,n1} & \Phi_{i,n2} & \cdots & \Phi_{i,nn} \\
    \end{bmatrix}.
\end{gather*}
The error $u_t = [u_{1t}, u_{2t}, \cdots, u_{nt}]^{\prime}$ is the component of $y_t$ which is unforecastable at time $t-1$.
However, the components of $y_t$ are contemporaneously correlated, therefore the contemporabeous covariance matrix $\Sigma = \mathbb{E}[u u^{\prime}]$ is non-diagonal.
\begin{definition}[VAR$(p)$]\label{def:varp}
    \

    A VAR(p) model is a system of $n$ equations of the form:
    \begin{gather}\label{eq:varp-def}
        y_t = \Phi_0 + \sum_{i=1}^{p} \Phi_i y_{t-i} + u_t, \quad u_t \sim WN(0, \Sigma)
    \end{gather}
    where $y_t$ is an $n$-dimensional vector, $\Phi_0$ is an $n$-dimensional vector of constants, $\Phi_i$ are $n \times n$ matrices of coefficients, and $u_t$ is an $n$-dimensional vector of white noise errors.
\end{definition}

\subsection{Stationarity of VAR\texorpdfstring{$(p)$}{(p)}}\label{sec:stationarity-var}
We are interested in conditions under which $y_t$ is a stationary process.
Let $\lambda_i(\Phi)$ denote the $i$-th eigenvalue of the matrix $\Phi$.
\begin{theorem}[SS and Ergodic of VAR(1)]\label{thm:stationarity-var1}
    \

    If $u_t$ is strictly stationary and ergodic, with $\mathbb{E}[u_t] < \infty$, and $\vert \lambda_i(\Phi_1) \vert < 1$,
    then the VAR$(1)$ process $y_t$ is also strictly stationary and ergodic.
\end{theorem}
\begin{eg}
    \

    Consider the VAR(2) model:
    \begin{gather*}
        y_t = \Phi_0 + \Phi_1 y_{t-1} + \Phi_2 y_{t-2} + u_t
    \end{gather*}
    or we write in companion form:
    \begin{gather*}
        \begin{bmatrix}
            y_t \\
            y_{t-1}
        \end{bmatrix} = \underset{F_0}{\underbrace{\begin{bmatrix}
            \Phi_0 \\
            0
        \end{bmatrix}}} + 
        \underset{F_1(np \times np)}{\underbrace{\begin{bmatrix}
            \Phi_1 & \Phi_2 \\
            I & 0
        \end{bmatrix}}}
        \begin{bmatrix}
            y_{t-1} \\
            y_{t-2}
        \end{bmatrix} + \begin{bmatrix}
             u_t \\
             0 \\
        \end{bmatrix}
    \end{gather*}
    we can simply write:
    \begin{gather*}
        Y_t = F_0 + F_1 Y_{t-1} + v_t
    \end{gather*}
    We take mean on both sides,
    \begin{gather*}
        \mathbb{E}[y_t] = \Phi_0 + \Phi_1 \mathbb{E}[y_{t-1}] + \Phi_2 \mathbb{E}[y_{t-2}] + 0
    \end{gather*}
    as $\mathbb{E}[y_t] = \mathbb{E}[y_{t-1}] = \mathbb{E}[y_{t-2}] = \mu$,
    we know that $\mu = \left(1 - \Phi_1 - \Phi_2\right)^{-1} \Phi_0.$
\end{eg}
For the $VAR(p)$ model \ref{eq:varp-def}, we can write using the lag operator notation:
\begin{gather*}
    \Phi (L) y_t = \Phi_0 + u_t
\end{gather*}
where
\begin{gather*}
    \Phi(z) = I_n - \Phi_1 z - \Phi_2 z^2 - \cdots - \Phi_p z^p.
\end{gather*}
The condition for stationarity of the system can be expressed as a restriction on the roots of the
determinantal equation of the autoregressive polynomial.
Recall that a root $r$ of $\det \left(\Phi(z) \right)$ is a solution to $\det \left(\Phi(r) \right) = 0$.
\begin{theorem}[SS and Ergodicity of VAR(p)]\label{thm:stationarity-varp}
    \

    If all roots $r$ of the polynomial $\det \left(\Phi(z) \right)$ satisfy $\vert r \vert > 1$,
    then the VAR(p) process $y_t$ is strictly stationary and ergodic.
\end{theorem}
Or, equivalently, we write the VAR$(p)$ model in companion form:
\begin{gather}\label{eq:varp-companion}
    \begin{bmatrix}
        y_t \\
        y_{t-1} \\
        \vdots \\
        y_{t-p+1}
    \end{bmatrix} = \begin{bmatrix}
         \Phi_0 \\
         0 \\
         \vdots \\
         0 \\
    \end{bmatrix} + 
    \underset{F_{np \times np}}{\underbrace{\begin{bmatrix}
        \Phi_1 & \Phi_2 & \cdots & \Phi_p \\
        I & 0 & \cdots & 0 \\
        0 & I & \cdots & 0 \\
        \vdots & \vdots & \ddots & \vdots \\
        0 & \cdots & I & 0 \\
    \end{bmatrix}}}
    \begin{bmatrix}
        y_{t-1} \\
        y_{t-2} \\
        \vdots \\
        y_{t-p}
    \end{bmatrix} + 
    \begin{bmatrix}
         u_t \\
         0 \\
         \vdots \\
         0 \\
    \end{bmatrix}
\end{gather}
We can simply write $Y_t = F_0 + F Y_{t-1} + v_t$,
hence Theorem \ref{thm:stationarity-var1} can be reorganize to:
\begin{theorem}[SS and Ergodicity of VAR(p)]\label{thm:stationarity-varp-2}
    \

    If all eigenvalues $\lambda_i(F)$ of the matrix $F$, given by:
    \begin{gather*}
        \vert I_n \lambda^p - \Phi_1 \lambda^{p-1} - \Phi_2 \lambda^{p-2} - \cdots - \Phi_p \vert = 0
    \end{gather*}
    satisfy $\vert \lambda_i(F) \vert < 1$,
    then the VAR$(p)$ process $y_t$ is strictly stationary and ergodic.
\end{theorem}

\subsection{Moments \& GLP-Representation}
Once we know that a VAR$(p)$ is Weakly Stationary, it is quite trivial to calculate its mean.

Consider first for a VAR$(1)$. Repeatedly inserting for lags $y_{t-1}$, we get:
\begin{align*}
    y_t &= \Phi_0 + \Phi_1 y_{t-1} + u_t \\
    &= \Phi_0 + \Phi_1 (\Phi_0 + \Phi_1 y_{t-2} + u_{t-1}) + u_t \\
    &= \Phi_0 + \Phi_1 \Phi_0 + \Phi_1^2 y_{t-2} + \Phi_1 u_{t-1} + u_t \\
    &= \cdots \\
    &= \sum_{l=0}^{\infty} \Phi_1^l \Phi_0 + \sum_{l=0}^{\infty} \Phi_1^l u_{t-l} + \lim_{k \to \infty} \Phi_1^k y_{t-k} \\
    &= \frac{\Phi_0}{1 - \Phi_1} + \sum_{l=0}^{\infty} \Phi_1^l u_{t-l}
\end{align*}
if $\Phi_1$ has all eigenvalues in the unit circle, then $\lim_{k \to \infty} \Phi_1^k y_{t-k} = 0$.
Based on this expresison, we can get the ACF:
\begin{align*}
    \Gamma(h) = \Cov[y_t, y_{t-h}] &= \mathbb{E}\left[ \left(\sum_{t=0}^{\infty}\Phi_1^l u_{t-l} \right) \left( \sum_{k=0}^{\infty}\Phi_1^k u_{t-h-k} \right)^{\prime} \right] \\
    &= \sum_{k=0}^{\infty} \mathbb{E}\left[ \left(\Phi_1^{k+h} u_{t-h-k} \right) \left(\Phi_1^k u_{t-h-k} \right)^{\prime} \right] \\
    &= \sum_{k=0}^{\infty} \Phi_1^{h+k}  \mathbb{E}\left[ u_{t-h-k} u_{t-h-k}^{\prime} \right] \Phi_1^{k^{\prime}} \\
    &= \sum_{k=0}^{\infty} \Phi_1^{h+k} \Sigma \Phi_1^{k^{\prime}}
\end{align*}
The previous discussion can be extended easily to VAR$(p)$ processes with $p>1$ because any VAR$(p)$ process can be written in VAR$(1)$ form. 
More precisely, if $y_t$ is a VAR$(p)$ process as in definition \ref{def:varp},
a corresponding $np$-dimensional VAR$(1)$ is shown as equation \ref{eq:varp-companion}: $Y_t = F_0 + F Y_{t-1} + v_t$.
Recall that
\begin{gather*}
    Y_t = \begin{bmatrix}
            y_t \\
            y_{t-1} \\
            \vdots \\
            y_{t-p+1} \\
         \end{bmatrix}, \quad 
    F = \begin{bmatrix}
            \Phi_1 & \Phi_2 & \cdots & \Phi_p \\
            I_n & 0 & \cdots & 0 \\
            0 & I_n & \cdots & 0 \\
            \vdots & \vdots & \ddots & \vdots \\
            0 & \cdots & I_n & 0 \\
        \end{bmatrix}, \quad
    v_t = \begin{bmatrix}
            u_t \\
            0 \\
            \vdots \\
            0 \\
        \end{bmatrix}
\end{gather*}
Following previous discussion, we know that $Y_t$ is strictly stationary and ergodic,
if $\vert \lambda_i(F) \vert < 1$ (Theorem \ref{thm:stationarity-varp-2}).
It's mean vector is:
\begin{gather*}
    \mathbf{\mu} = \mathbb{E}[Y_t] = \left(I_{np} - F\right)^{-1} F_0
\end{gather*}
and the autocovariances are
\begin{gather*}
    \Gamma_Y(h) = \sum_{k=0}^{\infty} F^{h+k} \Sigma^v F^{k^{\prime}}
\end{gather*}
where $\Sigma^v = \mathbb{E}[v_t v_t^{\prime}] = \mathbb{V}[u_t] = \begin{bmatrix}
    \Sigma & 0 \\
    0 & 0 \\
\end{bmatrix}$.
Using the $(n \times np)$ matrix
\begin{gather*}
    M = \begin{bmatrix}
        I_n & 0 & 0 & 0 \\
    \end{bmatrix},
\end{gather*}
the process $y_t$ is obtained as $y_t = M Y_t$.
Because $Y_t$ is strictly stationary and ergodic, $y_t$ is also strictly stationary and ergodic,
meaning that $\mathbb{E}[y_t] = M \mathbf{\mu}$ is constant for all $t$ and the autocovariances $\Gamma_y(h) = M \Gamma_Y(h) M^{\prime}$ are also time invariant.
And we can have the ACF for the original $y_t$:
\begin{align*}
    \Gamma_y(h) = \Cov[y_t, y_{t-h}] &= \Cov[MY_t, MY_{t-h}] \\
    &= M \Cov[Y_t, Y_{t-h}] M^{\prime} \\
    &= \sum_{k=0}^{\infty} M F^{h+k} \Sigma^v F^{k^{\prime}} M^{\prime} \\
    &= \sum_{k=0}^{\infty} \left(F^{k+h} \right)_{11} \Sigma^v \left( F^k \right)_{11}^{\prime}
\end{align*}
where $(F^k)_{11}$ denotes the upper-left $(n \times n)$ block of the $(np \times np)$ matrix $F^k$.
\subsection{Computation of Autocovariances and Autocorrelations of VAR Processes*}
In order to illustrate the computation of the autocovariances when the process
coefficients are given, suppose that $y_t$ is strictly stationary and ergodic VAR$(1)$ process:
\begin{gather*}
    y_t = \Phi_0 + \Phi_1 y_{t-1} + u_t, \quad u_t \sim WN(0, \Sigma)
\end{gather*}
We can rewrite the equation in mean-adjusted form:
\begin{gather*}
    y_t - \mu = \Phi_1 \left( y_{t-1} - \mu \right) + u_t
\end{gather*}
where $\mu = \mathbb{E}[y_t] = \frac{\Phi_0}{1-\Phi_1}.$
Postmultiplying by $(y_{t-h} - \mu)^{\prime}$ and taking the expection, we have:
\begin{gather*}
    \Cov[y_t ,y_{t-h}] = \mathbb{E}\left[(y_t - \mu)(y_{t-h} - \mu)^{\prime}\right] = \Phi_1 \mathbb{E}\left[(y_{t-1} - \mu)(y_{t-h} - \mu)^{\prime}\right] + \mathbb{E}[u_t (y_{t-h} - \mu)^{\prime}].
\end{gather*}
Thus, for $h=0$, we have:
\begin{gather}\label{eq:varp-gamma0}
    \Gamma_y(0) = \Cov[y_t, y_t] = \Phi_1 \Cov[y_{t-1}, y_t] + \mathbb{E}[u_t u_t^{\prime}] = \Phi_1 \Gamma_y(-1) + \Sigma = \Phi_1 \Gamma_y(1)^{\prime} + \Sigma
\end{gather}
and for $h > 0$,
\begin{gather}\label{eq:varp-gammarecursive}
    \Gamma_y(h) = \Phi_1 \Gamma_y(h-1) + \mathbb{E}\left[u_t \left(\sum_{j=0}^{\infty} \Phi_1^j u_{t-1-j}\right)\right] = \Phi_1 \Gamma_y(h-1)
\end{gather}
as $u_t$ is strict white noise, indicating that $\mathbb{E}[u_t u_s] = 0$ for $t \neq s$.

These equations are usually referred to as the \textit{Yule-Walker equations} (\ref{appendix:Yule-Walker}).
If $\Phi_1$ and the covariance matrix $\Gamma_y(0) = \Sigma_y$ of $y_t$ are unknown, the $\Gamma_y(h)$ can be computed recursively using the equation above.

If $\Phi_1$ and $\Sigma$ are given, $\Gamma_y(0)$ can be determined as follows.
For $h=1$, we get from equation \ref{eq:varp-gammarecursive} that $\Gamma_y(1) = \Phi_1 \Gamma_y(0)$,
and substitute back into equation \ref{eq:varp-gamma0} gives:
\begin{gather*}
    \Gamma_y(0) = \Phi_1 \Gamma_y(0) \Phi_1^{\prime} + \Sigma \footnotemark
\end{gather*}
\footnotetext{
Or using the stationarity of $y_t$, we know that $\mathbb{V}[y_t] = \mathbb{V}[y_{t-1}]$,
so \[\Gamma_y(0) = \mathbb{V}[y_t] = \mathbb{V}[\Phi_1 y_{t-1} + u_t] = \Phi_1 \Gamma_y(0) \Phi_1^{\prime} + \Sigma\].
}
This is a matrix equation in $\Gamma_y(0)$, which can be solved using the Kronecker product and vector operator (See appendix \ref{appendix:krproduct-vec}).
\begin{align*}
    \vecop \Gamma_y(0) &= \vecop (\Phi_1 \Gamma_y(0) \Phi_1^{\prime}) + \vecop \Sigma \\
                 &= \left(I_n \otimes \Phi_1\right) \vecop \Gamma_y(0) + \vecop \Sigma
\end{align*}
Hence,
\begin{gather*}
    \vecop \Gamma_y(0) = \left(I_{n^2} - \Phi_1 \otimes \Phi_1\right)^{-1} \vecop \Sigma
\end{gather*}
Note that the invertibility of $I_{n^2} - \Phi_1 \otimes \Phi_1$ follows from the strict stationarity of $y_t$
because the eigenvalues of $\Phi_1 \otimes \Phi_1$ are the products of the eigenvalues of $\Phi_1.$
Hence, the eigenvalues of $\Phi_1 \otimes \Phi_1$ are all less than 1 in absolute value.


\subsection{Reduced-Form vs. Structural Representation}
The above expressions for the VAR model are called reduced-form representation and the $u_t$ that appear in it are called reduced-form errors.
They are the forecasting errors obtained when predicting $y_t$ one step ahead:
\begin{align*}
    y_t &= \Phi_0 + \Phi_1 y_{t-1} + u_t \\
    \Rightarrow \mathbb{E}_{t-1} [y_t] &= \Phi_0 + \Phi_1 y_{t-1} + \mathbb{E}_{t-1} [u_t]\\
    \Rightarrow u_t &= y_t - \mathbb{E}_{t-1} [y_t]
\end{align*}
This is why $u_t$ is also referred to as the innovation to the process $y_t$ at time $t$;
it contains everything that affects $y_t$ and is not known to the researcher at time $t-1$.

Generally, in impulse response analysis the emphasis has shifted from specifying the relations
between the observable variables directly to interpreting the
unexpected part of their changes or the shocks. 
To make causal statements in VAR, we need to decompose the reduced-form errors into underlying, independent driving forces of $y_t$, referred to as shocks (structural innovations).
One way to do so is to think of the forecast errors as linear functions of the structural innovations.
In that case, we have the relations
\begin{gather*}
    u_t = \Phi_{\varepsilon} \varepsilon_t, \quad \varepsilon_t \sim WN(0, I_n)
\end{gather*}
where $\varepsilon_t$ is the vector of shocks. Hence $\Sigma = \Phi_{\varepsilon} \Sigma_{\varepsilon} \Phi_{\varepsilon}^{\prime}$,
as we normalize the variances of the structural innovations to one,
we have $\Sigma = \Phi_{\varepsilon} \Phi_{\varepsilon}^{\prime}$.

\begin{note}
    \

    Due to the symmetry of the covariance matrix, these relations specify only $\frac{n(n+1)}{2}$ different equations
and we need again $\frac{n(n-1)}{2}$ further relations to identify all $n^2$ elements of $\Phi_{\varepsilon}$.
We shall discuss this in a following section.
\end{note}

Let's review the two forms of VAR representation:
\begin{gather*}
    \text{Reduced-form: } y_t = \Phi_0 + \Phi_1 y_{t-1} + u_t, \quad u_t \sim WN(0, \Sigma) \\
    \text{Structural: } y_t = \Phi_0 + \Phi_1 y_{t-1} + \Phi_{\varepsilon} \varepsilon_t, \quad \varepsilon_t \sim WN(0, I_n)
\end{gather*}
For a general VAR$(p)$ model
\begin{gather*}
    y_t = \Phi_0 + \sum_{i=1}^{\infty} \Phi_i y_{t-i} + u_t, \quad u_t \sim WN(0, \Sigma)
\end{gather*}
Sometimes, we write the structural representation as:
\begin{gather*}
    A y_t = B_0 + B_1 y_{t-1} + \cdots + B_p y_{t-p} + \varepsilon_t, \quad \varepsilon_t \sim WN(0, I_n)
\end{gather*}
where $A = \Phi_{\varepsilon}^{-1} , B_0 = \Phi_{\varepsilon}^{-1} \Phi_0, B_i = \Phi_{\varepsilon}^{-1} \Phi_i$ and $\varepsilon_t = \Phi_{\varepsilon}^{-1} u_t \sim \left(0, \Sigma_{\varepsilon} = A \Sigma A^{\prime} \right).$
For the structural representation of VAR$(1)$, we can write:
\begin{gather*}
    y_t = \left(I - \Phi_1\right)^{-1} \Phi_0 + \sum_{l=0}^{\infty} \Phi_1^l \Phi_{\varepsilon} \varepsilon_{t-l}  \\
    \Gamma_y(0) = \sum_{l=0}^{\infty} \Phi_1^l \Sigma_v \Phi_1^{l^\prime}, \quad \Sigma_v = \mathbb{E}\left[\Phi_{\varepsilon}\varepsilon_t \varepsilon_t^{\prime} \Phi_{\varepsilon}^{\prime} \right] = \Phi_{\varepsilon} \Phi_{\varepsilon}^{\prime}
\end{gather*}

Suppose that we know all parameters in the VAR, including the matrix $\Phi_{\varepsilon}$ that maps reduced-form errors into structural shocks.
This allows us to compute impulse-response functions (IRFs), variance decompositions (VDs) and historical decompositions (HDs).

An IRF illustrates the dynamic effects of a shock on a series in $y_t$.
Having identified the structural shocks $\varepsilon_t$, our interest usually is not in the shocks themselves,
however, but in the responses of each element of $y_t$ to a one-time impulse of $\varepsilon_t$.
\begin{definition}[Impulse Response Function (IRF)]\label{def:IRF}
    \

    The effect of shock $j$, $\varepsilon_{jt}$ on series $i$, $y_{it}$, $h$ periods into the future is given by:
    \begin{gather}\label{eq:IRF}
        \Theta_{ij} = \pd{y_{i, t+h}}{\varepsilon_{j, t}} = \left[ \pd{y_{t+h}}{\varepsilon_t} \right]_{ij} = \left[(F^h)_{11} \Phi_{\varepsilon}\right]_{ij} = \left[\Phi_1^h \Phi_{\varepsilon}\right]_{ij}
    \end{gather}
    The sequence $\Theta_{ij}$ is referred to as the IRF of variable $i$ to shock $j$,
    which tells us the dynamic effect of $\varepsilon_{jt}$ on the series $y_{it}.$
    As by definition, $\varepsilon_{jt}$ is i.i.d., this effect has a causal interpretation.
\end{definition}

A variance decomposition of $y_t$ determines the contribution of each shock to the variance of $y_t$.
\begin{gather*}
    \Gamma_0 = \mathbb{V}[y_t] = \sum_{l=0}^{\infty} \Phi_1^l \Phi_\varepsilon I \Phi_\varepsilon^{\prime} \Phi_1^{l^\prime}.
\end{gather*}
Let $I^{(j)}$ be the identity matrix with all but the $j$-th diagonal element equal to zero. Then
\begin{gather*}
    \Gamma_0^{(j)} = \sum_{l=0}^{\infty} \Phi_1^l \Phi_\varepsilon I^{(j)} \Phi_\varepsilon^{\prime} \Phi_1^{l^\prime}
\end{gather*}
We have that $\Gamma_0 = \sum_{j=1}^{n} \Gamma_0^{(j)}$, as a result,
the fraction of the variance of a particular series $y_{it}$, $\mathbb{V}[y_{it}] = \left[\Gamma_0\right]_{ij}$ is explained by $\varepsilon_{jt}$ given by:
\begin{gather*}
    \left[\Gamma_0^{(j)}\right]_{ii} / \left[\Gamma_0\right]_{ii}.
\end{gather*}

\section{Estimation of Reduced-Form VARs}
\label{sec:estimation-var}

Consider the general VAR(p) model:
\begin{gather}\label{eq:varest}
    y_t = \Phi_0 + \Phi_1 y_{t-1} + \cdots + \Phi_p y_{t-p} + u_t, \quad u_t \sim WN(0, \Sigma)
\end{gather}
and we write it into a simple regression form:
\begin{gather*}
    y_t^{\prime} = x_t^{\prime} \Phi + u_t^{\prime}, \quad x_t = \begin{bmatrix}
         y_{t-1} \\
         y_{t-2} \\
         \vdots \\
         1 \\
    \end{bmatrix}, \Phi = \begin{bmatrix}
         \Phi_1^{\prime} \\
         \Phi_2^{\prime} \\
         \vdots \\
         \Phi_0^{\prime} \\
    \end{bmatrix}
\end{gather*}
in matrix form:
\begin{gather*}
    Y = X \Phi + U, Y = \begin{bmatrix}
         y_1^{\prime} \\
         y_2^{\prime} \\
         \vdots \\
         y_T^{\prime} \\
    \end{bmatrix}.
\end{gather*}
To simplify, we only analyze for the OLS estimator:
\begin{align*}
    \min_{\Phi} \sum_{t} \sum_{i} u_{it}^2 &= \min_{\Phi} \sum_{t} u_t^{\prime} u_t \\
    % &= \min_{\Phi} (Y - X\Phi)^{\prime} (Y - X\Phi) \\
    &= \min_{\Phi} \tr\left[ (Y - X\Phi)^{\prime} (Y - X\Phi) \right] \\
    \Rightarrow \hat{\Phi} &= (X^{\prime} X)^{-1} X^{\prime} Y \\
    \Sigma = \mathbb{V}[u_t] &= \begin{bmatrix}
        \mathbb{V}[u_{1t}] & \Cov[u_{1t}, u_{2t}] \\
        \cdot &  \mathbb{V}[u_{2t}] \\
    \end{bmatrix} \leftarrow \begin{bmatrix}
        \frac{1}{T} \sum_{t} \hat{u}_{it}^2 & \frac{1}{T} \sum_{t} \hat{u}_{1t} \hat{u}_{2t} \\
        \cdot & \frac{1}{T} \hat{u}_{2t} \\
    \end{bmatrix} = \hat{\Sigma} \\
    \hat{\Sigma} &= \frac{1}{T} \sum_{t=1}^{T} \hat{u}_t \hat{u}_t^{\prime} = \frac{1}{T} \hat{U}^{\prime} \hat{U} = \frac{1}{T} \left(Y - X \hat{\Phi} \right)^{\prime} \left(Y - X \hat{\Phi} \right)
\end{align*}
This holds whether or not $x_t$ is actually a VAR$(p)$ process.
By the properties of projection errors, $\mathbb{E}[x_t u_t^{\prime}] = 0$.
The projection coeicicent matrix $\hat{\Phi}$ is identified if $X^{\prime} X = \mathbb{E}[x_t x_t^{\prime}]$ is invertible.
\begin{theorem}[Identification of SVAR]\label{thm:identification-svar}
    \

    If $y_t$ is strictly stationary, and $\Sigma = \mathbb{E}[u_t u_t^{\prime}] < \infty$,
    then $Q = \mathbb{E}[x_t x_t^{\prime}] > 0$ and the coefficient vector
    $\hat{\Phi}$ is identified.
\end{theorem}


\section{Estimation of Structural VARs}
\label{sec:estimation-structural-var}

We  consider the structural VAR(1) model in the form:
\begin{gather*}
    y_t = \Phi_0 + \Phi_1 y_{t-1} + \Phi_{\varepsilon} \varepsilon_t, \quad \varepsilon_t \sim WN(0, I)
\end{gather*}
As we have previously discussed in a note, there are $\frac{n(n+1)}{2}$ unique elements in $\Sigma$, but $n^2$ elements in $\Phi_{\varepsilon}$.
So, \textit{point identify} $\Phi_{\varepsilon}$, we need $\frac{n(n-1)}{2}$ additional restrictions.
Even with fewer restrictions, we can decrease the identified set, and we can possibly point-identify the effects of some of the $n$ shocks in $\varepsilon_t$
(by point-identifying the corresponding columns of $\Phi_{\varepsilon}$) or the responses of some of the $n$ series in $y_t$ by point-identifying the corresponding rows of $\Phi_{\varepsilon}$.

We can further analyze the structural VAR model by decomposing $\Phi_{\varepsilon}$ into a part that we can point-identify and a part that we cannot.
Let $\Omega $ be an orthogonal matrix such that $\Omega \Omega^{\prime} = I$.
Then we can decompose $\Phi_{\varepsilon}$ as:
\begin{gather*}
    \Phi_{\varepsilon} = \Sigma_{\tr} \Omega
\end{gather*}
where $\Sigma_{\tr}$ is the Cholesky decomposition of $\Sigma$, satisfying $\Sigma_{\tr} \Sigma_{\tr}^{\prime} = \Sigma$,
and $\Omega$ is any orthogonal matrix with $\Omega \Omega^{\prime} = I.$

Because we can point-identify $\Sigma$, and because Cholesky decomposition is unique up to the ordering of the diagonal elements,
we can point-identify the diagonal elements of $\Sigma_{\tr}$ from the data.
And that any $\Omega$ is consistent with the data (gives the same likelihood), as for any $\Omega$,
$\Phi_{\varepsilon} \Phi_{\varepsilon}^{\prime} = \Sigma_{\tr} \Omega \Omega^{\prime} \Sigma_{\tr} = \Sigma$.

From a frequentist point of view, this means that we cannot find a
(unique) point estimator for $\Omega$ from the data.

\begin{eg}
    \

    For $n=2$, we can obtain a  nice representation of the space of orthogonal matrices:
    \begin{gather*}
        \Omega(\theta, \xi) = \begin{bmatrix}
            \cos \theta & -\xi \sin \theta \\
            \sin \theta & \cos \theta \\
        \end{bmatrix}, \quad
        \text{where } -\pi < \theta < \pi, \quad \xi = \{-1,1\}
    \end{gather*}
    We notice that $\Omega(\frac{\pi}{2}) = \Omega(-\frac{\pi}{2})$,
    which means that only the signs of the impulse responses change but not the shape.
\end{eg}

\section{Point Restrictions}
\label{sec:point-restrictions}

A point-restriction fixes some of the elements of $\Omega $.
For example, one might impose a point restriction on short-term dynamics,
such as real activity reacting to monetary policy shocks only with a lag.
If $y_{it}$ is GDP and $\varepsilon_{jt}$ is the MP shock, this implies that:
\begin{gather*}
    \pd{y_{it}}{\varepsilon_{jt}} = \left( (F^0)_{11} \Phi_{\varepsilon} \right)_{ij} = \left(\Sigma_{\tr} \Omega \right) = 0
\end{gather*}
and restrict the dot product of the $i$-th row of $\Sigma_{\tr}$ and the $j$-th column of $\Omega$ to be zero.

\begin{eg}[Short-run Rerictions]
    \

    Consider the following example: the central bank does not react contemporaneously to technology shocks
    because data on aggregate output only become available with a one-quarter lag.
    Suppose that
    \begin{gather*}
        y_t = \begin{bmatrix}
             \text{Fed Funds} \\
             \text{Output} \\
        \end{bmatrix}, \quad
        \varepsilon_t = \begin{bmatrix}
             \varepsilon_{R,t} \\
             \varepsilon_{z,t} \\
        \end{bmatrix}
        = \begin{bmatrix}
             \text{Technology shock} \\ 
             \text{Monetary Policy shock} \\
        \end{bmatrix}
    \end{gather*}
    Formalization in our framework:
    \begin{enumerate}
        \item[(i)] $\theta =0, \xi =1$;
        \item[(ii)] $\theta = 0, \xi =-1$;
        \item[(iii)] $\theta = \pi, \xi =1$;
        \item[(iv)] $\theta = \pi, \xi =-1$;
    \end{enumerate}
    Under (i), we have:
    \begin{gather*}
        u_t = \begin{bmatrix}
            \Sigma_{11}^{\tr}  &  0 \\
            \Sigma_{21}^{\tr} & \Sigma_{22}^{\tr} \\
        \end{bmatrix}
        \begin{bmatrix}
            1 &  0 \\
            0 &  1 \\
        \end{bmatrix}
        \begin{bmatrix}
             \varepsilon_{R,t} \\
             \varepsilon_{z,t} \\
        \end{bmatrix}
    \end{gather*}
    Choosing between (i)-(iv) normalizes the direction of responses.
\end{eg}

\begin{eg}[Long-run Restrictions]
    \

    Now, let's assume that monetary policy shocks do not raise output in the long-run.
    Suppose that
    \begin{gather*}
        y_t = \begin{bmatrix}
             \text{Inflation} \\
             \text{Output Growth} \\
        \end{bmatrix}, \quad
        \varepsilon_t = \begin{bmatrix}
             \varepsilon_{R,t} \\
             \varepsilon_{z,t} \\
        \end{bmatrix}
        = \begin{bmatrix}
             \text{Technology shock} \\ 
             \text{Monetary Policy shock} \\
        \end{bmatrix}
    \end{gather*}
    Moreover, we have (as in the definition \ref{def:MAq}): 
    \begin{gather*}
        y_t = \left(\sum_{j=0}^{\infty} \phi_j L^j\right) u_t = \phi(L) u_t
    \end{gather*}
    Notice that $\xi $ only affects the response to the second shock. Hence we set it to 1.

    Let's examine the moving average representation of $y_t$ in terms of the structural shocks:
    \begin{align*}
        y_t &= \begin{bmatrix}
            \phi_{11}(L) & \phi_{12}(L) \\
            \phi_{21}(L) & \phi_{22}(L) \\
        \end{bmatrix}
        \begin{bmatrix}
            \Sigma_{11}^{\tr}  &  0 \\
            \Sigma_{21}^{\tr} & \Sigma_{22}^{\tr} \\
        \end{bmatrix}
        \begin{bmatrix}
            \cos \theta & -\sin \theta \\
            \sin \theta & \cos \theta \\
        \end{bmatrix}
        \begin{bmatrix}
             \varepsilon_{R,t} \\
             \varepsilon_{z,t} \\
        \end{bmatrix} \\
        &=
        \begin{bmatrix}
            \cdot &  \cdot \\
            \Sigma_{11}^{\tr} \cos \theta \phi_{21}(L) + \left(\Sigma_{21}^{\tr} \cos \theta + \Sigma_{22}^{\tr} \sin \theta \right)\phi_{22}(L) & \cdot \\
        \end{bmatrix}
        \begin{bmatrix}
             \varepsilon_{R,t} \\
             \varepsilon_{z,t} \\
        \end{bmatrix} \\
        &=
        \begin{bmatrix}
            d_{11}(L) & d_{12}(L) \\
            d_{21}(L) & d_{22}(L) \\
        \end{bmatrix}
        \begin{bmatrix}
             \varepsilon_{R,t} \\
             \varepsilon_{z,t} \\
        \end{bmatrix}
    \end{align*}
    Suppose that in period $t = 0$ log output and log prices are equal to zero.
    Then the log-level of output and prices in period $t = T > 0$ is given by:
    \begin{gather*}
        y_T = \sum_{t=1}^{T}y_t = \sum_{t=1}^{T} \sum_{j=0}^{\infty} D_j \varepsilon_{t-j}
    \end{gather*}
    Now consider the derivative
    \begin{gather*}
        \frac{\partial y_T}{\partial \varepsilon_1^{\prime}} = \sum_{j=0}^{T-1} D_j
    \end{gather*}
    Letting $T \to \infty$ gives us the long-run response of the level of prices and output to the shock $\varepsilon_1$:
    \begin{gather*}
        \pd{y_\infty}{\varepsilon_1^{\prime}} = \sum_{j=0}^{\infty} D_j = D(1)
    \end{gather*}
    Here, we want to restrict the long-run effect of monetary policy shocks on output: $d_{21}(1) = 0$.
    This leads us to the equation
    \begin{gather*}
        \left[\Sigma_{11}^{\tr}\phi_{21}(1) + \Sigma_{21}^{\tr}\phi_{22}(1) \right]\cos \theta + \Sigma_{22}^{\tr}\phi_{22}(1) \sin \theta = 0.
    \end{gather*}
    Notice that the equation has two solutions for $\theta \in \left(-\pi, \pi \right].$
    Under one solution a positive monetary policy shock is contractionary,
    under the other solution it is expansionary.
    The shape of the responses is, of course, the same.
\end{eg}