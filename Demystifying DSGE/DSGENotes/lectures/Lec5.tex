\section{Random Effects}

As with pooled OLS, a random effects analysis puts $\alpha_i$ into the error term.
In fact, random effects analysis imposes more assumptions than those needed for pooled OLS:
\textbf{strict exogeneity} in addition to orthogonality between $\alpha_i$ and $x_{it}$. 

\subsection{Basic Assumptions and POLS}

Stating the assumption in terms of conditional means, we have:
\begin{assumption}[Random Effect]\label{assumption:RE}
    \

    \begin{enumerate}
        \item[(a)] $\mathbb{E}[u_{it} | \tilde{X}_i, \tilde{\alpha}_i] = 0, \forall t.$
        \item[(b)] $\mathbb{E}[\tilde{\alpha}_i | \tilde{X}_i] = \mathbb{E}[\tilde{\alpha}_i] = 0.$
    \end{enumerate}
    where $\tilde{X}_i = (x_{i1}, \cdots, x_{iT})$.
\end{assumption}
Assumption \ref{assumption:RE}(a) is the strict exogeneity condition and Assumption \ref{assumption:RE}(b) is is how we will state the orthogonality.

\begin{remark}[Why Strict Ecogeneity?\cite{wooldridge2010econometric}]
    \

    Why do we maintain Assumption \ref{assumption:RE}(a) when it is more restrictive than needed for
    a pooled OLS analysis? Because the random effects approach exploits the serial correlation in
    the composite error, $v_{it} = \alpha_i + u_{it}$, in a generalized least squares (GLS) framework. In
    order to ensure that feasible GLS is consistent, we need some form of strict exogeneity between the explanatory variables and the composite error.

    Under this assumption, we can write:
    \begin{align*}
        & y_{it} = x_{it}^{\prime} \beta + v_{it} \\
        & \mathbb{E}[v_{it} | X_i] = 0, t=1, \cdots, T
    \end{align*}
    The conditions shows that our model satisfies the GLS assumption, which confirms that
    we can apply GLS methods that account for the particular error structure $v_{it} = \alpha_i + u_{it}.$
\end{remark}


By defining $v_{it} = u_{it} + \alpha_i - \beta_0$, we can transform the random effect model to the following:
\begin{align*}
    y_{it} &= \alpha_i + x_{it}^{\prime} \beta + u_{it} \\
    &= \underset{\tilde{x}_{it}^{\prime} \tilde{\beta}}{\underbrace{\beta_0 + x_{it}^{\prime}\beta}} + \underset{\equiv v_{it}}{\underbrace{u_{it} + \alpha_i - \beta_0}}
\end{align*}
Defining again $\tilde{x}_{it} = (1, x_{it}^{\prime})^{\prime}$, $\tilde{\beta} = (\beta_0, \beta^{\prime})^{\prime}$, we can rewrite the model as:
\begin{align*}
    y_{it} &= \tilde{x}_{it}^{\prime} \tilde{\beta} + v_{it} \Leftrightarrow y_i = \tilde{X}_i^{\prime} \tilde{\beta} + v_i \\
    \rightarrow \hat{\tilde{\beta}} &= \left(\sum_i \tilde{X}_i^{\prime} \tilde{X}_i \right)^{-1} \sum_i \tilde{X}_i^{\prime} y_i
\end{align*}

With this intercept $\beta_0$, $\mathbb{E}[v_i] = 0$ is guaranteed to hold.
Define $\tilde{\alpha}_i = \alpha_i - \beta_0$ as the mean-zero unit-specific heterogeneity so that $v_i = u_i + \tilde{\alpha}_i.$


\begin{note}[POLS]
    \

    Homogenous spec: $y_{it} = \alpha  + x_{it}^{\prime} \beta + u_{it} = \tilde{x}_{it}^{\prime} \tilde{\beta} + v_{it}.$
    $\hat{\tilde{\beta}}$ is consistent if $\mathbb{E}[v_{it} x_{it}]=0, \forall t.$
\end{note}
Using pooled OLS to estimate $\hat{\tilde{\beta}}$,
\begin{align*}
    \hat{\tilde{\beta}}_{RE-OLS/POLS} &= \left(\frac{1}{n} \sum_i \tilde{X}_i^{\prime} \tilde{X}_i \right)^{-1} \frac{1}{n} \sum_i \tilde{X}_i^{\prime} y_i \\
    &= \tilde{\beta} + \left(\frac{1}{n} \sum_i \tilde{X}_i^{\prime} \tilde{X}_i \right)^{-1} \frac{1}{n} \sum_i \tilde{X}_i^{\prime} v_i \\
    &\overset{p}{\rightarrow} \tilde{\beta} + \mathbb{E}[\tilde{X}_i^{\prime} \tilde{X}_i]^{-1} \mathbb{E}[\tilde{X}_i^{\prime} v_i] \\
    \text{where} \quad \mathbb{E}[\tilde{X}_i^{\prime} v_i] &= \mathbb{E}\left[\sum_t \tilde{x}_{it}^{\prime} v_{it} \right] \\
    &= \sum_t \mathbb{E}\left[\tilde{x}_{it}^{\prime} v_{it}  \right] \\
    &= \sum_t \mathbb{E}\left[\tilde{x}_{it}(u_{it} + \alpha_i - \beta_0)\right]
\end{align*}
Here, the error term $v_i$ is not equal to the original error term $u_{it}$.
\begin{note}
    \

    Under the random effect, you have to use the heteroskedasticity-robust methods.
    Because even if we assume $u_{it}$ to be homoskedastic, $v_{it}$ is not,
    as it includes also the unit-specific heterogeneity $\alpha_i$.
\end{note}

\subsection{From POLS to GLS}

So, to obtain consistency, we need to assume that:
\begin{assumption}[Random Effect Independence]\label{assumption:RE1}
    \begin{enumerate}
        \item[(a)] $\mathbb{E}[u_{it} | \tilde{x}_{it}, \tilde{\alpha}_i] = 0, \forall t$.
        \item[(b)] $\mathbb{E}[\tilde{\alpha_i} | \tilde{x}_{it}] = 0, \forall t$.
    \end{enumerate}
\end{assumption}

And, we are also obliged to use HAC-robust standard error because:
\[\Omega \equiv \mathbb{E}[v_i v_i^{\prime} | \tilde{X}_i] = \mathbb{E}[(\alpha_i \mathbf{1}_i + u_i)(\tilde{\alpha}_i \mathbf{1}_i + u_i)^{\prime} | \tilde{X}_i] = \mathbb{E}[\tilde{\alpha}_i^2 \mathbf{1}_i \mathbf{1}_i^{\prime} | \tilde{X}_i] + \mathbb{E}[u_i u_i'| \tilde{X}_i] \]
is not diagonal.

\begin{assumption}[Random Effect Rank]\label{assumption:RE2}
    \

    $\rank \mathbb{E}\left[X_i^{\prime} \Omega^{-1} X_i \right] = K$
\end{assumption}
We know that both GLS and feasible GLS estimator would be consistent under Assumption \ref{assumption:RE} and \ref{assumption:RE2}.
A general FGLS analysis, using an unrestricted variance estimator $\Omega$,
is consistent and asymptotically normal as $N \to \infty.$

But, we won't exploit the unobserved effects structure $v_{it}.$
A standard random effects analysis adds assumptions on the idiosyncratic errors that
give $\Omega$ a special form. The first assumption is that the idiosyncratic errors
$u_{it}$ have a constant unconditional variance across $t$:
\begin{assumption}[RE-Homoskedasticity]\label{assumption:RE-homoskedasticity}
    \

    $\mathbb{E}[u_{it}^2] = \sigma_u^2, \forall t$
\end{assumption}
The second assumption is that the idiosyncratic errors are serially uncorrelated:
\begin{assumption}[RE-Serial Uncorrelated]\label{assumption:RE-serial_uncorrelated}
    \

    $\mathbb{E}[u_{it} u_{is}] = 0, \forall t \neq s$
\end{assumption}
Under these two assumptions, we can derive the variances and covariances of the
elements of $v_i$.
Given the error structure the natural estimator for $\beta$ is GLS. The GLS eimator for $\beta$ is:
\begin{gather*}
    \hat{\tilde{\beta}}_{RE-GLS} = \left( \sum_i \tilde{X}_i^{\prime} \Omega^{-1} \tilde{X}_i \right)^{-1} \sum_i \tilde{X}_i^{\prime} \Omega^{-1} y_i
\end{gather*}
where $\Omega ^{-\frac{1}{2}} y_i = \Omega ^{-\frac{1}{2}} \tilde{X}_i^{\prime} \tilde{\beta} + \Omega^{-\frac{1}{2}}v_i.$
\begin{align*}
    \Omega &= \mathbb{E}[v_i v_i^{\prime} | \tilde{X}_i] = \mathbb{E}\left[ \begin{bmatrix}
        v_{i1} \\
        v_{i2} \\
        \vdots \\
        v_{iT}
    \end{bmatrix} \begin{bmatrix}
        v_{i1} & v_{i2} & \cdots & v_{iT}
    \end{bmatrix} | \tilde{X}_i \right]\\
    &= \mathbb{E}\begin{bmatrix}
        \mathbb{E}[v_{i1}^2 | \tilde{X}_i] & \mathbb{E}[v_{i1}v_{i2} | \tilde{X}_i] & \cdots & \mathbb{E}[v_{i1}v_{iT} | \tilde{X}_i] \\
        \mathbb{E}[v_{i2}v_{i1} | \tilde{X}_i] & \mathbb{E}[v_{i2}^2 | \tilde{X}_i] & \cdots & \mathbb{E}[v_{i2}v_{iT} | \tilde{X}_i] \\
        \vdots & \vdots & \ddots & \vdots \\
        \mathbb{E}[v_{iT}v_{i1} | \tilde{X}_i] & \mathbb{E}[v_{iT}v_{i2} | \tilde{X}_i] & \cdots & \mathbb{E}[v_{iT}^2 | \tilde{X}_i]
    \end{bmatrix} \\
    &= \begin{bmatrix}
        \mathbb{E}\left[\alpha_i^2 | \tilde{X}_i \right] + \mathbb{E}[u_{i1}^2 | \tilde{X}_i] & \mathbb{E}\left[\alpha_i^2 | \tilde{X}_i \right] + \mathbb{E}[u_{i1}u_{i2} | \tilde{X}_i] & \cdots & \mathbb{E}\left[\alpha_i^2 | \tilde{X}_i \right] + \mathbb{E}[u_{i1}u_{iT} | \tilde{X}_i] \\
        \mathbb{E}\left[\alpha_i^2 | \tilde{X}_i \right] + \mathbb{E}[u_{i2}u_{i1} | \tilde{X}_i] & \mathbb{E}\left[\alpha_i^2 | \tilde{X}_i \right] + \mathbb{E}[u_{i2}^2 | \tilde{X}_i] & \cdots & \mathbb{E}\left[\alpha_i^2 | \tilde{X}_i \right] + \mathbb{E}[u_{i2}u_{iT} | \tilde{X}_i] \\
        \vdots & \vdots & \ddots & \vdots \\
        \mathbb{E}\left[\alpha_i^2 | \tilde{X}_i \right] + \mathbb{E}[u_{iT}u_{i1} | \tilde{X}_i] & \mathbb{E}\left[\alpha_i^2 | \tilde{X}_i \right] + \mathbb{E}[u_{iT}u_{i2} | \tilde{X}_i] & \cdots & \mathbb{E}\left[\alpha_i^2 | \tilde{X}_i \right] + \mathbb{E}[u_{iT}^2 | \tilde{X}_i]
    \end{bmatrix}\\
    &= \begin{bmatrix}
        \sigma_u^2 + \sigma_{\alpha}^2 & \sigma_{\alpha}^2 & \cdots & \sigma_{\alpha}^2 \\
        \sigma_{\alpha}^2 & \sigma_u^2 + \sigma_{\alpha}^2 & \cdots & \sigma_{\alpha}^2 \\
        \vdots & \vdots & \ddots & \vdots \\
        \sigma_{\alpha}^2 & \sigma_{\alpha}^2 & \cdots & \sigma_u^2 + \sigma_{\alpha}^2
    \end{bmatrix} \\
    &= \sigma_{\alpha}^2 \mathbf{1}_i \mathbf{1}_i^{\prime} + \sigma_u^2 I \\
    \text{beacuse } &\mathbb{V}[\tilde{\alpha}_i|\tilde{X}_i] = \sigma_{\alpha_i}^2 = \sigma_{\alpha}^2 \\
    &\mathbb{V}[u_{it} | \tilde{X}_i] = \sigma_u^2, \forall i.
\end{align*}
where $I$ is an identity matrix of dimention $T_i$.
Under the assumption $\mathbb{E}[u_{it} x_{is}] = 0$, we now describe some statistical properties of $\hat{\tilde{\beta}}_{RE-GLS}.$

\subsubsection{RE Consistency}
\begin{align*}
    \hat{\tilde{\beta}}_{RE-GLS} - \tilde{\beta} &= \Bigl( \sum_i \tilde{X}_i^{\prime} \Omega^{-1} \tilde{X}_i \Bigr)^{-1} \Bigl( \sum_i \tilde{X}_i^{\prime} \Omega^{-1} v_i \Bigr) \\
    & \rightarrow \mathbb{E}\Bigl[ \sum_i \tilde{X}_i^{\prime} \Omega^{-1} \tilde{X}_i \Bigr] \mathbb{E}\Bigl[ \sum_i \tilde{X}_i^{\prime} \Omega^{-1} v_i \Bigr] \\
    \text{where } \mathbb{E}\Bigl[ \sum_i \tilde{X}_i^{\prime} \Omega^{-1} v_i \Bigr] &= \sum_i \mathbb{E}\Bigl[ \tilde{X}_i^{\prime} \Omega^{-1} v_i \Bigr] \\
    &= \sum_i \tilde{X}_i^{\prime} \Omega^{-1} \mathbb{E}[v_i | \tilde{X}_i] \\
    &= \sum_i \tilde{X}_i^{\prime} \Omega^{-1} \mathbb{E}[u_i + \tilde{\alpha}_i | \tilde{X}_i] \\
    &=0
\end{align*}
Thus, $\hat{\tilde{\beta}}_{RE-GLS} $ is conditionally unbiased for $\tilde{\beta}$.
The conditional variance of $\hat{\tilde{\beta}}_{RE-GLS}$ is:
\begin{align*}
    \mathbb{V}\Bigl[\hat{\tilde{\beta}}_{RE-GLS}\Bigr] = \Bigl( \sum_i \tilde{X}_i^{\prime} \Omega^{-1} \tilde{X}_i \Bigr)^{-1} \sigma_{u}^2
\end{align*}

\subsubsection{RE Asymptotic Distribution}
The asymptotic variance of $\hat{\tilde{\beta}}_{RE-GLS}$ is:
\begin{align*}
    &\sqrt{n} \left( \hat{\tilde{\beta}}_{RE-GLS} - \tilde{\beta} \right) \overset{d}{\rightarrow} \mathcal{N}\left(0, V \right) \\
    \text{where } V_{GLS}  &= \mathbb{E}\left[\tilde{X}_i^{\prime} \Omega^{-1} \tilde{X}_i \right]^{-1} \mathbb{E}\left[\tilde{X}_i^{\prime} \Omega^{-1} v_i v_i^{\prime} \Omega^{-1} \tilde{X}_i \right] \mathbb{E}\left[\tilde{X}_i^{\prime} \Omega^{-1} \tilde{X}_i \right]^{-1} \\
    &= \mathbb{E}\left[\tilde{X}_i^{\prime} \Omega^{-1} \tilde{X}_i \right]^{-1} \underset{\equiv \Omega}{\underbrace{\mathbb{E}[v_i v_i^{\prime} | \tilde{X}_i]}}
\end{align*}
Because we do not know $\Omega$, the RE-GLS estimator is infeasible. 

% If indeed we have:
% \begin{align*}
%     \Omega &= \mathbb{E}[v_i v_i^{\prime}  | \tilde{X}_i] \\
%     &= \mathbb{E}[(\alpha_i \mathbf{1}_i + u_i)(\alpha_i \mathbf{1}_i  + u_i)^{\prime}  | \tilde{X}_i] \\
%     &= \mathbb{E}[\alpha_i^2] \mathbf{1}_i \mathbf{1}_i^{\prime} + \mathbb{E}[u_i u_i^{\prime} | \tilde{X}_i]
% \end{align*}
% which implies homoskedasticity.

The previous assumptions \ref{assumption:RE-homoskedasticity} and \ref{assumption:RE-serial_uncorrelated} are special to random e¤ects.
For efficiency of feasible GLS, we assume that the variance matrix of $v_i$ conditional on $\tilde{X}_i$ is constant:
\[\mathbb{E}[v_i v_i^{\prime} | \tilde{X}_i] = \mathbb{E}[v_i v_i^{\prime} ].\]
The two conditions are also implied by the following stronger version of assumption:
\begin{assumption}[RE General Homoskedasticity and Serial Uncorrelation]\label{assumption:RE3}
    \
    \begin{enumerate}
        \item[(a)] $\mathbb{E}[u_i u_i^{\prime} | \tilde{X}_i, \tilde{\alpha}_i] = 0$
        \item[(b)] $\mathbb{E}[\tilde{\alpha}_i^2 | \tilde{X}_i] = \sigma_{\alpha}^2 $  
    \end{enumerate}
\end{assumption}
Under assumption \ref{assumption:RE3}(a), $\mathbb{E}[u_{it}^2 | \tilde{X}_i, \tilde{\alpha}_i] = \sigma_u^2$,
which implies assumption \ref{assumption:RE-homoskedasticity}, and $\mathbb{E}[u_{it} u_{is} | \tilde{X}_i, \tilde{\alpha}_i] = 0$
which implies assumption \ref{assumption:RE-serial_uncorrelated}. 
But this new assumption is stronger because it implies that the conditional variances are constant and the conditional covariances are zero.

Together with assumption \ref{assumption:RE1}(b), assumption \ref{assumption:RE3}(b) is the same as: $\mathbb{V}[\tilde{\alpha}_i | \tilde{X}_i] = \mathbb{V}[\tilde{\alpha}_i]$,
which is a homoskedasticity assumption for the unobserved effects $\tilde{\alpha}_i$.

A feasible version replaces $\Omega$ with an estimator $\hat{\Omega}$. To implement an FGLS procedure, define:
\[\sigma_v^2 = \sigma_u^2 + \sigma_{\alpha}^2 \]
then we can obtain: $\hat{\Omega} = \hat{\sigma}_{\alpha}^2 \mathbf{1}_i \mathbf{1}_i^{\prime} + \hat{\sigma}_u^2 I_T$, 
a $T \times T$ matrix that we assume to be positive definite.
In a panel data context, the FGLS estimator that uses this variance matrix is what is known as the \textbf{random effects estimator}.
\[\hat{\beta}_{RE} = \Bigl( \sum_{i} \tilde{X}_i^{\prime} \hat{\Omega}^{-1} \tilde{X}_i \Bigr)^{-1} \sum_{i} \tilde{X}_i^{\prime} \hat{\Omega}^{-1} y_i \]
% Assuming homoskedasticity of the original errors:
% \begin{align*}
%     \mathbb{E}[u_i u_i^{\prime}  | \tilde{X}_i, \tilde{\alpha}_i] &= \sigma_u^2 I_T \\
%     \mathbb{E}[\tilde{\alpha}_i^2 | \tilde{X}_i] &= \sigma_{\alpha}^2
% \end{align*}

Hence, the motivation for using GLS is different than under a cross-sectional regression with heteroskedasticity.
We use GLS because of the autocorrelation in $v_{it}$ induced by the presence of time variant $\alpha_i$.

\subsection{Comparing POLS and GLS}

Now, let's compare the $\hat{\beta}_{RE-GLS}$ with the pooled estimator $\hat{\beta}_{POLS}$.

Under the assumptions of the random effects model, POLS estimator is also unbiased for $\beta$ and has conditional variance:
\begin{gather*}
    V_{POLS} = \Bigl(\sum_{i} X_i^{\prime} X_i \Bigr)^{-1} \Bigl( \sum_{i}  X_i^{\prime} \Omega X_i \Bigr) \Bigl( \sum_{i}  X_i^{\prime} X_i \Bigr)^{-1} 
\end{gather*}
Using the algebra of the Gauss-Markov Theorem we deduce that:
\[V_{RE-GLS} \leq V_{POLS} \]
and thus the random effects estimator $\hat{\beta}_{RE-GLS}$ is more efficient than $\hat{\beta}_{POLS} $ under the strict exogeneity assumption \ref{assumption:RE}.
The two variance matrices are identical when there is no individual-specific effect $\sigma_{\alpha}^2 = 0$ for then $V_{RE-GLS} = V_{POLS} = \left(X^{\prime} X\right)^{-1} \sigma_u^2.$

Under the assumption that the random effects model is a useful approximation but not literally true, 
we may use the cluster-robust covariance matrix estimator such as:
\begin{gather*}
    \hat{V}_{RE-GLS} = \Bigl(\sum_{i} X_i^{\prime} \Omega^{-1} X_i\Bigr)^{-1} \Bigl( \sum_{i} X_i^{\prime} \Omega^{-1} \hat{v}_i \hat{v}_i^{\prime} \Omega^{-1} X_i \Bigr) \Bigl( \sum_{i} X_i^{\prime} \Omega^{-1} X_i \Bigr)^{-1} 
\end{gather*}
where $\hat{v}_i = y_i - X_i \hat{\beta}_{RE-GLS}$, This may be re-scaled by a degree of freedom adjustment if desired.

\section{Fixed Effects}
In the econometrics literature if the stochastic structure of $\alpha_i$ is treated as unknown
and possibly correlated with $x_{it}$, then $\alpha_i$ is called a \textbf{fixed effect}.

Correlation between $\alpha_i$ and $x_{it}$ will cause both pooled and random effect estimators to be biased.
This is due to the classic problems of omitted variables bias and endogeneity.

The presence of the unstructured individual effect $\alpha_i$ means that it is not possible to identify $\beta$ under a simple projection assumption such as $\mathbb{E}[u_{it} x_{it}] = 0$.
It turns out that a sufficient condition for identification is the following.

\begin{definition}[Strictly exogeneity]\label{def:strictly_exogeneity}
    \

    A regressor $x_{it}$ is said to be strictly exogeneity if $\mathbb{E}[x_{it} u_{is}] = 0, \forall t, s = 1, \cdots, T$.
\end{definition}
Strict exogeneity is a strong projection condition,  meaning that is a $X_{is}, s \neq t$ is added into the regression model,
it would have a zero coefficient. Strict exogeneity is a projection analog of the \textbf{strict mean independence}\label{FE:SMI}:
\[\mathbb{E}[u_{it} | X_i] = 0\] 
which implies the strict exogeneity but not vice versa.

The strict exogeneity assumption \ref{def:strictly_exogeneity} is sufficient for identification and
asymptotic theory, we'll also use the strict mean independence assumption for finite sample analysis.

\begin{remark}[About strict exogeneity\cite{hansen2022econometrics}]
    \

    Strict ecogeneity(assumption \ref{def:strictly_exogeneity}) is typically inappropriate in dynamic models.
\end{remark}

\subsection{Within Transformation}

In previous steps, we showed that if $x_{it}$ and $\alpha_i$ are correlated, then pooled OLS and RE-GLS estimator would be biased and inconsistent.
If we leave the relationship between $\alpha_i$ and $x_{it}$ fully unstructured,
then the only way to consistently estimate the coefficient $\beta$ is by an estimator
which is invariant to $\alpha_i$.

The first fixed effects (FE) assumption is strict exogeneity of the explanatory variables conditional on $\alpha_i$:
\begin{assumption}[FE Strict Exogeneity]\label{assumption:FE-strictexogeneity}
    \

    $\mathbb{E}[u_{it} | X_i, \alpha_i] = 0, \forall t=1,\cdots, T$
\end{assumption}
This assumption is identical to the assumption \ref{assumption:RE}(a), we maintain strict exogeneity of $x_{it}, t=1, \cdots, T$
conditional on the unobserved effect.
The key difference is that \emph{we do not assume assumption \ref{assumption:RE}(b), which means that,
for FE analysis, $\mathbb{E}[\alpha_i | X_i]$ can be any function of $X_i$.}

By relaxing assumption \ref{assumption:RE}(b), we can conssitently estimate partial effects in the presence of time-consistent omitted variables
that can be arbitrarily related to unobserved variables $x_{it}$.
\emph{Therefore, FE analysis is more robust than RE analysis.}

But this robustness has a cost: we can not include any time-constant variables in $x_{it}$ without further assumptions.
The reason is simple: if $\alpha_i$ can be arbitrarily correlated with each element of $x_{it}$, 
then there's no way to distinguish the effect of time-constant observables from the time-constant unobservable $\alpha_i$.

We transform the equation to get rid of $\alpha_i$: $y_{it} = \alpha_i + x_{it}^{\prime} \beta + u_{it}.$
The first transformation is the \textbf{within transformation}.
Define the mean of a variable for a given individual as
\begin{align*}
    \overline{y}_i &= \frac{1}{T} \sum_t y_{it} \\
    \overline{x}_i &= \frac{1}{T} \sum_t x_{it} \\
    \overline{u}_i &= \frac{1}{T} \sum_t u_{it}
\end{align*}
We call this the \textbf{individual-specific mean} since it is the mean of a given individual. \footnote{Some
authors call this the \textbf{time-average} or \textbf{time-mean} since it is the average over the time periods.}

Then, subtracting the individual-specific mean from the variable we obtain the deviations:
\begin{gather*}
    (y_{it} - \overline{y}_i) = (x_{it} - \overline{x}_i)^{\prime} \beta + (u_{it} -\overline{u}_i) + (\alpha_i - \alpha_i) \\
    \ddot{y}_{it} = \ddot{x}_{it}^{\prime} \beta + \ddot{u}_{it} \text{ or at individual level: } \ddot{y}_i = \ddot{X}_i^{\prime} \beta + \ddot{u}_i
\end{gather*}
This is the \textbf{within transformation}. We also refer to $\ddot{y}_{it}$ as the \textbf{demanded values} or \textbf{deviation from individual means}.
\emph{What is important is that the demeaning has occured at the individual level.}

Denote the time-averages method by $\hat{\beta}_{FE-W}$, in order to ensure that the FE estimator is consistent and 
well behaved asymptotically, we need a standard rank condition on the matrix of time-demeaned explanatory variables:
\begin{assumption}[FE full rank]\label{assumption:FE-rank}
    \

    $\rank \sum_{t} \mathbb{E}[\ddot{x}_{it}^{\prime} \ddot{x}_{it}] = \rank \mathbb{E}[\ddot{X}_i^{\prime} \ddot{X}_i] = K$
\end{assumption}

\subsubsection{FE Consistency}
\begin{align*}
    \hat{\beta}_{FE-W} &= \left( \sum_{i} \ddot{X}_i^{\prime} \ddot{X}_i \right)^{-1} \left( \sum_{i} \ddot{X}_i^{\prime} \ddot{y}_i \right) \\
    &= \left(\sum_i \sum_t \ddot{x}_{it} \ddot{x}_{it}^{\prime} \right)^{-1} \sum_i \sum_t \ddot{x}_{it} \ddot{y}_{it} \\
    &= \beta + \left(\sum_i \sum_t \ddot{x}_{it} \ddot{x}_{it}^{\prime} \right)^{-1} \sum_i \sum_t \ddot{x}_{it} \ddot{u}_{it} \\
    &\overset{p}{\rightarrow} \beta + \mathbb{E}\left[\sum_t \ddot{x}_{it} \ddot{x}_{it}^{\prime} \right]^{-1} \mathbb{E}\left[\sum_t \ddot{x}_{it} \ddot{u}_{it} \right] \\
    \text{where } \mathbb{E}\left[\sum_t \ddot{x}_{it} \ddot{u}_{it}\right] &= \sum_t \mathbb{E}\left[\ddot{x}_{it} \ddot{u}_{it} \right]\\
    \mathbb{E}\left[\ddot{x}_{it} \ddot{u}_{it} \right] &= \mathbb{E}\left[\left(x_{it} - \frac{1}{T}\sum_t x_{it} \right) \left(u_{it} - \frac{1}{T}\sum_t u_{it} \right)^{\prime} \right] \\
    &= 0 \quad \text{if } u_{it} \perp\!\!\!\perp x_{is}, \forall t, s = 1, \cdots, T.
\end{align*}
Then, let $\Sigma_i = \mathbb{E}[u_i u_i^{\prime} | X_i]$ denote the $T_i \times T_i$ covariance matrix of the idiosyncratic errors.
The variance of $\hat{\beta}_{FE-W}$ is:
\begin{gather*}
    V_{FE-W} = \mathbb{V}[\hat{\beta}_{FE-W} | X_i] = \Bigl( \sum_{i} \ddot{X}_i^{\prime} \ddot{X}_i \Bigr)^{-1} \Bigl( \sum_{i} \ddot{X}_i^{\prime} \Sigma_i \ddot{X}_i \Bigr) \Bigl( \sum_{i} \ddot{X}_i^{\prime} \ddot{X}_i \Bigr)^{-1} 
\end{gather*}
This expression simplifies when the idiosyncratic errors are homoskedastic and serially uncorrelated:
\begin{assumption}[FE homoskedasticity and Serial Uncorrelation]\label{FE-homoskedasticity}
    \

    \begin{enumerate}
        \item[(a)] $\mathbb{E}[u_{it}^2 | X_i] = \sigma_u^2$ 
        \item[(b)] $\mathbb{E}[u_{it} u_{is} | X_i] = 0, \forall s \neq t.$
    \end{enumerate}
\end{assumption}

\subsubsection{FE Asymptotic Distribution}
In this case, $\Sigma_i = \sigma_u^2 I_i$ and $V_{FE-W}$ simplifies to:
\begin{gather*}
    V_{FE-W}^0 = \sigma_u^2 \Bigl( \sum_{i} \ddot{X}_i^{\prime} \ddot{X}_i \Bigr)^{-1} 
\end{gather*}
 We can also write the asymptotic distribution as below
\begin{align*}
    \sqrt{n} (\hat{\beta}_{FE-W} - \beta) &= \left( \frac{1}{N} \sum_{i} \ddot{X}_i^{\prime} \ddot{X}_i \right)^{-1}  \left( \frac{1}{\sqrt{N}} \sum_{i} \ddot{X}_i^{\prime} \ddot{u}_i \right) \\
    &= \left( \frac{1}{N} \sum_{i} \ddot{X}_i^{\prime} \ddot{X}_i \right)^{-1}  \left( \frac{1}{\sqrt{N}} \sum_{i} \ddot{X}_i^{\prime} u_i \right) \footnotemark \\
    &\rightarrow \mathbb{E}[\ddot{X}_i^{\prime} \ddot{X}_i]^{-1} \cdot \mathcal{N} (0, \mathbb{V}[\ddot{X}_i^{\prime} u_i]) \\
    &\sim \mathcal{N}\left(0, V_{FE-W} \right) \\
    \text{where } V_{FE-W} &= \sigma_u^2 \left( \sum_{i} \ddot{X}_i^{\prime} \ddot{X}_i \right)^{-1} 
\end{align*}
\footnotetext{From the regression model $\ddot{y}_i = \ddot{X}_i \beta + \ddot{u}_i$,
where $\ddot{y}_i$ is $T \times 1$, $\ddot{X}_i$ is $T \times K$, and $\ddot{u}_i$ is $T \times 1$,
We can write the individual-specific mean as $\bar{y}_i = (\mathbf{1}_i^{\prime} \mathbf{1}_i)^{-1} \mathbf{1}_i y_i$.
Then, we can define a \textbf{individual-specific demeaning operator}:
\[M_i = I_i - \mathbf{1}_i (\mathbf{1}_i^{\prime} \mathbf{1}_i)^{-1} \mathbf{1}_i^{\prime}, \]
giving that
\[\ddot{y}_i = y_i - \mathbf{1}_i \bar{y}_i = y_i - \mathbf{1}_i (\mathbf{1}_i^{\prime} \mathbf{1}_i)^{-1} \mathbf{1}_i y_i = M_i y_i.\]
Notice that $M_i$ is idempotent ($M_i M_i = M_i, M_i^{\prime} = M_i$). Similarly for $\ddot{X}_i$ and $\ddot{u}_i$.
Thus, we have:
\begin{align*}
    \ddot{X}_i^{\prime} \ddot{u}_i = X_i^{\prime} M_i M_i u_i = X_i^{\prime} M_i u_i = \ddot{X}_i^{\prime} u_i. 
\end{align*}
}
\begin{remark}[About FE Asymptotic Distribution]
    \

    Actually, the asymptotic distribution of $\hat{\beta}_{FE-W}$ is not as obvious as it seems.
    We have to restress a few assumptions to guarantee its validity:
    \[
    \frac{1}{\sqrt{N}} \sum_{i} \ddot{X}_i^{\prime} u_i \overset{d}{\rightarrow} \mathcal{N}(0, \mathbb{V}[\ddot{X}_i^{\prime} u_i]).
    \]
    \begin{assumption}\label{FE Asymptotic Distribution}
        \

        \begin{enumerate}
            \item[(1)] Variables $(u_i, X_i), i=1, \cdots, N$ are independent and identically dis-
            tributed.
            \item[(2)] $\mathbb{E}[X_{it} u_{is}] = 0, \forall t=1, \cdots, T.$
            \item[(3)] $Q_T = \mathbb{E}[\ddot{X}_i^{\prime} \ddot{X}_i] > 0.$
            \item[(4)] $\mathbb{E}[u_{it}^4] < \infty.$
            \item[(5)] $\mathbb{E} \vert | X_{it} | \vert^4 < \infty.$
        \end{enumerate} 
    \end{assumption}
    Assumption \ref{FE Asymptotic Distribution}(2) implies that:
    \begin{gather*}
        \mathbb{E}[\ddot{X}_i u_i] = \sum_{t} \mathbb{E}[\ddot{X}_{it} u_{it}] = \sum_{t} \mathbb{E}[X_{it} u_{it}] - \sum_{t} \sum_{s=1}^T \mathbb{E}[X_{is} u_{it}] = 0
    \end{gather*}
    so they are mean zero.
    Assumption \ref{FE Asymptotic Distribution}(4) and (5) imply that $\ddot{X}_i u_i$ has a finite covariance matrix $\Omega.$
    The assumptions for the CLT hold, thus we have the result.
\end{remark}

\begin{remark}[FE VS. POLS]
    \

    It is instructive to compare the variances of the fixed-effects and pooled estimators under
    \begin{align*}
        \mathbb{E}[u_{it}^2 | X_i] &= \sigma_u^2 \\
        \mathbb{E}[u_{it} u_{is} | X_i] &= 0, \forall s \neq t.
    \end{align*}
    and the assumption that there is no individual-specific effect, $\alpha_i = 0$.
    In this case, we can see that:
    \begin{gather*}
        V_{FE-W}^0 = \sigma_u^2 \Bigl( \sum_{i} \ddot{X}_i^{\prime} \ddot{X}_i \Bigr)^{-1} \geq \sigma_u^2 \Bigl( \sum_{i} X_i^{\prime} X_i \Bigr)^{-1} = V_{POLS}.
    \end{gather*}
    The inequality holds since the demeaned variables $\ddot{X}_i$ have reduced variation compared to the original observations $X_i$.

    This shows the cost of using fixed effects relative to pooled estimation. 
    The estimation variance increases due to reduced variation in the regressors. 
    This reduction in efficiency is a necessary by-product of the robustness of the estimator to the individual effects $\alpha_i$.
\end{remark}

\subsection{First Difference Transformation}

Another important transformation which does the same as within transformation is \textbf{first-differencing}.
\emph{This can be applied to all but the first
observation (which is essentially lost).}
\begin{gather*}
    y_{it} - y_{i, t-1} = (x_{it} - x_{i, t-1})^{\prime} \beta + (u_{it} - u_{i, t-1}) \\
    \Delta y_{it} = \Delta x_{it}^{\prime} \beta + \Delta u_{it} \text{ or at individual level } \Delta y_i = \Delta X_i \beta + \Delta u_i, \quad i=1 \cdots N, t=2 \cdots T
\end{gather*}
We can see that the individual effect $\alpha_i$ has been eliminated.

Denote the first difference method by $\hat{\beta}_{FE-FD}$, 
the fixed effect estimator is consistent and asymptotically normal based on two assumptions.
\begin{assumption}[FD Strict exogeneity]\label{assumption:FD1}
    \

    It's the same as FE's assumption \ref{assumption:FE-strictexogeneity}.
\end{assumption}
\begin{assumption}[FD Full rank]\label{assumption:FD2}
    \

    $\rank \sum_{t=2}^{T} \mathbb{E}[\Delta x_{it}^{\prime} \Delta x_{it}] = K$
\end{assumption}

\subsubsection{FE-FD Consistency}
\begin{align*}
    \hat{\beta}_{FE-FD} &= \left(\sum_i \sum_t \Delta x_{it} \Delta x_{it}^{\prime} \right)^{-1} \sum_i \sum_t \Delta x_{it} \Delta y_{it} \\
    &= \beta + \left(\frac{1}{NT} \sum_i \sum_t \Delta x_{it} \Delta x_{it}^{\prime} \right)^{-1} \frac{1}{NT} \sum_i \sum_t \Delta x_{it} \Delta u_{it} \\
    &\overset{p}{\rightarrow} \beta + \mathbb{E}\left[\sum_t \Delta x_{it} \Delta x_{it}^{\prime} \right]^{-1} \mathbb{E}\left[\sum_t \Delta x_{it} \Delta u_{it} \right] \\
    \text{where } \mathbb{E}\left[\sum_t \Delta x_{it} \Delta u_{it}\right] &= \sum_t \mathbb{E}\left[\Delta x_{it} \Delta u_{it} \right]\\
    \mathbb{E}\left[\Delta x_{it} \Delta u_{it} \right] &= \mathbb{E}\left[\left(x_{it} - x_{i, t-1} \right) \left(u_{it} - u_{i, t-1} \right)^{\prime} \right] \\
    &= 0 \quad \text{if } x_{it} \perp\!\!\!\perp (u_{it}, u_{i, t-1}), \forall t.
\end{align*}
For $T = 2$, $\hat{\beta}_{FE-FD} = \hat{\beta}_{FE-W}$, equals the fixed effects estimator and they differ however, for $T>2$ (See Hanse, 2022\cite{hansen2022econometrics}).

\subsubsection{FE-FD Asymptotic Distribution}
We just use the standard calculation:
\begin{gather*}
    \sqrt{n} (\hat{\beta}_{FE-FD} - \beta) \overset{d}{\rightarrow} \mathcal{N} (0, V_{FE-FD})
\end{gather*}
where
\begin{gather*}
    V_{FE-FD} = \mathbb{E}\Bigl[ \sum_{t=2}^{T} \Delta x_{it} \Delta x_{it}^{\prime} \Bigr]^{-1} \mathbb{E}\Bigl[ \Bigl( \sum_{t=2}^{T} \Delta x_{it} \Delta u_{it} \Bigr) \Bigl( \sum_{s=2}^{T} \Delta x_{is} \Delta u_{is} \Bigr)^{\prime} \Bigr] \mathbb{E}\Bigl[ \sum_{t=2}^{T} \Delta x_{it} \Delta x_{it}^{\prime} \Bigr]^{-1} 
\end{gather*}
If we still assume that the first-difference error term $\Delta u_{it}$ is homoskedastic:
\begin{assumption}[FD homoskedasticity]\label{assumption:FD3}
    \

    Denote $e_{it} \equiv \Delta u_{it}$, $e_i$ is the stack of $e_{it}$ for $t=2, \cdots, T$. 
    $\mathbb{E}\left[e_i e_i^{\prime} | X_i, \alpha_i \right] = \sigma_e^2 I.$
\end{assumption}
then, we can write:
\begin{gather*}
    A\mathbb{V}[\hat{\beta}_{FE-FD}] = \hat{\sigma}_e^2 \Bigl(\sum_{i} \Delta X_i^{\prime} \Delta X_i \Bigr)^{-1} 
\end{gather*}
where $\hat{\sigma}_e^2$ is a consistent estimator of $\sigma_e^2$, and the simplest estimator is obtained by com-
puting the OLS residuals:
\[
\hat{e}_{it} = \Delta y_{it} - \Delta x_{it} \hat{\beta}_{FE-FD}
\]
from the pooled regression.

If the assumption \ref{assumption:FD3} is violated,
replacing expectations with sample means and $\Delta u_{it}$($e_{it}$) with $\widehat{\Delta u_{it}}$($\hat{e}_{it} $) yields the HAC-robust variance estimator $\hat{V}_{FE-FD}.$
\begin{gather*}
    \hat{V}_{FE-FD} = \Bigl( \sum_{i} \Delta X_i^{\prime} \Delta X_i \Bigr)^{-1}  \Bigl( \sum_{i} \Delta X_i^{\prime} \hat{e}_{it} \hat{e}_{it}^{\prime} \Delta X_i \Bigr) \Bigl( \sum_{i} \Delta X_i^{\prime} \Delta X_i \Bigr)^{-1} 
\end{gather*}

\begin{remark}[About FE-W and FE-FD (Hansen, 2022 \cite{hansen2022econometrics})]
    \

    The FD method is not as strong as the within method, because it only requires that the variable is
    uncorrelated with the error term in the same period and the previous period.

    If there is a correlation between the error term in current period and two periods ago, there is a problem of feedback loop,
    which we will imply the correlated random effect model.
\end{remark}

\subsection*{Matrix Notation for FE-FD}
The first-differencing transformation is $\Delta Y_{it} = Y_{it} - Y_{i,t-1}.$
This can be applied to all but the first observation (which is essentially lost).
At the level of the individual this can be written as:
\[\Delta Y_i = D_i Y_i\]
where $D_i$ is the $(T_{i-1}) \times T_i$ difference matrix differencing operator:
\begin{gather*}
    D_i = \begin{bmatrix}
        -1 & 1 & 0 & \cdots & 0 & 0\\
        0 & -1 & 1 & \cdots & 0 & 0\\
        \vdots & \vdots & \vdots & \ddots & \vdots & \vdots\\
        0 & 0 & 0 & \cdots & -1 & 1
    \end{bmatrix}
\end{gather*}
Applying the transformation $\Delta$ to $Y_{it} = \alpha_i + X_{it} \beta + u_{it}$ or
\begin{gather*}
    \Delta Y_i = \Delta X_i \beta + \Delta u_i 
\end{gather*}
We can see that the individual effect $u_i$ has been eliminated.

Least squares applied to the differenced equation gives:
\begin{align*}
    \hat{\beta}_{\Delta} &= \Bigl( \sum_{i} \sum_{t=2}^{T} \Delta X_{it}^{\prime} \Delta X_{it}  \Bigr)^{-1} \Bigl( \sum_{i} \sum_{t=2}^{T} \Delta X_{it}^{\prime} \Delta Y_{it} \Bigr) \\
                &= \Bigl( \sum_{i} \Delta X_i^{\prime} \Delta X_i \Bigr)^{-1} \Bigl( \sum_{i} \Delta X_i^{\prime} \Delta Y_i \Bigr) \\
                &= \Bigl( \sum_{i} X_i^{\prime} D_i^{\prime} D_i X_i \Bigr)^{-1} \Bigl( \sum_{i} X_i^{\prime} D_i^{\prime} D_i Y_i \Bigr)
\end{align*}
is called the differenced estimator.

When the errors $u_{it}$ are serially uncorrelated and homoskedastic, then the error $\Delta u_i = D_i u_i$ has the covariance matrix $H \sigma_u^2$,
where
\begin{gather*}
    H = D_i D_i^{\prime} = \begin{bmatrix}
        2 & -1 & 0 & 0 \\
        -1 & 2 & \ddots & 0 \\
        0 & \ddots & \ddots & -1 \\
        0 & 0 & -1 & 2
    \end{bmatrix}
\end{gather*}
We can reduce estimation variance by using GLS.
When errors are i.i.d. (serially uncorrelated and homoskedastic),
this is:
\begin{align*}
    \tilde{\beta}_{\Delta} &= \Bigl( \sum_{i} \Delta X_i^{\prime} H^{-1} \Delta X_i \Bigr)^{-1} \Bigl( \sum_{i} \Delta X_i^{\prime} H^{-1} \Delta Y_i \Bigr) \\
                  &= \Bigl( \sum_{i} X_i^{\prime} D_i^{\prime} (D_i D_i^{\prime}) D_i X_i \Bigr)^{-1} \Bigl( \sum_{i} X_i^{\prime} D_i^{\prime} (D_i D_i^{\prime}) D_i Y_i \Bigr) \\
                  &= \Bigl( \sum_{i} X_i^{\prime} M_i X_i \Bigr)^{-1} \Bigl( \sum_{i} X_i^{\prime} M_i Y_i \Bigr)
\end{align*}
where $M_i = D_i^{\prime} (D_i D_i^{\prime})^{-1} D_i$.
Recall that $D_i$ is $(T_i - 1) \times T_i$ with rank $T_i - 1$ and is orthogonal to the
vector of ones $\mathbf{1}_i$. This means that $M_i$ projects orthogonally to $\mathbf{1}_i$ 
and thus equals the within transformation matrix.
Hence $\tilde{\beta}_{\Delta} = \hat{\beta}_{FE-W}.$

What we have shown is that under i.i.d. errors, 
GLS applied to the first-differenced equation precisely
equals the fixed effects estimator.

Since the Gauss-Markov theorem shows that GLS has lower variance than least squares,
this means that the fixed effects estimator is more efficient than first differencing
under the assumption that $u_{it}$ is i.i.d.