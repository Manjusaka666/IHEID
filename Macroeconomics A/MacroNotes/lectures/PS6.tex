\section*{Question 1: The CES consumption aggregator}
Consider the CES consumption aggregator $c=\left(\int c(i)^{\frac{\epsilon-1}{\epsilon}}di\right)^{\frac{\epsilon}{\epsilon-1}}.$

\begin{problem*}[1]
    Find the demand functions that minimize the expenditure $\int p(i)c(i)di$ needed to obtain a particular
level of consumption $c$ of the basket of goods.
\end{problem*}

\begin{solution}
    The expenditure minimization problem is:
    \begin{align*}
        \min_{c(i)} &\int p(i)c(i)di\\
        \text{s.t.} &\left(\int c(i)^{\frac{\epsilon-1}{\epsilon}}di\right)^{\frac{\epsilon}{\epsilon-1}}=c.
    \end{align*}
    The Lagrangian is:
    \begin{align*}
        \mathcal{L}=\int p(i)c(i)di+\lambda\left(c-\left(\int c(i)^{\frac{\epsilon-1}{\epsilon}}di\right)^{\frac{\epsilon}{\epsilon-1}}\right).
    \end{align*}
    The first-order condition is:
    \begin{align*}
        \frac{\partial \mathcal{L}}{\partial c(i)} &= p(i) - \lambda c(i)^{-\frac{1}{\varepsilon}}\left(\int c(i)^{\frac{\varepsilon-1}{\varepsilon}} di\right)^{\frac{\varepsilon}{\varepsilon-1}-1} = 0 \\
        \Rightarrow p(i) &= \lambda c(i)^{-\frac{1}{\varepsilon}}c^{\frac{1}{\varepsilon}}
    \end{align*}
    From the first order conditions we get:
    \begin{align*}
        c(i) = c \left( \frac{p(i)}{\lambda} \right)^{-\varepsilon}
    \end{align*}
    Substitute this into the FOC to get:
    \begin{align*}
        c &= \left( \int \left(\frac{p(i)}{\lambda }\right)^{1-\varepsilon} c^{\frac{\varepsilon-1}{\varepsilon}} di \right)^{\frac{\varepsilon}{\varepsilon-1}} \\
        &= c\left(\frac{1}{\lambda}\right)^{-\varepsilon} \left(\int p(i)^{1-\varepsilon} di\right)^{\frac{\varepsilon}{\varepsilon-1}} \\
        \lambda &= \left( \int p(i)^{1-\varepsilon} di \right)^{\frac{1}{1-\varepsilon}}
    \end{align*}
    Interpret $\lambda$ as the shadow price:"If you tighten the
    constraint by one unit, the objective function increases by $\lambda$ units."
    $\lambda$ is the price of one basket, $p_t = \lambda$.
    
    Hence, the demand for good $i$ is:
    \begin{align*}
        c(i) = c \left( \frac{p(i)}{p} \right)^{-\varepsilon}
    \end{align*}
    where
    \begin{align*}
        p = \left( \int p(i)^{1-\varepsilon} di \right)^{\frac{1}{1-\varepsilon}}
    \end{align*}
    Interpretation: demand depends on the relative price $p(i)/p$.
\end{solution}

\begin{problem*}[2]
    Show that the minimum expenditure required is $\int p(i)c(i)di=pc$ where $p=(\int p(i)^{1-\epsilon}di)^{\frac1{1-\epsilon}}$ is the
price index.
\end{problem*}

\begin{solution}
    Plug in the demand function into the expenditure function:
    \begin{align*}
        \int p(i)c(i)di &= \int p(i)c \left( \frac{p(i)}{p} \right)^{-\varepsilon} di \\
        &= c \int p(i) \left( \frac{p(i)}{p} \right)^{-\varepsilon} di \\
        &= p^{\varepsilon} c \int p(i)^{1-\varepsilon} di \\
        &= p^{\varepsilon}c \left[\left( \int p(i)^{1-\varepsilon} di \right)^{\frac{1}{1-\varepsilon}}\right]^{1-\varepsilon} \\
        &= p^{\varepsilon}c p^{1-\varepsilon} \\
        &= pc
    \end{align*}
\end{solution}

\section*{Question 2: Deriving the IS curve}
Consider the New Keynesian model. Definc the output gap $x_t$ as
\[
x_t=\frac{y_t}{\hat{y}_t}
\]
where $\hat{y} _t$denotes the efficient level of output.
Recall the Euler cquation for consumption (with $u^{\prime}(c)=c_{t}^{-1/\sigma}$ 
and market clcaring, $y_t=c_t$)
\[
y_t^{-\frac1\sigma}=\mathbb{E}_t\left(\frac{1+r_t}{1+\bar{r}}y_{t+1}^{-\frac1\sigma}\right)
\]
where $\bar{r}$ is the real intercst rate in steady state, $\beta=1/(1+\bar{r}).$
\begin{problem*}[1]
    Log-lincarize the Euler cquation around the stcady state.
\end{problem*}
\begin{problem*}[2]
    Noting that the approximatod Euler cquation should also hold in an cfficient world
    (i.c. for the efficient level of output and the cfficient real interest rate), 
    usc approximations for the Fisher equation
    \[
    r_t=i_t-\pi_{t+1}
    \]
    and for the definition of the output gap (1) to show that
    \[
    \widetilde{x}_t=\mathbb{E}_t\left(\widetilde{x}_{t+1}-\sigma(\widetilde{i}_t-\widetilde{\pi}_{t+1})\right)+\sigma\widetilde{\hat{r}_t}.
    \]
\end{problem*}

\begin{solution}
    In the steady state, the Euler equation is:
    \[
    y_t^{-\frac1\sigma}=\beta \mathbb{E}_t\left((1+r_t)y_{t+1}^{-\frac1\sigma}\right)
    \]
    where $\beta = \frac{1}{1+i}.$
    Log-linearize the Euler equation around the steady state and take 1st order approximation:
    \begin{align*}
        \ln y_t^{-\frac1\sigma} &= \ln \beta + \mathbb{E}_t\left(\ln(1+r_t) + \ln y_{t+1}^{-\frac1\sigma}\right) \\
        -\frac1\sigma \ln y^* - \frac{1}{\sigma}\frac{1}{y^*}(y_{t}-y^*) &= \ln \beta + \ln(1+r^*) + \frac{1}{1+r^*}(1+r_t - 1-r^*) - \frac{1}{\sigma}\ln y^* \\
        -\frac{1}{\sigma}\frac{1}{y^*}(y_{t}-y^*) &= \frac{r_t - r^*}{1+r^*}\\
        -\frac{1}{\sigma}\tilde{y}_t &= \mathbb{E}_t\left(\tilde{r}_{t} - \frac{1}{\sigma}\hat{y}_{t+1} \right)\\
        \tilde{y}_t &= \mathbb{E}_t\left(\tilde{y}_{t+1} - \sigma \tilde{r}_t\right)\\
        x_t &= \frac{y_t}{\hat{y}_t} \overset{approx}{\sim} \tilde{x}_t = \tilde{y}_t -\tilde{\hat{y}}_t\\
        \Rightarrow \tilde{\hat{y}}_t &= \mathbb{E}_t\left(\tilde{\hat{y}}_{t+1} - \sigma \tilde{\hat{r}}_t\right)\\
        \tilde{y}_t - \tilde{\hat{y}}_t &= \mathbb{E}_t\left(\tilde{y}_{t+1} - \sigma \tilde{r}_t - \tilde{\hat{y}}_{t+1} + \sigma \tilde{\hat{r}}_t\right)\\
        \tilde{x}_t &= \mathbb{E}_t\left(\tilde{x}_{t+1} - \sigma(\tilde{i}_t - \tilde{\pi}_{t+1})\right) + \sigma \tilde{\hat{r}}_t\\
    \end{align*}
    where $\tilde{r}_t = \tilde{i}_t - \tilde{\pi}_{t+1}.$
\end{solution}

\section*{Question 3: The three-equation New Keynesian model}
Consider the New Keyncsian model, which consists of the New Keyncsian Phillips curve,
$$\pi_t=\beta\mathbb{E}_t\pi_{t+1}+\kappa x_t+e_t,$$
the IS curve
$$x_t=\mathbb{E}_t\left(x_{t+1}-\sigma(i_t-\pi_{t+1})\right)+v_t$$
where $v_t=\sigma\widetilde{\hat{r}}_t$ is a shock to the efficient real intercst rate, and a monctary policy rule of the form
$$i_t=\alpha_1\pi_t+u_t$$
where $u_t$ is a shock (with the usual properties: mean zero, uncorrelatcd with all other variables). 
All variables in the three equations, except the shocks, 
are in $\log$deviations from the stcady state 
(i.c. I dropped the $\sim$).

\begin{note}
    \begin{align*}
        r_t &=\gamma \mu_t^*\\
        \mu_t^* &= \frac{\varepsilon}{\varepsilon-1}\\
        v_t &= \sigma \hat{r}_t\\
        \hat{r}_t &= (1+\bar{r})\frac{a_{t+1}}{a_t}-1
    \end{align*}
\end{note}

\begin{problem*}[1]
    Interpret the monctary policy rule. Which sign would you cxpect $\alpha$ to have?
\end{problem*}

\begin{solution}
    The monetary policy rule is a Taylor rule where the interest rate is set as a function of inflation.
    The sign of $\alpha$ is positive since the central bank would want to raise interest rates when inflation is high.
\end{solution}

\begin{problem*}[2]
    State what effect the following havc on $e_t,v_t$, and $u_t:$
    \begin{enumerate}
        \item A temporary risc in productivity (TFP) $a_t$
        \item A permancnt rise in productivity productivity at (i.c. the percentage incrcasc in $a_t$ and $a_t+1$
        is the samc)
        \item A temporary rise in the competitivencss of markets as measurcd by the price clasticity $\varepsilon_t$
        \item A shift in monctary policy caused by new personncl on the monctary policy committec
    \end{enumerate}
\end{problem*}

\begin{solution}
    \begin{enumerate}
        \item A temporary rise in $a_t$ would cause a temporary decrease in $\hat{r}_t$, thus a decrease in $v_t$.
        \item A permanent rise in $a_t$ and $a_{t+1}$ makes no change to $v_t$, thus the three equations stays the same, 
        but since the TFP has increased permanently, the economy will rise to a new higher level equilibrium.  
        \item A temporary rise in the competitiveness of markets would cause a decrease in $\mu_t^*$, thus a decrease in $e_t$.
        \item A shift in monetary policy
    \end{enumerate}
\end{solution}

\begin{problem*}[3]
    Consider a temporary shock to $e_t$ and assumc that this has no cffcct on cxpectations of the future,
    so that $\mathbb{E}_t\pi_{t+1}=0$ and $\mathbb{E}_tx_{t+1}=0.$
    No other shock occurs at the same time, thus $v_t=0$ and $u_t=0.$
    Find expressions for $\pi_t$ and $x_t$ in terms of $e_t$ and interpret your results.
\end{problem*}

\begin{solution}
    In this case, we have:
    \begin{align*}
        \pi_t &= \kappa x_t + e_t\\
        x_t &= \sigma i_t\\
        i_t &= \alpha \pi_t
    \end{align*}
    Thus, 
    \begin{align*}
        x_t &= -\sigma \alpha_1 \pi_t\\
        &= \frac{-\alpha_1 \sigma}{1+\alpha \kappa \sigma}e_t\\
        \pi_t &= \frac{e_t}{1+\kappa \alpha \sigma}\\
        i_t &= \frac{\alpha_1}{1+\kappa \sigma \alpha}e_t
    \end{align*}
\end{solution}

\begin{problem*}[4]
    Suppose that the monctary policy rule is modified so that $i_t=\hat{r}_t+\alpha\pi_t+u_t.$
    If this interest rate rule is used then what are the effects of a temporary shock 
    to the natural interest rate (and hence $v_t)$ on inflation and the output gap?
\end{problem*}

\begin{solution}
    We know that:
    \[
    x_t = \mathbb{E}_t x_{t+1} - \sigma(i_t - \mathbb{E}_t \pi_{t+1}) + v_t,
    \]
    \[
    \pi_t = \mathbb{E}_t \pi_{t+1} + \kappa x_t + e_t.
    \]
    and
    \[
    i_t = \alpha  \pi _t + \hat{r}_t + u_t.
    \]
    We have:
    \[ \mathbb{E}_t x_{t+1} = \mathbb{E}_t \pi_{t+1}=0,\]
    and $u_t = e_t = 0$, $v_t = \sigma \hat{r}_t.$
    So we have:
    \[
    x_t = -\sigma (\alpha \pi_t + \hat{r}_t) + \sigma \hat{r}_t.
    \]

\end{solution}

\begin{problem*}[5]
    Is it possible to redcsign the monctary policy rule to achicve the same cffect found in part (d) but
    now in responsc to a shock to $e_t?$ Explain why or why not.
\end{problem*}