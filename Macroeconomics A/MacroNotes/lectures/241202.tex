\section{Introduction and Motivation}
\underline{Consumption is the largest part of GDP.}

So, probably utility and welfare depend the most heavily on
consumption, and consumption is a part of a dynamic decision problem.

\section{Benchmark consumption model without uncertainty}

Assume perfect foresight and no uncertainty. Consumers can save
in a risk-free one period bond, the budget constraint in period $t$ is:
\[c_t + b_{t+1} \leq (1+r_t)b_t + y_t. \]

Imagine they can also 'buy' period $t$ consumption in period 0:
the price of period $t$ consumption relative to period 0 consumption, for a given price of perios 0 cunsumption $q_0 > 0$, is:
\[q_t = \frac{q_0}{(1+r_1)(1+r_2)\cdots(1+r_t)}. \]
Using this assumption, we can aggregate the per period budget constraint until $T$:
\[\sum_{t=0}^{T} q_t c_t + q_T b_{T+1} \leq q_0(1+r_0)b_0 + \sum_{t=1}^{T} q_t y_t \]
this inter-temporal BC is equivalent to the stream of per-period BCs.

If the agent dies in period $T$, then it is optimal to have $b_{T+1} = 0$;

or if the agent is immortal, then we impose $T \to \infty$, $q_T b_{T+1} \to 0$.

The inter-temporal BC is given by:
\[\sum_{t=0}^{\infty} q_t c_t \leq q_0 (1+r_0) b_0 + \sum_{t=1}^{\infty} q_t y_t = q_0 ((1+r_0)b_0 + h_0) = q_0 x_0. \]
where:
$(1+r_0)b_0$ is the \textit{financial wealth} of the agent at time 0,

$h_0 = \sum_{t=1}^{T}\frac{q_t}{q_0}y_t$ is the \textit{human wealth} of the agent at time 0,

$x_0$ is the \textit{effective wealth} of the agent at time 0.

The agent’s problem can then be re-written as:
\begin{align*}
    \max_{c_t} &\quad \sum_{t=0}^{T} \beta^t U(c_t) \\
    \text{s.t.} &\quad \sum_{t=0}^{T} q_t c_t \leq q_0 x_0
\end{align*}

\begin{note}
    \

    This is equivalent to cake-eating problem, as $q_0 x_0$ is given at 0,
    due to the fact that we are not looking at general equilibrium, i.e. the consumption-saving choice 
    does not affect the interest rates.

    This can be viewed as a static consumption problem where
    we can interpret $c_t$ as different consumption goods, and $q_t$ as the price of these goods.
\end{note}

\section{Introducing uncertainty}

What is the optimal path of consumption if there is uncertainty
about the realization of income, $y_t$?

We introduce the \underline{idiosymcratic shocks}.

\underline{Some notation:}
\begin{itemize}
    \item $s_t \in S_t$ is the current state of the economy.
    \item $s^t = \{s_0, s_1, s_2, \dots, s_t\}$ is the history of the economy up to time $t$, where $s^t \in S^t = S_0 \times S_1 \times \cdots \times S_t$.
    \item $\pi(s^t)$ is the probability of history $s^t$ occurring.
    \item $y_t^i(s^t)$ is the income of individual $i$ in period $t$ if history $s^t$ occurred,
    and $\sum_{i \in I} y_t^i(s^t) = Y_t(s^t)$ is the aggregate income in the economy upon the realization of $s^t$.
\end{itemize}

\subsection{Benchmark 1 - Autarky}
Assuming in an endowment economy, where agents cannot trade with each other,
and there's no storeage tech to transfer resources across time. ($b_t = 0$)

In such a setup it is optimal for the agent to consume all his
income in each period:
\[c_t^i(s^t) = y_t^i(s^t)\]

\subsection{Benchmark 2 - Complete Markets}
Still in an endowment economy, but with full set of insurance markets, 
where agents can trade assets that pay off one unit of consumption if a
particular history $s^t$ occurs.(\textit{Arrow-Debreu securities})

With $S$ states in each period, $T$ periods, there are $1 + S + S^2 + \cdots + S^T$ markets.

Let $p_t(s^t)$ be the price in terms of period 0 consumption of a unit of income(or consumption,
i.e. the price of the AD-security) in period after history $s^t$.

For every individual $i \in I$, the budget constraint at time 0 is:
\[\sum_{t=0}^{T} \sum_{s^t \in S^t} p_t(s^t) \left( c_t^i(s^t) - y_t^i(s^t) \right) = 0. \]
every transfer of income across states and time is possible, as long
as the expected discounted consumption expenditures equal (or do
not exceed) the expected discounted income for each individual.

\subsection{The AD competitive equilibrium}
The equilibrium is an allocation of consumption and prices of AD-securities:
\begin{align*}
    \{c_t^i(s^t)\}_{t=0}^{T}, \{p_t(s^t)\}_{t=0}^{T}, \forall i \in I, \forall s^t \in S^t
\end{align*}
such that:

given $p_t(s^t)$ for every $i$, $c_t^i(s^t)$ solves the utility maximization problem:
\begin{align*}
    \max_{c_t^i(s^t)} &\quad \sum_{t=0}^{\infty} \beta^t \sum_{s^t} \pi(s^t) U(c_t^i(s^t)) \\
    \text{s.t.} &\quad \sum_{t=0}^{T} \sum_{s^t \in S^t} p_t(s^t) \left( c_t^i(s^t) - y_t^i(s^t) \right) = 0
\end{align*}
while the markets clear:
\[\sum_{i} c_t^i(s^t) = Y_t(s^t) \forall s^t \in S^t, t \geq 0.\]

Take the FOC, we get:
\[\beta^t \pi(s^t) U^{\prime} (c_t^i(s^t)) = \lambda^i p_t(s^t) \forall i, s^t, t.\]

\begin{enumerate}
    \item \underline{Perfect risk sharing:}
    \[ \frac{U^{\prime} (c_t^i(s^t))}{U^{\prime} (c_t^j(s^t))} = \frac{\lambda^i}{\lambda^j} \]
    The relative marginal utility does not depend on state/history.

    Given that
    \[\sum_{i} (U^{\prime} )^{-1} \left( \frac{\lambda^i}{\lambda^j}U^{\prime} (c_t^j(s^t)) \right) = \sum_i c_t^i(s^t) = \sum_i y_t^i(s^t) = Y_t(s^t) \]
    \textit{only aggregate income matters for individual consumption}

    Hence, agents fully insure against idiosyncratic risk (but cannot insure against aggregate risk!)

    \item \underline{Consumption smoothing:}
    \[ \beta ^{t- \tau} \frac{\pi(s^t) U^{\prime} (c_t^i(s^t))}{\pi(s^{\tau} s^tau) U^{\prime} (c_t^j(s^t))} = \frac{p_t(s^t)}{p_{\tau}(s^{\tau})} \]
    Desire to smooth consumption across time and states attenuated by prices and probabilities.
    
    Marginal rate of substitution = Marginal rate of transformation
\end{enumerate}

For conditions of the First Welfare Theorem hold, the competitive equilibrium is Pareto efficient.

Hence, it can be modeled as the outcome of a social planner's problem:
\begin{align*}
    \max_{\{c_t^i(s^t)\}} &\quad \sum_{t=0}^{T} \sum_{s^t} \pi(s^t) \beta^t \pi(s^t) \sum_{i \in I} \alpha^i U(c_t^i(s^t)) \\
    \text{s.t.} &\quad \sum_{i} c_t^i(s^t) = Y_t(s^t) \forall s^t \in S^t, t \geq 0
\end{align*}
yields identical outcome to decentralized Arrow-Debreu model when $\alpha ^i = (\lambda^i)^{-1}. $

\section{empirical implications}
Both autarky and complete markets benchmarks have empirically
testable implications. 

We estimate the following form micro data:

\[\Delta \log c_t^i = \beta_1 \Delta \log C_t + \beta_2 \Delta \log y_t^i + \varepsilon_t\]

Autareky implies that $\beta_1 = 0$, $\beta_2 = 1$.

Arrow-Debreu implies that $\beta_1 = 1$, $\beta_2 = 0$.

Both hypotheses are rejected, we have to look for a model of partial consmuption insurance.

\section{The permanaent income hypothesis}
Assume instead of complete markets that the household can only
trade in a non state-contingent asset, i.e. they can only buy
one-period bonds (which have return $r$).

The consumer has an incentive to hold some bonds in order to
smooth consumption, with the budget constrant:
\[b_t + y_t = c_t + \frac{b_{t+1}}{1+r}\]
which is the 'standard' setup in the RBC model.

Euler equation:
\[u^{\prime} (c_t) = \beta (1+r) \mathbb{E}_t (u^{\prime} (c_{t+1} ))\]
Make two additional assumptions:
\begin{enumerate}
    \item quadratic preferences, $U(c) = a_{1}c - \frac{1}{2}a_2 c^2 $.
    \item the interest rate on the bond equals the inverse of the
    discount rate $\beta (1+r) = 1$.
\end{enumerate}
Then, the Euler equation becomes:
\[a_1 - a_2 c_t = \beta (1+r) \mathbb{E}_t (a_1 - a_2 c_{t+1} ) \Rightarrow \mathbb{E}_t c_{t+1} = c_t \]
which implies that consumption is a random walk.

Law of iterated expectations yields $\mathbb{E}_t c_{t+j} = c_t \forall j \geq 0.$

Using the budget constraint, we can get:
\begin{align*}
    \sum_{j=0}^{\infty }\left(\frac{1}{1+r}\right)^j \mathbb{E}_t c_{t+j} &= b_t + \underset{\equiv h_t}{\underbrace{\sum_{j=0}^{\infty }\left(\frac{1}{1+r}\right)^j \mathbb{E}_t y_{t+j} }} \\
    \Rightarrow c_t &= \frac{r}{1+r}(b_t + h_t)
\end{align*}
consumption today only depends on the annuity value of income,
on the \textit{permanent income}.

\subsection{Consumption dynamics}
\begin{align*}
    \Delta c_t &= c_t - c_{t-1} = c_t - \mathbb{E}_{t-1} c_t \\
    &= \frac{r}{1+r} \left( b_t - \mathbb{E}_{t-1} b_t + \sum_{j=0}^{\infty } \left(\frac{1}{1+r}\right)^j (\mathbb{E}_t y_{t+j} - \mathbb{E}_{t-1}(\mathbb{E}_t y_{t+j} ) ) \right) \\
    &= \frac{r}{1+r} \sum_{j=0}^{\infty } \left(\frac{1}{1+r}\right)^j (\mathbb{E}_t y_{t+j} - \mathbb{E}_{t-1} y_{t+j} )
\end{align*}
the change in consumption between period $t-1$ and t is due to
the revision of permanent income, and it is proportional to the
revision of expected earnings due to information that arrived in the
same period.


\section{Implications}
\begin{enumerate}
    \item \underline{Consumption is a random walk:}
    \[\mathbb{E}_{t-1} \Delta c_t = 0 \]
    there is no insurance, just consumption smoothing.

    \item The marginal propensity to consume from wealth is
    \[\frac{r}{1+r}\]
    the marginal propensity to consume from an innovation to 
    current income depends on the persistence of the income process.

    Lower persistence $\Rightarrow$ smaller reaction in consumption to
    current income changes.
\end{enumerate}