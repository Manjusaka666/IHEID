\section*{Monopolistically competitive firm}
Consider a monopolistically competitive firm with real unit cost of $z_t$ per unit sold at time $t$ (marginal cost is constant at the firm level). The firm (firm $i$) sells output at price $p_t(i)$ and faces the following demand function for its output $y_t(i)$:

\[
y_t(i) = \left(\frac{p_t(i)}{p_t}\right)^{-\varepsilon_t} y_t
\]

where $p_t$ is the general price level, $y_t$ is aggregate output, and $\varepsilon_t$ is the price elasticity of demand. The firm's price $p_t(i)$ implies a (gross) markup of $\mu_t(i) = \frac{p_t(i)}{p_t z_t}$ on marginal cost. Profits (in real terms) made by the firm at time $t$ are therefore given by

\[
\text{Profits}_{i,t} = z_t^{1-\varepsilon_t} y_t \left(\mu_t(i)^{1-\varepsilon_t} - \mu_t(i)^{-\varepsilon_t}\right)
\]

\begin{problem*}[a]
    Show that the profit-maximizing markup for a firm with flexible prices is $\mu_t^* = \frac{\varepsilon_t}{\varepsilon_t - 1}$.
\end{problem*}

\begin{solution}
    \

    Take the first order derivative with respect to $\mu_t(i)$ and set it to zero, we have:
    \begin{align*}
        & \frac{\partial \text{Profits}_{i,t}}{\partial \mu_t(i)} = \frac{\partial}{\partial \mu_t(i)} \left(z_t^{1-\varepsilon_t} y_t \left(\mu_t(i)^{1-\varepsilon_t} - \mu_t(i)^{-\varepsilon_t}\right)\right) = 0 \\
        & \Rightarrow z_t^{1-\varepsilon_t} y_t \left((1-\varepsilon_t) \mu_t(i)^{-\varepsilon_t} + \varepsilon_t \mu_t(i)^{-\varepsilon_t-1} \right) = 0 \\
        & \Rightarrow (1-\varepsilon_t) \mu_t(i)^{-\varepsilon_t} + \varepsilon_t \mu_t(i)^{-\varepsilon_t-1} = 0 \\
        & \Rightarrow \mu_t^* = \frac{\varepsilon_t}{\varepsilon_t - 1}
    \end{align*}
\end{solution}

The firm hires labor in a perfectly competitive market at wage $w_t$ and each unit of labor has
productivity $a_t$. Thus $z_t = w_t/a_t$. The wage is determined by the household labor supply condition
\[w_t = \frac{\nu^{\prime}(l_t)}{u^{\prime} (c_t)}\]
where $l_t$ is labor supply and $c_t$ is consumption.

\begin{problem*}[b]
    Assume that $u(c_t) = 1 - c_t^{-1}$ and $\nu(l_t) = l_t$. Show that the natural level of output $y_t^*$ is
    \[
    y_t^* = \sqrt{\frac{a_t}{\mu_t^*}}
    \]

    and explain why this means that the efficient level of output is $\hat{y}_t = \sqrt{a_t}$.
\end{problem*}

\begin{solution}
    From the household's labor supply condition:
    \[w_t = \frac{\nu ^{\prime} (l_t)}{u^{\prime} (c_t)} = \frac{1}{c_t^{-2}} = c_t^2.\]
    In the flexible price equilibrium, firms set prices to maximize profits, leading to
    \[\mu_t(i) = \frac{\varepsilon_t}{\varepsilon_t - 1} = \mu_t^*.\]
    Hence the price would also be the same: $p_t(i) = p_t^*$ and the markups become:
    \[\mu_t^* = \frac{1}{z_t}.\]
    In a perfectly competitive market, the market clearing condition is: $y_t = c_t$. So, we have:
    \begin{align*}
        & z_t = \frac{w_t}{a_t} = \frac{c_t^2}{a_t} = \frac{y_t^2}{a_t} = \frac{1}{\mu_t^*} \\
        & \Rightarrow y_t^* = \sqrt{\frac{a_t}{\mu_t^*}}
    \end{align*}
    The efficient level of output occurs when there are no distortions, i.e. when the markup is eliminated: $\mu_t^* = 1$.
    Thus, $\hat{y}_t = \sqrt{a_t}.$
    The natural level of output $y_t^*$ is lower than the efficient level $\hat{y}_t$due to the presence of monopolistic competition,
    which introduces a markup over marginal cost($\mu_t^* > 1$). This markup leads to higher prices and reduced output compared to the efficient (perfect competition) case. 
    The efficient level of output is achieved when $\mu_t^* = 1$, eliminating the distortion caused by the markup, and thus maximizing output given the technology level $a_t$.
\end{solution}

When firms can adjust their prices at random staggered intervals (Calvo pricing), inflation is determined by the New Keynesian Phillips curve:
\[
\tilde{\pi}_t = \beta E_t \tilde{\pi}_{t+1} + \kappa \tilde{x}_t + \gamma \tilde{\mu}_t^*
\]
where $\tilde{\pi}_t$ is inflation and $\tilde{x}_t$ is the output gap (defined relative to efficient output). The notation $\tilde{x}_t$ indicates the percentage deviation of $x_t$ from its steady-state value.

\begin{problem*}[c]
    In this environment, should monetary policy aim to stabilize the output gap $\tilde{x}_t$ completely following an unexpected temporary decrease in $a_t$? Why or why not?
\end{problem*}

\begin{solution}
    \

    Yes. The monetary policy should aim to stabilize the output gap.

    $\tilde{x}_t$ is defined as the difference between actual output $y_t$ and efficient level of output $\hat{y}_t$:
    \[\tilde{x}_t = \frac{y_t}{\hat{y}_t}.\]
    Recall $z_t = \frac{w_t}{a_t}$, $w_t = y_t^2$ and $\hat{y}_t = \sqrt{a_t} $,
    we know that:
    \[z_t = \tilde{x}_t^2\]
    As $a_t$ shock is exogenous, it won't affect the markup term $\tilde{\mu}_t^*$, so just cancel it out in
    this case, we have:
    \[\tilde{\pi}_t = \beta \mathbb{E}_t \tilde{\pi}_{t+1} + \kappa \tilde{x}_t \]
    
    If we stabilize the output gap completely in response to a temporary decrease in $a_t$, we have:
    \[\tilde{x}_t = 0 \Rightarrow \tilde{\pi}_{t+1} = 0 \Rightarrow \tilde{\pi}_t = \beta \mathbb{E}_t \tilde{\pi }_{t+1} = 0\]
    
    In this case, output is kept equal to the flexible-price equilibrium level of output. 
    This also guarantees inflation equals zero, thereby eliminating the costly dispersion of relative prices that arises with inflation. 
    When firms do not need to adjust their prices, the fact that prices are sticky is no longer relevant. 

\end{solution}

Now suppose that markets become more competitive when output is higher (perhaps because of
more competition for new customers). In particular, assume that $\varepsilon_t = 1 + \frac{1}{2}y_t$. Price adjustment is
staggered according to the Calvo model.

\begin{problem*}[d]
    Derive an expression for the desired flexible-price markup $\mu_t^*$ in terms of the output gap $x_t \equiv y_t / \hat{y}_t$ and productivity $a_t$. 
    (Hint: the efficient level of output is still $\hat{y}_t = \sqrt{a_t}$.) 
    Find a log-linear approximation of this equation in terms of $\tilde{\mu}_t^*$, $\tilde{a}_t$, 
    and $\tilde{x}_t$, and interpret it.
\end{problem*}

\begin{solution}
    As given, $\varepsilon_t = 1 + \frac{1}{2}y_t$. The profit-maximizing markup is $\mu_t^* = \frac{\varepsilon_t}{\varepsilon_t - 1}$.
    So, we have:
    \[\mu_t^* = \frac{1 + \frac{1}{2}y_t}{\frac{1}{2}y_t} = 1 + \frac{2}{y_t} = 1 + \frac{2}{x_t \sqrt{a_t} }.\]
    We also know that $y_t = x_t \hat{y}_t = x_t \sqrt{a_t}.$
    Substitute this into the equation, we have:
    \[\mu_t^* = 1 + \frac{2}{y_t} \Rightarrow y_t \mu_t^* = y_t + 2.\]
    
    Take the log-linear approximation, we get:
    \begin{align*}
        \tilde{y}_t & \approx \tilde{x_t} + \frac{1}{2}\tilde{a}_t \\
        \tilde{y}_t + \tilde{\mu}_t^* & \approx \frac{\bar{y}}{2 + \bar{y}}\tilde{y}_t \\
        \Rightarrow \tilde{\mu}_t^* & \approx -\frac{2}{2 + \bar{y}} \left(\tilde{x}_t + \frac{1}{2} \tilde{a}_t\right) \\
        &= -\frac{2}{2 + \sqrt{\bar{a}}} \left(\tilde{x}_t + \frac{1}{2} \tilde{a}_t\right) \\
        & \approx -\tilde{x}_t - \frac{1}{2} \tilde{a}_t
    \end{align*}
    This indicates that the desired markup $\tilde{\mu}_t^*$ decreases 
    when the output gap increases. 
    This reflects the assumption that markets become more competitive 
    when output is higher.

    Higher productivity can enhance competition by enabling firms to produce more efficiently, 
    which may encourage them to lower prices to gain market share.
\end{solution}

\begin{problem*}[e]
    In this environment, should monetary policy aim to stabilize the output gap $\tilde{x}_t$ completely following an unexpected temporary decrease in $a_t$? 
    Why or why not?
\end{problem*}

\begin{solution}
    \
    
    No, in this case, the monetary policy should not aim to stabilize the output gap completely following an unexpected temporary decrease in $a_t$.

    As we have a negative coefficient, if we stabilize the output gap completely in response to a temporary decrease in $a_t$, we have:
    \[\tilde{\mu}_t^* = -\frac{1}{2} \tilde{a}_t > 0.\]
    Then, if we still expect the future inflation to be zero, we have:
    \[\tilde{\pi}_t = \beta \mathbb{E}_t \tilde{\pi}_{t+1} + \kappa \tilde{x}_t + \gamma \tilde{\mu}_t^* = -\frac{\gamma}{2}\tilde{a}_t > 0. \]
    This shows that we have a tradeoff between en stabilizing inflation and closing the output gap.
\end{solution}

\begin{problem*}[f]
    Compare the size of the response of inflation $\tilde{\pi}_t$ to the output gap $\tilde{x}_t$ 
    in the two cases where the price elasticity $\varepsilon_t$ is exogenous and where it depends positively on output $y_t$. 
    Assuming the central bank is minimizing the same standard loss function in both cases, 
    what implications does the dependence of $\varepsilon_t$ on $y_t$ have for the optimal balance 
    between stabilizing inflation and stabilizing the output gap 
    (when shocks shift the New Keynesian Phillips curve)? Why?
\end{problem*}

\begin{solution}
    \
    
    With exonogenous $\varepsilon_t$, we have:
    \[\tilde{\pi}_t = \beta \mathbb{E}_t \tilde{\pi}_{t+1} + \kappa \tilde{x}_t.\]
    With $\varepsilon_t = 1 + \frac{1}{2}y_t$, we have:
    \[\tilde{\pi}_t = \beta \mathbb{E}_t \tilde{\pi}_{t+1} + \kappa \tilde{x}_t + \gamma (-\tilde{x}_t - \frac{1}{2} \tilde{a}_t) = \beta \mathbb{E}_t\tilde{\pi}_{t+1} + (\kappa - \gamma)\tilde{x}_t -\frac{1}{2} \gamma \tilde{a}_t. \]
    Thus, the response of inflation to the output gap is smaller when $\varepsilon_t$ depends positively on output $y_t$.

    Assuming the loss function is:
    \[L_t = \frac{1}{2} \left(\tilde{\pi}_t^2 + a \tilde{x}_t^2\right).\]
    The optimal balance problem is:
    \begin{align*}
        \min_{\pi_t, x_t} &\quad L_t = \frac{1}{2} (\tilde{\pi}_t^2 + a\tilde{x}_t^2) \\
        s.t. & \quad \tilde{\pi}_t = \beta \mathbb{E}_t \tilde{\pi}_{t+1} + \kappa \tilde{x}_t + \gamma (-\tilde{x}_t - \frac{1}{2} \tilde{a}_t) \\
        & \quad \tilde{\pi}_t = \beta \mathbb{E}_t \tilde{\pi}_{t+1} + \kappa \tilde{x}_t
    \end{align*}

    Take the first order condition with respect to $\tilde{\pi}_t$ and $\tilde{x}_t$, we have:
    \begin{align*}
        \tilde{\pi}_t &= -\frac{a}{\kappa} \tilde{x}_t \quad \text{under exogenous $\varepsilon_t$}; \\
        \tilde{\pi}_t &= -\frac{a}{\kappa - \gamma} \tilde{x}_t \quad \text{under $\varepsilon_t = 1 + \frac{1}{2}y_t$}.
    \end{align*}
    So, the optimal balance between stabilizing inflation and stabilizing the output gap is affected by the dependence of $\varepsilon_t$ on $y_t$.
    When $\varepsilon_t$ depends positively on output $y_t$, the inflation rate is more volatile to the output gap. 
    The monetary policy will prioritize inflation control over output gap stabilization.
\end{solution}

\section*{Short Questions}

\begin{problem*}[2]
    Explain the channel system of conducting monetary policy, and illustrate its advantages and disadvantages.
\end{problem*}

\begin{solution}
    \

    In a channel system, the central bank sets lower and upper bounds 
    for the interest rate. 
    \begin{enumerate}
        \item \textbf{Deposit Facility Rate (Floor):} The interest rate the central bank pays on excess reserves deposited by commercial banks.
        \item \textbf{Lending Facility Rate (Ceiling):} The interest rate at which the central bank lends to commercial banks (typically through overnight loans).
    \end{enumerate}
    The target policy rate is kept within this corridor, 
    and the central bank uses open market operations to manage 
    liquidity and guide the market rate towards the target.

    \textbf{Advantages:}
    \begin{itemize}
        \item \textbf{Interest Rate Control:} The corridor provides clear boundaries for short-term interest rates, enhancing the central bank's ability to control market rates.
        \item \textbf{Liquidity Management:} By providing standing facilities, banks have predictable access to liquidity, reducing uncertainty in interbank markets.
        \item \textbf{Stability:} The system can mitigate volatility in interest rates by absorbing shocks through the corridor's floor and ceiling rates.
    \end{itemize}

    \textbf{Disadvantages:}
    \begin{itemize}
        \item \textbf{Interest Rate Volatility:} If the corridor is wide, market rates can fluctuate significantly within it, potentially leading to uncertainty.
        \item \textbf{Ineffective Transmission:} In times of excess reserves (e.g., during quantitative easing), the floor may become the de facto policy rate, weakening the central bank's control over other interest rates.
        \item \textbf{Dependency on Market Operations:} The system requires active management of liquidity through frequent open market operations, which can be resource-intensive.
    \end{itemize}
\end{solution}

\begin{problem*}[3]
    Discuss whether (and in which way) optimal monetary policy with commitment
    is preferable over discretionary optimal monetary policy. 
    Which assumption(s) of the New Keynesian model generate the dynamic 
    externality that drives the differences between discretion and commitment?
\end{problem*}

\begin{solution}
    \

    \begin{itemize}
        \item \textbf{Commitment Policy:}
        \begin{itemize}
            \item Under commitment, the central bank commits to future policy actions, influencing expectations and economic outcomes today.
            \item It can implement time-inconsistent policies that are optimal in the long run but may not be optimal in the short run.
        \end{itemize}
        \item \textbf{Discretionary Policy:}
        \begin{itemize}
            \item The central bank optimizes policy period by period without committing to future actions.
            \item This can lead to suboptimal outcomes due to the inability to influence expectations effectively.
        \end{itemize}
    \end{itemize}
    If we suppose a timporary positive cost-push shock at time $t=0$, then for the commitment policy, the central bank will increase the interest rate to reduce inflation, while for the discretionary policy, 
    $x_{t+j}$ would be negative for a while and the inflation expectations will fall $\Rightarrow$ 
    $\pi_0$ increases by less than under discretion without large decrease in $x_0$. 
    
    Hence, outcome leads to higher welfare.    
    % \textbf{Preferability of Commitment:}
    % \begin{itemize}
    %     \item \textbf{Improved Credibility and Expectations Management:}
    %     \begin{itemize}
    %         \item Commitment enhances the central bank's credibility, leading to better-anchored inflation expectations.
    %         \item It reduces the inflation bias and can achieve lower inflation without sacrificing output.
    %     \end{itemize}
        
    %     \item \textbf{Dynamic Consistency:}
    %     \begin{itemize}
    %         \item Commitment policies are dynamically consistent, aligning short-term actions with long-term objectives.
    %         \item They can mitigate the time inconsistency problem inherent in discretionary policies.
    %     \end{itemize}
    % \end{itemize}
    
    \textbf{Assumptions Generating the Dynamic Externality:}
    \begin{itemize}
        \item \textbf{Forward-Looking Expectations:}
        \begin{itemize}
            \item In the New Keynesian model, prices and wages are set based on expectations of future economic conditions.
            \item This creates a dynamic externality where current policy affects future expectations and, therefore, current economic outcomes.
        \end{itemize}
        
        \item \textbf{Sticky Prices and Wages:}
        \begin{itemize}
            \item Price and wage rigidities cause adjustments to be gradual, making expectations about future policy more impactful.
            \item The slow adjustment amplifies the importance of commitment in shaping expectations.
        \end{itemize}
        
        \item \textbf{Monopolistic Competition and Price Setting:}
        \begin{itemize}
            \item Firms set prices considering future marginal costs and the anticipated policy path.
            \item Commitment allows the central bank to influence these expectations, improving allocation efficiency.
        \end{itemize}
    \end{itemize}
\end{solution}