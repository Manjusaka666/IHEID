The NK model takes an RBC model as its backbone and adds some nominal rigidities - 
here, we will add sticky prices - which allows shocks to affect the real economy.

But price stickiness only makes sense when firms can set prices;
hence, need some degree of market power. Special case of
Rotemberg \& Woodford.

Nominal rigidities: Calvo pricing: firm-specific shock that tells
you whether you're allowed to reset your price.
\footnote{Taylor's (1980) overlapping-contracts model adopts such a view of nominal
rigidities and assumes that each price is fixed for some number n periods and therefore each
period $1/n$ of the prices in the economy is updated. This assumption requires the modeler to
keep track of n different prices. A simpler approach is to assume that a fraction $1-\theta \in [0, 1]$
of the prices in the economy is updated each period, but unlike Taylor's model, the prices
that are updated each period are randomly drawn from the existing prices. This assumption,
which was introduced by Calvo (1983), leads to a much more tractable model because it is
no longer necessary to keep track of the previously set prices as we will explain below.\\
We'll discuss this later.}

\section{Households and Consumption}
For representative household, they maximize the utility:
\begin{align*}
    \max &\quad \mathbb{E}_0 \sum_{i=0}^\infty \beta^i \left[ u(c_{t+1}) - \nu(l_{t+i} ) \right] \\
    \text{s.t.} &\quad p_tc_t + b_t = (1+i_{t-1})b_{t-1} + p_t w_t l_t + p_t d_t
\end{align*}
where

$c_t$ is quantity of baskets of goods consumed, $p_t$ is the prive of each basket in terms of money,

$d_t$ is the real wage, $d_t$ are dividends(firms' profits) received from owning firms,

$i_t$ is the nominal interest rate, 

$b_t$ is the quantity of nominal bonds held by the household.

The intertemporal optimality condition is as follows:
\begin{align*}
    u'(c_t) = \beta \mathbb{E}_t(1+r_t)u'(c_{t+1})
\end{align*}
where $r_t$ is the real interest rate given by: $1 + r_t = \frac{1+i_t}{1+\pi_{t+1} }$
and $\pi_{t+1} = \frac{p_{t+1}-p_t}{p_t}$ is the inflation rate between $t$ and $t+1$.

Labor supply is given by:
\begin{align*}
    MRS_{l,c} = \frac{\nu'(l_t)}{u'(c_t)} = w_t
\end{align*}

Introduce the basket of imperfectly substitutable goods
through a consumption aggregator, a description of consumer
preferences over the whole set of goods (indexed by $i$).
Formula linking consumption $c(i)$ of each good $i \in [0, 1]$ to
an 'amount' of total baskets c enjoyed by the household.

The most common aggregator used is the CES(constant
elasticity of substitution; also called Dixit-Stiglitz):
\[ 
c = \left(\int c(i)^{\frac{\varepsilon-1}{\varepsilon}} di\right)^{\frac{\varepsilon}{\varepsilon-1}}
\]
where $\varepsilon$ is the elasticity of substitution between goods.

\subsection{Expenditure minimization}
Suppose that the household faces prices $p(i)$. The expenditure minimization problem is:
\begin{align*}
    \min &\quad \int p(i) c(i) di \\
    \text{s.t.} &\quad \left(\int c(i)^{\frac{\varepsilon-1}{\varepsilon}} di\right)^{\frac{\varepsilon}{\varepsilon-1}} = c
\end{align*}
Define the Lagrangian:
\begin{align*}
    \mathcal{L} = \int p(i) c(i) di + \lambda \left( c - \left(\int c(i)^{\frac{\varepsilon-1}{\varepsilon}} di\right)^{\frac{\varepsilon}{\varepsilon-1}} \right)
\end{align*}
The first order conditions are:
\begin{align*}
    \frac{\partial \mathcal{L}}{\partial c(i)} &= p(i) - \lambda c(i)^{-\frac{1}{\varepsilon}}\left(\int c(i)^{\frac{\varepsilon-1}{\varepsilon}} di\right)^{\frac{\varepsilon}{\varepsilon-1}-1} = 0 \\
    \Rightarrow p(i) &= \lambda c(i)^{-\frac{1}{\varepsilon}}c^{\frac{1}{\varepsilon}}
\end{align*}
From the first order conditions we get:
\begin{align*}
    c(i) = c \left( \frac{p(i)}{\lambda} \right)^{-\varepsilon}
\end{align*}
Substitute this into the FOC to get:
\begin{align*}
    c &= \left( \int \left(\frac{p(i)}{\lambda }\right)^{1-\varepsilon} c^{\frac{\varepsilon-1}{\varepsilon}} di \right)^{\frac{\varepsilon}{\varepsilon-1}} \\
    \lambda &= \left( \int p(i)^{1-\varepsilon} di \right)^{\frac{1}{1-\varepsilon}}
\end{align*}
Interpret $\lambda$ as the shadow price:'If you tighten the
constraint by one unit, the objective function increases by $\lambda$ units.'
$\lambda$ is the price of one basket, $p_t = \lambda$.

Hence, the demand for good $i$ is:
\begin{align*}
    c(i) = c \left( \frac{p(i)}{p} \right)^{-\varepsilon}
\end{align*}
where
\begin{align*}
    p = \left( \int p(i)^{1-\varepsilon} di \right)^{\frac{1}{1-\varepsilon}}
\end{align*}
Interpretation: demand depends on the relative price $p(i)/p$.
Price elasticity of demand is now constant $\varepsilon$.

\subsection{Production technology and market clearing}
A typical intermediate firm produces output according a constant returns to scale technology in labour,
with a common productivity shock, $a_t$:
Output $y(i)$ is given by production function:
\begin{align*}
    y_t(i) = a_t h_t(i)
\end{align*}
where

$a_t$ is the total factor productivity,

$h_t(i)$ is the hours of labor used in production of good $i$.

\subsubsection{\textcolor{blue}{\textbf{Market Clearing}}}

There are three markets at each date: \textit{the labor market, the goods
market, and the bond market.}

\underline{\textbf{Labor market}} clearing requires that the total labor supplied
by households equals the total labor used in production.
Interpret $l_t$ as average labor supplied per good, the labor market clearing:
\begin{align*}
    l_t = \int h_t(i) di
\end{align*}

\underline{\textbf{Goods market}} clearing assumes that there's no investment of gov spending,
and as consumption is the only use for final goods, 
goods market clearing requires that production equals total consumption.
\begin{align*}
    y_t(i) = c_t(i), \quad c_t = y_t
\end{align*}

\underline{\textbf{Bond market}} clearing requires that the net demand for bonds is zero.


\subsubsection{Demand, revenues and costs}

Firm $i$ faces demand function:
\begin{align*}
    y_t(i) = c_t(i) = y_t \left( \frac{p(i)}{p} \right)^{-\varepsilon}, \quad p_t = \left( \int p_t(i)^{1-\varepsilon} di \right)^{\frac{1}{1-\varepsilon}}
\end{align*}
Firm $i$ has the power to set price $p_t(i)$, but it is too small to affect
the general price level $p_t$ or aggregate demand $y_t$ (monopolistic
competition).

In a competitive market, firms can hire labor at the wage rate $w_t$, let $z_t$ denote the real cost of production
per unit of output. With linear production function for $y(i)$, we have\footnote{The cost minimisation problem is to minimise
total cost subject to the constraint producing enough to meet demand (see Blanchard and Kiyotaki (1987) for the derivation of this problem):
\begin{align*}
    \min_{h_t(i)} & \quad w_t h_t(i) \\
    \text{s.t.} & \quad  a_t h_t(i) \geq y_t(i) = \left(\frac{p_t(i)}{p_t}\right)^{-\varepsilon}y_t
\end{align*}
define the Lagranrian(use $z_t$ to replace $\lambda$) and take the FOC, we'll have:
\[\pd{\mathcal{L}}{h_t(i)} = w_t - z_t a_t = 0\]}:
\[
z_t = \frac{w_t}{a_t}
\]
with total revenue: $\frac{p_t(i)}{p_t} y_t(i)$, total real cost of production: $z_t y_t(i)$.

\subsubsection{Profits}

The profits(in real terms) made by firm $i$ at time $t$ is:
\[
d_t(i) = \frac{p_t(i)}{p_t} y_t(i) - z_t y_t(i) = \left(\frac{p_t(i)}{p_t}-z_t\right)\left(\frac{p_t(i)}{p_t}\right)^{-\varepsilon}y_t
\]
Let $\mu_t(i) = \frac{p_t(i)}{p_t z_t}$ be the firm $i$'s (gross) markup of price over marginal cost.

The expression for profits can be rewritten as:
\[
d_t(i) = z_t^{1-\varepsilon}y_t (\mu_t(i)^{1-\varepsilon} - \mu_t(i)^{-\varepsilon})
\]

Let's first consider a world with flexible prices. To maximize profits, firm $i$ chooses $\mu_t(i)$ to solve:
\[
\max_{\mu_t(i)} \quad d_t(i) = z_t^{1-\varepsilon}y_t (\mu_t(i)^{1-\varepsilon} - \mu_t(i)^{-\varepsilon})
\]
The first order condition is:
\[
\frac{\partial d_t(i)}{\partial \mu_t(i)} = z_t^{1-\varepsilon}y_t ((1-\varepsilon)\mu_t(i)^{-\varepsilon} + \varepsilon \mu_t(i)^{-\varepsilon-1}) = 0 \Rightarrow \mu_t^* = \frac{\varepsilon}{\varepsilon-1}
\]
So, a higher price elasticity $\varepsilon$ will cause a lower markup.

\subsection{General Equilibrium with flexible prices}

Assuming the following utility function:
\[
u(c_t) = \frac{c_t^{1-\frac{1}{\sigma}}}{1-\frac{1}{\sigma}}, \quad \nu(l) = l
\]
The labor supply condition is:
\[
w_t = \frac{\nu^{\prime}(l_t)}{u^{\prime}(c_t)} = y_t^{\frac{1}{\sigma}}
\]
Assuming the market clearing condition holds, $c_t = y_t$.
The firm's profit maximization implies that 
\[
w_t = \frac{a_t}{\mu_t^*}.
\]
\textcolor{red}{Wage equals to the marginal revenue product of capital.}

\subsection{Flexible-price world: natural level of output}
With flexible prices, all firms set the samse markup: $\mu_t(i) = \mu_t^*$.
Hence the price would also be the same: $p_t(i) = p_t^*$ and the markups
become:
\[
\mu_t^* = \frac{1}{z_t}
\]
The \underline{\textcolor{blue}{natural level of output}} is the level of output that would be produced if all prices were flexible.

Equating labor supply and labor demand with flexible prices:
\[
y_t^{\frac{1}{\sigma}} = w_t = \frac{a_t}{\mu_t^*}
\]
hence the natural level of output $y_t^*$ is:
\[
y_t^* = \left( \frac{a_t}{\mu_t^*} \right)^{\sigma}
\]
\begin{itemize}
    \item The natural output is higher when productivity $a_t$ is higher.
    \item The natural output is lower when the markup(market power) $\mu_t^*$ is higher.
\end{itemize}

\subsection{Natural real interest rate}
The \underline{\textcolor{blue}{natural real interest rate}} is the real interest rate consistent with the natural level of output.

This requires $c_t = y_t = y_t^*$, which implies by Euler equation:
\[
u^{\prime} (y_t^*) = \beta (1+r_t^*)u^{\prime} (y_{t+1}^*)
\]
with $r_t^*$ being the natural real interest rate.

Using $u^{\prime}(c) = c^{-\frac{1}{\sigma}}$, we get:
\begin{align*}
    \left(y_t^{*}\right)^{-\frac{1}{\sigma}} &= \beta (1+r_t^*)\left(y_{t+1}^*\right)^{-\frac{1}{\sigma}}\\
    \frac{\mu_t^*}{a_t} &= \beta (1+r_t^*)\frac{\mu_{t+1}^*}{a_{t+1}}
\end{align*}
The natural interest rate is given by:
\[
1 + r_t^* = (1+\bar{r}) \frac{a_{t+1}}{a_t} \frac{\mu_t^*}{ \mu_{t+1}^*}
\]
where $\bar{r}$ is the average interest rate in the steady state with $\beta = \frac{1}{1+\overline{r}}$.

\subsection{Efficiency of output}

Imperfect competition makes market allocation of output not Pareto efficient.

Perfect competition is consistent with efficiency, so we can ask what level of output is efficient by 
finding the equilibrium in the special case of perfect competition:
\begin{itemize}
    \item perfect competition: perfectly elastic demand curve because of perfect substitutability(no market power): $\varepsilon \to \infty$
    \item zero markup: $\mu_t^* \to 1$.
\end{itemize}

\textbf{Efficient output} $\hat{y}_t$ found using formula for natural output, and setting $\mu_t^* = 1$:
\[
\hat{y}_t = a_t^{\sigma}
\]

\textbf{Efficient interest rate} $\hat{r}_t$ is the real interest rate consistent with output as its efficient level:
\[
1 + \hat{r}_t = (1+\bar{r}) \frac{a_{t+1}}{a_t}.
\]

\section{Nominal Rigidities}

\begin{assumption}[Calvo pricing, Calvo and Yun]
    \ 

    Each period, a random selected fraction of $1-\phi$ of firms has an opportunity to reset its price.
    All other firms must continue to use the same price as before.

    This is a simple idea in a complicated approach:
    \begin{itemize}
        \item firms face a fixed physical cost of adjusting prices
        \item need to calculate the points in time when gains from adjusting prices exceed the costs
        \item hard to do in a dynamic model
    \end{itemize}
\end{assumption}

\subsection{Setting a new price}

Suppose firm $i$ set the new price $s_t$ at time $t$.
In each subsequent period, probability $\phi $ that the price won't be changed.
Hence, the probability that the price is still used at time $T>t$ is $\phi^{T-t}$.

The expected present value of profits from setting price $s_t$ at time $t$ is:
\[
d_t(i) + \phi \frac{d_{t+1}(i)}{1+r_t} + \phi^2 \frac{d_{t+2}(i)}{(1+r_t)(1+r_{t+1})} + \ldots
\]
where dividends(=profits) $d_t, d_{t+1}, \ldots$ are calculated assuming the price remains 'sticky' at $s_t$:
$p_t = p_{t+1} = \ldots = s_t.$

From the firms' profits equation, we have:
\[
d_t(i) = \left(\frac{p_t(i)}{p_t} - z_t\right)\left(\frac{p_t(i)}{p_t}\right)^{-\varepsilon}y_t
\]
we get the objective function to maximize:
\[
\max \quad \mathbb{E}_t \sum_{i=0}^{\infty} \left(\prod_{j=1}^{i}\frac{1}{1+r_{t+j-1} }\right) \phi^{i} \left(\frac{p_t(i)}{p_{t+i}} - z_{t+i}\right)\left(\frac{p_t(i)}{p_{t+i}}\right)^{-\varepsilon}y_{t+i}
\]
the first order condition with respect to $\varepsilon$ is:
\[
\mathbb{E}_t \sum_{i=0}^{\infty} \left(\prod_{j=1}^{i}\frac{1}{1+r_{t+j-1} }\right) \phi^{i} \left((1-\varepsilon)\frac{p_t(i)}{p_{t+i}} + \varepsilon z_{t+i}\right) \frac{1}{p_t(i)} \left(\frac{p_t(i)}{p_{t+i}}\right)^{-\varepsilon}y_{t+i} = 0
\]
plug in the Euler equation to get rid of the interest rates:
\[
\mathbb{E}_t \sum_{i=0}^{\infty} y_{t+i}^{1+\frac{1}{\sigma}}  \beta^{i} \phi^{i} \left((1-\varepsilon)\frac{p_t(i)}{p_{t+i}} + \varepsilon z_{t+i}\right) \frac{1}{p_t(i)} \left(\frac{p_t(i)}{p_{t+i}}\right)^{-\varepsilon}= 0
\]
The optimal reset price $p_t(i)$ doesn't depend on the current price level $p_t$, denote by $s_t$ for simplicity.
We have:
\[
\mathbb{E}_t \sum_{i=0}^{\infty} y_{t+i}^{1+\frac{1}{\sigma}} (\beta\phi)^{i} \left(\frac{s_t}{p_{t+i}}\right)^{1-\varepsilon} = \frac{\varepsilon}{1-\varepsilon} \mathbb{E}_t \sum_{i=0}^{\infty} y_{t+i}^{1+\frac{1}{\sigma}} (\beta\phi)^{i} z_{t+i} \left(\frac{s_t}{p_{t+i}}\right)^{-\varepsilon}
\]
then log-linearize the above equation to get:
\[
\tilde{s_t} = (1 - \beta \phi)\mathbb{E}_t\left(\sum_{i=0}^{\infty}(\beta \phi)^i(\tilde{p}_{t+i}+\tilde{z}_{t+i}) \right)
\]

\begin{note}
    \ 

    This indicates that firms set price in accordance with current and expected future price level and cost deviations.
\end{note}

Write the equation for time $t+1$ and subtract from this equation to get:
\[
\tilde{s_t} - \beta \phi\mathbb{E}_t \tilde{s}_{t+1} = (1-\beta \phi)(\tilde{p}_t + \tilde{z}_t)
\]

\subsection{Overall price level}
Recall the overall price level:
\[
p_t = \left( \int p_t(i)^{1-\varepsilon} di \right)^{\frac{1}{1-\varepsilon}}
\]
The log-linearized version\footnote{We define $\tilde{x} = \frac{x_{t}-x^*}{x_t}$ for any variable $x$ as the percent deviation under log-linearization.} of the above equation is:
\[
\tilde{p}_t = \int \tilde{p}_t(i)di
\]
How old are the prices of the firms?
\begin{itemize}
    \item $1-\phi$ reset prices at time $t$,
    \item $\phi(1-\phi)$ reset prices at time $t-1$,
    \item $\phi^2(1-\phi)$ reset prices at time $t-2$, etc.
\end{itemize}
Hence,
\[
\tilde{p_t} = (1-\phi)\tilde{s}_t + \phi(1-\phi)\tilde{s}_{t-1} + \phi^2(1-\phi)\tilde{s}_{t-2} + \ldots
\]
Write for $t-1$ and subtract, we get:
\[\tilde{p_t} - \phi \tilde{p}_{t-1} = (1-\phi)\tilde{s_t}\]

\begin{note}
    \ 

    The price level today is a weighted average of past price level and the reset prices.
\end{note}

Log-linearize the inflation rate:
\[\tilde{\pi_t} = \tilde{p_t} - \tilde{p}_{t-1}\]
plug in the above equation, we get:
\[\phi \tilde{\pi _t} = (1-\phi)(\tilde{s_t} - \tilde{p_t}).\]

Taking the firm's FOC,
\[\tilde{s_t} = (1-\beta \phi )(\tilde{p_t} + \tilde{z_t}) + \beta \phi \mathbb{E}_t \tilde{s}_{t+1} \]
subtract $\tilde{p_t}$ and times $1-\phi $, we get:
\[(1-\phi )(\tilde{s_t} - \tilde{p_t}) = (1-\beta \phi)(1-\phi)\tilde{z_t} + \beta \phi (1-\phi)(\mathbb{E}_t \tilde{s}_{t+1} - \tilde{p_t}) \]
as we assumed prices are sticky, $\tilde{p_t} = \tilde{p}_{t+1} $, we have:
\[\phi \tilde{\pi _t} = \beta \phi \mathbb{E}_t \tilde{\pi}_{t+1} + (1-\beta \phi)(1-\phi)\tilde{z_t}\]
so that
\[\tilde{\pi _t} = \beta \mathbb{E}_t \tilde{\pi}_{t+1} + \underset{\equiv \gamma}{\underbrace{\frac{(1-\beta \phi)(1-\phi)}{\phi}}} \tilde{z_t}\protect\label{NKPC1}.\]

This expression is called the \textcolor{blue}{\textit{'New Keynesian Phillips Curve' (NKPC)}}.
It is 'new' because it is forward-looking unlike classic Phillips Curves, 
but it's a Phillips Curve in the sense that it captures a relationship between inflation and some real measure. 

\subsection{Output gap}
We define the output gap as the deviation of actual output from its efficient level:
\[x_t \equiv \frac{y_t}{\hat{y_t}}.\]

\begin{remark}
    \ 

    Even in the nonstochastic steady state, $y_t < \hat{y_t}$. Why?

    because output responds significantly less than the flexible price level of output to the productivity shock, 
    we see a large negative output gap opening up following the positive productivity shock.
\end{remark}

Recall the following:
\begin{itemize}
    \item Real marginal cost: $z_t = \frac{w_t}{a_t}$;
    \item Labor supply: $w_t = y_t^{\frac{1}{\sigma}}$;
    \item Efficient output: $\hat{y_t} = a_t^{\sigma}$;
\end{itemize}
We can get:
\begin{align*}
    z_t &= \left(\frac{y_t}{\hat{y_t}}\right)^{\frac{1}{\sigma}} = x_t^{\frac{1}{\sigma}} \\
    \Rightarrow \tilde{z_t} &= \frac{1}{\sigma} \tilde{x_t}
\end{align*}

We can re-write the NKPC(\ref{NKPC1}) in terms of the output gap, we get:
\[\tilde{\pi _t} = \beta \mathbb{E}_t \tilde{\pi}_{t+1} + \underset{\equiv \kappa}{\underbrace{\frac{\gamma}{\sigma}}} \tilde{x_t} \protect\label{NKPC2}.\]
In our derivation we assumed that $\varepsilon$ is constant.

\subsection{The New Keynesian model with Rotemberg (1982) pricing}
If instead we make $\varepsilon$ stochastic and allow shocks to it(as done by Rotemberg 1982),
would have shocks to markups even in a frictionless world:
\[\mu_t^* = \frac{\varepsilon_t}{\varepsilon_t - 1}.\]
Then, we use $\tilde{z_t} + \tilde{\mu_t^*}$ to replace $\tilde{z_t}$, and we get:
\begin{equation}\label{NKPC}
    \tilde{\pi _t} = \beta \mathbb{E}_t \tilde{\pi}_{t+1} + \kappa \tilde{x_t} + \underset{\equiv e_t}{\underbrace{\gamma \tilde{\mu_t^*}}}.
\end{equation}

\section{The New Keynesian Phillips curve}
The New Keynesian Phillips curve is a forward-looking relationship between inflation and the output gap.
Inflation is forward-looking and increasing in the current output gap.

The slope of the Phillips curve, $\kappa$, is larger if prices are more
flexible (lower $\phi$) or if marginal cost is more sensitive to the level of production.

\begin{itemize}
    \item Expected future inflation $\beta \mathbb{E}_t \tilde{\pi}_{t+1}$: firms try to anticipate future
    inflation and increase current price in line with expected future inflation;
    \item Current output gap $\kappa\tilde{x_t}$: higher real marginal cost means higher prices, $\phi =0$ means $\kappa \to \infty$,
    gives a vertical NKPC, $\phi = 1$ means $\kappa = 0$, gives a horizontal NKPC;
    \item 'Cost-push shock' $e_t$: shock to magnitude of markups over real marginal cost.
\end{itemize}

\subsection{The IS equation}
The NKPC explains inflation in terms of the output gap, but how do we find the output gap?

Take consumption Euler equation with market clearing condition($c_t = y_t$):
\begin{align*}
    y_t^{-\frac{1}{\sigma}} &= \beta \mathbb{E}_t \left(y_{t+1}^{-\frac{1}{\sigma}}(1+r_{t})\right) \\
    \Rightarrow y_t^{-\frac{1}{\sigma}} &= \mathbb{E}_t \left(y_{t+1}^{-\frac{1}{\sigma}} \frac{(1+r_{t})}{1+\bar{r}}\right)
\end{align*}
where $\beta  = \frac{1}{1+\bar{r}}$, log-linearize it, we get:
\begin{align*}
    -\frac{1}{\sigma}\log y_t &= \mathbb{E_t}\left(-\frac{1}{\sigma}\log y_{t+1} + \log{(1+r_t)} - \log{(1+\bar{r})} \right) \\
    \Rightarrow \tilde{y_t} &= \mathbb{E}_t \left(\tilde{y}_{t+1} - \sigma \tilde{r_t} \right)
\end{align*}

Plug in approximate Fisher equation $\tilde{r_t} = \tilde{i_t} - \tilde{\pi}_{t+1}$ to get:
\[\tilde{y_t} = \mathbb{E}_t \left(\tilde{y}_{t+1} - \sigma (\tilde{i_t} - \tilde{\pi}_{t+1} ) \right)\]
Similarly, for the efficient level of output:
\[\tilde{\hat{y_t}} = \mathbb{E}_t \left(\tilde{\hat{y}}_{t+1} - \sigma \tilde{\hat{r_t}}\right).\]
Use the definition of the output gap: $\tilde{x_t} = \tilde{y_t} - \tilde{\hat{y_t}}$, we have:
\begin{equation*}\label{NKIS}
    \tilde{x_t} = \mathbb{E}_t \left(\tilde{x}_{t+1} - \sigma (\tilde{i_t} - \tilde{\pi}_{t+1}) \right) + \underset{\equiv v_t}{\underbrace{\sigma \tilde{\hat{r_t}}}}
\end{equation*}
which is the \textcolor{blue}{\textit{New Keynesian IS equation}}, $v_t$ is a shock to the efficient real interest rate.

\subsection{The New Keynesian model}
We can write the approximated dynamics of the model as:
\begin{align}
    \text{(NKPC)} \quad \tilde{\pi _t} &= \beta \mathbb{E}_t \tilde{\pi}_{t+1} + \kappa \tilde{x_t} + e_t \nonumber \\
    \text{(NKIS)} \quad \tilde{x_t} &= \mathbb{E}_t \left(\tilde{x}_{t+1} - \sigma (\tilde{i_t} - \tilde{\pi}_{t+1}) \right) + v_t \nonumber \\
    \text{(Taylor rule)} \quad \tilde{i_t} &= \alpha_1 \tilde{\pi_t} + \alpha_2 \tilde{x_t} + u_t \nonumber
\end{align}
Sometimes researches specify that monetary policy (through the
interest rate) follows the Taylor rule, i.e. interest rates respond to
inflation and output gap deviations, subject to a shock $u_t$,
which generally delivers a pretty good description of how MP is
done.

\subsection{Problem of the New Keynesian model}
\begin{itemize}
    \item Calvo pricing does not really make sense;
    \item Ideally price setting should be coming from firm's optimal
    responses to policy
    \item Large literature (even before NK model) where price rigidities
    are coming from adjustment costs to prices (menu costs)
    \item Alternative approach: imperfect information
    \item A large number of ad-hoc economic mechanisms designed mainly to fit persistence properties of
    the data rather than because economists have a strong belief in these particular stories;
    \item A large amount of unexplained shocks which are often highly persistent;
    \item Very limited modelling of policy tools or details of national accounts.
\end{itemize}