The study of the role of financial market frictions for the dynamics of the macro-economy
has a long tradition that started well before the 2008 financial crisis. Bernanke and Gertler
(1989) and Kiyotaki and Moore (1997) are two well-known studies that incorporate financial
frictions in general equilibrium models and became the standard references for this literature.
However, before the 2008 financial crisis, the mainstream approach to the study of the
macro-economy abstracted from financial frictions.

This was, in part, motivated by the view that, although markets are clearly incomplete, 
the importance of financial frictions for the dynamics of the macro-economy is somewhat negligible. 
Based on this view, it became preferable to use models with complete markets because they allow 
for simpler analytical characterization. These models featured many frictions such as sticky prices, 
sticky wages, adjustment costs in investment, variable capital utilization, matching frictions in the labor
market, but not financial frictions.

Prior to the 2008 financial crisis, the main idea was that financial frictions could make the real
impact of a shock bigger (amplification) or smaller (dampening). However, financial frictions
are not the initial ‘cause’ of macroeconomic expansions and contractions: something else has
to happen first in the nonfinancial sector. The analysis of financial shocks as a ‘source’ of
macroeconomic fluctuations has received more attention after the 2008 financial crisis.

\begin{itemize}
    \item The financial crisis can be analysed from many different angles
    (labor: banker’s pay and incentives, intl macro: savings glut,
    micro: coordination failures/bank runs, political economy).
    \item We are going to model one mechanism that gives a role for
    asset prices in the transmission of shocks.
    \item This mechanism gives an answer to the question: why were
    the real effects of the bursting of the housing bubble so big?
\end{itemize}

\section{Credit markets and asymmetric information}

Fundamental problem that borrowers have more information
about (and more control over) the projects they undertake
than the lenders who finance them.

WE assume that lender cannot verify the outcome of an investment project
without some monitoring cost.

Known as the \textit{costly state verification} problem.

Leads to a positive external finance premium.

