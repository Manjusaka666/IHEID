\section*{Question 1}

Consider the model of the interbank market presented in the lectures. i is the interbank interest rate, id
is the interest rate offered by the central bank on positive balances on banks' reserve accounts (deposit
rate), ib is the interest rate charged by the central bank on negative reserve balances (borrowing rate).
Banks face uncertainty about the payments they must make after the interbank market has closed. The
probability that a bank's required payment is less than $T$ is denoted by $F(T)$. Optimization by banks
implies the following condition:
\[(i_b - i)\left(1 - F(R/n)\right)= (i -i_d)F(R/n)\] 
or equivalently:
\[F(R/n) = \frac{i_b - i}{i_{b} - i_d}\]
where $R$ is the total quantity of reserves and $n$ is the number of banks.

\begin{problem*}[1]
    Provide an intuitive explanation of the optimality condition for banks.
\end{problem*}

\begin{solution}
    
\end{solution}

\begin{problem*}[2]
    Show how the equilibrium interbank interest rate is determined using a diagram.
\end{problem*}

\begin{solution}
    The conditions gives that:
    \[
    F(R/n) = \frac{i_{b} -i}{i_{b} -i_d} \in [0, 1] \Rightarrow i_d \leq i \leq i_{b} .
    \]
    If the inter-bank rate 
    \begin{figure*}[h!]
        \centering
        \includegraphics*{figures/PS5(2).png}
    \end{figure*}
\end{solution}

\begin{problem*}[3]
    Suppose the central bank would like to raise the market interest rate.
    Explain how this can be done through:

        (a) Open market operations (only R is adjusted);

        (b) Adjusting standing facility terms (only ib and id are adjusted)
\end{problem*}

\begin{solution}
    Originally, the graph changes as the following graph:
    
    \begin{figure*}[h!]
        \centering
        \includegraphics*{PS(3).png}
    \end{figure*}

    Let's suppose the bank increase rate by $x$, say $i^{\prime} =i+ x$.
    \[ 
    F(R/n) = \frac{(i_{b} +x) - (i+x)}{(i_{b} +x) - (i_d +x)} = \frac{i_{b} ^{\prime} -i^{\prime} }{i_{b} ^{\prime} -i_d^{\prime}}
    \]
    \begin{figure*}[h!]
        \centering
        \includegraphics*{figures/PS(3)-2.png}
    \end{figure*}
\end{solution}


\section*{Question 2}
You are hired by the International Monetary Fund, and you're being assigned on a mission to provide
technical assistance to Bank al-Maghrib, the central bank of Morocco. You're being asked to estimate
the impact of monetary policy on unemployment. Knowing that your boss is not fond of DSGE models,
but also distrusts the results from Christiano-Eichenbaum-Evans (because they were obtained with data
from high-income countries), you decide to estimate a Vector Autoregression model (with interest rates
and unemployment as the endogenous variables) yourself.

\begin{problem*}[1]
    Write down the system of equations for a first-order Vector Autoregression 
    where the nominal interest rate and unemployment depend on each other 
    and a lagged variable for each. Both variables are also determined by a shock, 
    $\eta_t^i$ and $\eta_t^u$ , which are orthogonal to past values of $i$ and $u$.
\end{problem*}

\begin{solution}
    \begin{align*}
        i_t &= \phi_1 i_{t-1} + \phi_2 u_t + \phi_3 u_{t-1} + \eta_t^i \\
        u_t &=\phi_4 i_t + \phi_5 u_{t-1} + \phi_6 i_{t-1} + \eta_t^u \\
        & cov(x_{i_t}, \epsilon_{i_t}) = 0
    \end{align*}
    \[
    \Rightarrow i_t = \phi_1 i_t + \phi_2(\phi_4 i_{t-1} + \phi_5 u_{t-1} + \phi_6 i_{t-1} + \eta_t^u) + \phi_3 u_{t-1} + \eta_t^i
    \]
    Solve the equation, we have:
    \begin{align*}
        i_t &= \frac{1}{1-\phi_2 \phi_4} \left[(\phi_{1}+\phi_2 \phi_6)i_{t-1} + (\phi_{3}+\phi_2 \phi_5)u_{t-1} + \phi_2 \eta_t^u + \eta_t^i\right] \\
        u_t &= \frac{1}{1-\phi_2 \phi_4} \left[(\phi_{1} \phi_4 +\phi_6)i_{t-1} + (\phi_{3}\phi_4 +\phi_5)u_{t-1} + \phi_4 \eta_t^i + \eta_t^u\right] \\
        \epsilon_{i_t} &= \frac{1}{1-\phi_2 \phi_4}\left[\phi_2 \eta_t^u + \eta_t^i\right] \\
        \epsilon_{u_t} &= \frac{1}{1-\phi_2 \phi_4}\left[\phi_4 \eta_t^i + \eta_t^u\right]
    \end{align*}
    If we guess $\phi_4 = 0$, we have:
    \begin{align*}
        \epsilon_t^i &= \eta_t^i + \phi_2 \eta_t^u\\
        \epsilon_t^u &= \eta_t^u \\
        \Rightarrow \epsilon_t^i &= \phi_2 \epsilon_t^u + \eta_t^i 
    \end{align*}

\end{solution}