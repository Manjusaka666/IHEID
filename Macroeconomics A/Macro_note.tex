\documentclass{article}
\usepackage[english]{babel}
\usepackage{geometry,amsmath,amssymb,enumerate,xcolor,theorem}
\geometry{top=20mm, left=15mm, left=15mm, bottom=20mm, right=15mm}

%%%%%%%%%% Start TeXmacs macros
\newcommand{\backassign}{=:}
\newcommand{\chapter}[1]{\medskip\bigskip--\noindent\textbf{\huge #1}}
\newcommand{\tmaffiliation}[1]{\\ #1}
\newcommand{\tmcolor}[2]{{\color{#1}{#2}}}
\newcommand{\tmop}[1]{\ensuremath{\operatorname{#1}}}
\newcommand{\tmtextbf}[1]{\text{{\bfseries{#1}}}}
\newcommand{\tmtextit}[1]{\text{{\itshape{#1}}}}
\newenvironment{descriptioncompact}{\begin{description} }{\end{description}}
\newenvironment{enumeratealpha}{\begin{enumerate}[a{\textup{)}}] }{\end{enumerate}}
\newcounter{nnnote}
\def\thennnote{\unskip}
{\theorembodyfont{\rmfamily}\newtheorem{note*}[nnnote]{Note}}
{\theorembodyfont{\rmfamily}\newtheorem{question}{Question}}
\newtheorem{theorem}{Theorem}
%%%%%%%%%% End TeXmacs macros

\begin{document}

\title{Macroeconomics A}

\author{
  Christopher Fu
  \tmaffiliation{\tmtextit{Version}: 23 Sep. 2024}
}

\maketitle

{\tableofcontents}

\chapter{Solow Model \& Dev Accounting}

\section{Solow model}

\begin{eqnarray*}
  Y & = & F (K, L, A)\\
  & = & w_t L^s (t) + R_t K^s (t) + \Pi (t)\\
  & = & \tmop{labor} \tmop{income} + \tmop{capital} \tmop{income} +
  \tmop{profits}
\end{eqnarray*}
Rental rate of capital:
\[ R_t = F_K (K (t), L, A) = f' (k (t)) \]
Wage rate:
\[ w_t = F_L (K, L (t), A) = f (k (t)) - f' (k (t)) k (t) \]
\[ \tmcolor{red}{k (t + 1) = (1 - \delta) k (t) + s f (k (t))} \]
\tmtextbf{Steady state:} All variables grow at constant rates.

For the steady state $k^{\ast}$,
\[ \delta k^{\ast} = s f (k^{\ast}) \]
\[ \frac{f (k^{\ast})}{k^{\ast}} = \frac{\delta}{s} \]

\section*{Reveiw Session}

\begin{question}
  \tmtextbf{AK Model} $Y = A K + B L$
\end{question}
\begin{enumeratealpha}
  \item Is it neoclassical? ------\tmcolor{red}{NO}
  
  Constant return: \tmcolor{red}{YES}. $\lambda Y = A (\lambda K) + B (\lambda
  L)$
  
  Positive \& diminishing marginal return: \tmcolor{red}{NO}. $\frac{\partial
  Y}{\partial K} = A > 0$, $\frac{\partial^2 Y}{\partial K^2} =
  0$(\tmcolor{red}{{\times}})
  
  Inada conditions: \tmcolor{red}{NO}. $\underset{K \rightarrow 0}{\lim} A =
  A$(\tmcolor{red}{{\times}})
  
  \item $y = k A + B$, where $y = \frac{Y}{L}$ and $k = \frac{K}{L}$
  
  marginal product of $k$: $\frac{\partial y}{\partial k} = A$
  
  average product of $k$: $\frac{Y}{K} = A + B \cdot \frac{L}{K} = A +
  \frac{B}{k}$
  
  \item fundamental equation of the Solow model: $K_{t + 1} = (1 - \delta) K_t
  + I_t$, $I_t = s Y_t = s f (k_t)$
  
  $\dot{k_t} = s y_t - (\delta + n) k_t$ and $\dot{K_t} = s Y_t - (\delta + n)
  K_t$, given $k = \frac{K}{L}$, capital depreciating at speed $\delta$ and
  population growth rate $n$: $\frac{\dot{L}}{L} = n$.
  
  \item ($Y = A K + B L$) At the steady state, $\dot{k} = 0$, thus $\dot{k} =
  s y_t - (n + \delta) k_t = 0$, $k = \frac{s B}{n + \delta - s A}$
  
  If $n + \delta > s A$, there exists the steady state, yet if $n + \delta < s
  A$, there's no steady state, it's the endogenous growth. Increase of
  population and depreciation of capital never counters investment, it forms
  an endogenous growth.
  
  \item ($Y = A K + B L$) In the case of $n + \delta < s A$, over time,
  $\frac{\dot{K}}{K} = s \frac{Y_t}{K_t} - \delta = s \left( A + \frac{B}{k}
  \right) - \delta$, as $\dot{k} = s y_t - (n + \delta) k_t = (s A - n -
  \delta) k_t + s B > 0$, $k$ keeps increasing. Thus, in the long run, $k
  \rightarrow \infty$, and $\frac{B}{k} \rightarrow 0$, $\frac{\dot{K}}{K}
  \rightarrow s A - \delta$.
  
  For output, $\frac{\dot{y}}{y} = \frac{\dot{k} A}{k A + B} \approx
  \frac{\dot{k} A}{k A} = \frac{\dot{k}}{k} = s A - (n + \delta)$.
  
  For consumption, $c_t = (1 - \delta) y_t$, thus $\frac{\dot{c}}{c} =
  \frac{\dot{y}}{y} \approx s A - (n + \delta)$.
  
  \item $s = 0.4$, $A = 1$, $B = 2$, $\delta = 0.08$, and $n = 0.02$, then the
  long-run growth rate of this economy is $s A - (n + \delta) = 0.4 - (0.02 +
  0.08) = 0.3$, if $B$ changes, there's no difference.
\end{enumeratealpha}
\begin{question}
  \tmtextbf{Government in the Solow Model} $Y_t = C_t + I_t + G_t$
  
  $G (t)$ denoting government spending at time $t$. Imagine that government
  spending is given by $G (t) = \sigma Y (t)$.
  \begin{enumeratealpha}
    \item Assuming $C (t) = s Y (t)$ is not very reasonable since it implies
    that consumption for a given level of aggregate income would be
    independent of government spending. Since government spending is financed
    by taxes, it is more reasonable to assume that higher government spending
    would reduce consumption to some extent. As an alternative, we may assume
    that consumers follow the rule of consuming a constant share of their
    after tax income, captured by the functional form $C (t) = s (Y (t) - G
    (t))$. Using $G (t) = \sigma Y (t)$, this functional form is also
    equivalent to $C (t) = (s - s \sigma) Y (t)$. In Part (b), we assume a
    more general consumption rule $C (t) = (s - \lambda \sigma) Y (t)$ with
    the parameter $\lambda \in [0, 1]$ controlling the response of consumption
    to increased taxes. The case $\lambda = 0$ corresponds to the extreme case
    of no response, $\lambda = s$ corresponds to a constant after-tax savings
    rule, and $\lambda \in [0, 1]$ correspond to other alternatives.
    
    \begin{note*}
      Higher $\sigma$ means lower consumption
    \end{note*}
    
    \item $C_t = (c - \lambda \sigma) Y_t$, $I_t = (1 - c - (1 - \lambda)
    \sigma) Y_t$. So, if government spending increases ($\sigma$ is higher),
    
    $K_{t + 1} = (1 - \delta) K_t + (1 - c - (1 - \lambda) \sigma) Y_t$
    {\Rightarrow} $\dot{k_t} = (1 - c - (1 - \lambda) \sigma) y_t - \delta
    k_t$
    
    Assuming there's no population growth, $k_{t + 1} = (1 - \delta) k_t + (1
    - c - (1 - \lambda) \sigma) y_t$.
    
    With higher government spending, capital per labor $k$ will decrease,
    which is that $k_{\sigma} (t) > k_{\sigma'} (t)$ for $\sigma' > \sigma$.
    
    Intuitively, higher government spending reduces net income and savings in
    the economy and depresses the equilibrium capital-labor ratio in the Solow
    growth model.
    
    As in the baseline Solow model, the capital-labor ratio in this economy
    converges to a unique positive steady state level $k^{\ast}$ characterized
    by
    
    $\frac{y^{\ast}}{k^{\ast}} = \frac{\delta}{1 - c - \sigma (1 - \lambda)}$
    
    The unique solution $k^{\ast}$ is decreasing in $\sigma$ and increasing in
    $\lambda$ since $f (k) / k$ is a decreasing
    
    function of $k$. In the economy with higher government spending (higher
    $\sigma$), the capital-labor
    
    ratio is lower at all times, and in particular, is also lower at the
    steady state. Also, the more
    
    individuals reduce their consumption in response to government spending
    and taxes (higher
    
    $\lambda$), the more they save, the higher the capital-labor ratio at all
    times and, in particular, the
    
    higher the steady state capital-labor ratio.
    
    \item a fraction $\phi$ of $G (t)$ is invested in the capital stock, so
    that total investment at time $t$ is given by $I_t = (1 - c - (1 - \lambda
    - \phi) \sigma) Y_t$
    
    $\dot{k_t} = (1 - c - (1 - \lambda - \phi) \sigma) y_t - \delta k_t $,
    $\frac{y^{\ast}}{k^{\ast}} = \frac{\delta}{1 - c - (1 - \lambda - \phi)
    \sigma}$
    
    $\frac{\partial}{\partial \sigma} = \frac{\delta (1 - \lambda - \phi)}{1 -
    c - (1 - \lambda - \phi) \sigma}$
    
    SO, if $\phi > 1 - \lambda$, $\frac{\partial}{\partial \sigma} < 0$, which
    means government spends a lot, it's not very reasonable.
  \end{enumeratealpha}
\end{question}

\section{Development Accounting}

Whether the factors of production can explain income levels?

Use Cobb-Douglas production function in physical and human capital

({\rightarrow} more general than labor):
\[ Y_j = A_j K_j^{\alpha} (L_j h_j)^{1 - \alpha} \]
\[ y_j = A_j K_j^{\alpha} h_j^{1 - \alpha} \backassign A_j y_j^{K H} \]
Caselli's measure of success:
\[ \tmop{success} = \frac{\tmop{Var} (\log (y_j^{K H}))}{\tmop{Var} (\log
   (y_j))} \]
If all countries had the same technology $A_j$, then $\tmop{success} = 1$.

\

\section{Neoclassical Growth Model}


The same basic environment as Solow model without the assumption of the constant exogenous saving rate.
\[ Y = F(K(t), L(t), Z(t)) \]

As in basic general equilibrium theory, let us suppose that preference orderings can be represented by utility functions. In particular, suppose that there is a unique consumption good, and each household $h$ has an \textit{\textcolor{blue}{instantaneous utility function}} given by: $u(c^h(t))$, where $c^h(t)$ is the consumption and $u : \mathbb{R}_+ \rightarrow \mathbb{R}$ is increasing and concave.

\begin{note}
\textcolor{red}{Negative levels of consumption are not allowed.}
\end{note}

We shall make 2 major assumptions:

\begin{description}[nosep]
\item[] Household does not derive any utility from the consumption of other households, so consumption externalities are ruled out.
\item[] Impose the condition that overall utility is \textcolor{blue}{\textit{time-separable and stationary}}; that is, instantaneous utility at time $t$ is independent of the consumption levels at past or future dates and is represented by the same utility function $u^h$ at all dates.
\end{description}

Representative household with preferences:
\[ U^h(T) = \sum_{t = 0}^T (\beta^h)^t u^h(c^h(t)) \]
where $\beta^h \in (0, 1)$ is the time discount factor of household $h$.

A solution $\{x(t)\}_{t = 1}^T$ to a dynamic optimization problem is \textit{\textcolor{blue}{time-consistent}} if the following is true: \textit{When $\{x(t)\}_{t = 1}^T$ is a solution to the continuation dynamic optimization problem starting from $t = 0$, $\{x(t)\}_{t = t'}^T$ is a solution to the continuation dynamic optimization starting from the time $t = t' > 0$}.

Let's consider the simplest case and suppose all households are infinitely-lived and identical. Then the demand side of the economy can be represented as the solution of the following maximization problem at time $t = 0$:
\[ \max \sum_{t = 0}^{\infty} \beta^t u(c(t)) \]
where $\beta \in (0, 1)$.

Budget constraint:
\[ C(t) + K(t + 1) = (1 + r(t))K(t) + w(t)L(t) \]
$r(t)$ is the rate of return on lending capital to firms.

\begin{note}
\textbf{\textcolor{blue}{Optimal Growth}}

\textbf{If the economy consists of a number of identical households, then this problem corresponds to the Pareto optimal allocation giving the same (Pareto) weight to all households. Therefore the optimal growth problem in discrete time with no uncertainty, no population growth, and no technological progress can be written as follows:}
\begin{align*}
    \max_{\{c(t), k(t)\}_{t=1}^{\infty}} &\sum_{t = 0}^{\infty} \beta^t u(c(t)) \\
    \text{s.t.} \quad &k(t + 1) = f(k(t)) + (1 - \delta)k(t) - c(t)
\end{align*}
with $k(t) \geq 0$ and given $k(0) > 0$.

The constraint is straightforward to understand: \textit{\textcolor{blue}{total output per capita produced with capital-labor ratio $k(t)$, $f(k(t))$, together with a fraction $(1 - \delta)$ of the capital that is undepreciated make up the total resources of the economy at date $t$.}}

Assuming that the representative household has $L(t)$ unit of labor supplied inelastically and denoting its assets at time $t$ by $a(t)$, this problem can be written as:
\begin{align*}
    \max_{\{c(t), k(t)\}_{t=1}^{\infty}} &\sum_{t = 0}^{\infty} \beta^t u(c(t)) \\
    \text{s.t.} \quad &a(t + 1) = (1 + r(t))a(t) - c(t) + w(t)L(t)
\end{align*}
where $r(t)$ is the net rate of return on assets, so that $1 + r(t)$ is the gross rate of return, and $w(t)$ is the equilibrium wage rate. \textcolor{red}{Market clearing then requires $a(t) = k(t)$.}
\end{note}

And, we have the non-Ponzi condition:
\[ \lim_{t \to \infty} \left\{K(t)\left[\prod_{s = 1}^{t-1} \left(\frac{1}{1 + r_s}\right)\right]\right\} \geq 0 \]
which refers to: limit of the PDV of debt has to be nonnegative (you cannot die in debt)

\begin{theorem}[\textcolor{blue}{\textbf{Euler Equation}}]
\[ \forall t : u'(c_t) = \beta(1 + r_{t + 1})u'(c_{t + 1}) \]
\end{theorem}

With an infinite horizon ($T \to \infty$), the Euler equations still determine the relative consumption levels. The terminal condition becomes \textcolor{blue}{transversality condition} is the limit of the terminal condition:
\[ \lim_{T \to \infty} K_{T + 1}\beta^T u'(c_T) = \lim_{T \to \infty} K_{T + 1}\beta^{T + 1}u'(c_{T + 1})(1 + r_{T + 1}) = 0 \]
Hence
\[ \lim_{t \to \infty} \beta^t u'(c_t)(1 + r_t)K_t = 0 \]
Then, we have two conditions involve the infinity:
\begin{equation}
    \lim_{t \to \infty} \beta^t u'(c_t)(1 + r_t)K_t = 0 \label{eq:trans}
\end{equation}
\begin{equation}
    \lim_{t \to \infty} \left\{K(t)\left[\prod_{s = 1}^{t-1} \left(\frac{1}{1 + r_s}\right)\right]\right\} \geq 0 \label{eq:nonponzi}
\end{equation}
The \eqref{eq:trans} is a condition for optimal behavior. The \eqref{eq:nonponzi} is a condition to make sure that the flow budget constraints are consistent with a lifetime budget constraint and no perpetual debt.

\textbf{Steady State} $(C^*, K^*)$

From the Euler equation:
\[ u'(C^*) = \beta(1 + F_K(K^*, L))u'(C^*) \]
hence
\[ 1 + r^* = 1 + F_K(k^*, 1) - \delta = \frac{1}{\beta} \]
which fully determines $k^*$.

\begin{note}
This shows that $k^*$ does not depend on $u(\cdot)$.
\end{note}

Level of per-capita consumption is then given by the economy's resource constraint:
\[ C^* + K^* = (1 - \delta)K^* + F(K^*, L) \]
hence,
\[ c^* = F(k^*, 1) - \delta k^* \]
\[ F_K(k^*, 1) = \delta + \left(\frac{1}{\beta} - 1\right) \]
In the Solow model the golden-rule level of capital (which maximizes steady-state consumption) satisfies
\[ F_K(k_S^{\text{GR}}, 1) = \delta \]
hence
\[ k^*_{\text{NGM}} < k_S^{\text{GR}} \]

\begin{note}
\textbf{Intuition:}

In NGM model, people tend to consume more on the near future rather than in the far steady state.
\end{note}

\subsection{NGM in continuous time}

Assume labor supply $L$ growth at rate $n$:
\[ L(t) = e^{nt} \]
Households choose consumption/saving to maximize
\[ U = \int_0^{\infty} e^{-\rho t}e^{nt}u(c(t))dt = \int_0^{\infty} e^{(n-\rho)t}u(c(t))dt \]
Intertemporal budget constraint: as in the discrete time case, but treat $\Delta K \to 0$:
\[ \dot{K}(t) = w(t)L(t) + (1 + r(t))K(t) - C(t) \]
Write in per-capita units:
\[ \dot{k}(t) = w(t) + (1 + r(t))k(t) - c(t) - nk(t) \]
Non-Ponzi condition:
\[ \lim_{t \to \infty} \left\{k(t)\exp\left[-\int_0^t(r(v) - n)dv\right]\right\} \geq 0 \]
To solve this, we define the \textbf{Hamiltonian}:
\[ H(c, k, \mu) = e^{-(\rho - n)t}u(c(t)) + \mu(t)[w(t) + (1 + r(t) - n)k(t) - c(t)] \]
optimality conditions are
\begin{align*}
    H_c(c, k, \mu) &= 0 \\
    H_k(c, k, \mu) &= -\dot{\mu}(t) \\
    H_{\mu}(c, k, \mu) &= \dot{k}(t) \\
    \lim_{t \to \infty} \mu(t)k(t) &= 0
\end{align*}
Hence:
\begin{align*}
    e^{-(\rho - n)t}u'(c(t)) - \mu(t) &= 0 \\
    \mu(t)(1 + r(t) - n) &= -\dot{\mu}(t)
\end{align*}
and we get:
\[ e^{-(\rho - n)t}u'(c(t))(1 + r(t) - n) = -e^{-(\rho - n)t}u''(c(t))\dot{c}(t) + (\rho - n)e^{-(\rho - n)t}u'(c(t)) \]
Denote $\sigma(c) = -\frac{u'(c)}{u''(c)c}$, we have the Euler equation in continuous time:
\[ \sigma(c)(1 + r(t) - \rho) = \frac{\dot{c}(t)}{c(t)} \]
$\sigma(c)$ is the intertemporal elasticity of substitution (IES).

\section{Business Cycle Facts and RBC Model}

\tmtextbf{Business Cycle Facts I}

We'll study the \tmtextbf{detrended macro times series:}
\[ x_t = \log (X_t) - \log (X_t^{\ast}) \]
is the percentage deviation of variable $X$ from its trend $X^{\ast}$.

How's the trend?

\begin{descriptioncompact}
  \item[] First linear,
  
  \item[] more sophisticated filters: Baxter-King(Bankpass) filter,
  Hodrick-Prescott(HP) filter
\end{descriptioncompact}

Linear: just run OLS regression and use residuals.

\

\tmtextbf{Business Cycle Facts III:}

Governmetn spending, not strongly correlated with level of output, main driver
of business cycles.

Real wage is mildly pro-cyclical: average wage evolves differently than the
wage of a continuously employed worker

\

$U_C (C, 1 - L) > 0, U_{C C} (C, 1 - L) < 0$ and $U_L (C, 1 - L) > 0$, $U_{L
L} (C, 1 - L) < 0$

\end{document}
