\documentclass[12pt,a4paper]{article}

\usepackage[margin=3cm]{geometry}
\usepackage{setspace}
\setstretch{1.25}

\usepackage[T1]{fontenc}
\usepackage[utf8]{inputenc}

\usepackage{amsmath,amssymb,mathtools}
\usepackage{booktabs}
\usepackage{enumitem}
\usepackage{microtype}
\usepackage[hidelinks]{hyperref}
\usepackage{fancyhdr}

\pagestyle{empty}

\begin{document}

\begin{center}
{\Large \textbf{Forecasting Horse Races and ``Belief Distortions'':\\
A Hierarchical Bayesian VAR Study with Sentiment Signals}}\\[2pt]
% {\normalsize Term Paper Proposal --- Topics in Econometrics (EI137)}\\
% {\normalsize Geneva Graduate Institute (IHEID)}\\[2pt]
{\normalsize \textit{Jingle Fu}}\\
{\normalsize \today}
\end{center}

% --- Section 1: Motivation ---
\section*{1. Research Question and Motivation}
% Vector Autoregressions (VARs) face a fundamental trade-off between information content and parameter proliferation.
% While Hierarchical BVARs addresses overfitting,
% it remains an open question whether the resulting forecasts exhibit systematic biases akin to human ``behavioral'' anomalies.
Recent literature on Diagnostic Expectations (DE) suggests that economic agents overreact to news,
whereas information rigidity models predict underreaction. This project investigates two precise questions:
% \begin{enumerate}
%     \item \textbf{Forecasting Performance:} Does expanding the information set of a hierarchical BVAR to include forward-looking financial prices and consumer sentiment reduce Root Mean Squared Forecast Error (RMSFE) relative to smaller baselines?
%     \item \textbf{Behavioral Diagnostics:} Do the resulting forecast revisions predict forecast errors?
%     Specifically, does the inclusion of sentiment shift the ``Coibion-Gorodnichenko'' coefficient ($\beta$) towards negative values (implying overreaction) or positive values (implying rigidity),
%     thereby locating the algorithm on the behavioral spectrum?
% \end{enumerate}

\textit{1. Does expanding the information set of a hierarchical BVAR to include forward-looking financial prices and consumer sentiment reduce Root Mean Squared Forecast Error (RMSFE) relative to smaller baselines?}

\textit{2. Whether adding sentiment changes the Coibion-Gorodnichenko(CG) error-revision coefficient in a direction consistent with diagnostic overreaction.}
% --- Section 2: Data ---
\section*{2. Data}

I use monthly U.S. macroeconomic series from the FRED-MD database, covering the period \textbf{1985M1--2019M12}.
The sample ends in 2019 to avoid COVID-19 outliers that would require complex volatility modeling beyond the scope of this term paper.
To ensure consistent evaluation, variables are estimated in log-levels (to preserve cointegration) but evaluated in growth rates.
The analysis compares three nested information sets:
\begin{itemize}[leftmargin=*]
    \item \textbf{Small Model (Baseline):} Industrial Production (INDPRO), Consumer Price Index (CPIAUCSL), Unemployment Rate (UNRATE), and Federal Funds Rate (FEDFUNDS).
    \item \textbf{Medium Model (Financial Extension):} Adds the 10-Year Treasury Yield (GS10) and S\&P 500 Index (S\&P500) to capture forward-looking financial cycles.
    \item \textbf{Full Model (Sentiment Extension):} Adds the University of Michigan Consumer Sentiment Index (UMCSENT) to test the marginal predictive power of ``soft'' data.
\end{itemize}


% --- Section 3: Econometric Framework ---
\section*{3. Econometric Framework}
The core methodology relies on a reduced-form VAR estimated with a Minnesota-style Normal-Inverse-Wishart prior.
\begin{enumerate}
    \item \textbf{Hierarchical BVAR:} Let $y_t$ be the vector of endogenous variables. We estimate three nested BVAR systems with $p=12$ lags:
    \begin{equation}
        y_t^{Small, Medium, Full} = c + \sum_{\ell=1}^{p} B_{\ell} y_{t-\ell} + u_t, \quad u_t \sim \mathcal{N}(0, \Sigma)
    \end{equation}
    across three distinct information sets.
    Shrinkage is selected endogenously by treating $\lambda $ as a hyperparameter with a hyperprior and choosing it by marginal likelihood.
    We compute RMSFE for $h = {1,3,12}$ and report RMSFE ratios relative to the benchmark, together with Diebold-Mariano tests for pairwise comparisons.
    Benchmarks are a random-walk-type forecast and an AR(1) forecast defined on the same evaluation transforms used for the BVAR outputs.
    % \begin{itemize}
    %     \item \textbf{Model 1: Baseline (Small) BVAR.} Focuses on standard real activity and price data. The vector of endogenous variables is:
    %     \begin{equation}
    %         Y_t^{Small} = [\text{INDPRO}_t, \text{CPI}_t, \text{UNRATE}_t, \text{FEDFUNDS}_t]'
    %     \end{equation}
    %     \item \textbf{Model 2: Financial (Medium) BVAR.} Augments the baseline with forward-looking financial indicators to capture market expectations:
    %     \begin{equation}
    %         Y_t^{Med} = [Y_t^{Small'}, \text{GS10}_t, \text{S\&P500}_t]'
    %     \end{equation}
    %     \item \textbf{Model 3: Sentiment (Full) BVAR.} Further expands the system to test if ``soft'' sentiment data reduces forecast errors or introduces behavioral bias:
    %     \begin{equation}
    %         Y_t^{Full} = [Y_t^{Med'}, \text{UMCSENT}_t]'
    %     \end{equation}
    % \end{itemize}

    \item \textbf{Identification of ``Behavioral'' Bias:} 
    Let $z_{t+h}$ denote the realized growth rate of a target variable ($z \in \{\text{INDPRO}, \text{CPI}\}$) at horizon $h$.
    Let $\hat{z}_{t+h|t}^{(m)}$ denote the forecast generated by Model $m \in \{\text{Small, Med, Full}\}$ at time $t$.
    We estimate the following regression linking ex-post forecast errors to forecast revisions:
    \begin{equation}
        (z_{t+h} - \hat{z}_{t+h|t}^{(m)}) = \alpha_h + \beta_h (\hat{z}_{t+h|t}^{(m)} - \hat{z}_{t+h|t-1}^{(m)}) + \varepsilon_{t+h}
    \end{equation}
    where the term in the first parenthesis represents the forecast error and the term in the second parenthesis represents the forecast revision.
    Since the forecast horizon $h$ creates overlapping observations, inference on $\beta_h$ relies on Newey-West HAC standard errors.
\end{enumerate}

% --- Section 4: Interpretation ---
\section*{4. Interpretation and Expected Results}
% The analysis will be conducted using a recursive pseudo-out-of-sample scheme (2001--2019).
% \begin{itemize}
%     \item \textbf{The Horse Race:} 
%     \item \textbf{The Diagnostic:} By comparing $\beta_h$ across the three nested models, we can isolate the marginal effect of sentiment.
%     A coefficient $\beta_h = 0$ implies rational expectations (null hypothesis);
%     $\beta_h > 0$ indicates information rigidity (underreaction);
%     and $\beta_h < 0$ indicates diagnostic expectations (overreaction).
%     The key empirical object is the change in $\beta_h$ when adding sentiment (Medium-Finance vs Medium-Finance+Sentiment), which might connect to diagnostic expectations.
% \end{itemize}
First, We verify whether the Hierarchical BVAR outperforms AR(1) benchmarks.
% Improvements at $h=1$ are consistent with a nowcasting channel,
% while improvements concentrated at $h=12$ are more consistent with extracting medium-run cyclical information from forward-looking variables.
The key test is whether the \textit{Full Model} lowers RMSFE for real activity variables at short horizons ($h=1, 3$) (and at long horizens $h=12$).
% Then, by comparing $\beta_h$ across the three nested models, we can isolate the marginal effect of sentiment.
% The key empirical object is the change in $\beta_h$ when adding sentiment (Medium-Finance vs Medium-Finance+Sentiment), which might connect to diagnostic expectations.
A shift in $\beta_h$ \emph{toward zero} when adding UMCSENT is interpreted as sentiment improving the informational content of revisions,
making updates less systematically biased.
By contrast, if this pushes $\beta_h$ \emph{further away from zero}---especially into negative---it suggests that sentiment
% primarily introduces high-frequency noise that the model partially extrapolates,
generating an overreaction pattern consistent with the diagnostic-expectations sign prediction.
% % --- Section 5: Preliminary Structure ---
% \section*{5. Preliminary Structure of the Term Paper}
% The final paper will be approximately 15--20 pages (including appendix) and structured as follows:

% \begin{enumerate}[leftmargin=*, label=\textbf{\arabic*}.]
%     \item \textbf{Introduction}
%     \begin{itemize}
%         \item \textbf{Motivation:} Briefly discuss the tension between ``black box'' machine learning methods in macroeconomics and the need for economic interpretability through the lens of behavioral diagnostics.
%         \item \textbf{Research Questions:} Explicitly state the two goals: (1) assessing the forecasting value of sentiment in a BVAR, and (2) testing whether this induces overreaction.
%         \item \textbf{Contribution:} Highlight the novelty of applying human-behavioral tests (Coibion-Gorodnichenko) to machine-learning forecasts.
%     \end{itemize}

%     \item \textbf{Data and Experimental Design}
%     \begin{itemize}
%         \item \textbf{Data Description:} Detail the FRED-MD dataset selection, the rationale for the 1985--2019 sample period, and the specific transformations (log-levels for estimation vs. growth rates for evaluation).
%         \item \textbf{Forecasting Scheme:} Define the recursive expanding window approach and the specific forecast horizons ($h=1, 3, 12$) chosen for the analysis.
%         \item \textbf{Model Specifications:} Clearly define the variables included in the Small, Medium, and Full information sets.
%     \end{itemize}

%     \item \textbf{Econometric Methodology}
%     \begin{itemize}
%         \item \textbf{The BVAR Model:} Present the VAR likelihood and the structure of the Conjugate Normal-Inverse-Wishart prior (Minnesota prior).
%         \item \textbf{Hierarchical Selection (GLP):} Explain the mechanism for selecting optimal shrinkage hyperparameters ($\lambda$) via Marginal Likelihood maximization, avoiding ad-hoc choices.
%         \item \textbf{Behavioral Diagnostic Test:} Derive the Coibion-Gorodnichenko (CG) regression equation. Discuss the econometric issues arising from overlapping forecast horizons and the use of Newey-West HAC standard errors to correct for serial correlation.
%     \end{itemize}

%     \item \textbf{Empirical Results}
%     \begin{itemize}
%         \item \textbf{Forecasting Horse Race:} Present a table of RMSFE ratios relative to the AR(1) benchmark. Discuss whether the inclusion of financial and sentiment variables improves accuracy for specific targets (INDPRO, CPI) or horizons.
%         \item \textbf{Algorithmic Behavior Analysis:} Present the estimates of the $\beta_h$ coefficients for each model. Analyze the shift in $\beta_h$ when moving from the Medium to the Full model. Does the algorithm become more ``rigid'' or more ``volatile'' when fed with sentiment data?
%         \item \textbf{Robustness Checks:} Briefly discuss sensitivity to lag length ($p=12$ vs. $p=6$) or alternative evaluation metrics (e.g., Mean Absolute Error).
%     \end{itemize}

%     \item \textbf{Conclusion}
%     \begin{itemize}
%         \item Summarize the main findings regarding the trade-off between forecast accuracy and behavioral consistency.
%         \item Discuss limitations (e.g., the exclusion of the COVID-19 period) and potential avenues for future research, such as time-varying parameters.
%     \end{itemize}
% \end{enumerate}

\end{document}
