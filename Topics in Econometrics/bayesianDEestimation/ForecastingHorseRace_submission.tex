\documentclass[12pt,a4paper]{article}

\usepackage[margin=3cm]{geometry}
\usepackage{setspace}
\setstretch{1.25}

\usepackage[T1]{fontenc}
\usepackage[utf8]{inputenc}

\usepackage{amsmath,amssymb,mathtools}
\usepackage{booktabs}
\usepackage{graphicx}
\usepackage{caption}
\usepackage{threeparttable}
\usepackage{float}
\usepackage{siunitx}
\usepackage{enumitem}
\usepackage{microtype}
\usepackage[hidelinks]{hyperref}
\usepackage[natbibapa]{apacite}
\usepackage{fancyhdr}

\graphicspath{{results/figures/}}
\sisetup{group-separator = {,}, group-minimum-digits = 4, input-symbols = ()}

\newif\ifsubmission
% Standalone submission source: always compile submission branch.
\submissiontrue

\begin{document}
% Title page (unnumbered)
\begin{titlepage}
    \centering
    \vspace*{1.5cm}
    {\Large Geneva Graduate Institute (IHEID)\par}
    \vspace{0.4cm}
    {\large Topics in Econometrics (EI137)\par}
    \vspace{0.2cm}
    {\large Term Paper\par}
    \vspace{1.0cm}

    {\LARGE\bfseries The Incremental Predictive Power of Consumer Sentiment in Macroeconomic Forecasting\par}
    \vspace{0.25cm}
    {\large Evidence from a Hierarchical Bayesian VAR and Forecast-Revision Diagnostics\par}

    \vfill

    {\large Jingle Fu\par}
    \vspace{0.2cm}
    {\large Professor: Marko Mlikota\par}
    \vspace{0.2cm}
    {\large Spring 2025\par}
    \vspace{0.6cm}
    {\large \today\par}
\end{titlepage}

% Abstract page (roman numbering allowed before Introduction)
\pagenumbering{roman}
\setcounter{page}{1}
\pagestyle{plain}

\begin{abstract}
    Does consumer sentiment add incremental predictive content for U.S.\ inflation and industrial production once standard macro aggregates and financial prices are already included? Using monthly data (1985M1--2019M12) and a hierarchical Bayesian VAR with three nested information sets---\emph{Small} (core macro), \emph{Medium} (+ financial variables), and \emph{Full} (+ sentiment)---we estimate pseudo out-of-sample forecasts recursively from 2001M1 through 2019M11. Forecast accuracy is evaluated by RMSFE at horizons $h \in \{1,3,12\}$; forecast-revision patterns are assessed via error-on-revision regressions following \cite{CG2015}.

    Two findings emerge. First, sentiment adds little incremental point-forecast accuracy (Table~\ref{tab:rmsfe}): all specifications substantially outperform naive benchmarks for inflation, but the \emph{Small} model attains the lowest long-horizon inflation RMSFE, while financial variables dominate short-horizon industrial production forecasts. Second, revision diagnostics (Table~\ref{tab:cg_regression}) show inflation forecasts exhibit short-horizon underreaction and long-horizon overreaction; richer information sets attenuate long-horizon coefficients toward zero, though differences are estimated imprecisely. These revision patterns measure the \emph{internal consistency} of model probability updates and are consistent with---but do not prove---sentiment capturing belief distortion or diagnostic-expectations dynamics. Because models are nested, we report Clark--West adjusted tests and emphasize RMSFE magnitudes over sharp statistical claims.
\end{abstract}
\noindent\textit{Keywords:} Bayesian VAR; hierarchical shrinkage; forecasting; consumer sentiment; forecast revisions.\\
\noindent\textit{JEL codes:} C11; C53; E37.

\clearpage
% Main text starts here (arabic numbering from Introduction)
\pagenumbering{arabic}
\setcounter{page}{1}

% --- Section 1: Introduction ---
\section{Introduction}

\ifsubmission
    Macroeconomic forecasting systems face a perennial trade-off: incorporating additional information may capture forward-looking signals, but in finite samples it increases parameter uncertainty and risks overfitting unless regularization is sufficiently aggressive. This tension is particularly acute for soft information such as consumer sentiment, which may reflect households' inflation expectations and spending intentions but could also overlap with signals already embedded in financial asset prices and realized aggregates. This paper investigates whether consumer sentiment adds incremental predictive content for U.S.\ inflation and industrial production once standard macro variables and financial prices are included, and whether sentiment alters forecast-revision dynamics in ways consistent with expectation-formation mechanisms such as diagnostic expectations \citep{BordaloGS2018,Bordalo2020}.

    I conduct a transparent forecasting horse race using hierarchical Bayesian VARs estimated recursively on monthly data (1985M1--2019M12) across three nested information sets: \emph{Small} (core macro aggregates), \emph{Medium} (+ financial variables), and \emph{Full} (+ sentiment). Forecast accuracy is evaluated by root mean squared forecast error (RMSFE) at horizons $h\in\{1,3,12\}$ months on pseudo out-of-sample forecasts spanning 2001M1--2019M11. To assess whether expanding the information set disciplines forecast updating, I estimate forecast-error-on-revision regressions following \citet{CG2015}, which reveal systematic underreaction or overreaction in model probability updates.

    Two disciplined results emerge. First, sentiment adds little incremental \emph{point-forecast accuracy}: all BVAR specifications substantially outperform no-change benchmarks for inflation (Table~\ref{tab:rmsfe}), but the \emph{Small} model attains the lowest long-horizon (h=12) inflation RMSFE, while sentiment's inclusion does not improve upon the \emph{Medium} specification. For industrial production, financial variables improve short-horizon forecasts, but all models underperform benchmarks at h=12, reflecting the inherent difficulty of predicting long-run productivity trends. Second, revision diagnostics (Table~\ref{tab:cg_regression}) show inflation forecasts exhibit short-horizon underreaction (positive error-on-revision coefficients) and long-horizon overreaction (negative coefficients); richer information sets attenuate long-horizon overreaction, moving coefficients closer to the rational-expectations benchmark of zero, though differences are statistically imprecise. These revision patterns measure the \emph{model's internal probability-update consistency} rather than economic agents' behavioral frictions, and are consistent with---but do not causally identify---mechanisms such as belief distortion or diagnostic expectations.

    This paper makes two contributions. First, I provide a transparent mapping from hierarchical BVAR estimation with data-driven shrinkage to pseudo out-of-sample forecast evaluation and revision diagnostics, all anchored to the same forecasting model. This design ensures alignment between data transformations, information sets, and horizon definitions, making empirical claims auditable against project outputs (all numerical statements trace to Tables~\ref{tab:rmsfe} and \ref{tab:cg_regression}). Second, I document a sharp distinction between forecast \emph{accuracy} and forecast \emph{discipline}: sentiment contributes little to minimizing squared forecast errors but alters revision dynamics in ways suggestive of capturing household belief distortion. If sentiment primarily reflects households' diagnostic beliefs---whereby recent inflation experiences are overweighted when forming expectations \citep{BordaloGS2018}---then its forecasting value lies in understanding expectation formation rather than optimizing point predictions. This interpretation is consistent with financial variables subsuming sentiment's informational content for point forecasting (via market aggregation of diverse expectations) while sentiment retains independent value as a proxy for belief dynamics.

    Because the information sets are nested (Small $\subset$ Medium $\subset$ Full), standard equal-accuracy tests exhibit size distortions \citep{ClarkMcCracken2001}. I therefore report Clark--West MSPE-adjusted tests \citep{ClarkWest2007} as robustness and emphasize the \emph{magnitude and stability} of RMSFE differences rather than sharp statistical rejection claims.

    \subsection*{Related Literature}

    This paper contributes to five research strands.

    \emph{Diagnostic expectations and forecast-revision dynamics.} \citet{BordaloGS2018} formalize diagnostic expectations (DE), a model of belief formation in which agents overweight recent or salient signals when updating. \citet{Bordalo2020} show DE can generate systematic forecast-error patterns such as overreaction to news. Our revision diagnostics (CG regressions) reveal horizon-dependent patterns---short-horizon underreaction and long-horizon overreaction---consistent with DE-style dynamics, though we emphasize these are \emph{model-based forecast diagnostics} rather than structural estimates of household behavior. Sentiment's potential role as a belief-distortion proxy aligns with recent work emphasizing households' non-rational inflation expectations \citep{CoibionGorodnichenko2012,CG2015}.

    \emph{Consumer confidence in macroeconomic forecasting.} Early work documents predictive content of sentiment indices for consumption and output \citep{StockWatson2002}. More recent evaluations are mixed: \citet{LahiriZhao2015} find limited incremental value conditional on financial variables, while \citet{ChristiansenEtAl2014} document sentiment's role in recession forecasting. Our RMSFE results align with the skeptical view for \emph{point forecasts} conditional on financial prices, but our revision diagnostics suggest sentiment may still matter for \emph{expectation dynamics}.

    \emph{Inflation forecasting difficulty and parsimony.} \citet{AtkesonOhanian2001} show a naive random-walk model is hard to beat for long-horizon inflation forecasting; \citet{StockWatson2007} document that inflation has become harder to forecast over time. Our finding that the \emph{Small} model attains the lowest h=12 inflation RMSFE reflects this well-known parsimony advantage: at long horizons, adding variables increases parameter uncertainty faster than it adds signal, even with aggressive shrinkage.

    \emph{Hierarchical Bayesian VAR shrinkage.} \citet{GLP2015} develop hierarchical prior selection, treating shrinkage intensity as a hyperparameter learned from marginal likelihood. \citet{BanburaGLP2010} show this approach makes high-dimensional VARs feasible. I adopt hierarchical shrinkage to ensure fair comparisons across nested information sets: the procedure endogenously tightens priors as model size grows, mitigating overfitting without manual recalibration.

    \emph{Nested forecast evaluation.} \citet{ClarkMcCracken2001} show standard Diebold--Mariano tests have nonstandard distributions when comparing nested models; \citet{ClarkWest2007} propose MSPE-adjusted tests to correct size distortions. Because our information sets are strictly nested, I report both standard DM tests (as suggestive evidence) and Clark--West tests (Appendix Table~\ref{tab:clark_west}), and emphasize RMSFE magnitudes in the main text.
\else

    Building an effective macroeconomic forecasting model requires balancing two fundamental tensions. First, incorporating additional information can in principle improve predictions by capturing forward-looking signals, but in finite samples it increases parameter uncertainty and can worsen forecast performance unless regularization is sufficiently aggressive. Second, even if a model fits historical data well, its forecast revisions may exhibit systematic biases that reveal whether the underlying probability updates are well-calibrated or subject to cognitive frictions. This paper investigates both dimensions simultaneously: whether soft information (consumer sentiment) contains incremental predictive value for key macro targets once one already conditions on standard aggregates and financial prices, and whether sentiment helps align the model's updating behavior with rational expectations.

    The specific research context is as follows. Consumer sentiment indices have long been recognized as containing information about households' perceptions of economic conditions and future prospects \cite{StockWatson2002,KoopKorobilis2010}. Asset prices, by contrast, are forward-looking summaries of market expectations about fundamentals and risk premia \cite{Estrella1998}. A natural question is whether these two information sources---sentiment (household expectations) and financial prices (market pricing)---are redundant once one conditions on standard macro aggregates like unemployment and inflation, or whether they each contain independent information about different frequencies of macro fluctuations. Sentiment may be particularly informative about the persistent (low-frequency) component of inflation if households' wage-setting and pricing behavior responds to their own inflation expectations, which sentiment may better measure than high-frequency financial variables. Conversely, financial variables may excel at capturing near-term cyclical shifts because stock prices and yield spreads are sensitive to quarterly or monthly demand revisions.

    A complementary diagnostic asks whether forecast-error patterns exhibit the signatures of rational expectation formation or reveal systematic cognitive biases. Following \citet{CG2015}, we implement the forecast-error-on-revision regression, which relates ex-post forecast errors to contemporaneous forecast revisions. Under rational expectations, this coefficient should be zero; a positive coefficient signals underreaction (information rigidity or gradual belief updating), while a negative coefficient suggests overreaction (extrapolation or overfitting to transitory movements). Applied to a model-based forecasting system, this diagnostic measures the internal consistency of the model's probability updates: do revisions move in the right direction but with insufficient magnitude, or do they overshoot subsequent realizations? The hypothesis is that adding sentiment---if it provides a disciplining signal about inflation persistence---may reduce systematic updating biases, particularly at short horizons where standard models might otherwise place excessive weight on high-frequency fluctuations.

    The contribution of this paper is twofold. First, we provide a transparent mapping from hierarchical BVAR estimation (with endogenous shrinkage) to pseudo out-of-sample forecast evaluation and behavioral diagnostics, all computed from the same underlying forecasting model. This forces alignment between data transformations, information sets, benchmarks, and horizon definitions, making the empirical claims auditable against project outputs. Second, we document a horizon- and target-specific pattern of information roles: sentiment's incremental value is most pronounced for inflation at long horizons, while financial variables dominate short-horizon real activity prediction. These patterns are consistent with sentiment capturing low-frequency information about expectation anchoring and inflation persistence, while financial variables measure near-term demand pressures.

    The analysis is deliberately focused on internal consistency and transparent implementation rather than methodological novelty. Our approach will be useful both for practitioners building production forecasting systems and for researchers interested in how different information types contribute to forecast discipline and accuracy.

    % Paragraph on empirical hypotheses
    \paragraph{Empirical hypotheses.}
    We structure our investigation around two complementary hypotheses. \emph{First}, if sentiment contains genuine information about inflation persistence, then adding it should improve forecast accuracy primarily at longer horizons ($h=3$ and especially $h=12$), where low-frequency dynamics dominate, while having limited incremental value at very short horizons. Conversely, financial variables should matter more at short horizons where they encode near-term cyclical pressures. \emph{Second}, if richer information sets discipline internal updating, then revisions should become less systematically related to subsequent forecast errors in the \cite{CG2015} regression (coefficients closer to zero), with the most plausible improvements at long horizons where trend extrapolation is a risk.
\fi
% --- Section 2: Data ---
\section{Data}

\ifsubmission
    The dataset comprises monthly U.S.\ series spanning 1985M1--2019M12, ending before the COVID-19 period to avoid structural shifts requiring separate treatment. The three nested information sets are \emph{economically motivated}. The \emph{Small} model includes industrial production, CPI, unemployment, and the federal funds rate, capturing \emph{Phillips-curve dynamics} (unemployment-inflation linkage) and \emph{monetary policy stance}. The \emph{Medium} model adds the 10-year Treasury yield, the S\&P 500, and oil prices, incorporating \emph{forward-looking market expectations} embedded in yield spreads (growth and inflation risks) and equity valuations (earnings expectations). The \emph{Full} model further adds the University of Michigan sentiment index, a \emph{household-expectations proxy} that may capture low-frequency inflation beliefs not fully reflected in financial prices. This nesting isolates the incremental role of sentiment conditional on financial variables.

    Following the standard BVAR forecasting literature, the model is estimated in levels or log-levels \cite{Sims1980,GLP2015}. Forecasts are evaluated on annualized cumulative growth rates (constructed from the same origin date as the forecast), ensuring comparability across horizons. Full transformation and implementation details are reported in the extended version.
\else

    The dataset consists of monthly U.S.\ time series over 1985M1--2019M12. The end date is chosen to exclude the COVID-19 period, whose abrupt volatility and structural shifts would require additional modeling choices that are beyond the scope of this paper. The series are obtained from FRED (industrial production \texttt{INDPRO}, CPI \texttt{CPIAUCSL}, unemployment \texttt{UNRATE}, federal funds rate \texttt{FEDFUNDS}, the 10-year Treasury yield \texttt{GS10}, and WTI crude oil prices \texttt{DCOILWTICO}) and from Yahoo Finance for the S\&P 500 index (mapped to \texttt{SP500} in the code). Consumer sentiment is measured by the University of Michigan index \texttt{UMCSENT}.

    I compare three nested information sets. The \emph{Small} model includes \texttt{INDPRO}, \texttt{CPIAUCSL}, \texttt{UNRATE}, and \texttt{FEDFUNDS}. The \emph{Medium} model augments the small model with \texttt{GS10}, \texttt{SP500}, and \texttt{DCOILWTICO}. The \emph{Full} model further adds \texttt{UMCSENT}. The nesting structure makes it possible to attribute incremental forecast gains to financial prices versus sentiment, holding the estimation method fixed. Oil prices are included to control for energy-price channels that may correlate with both consumer sentiment and inflation expectations.

    \subsection{Data transformation and evaluation targets}

    In time-series analysis, (weak) stationarity is often crucial. Many macroeconomic databases (including FRED-MD) provide recommended transformations intended to remove unit roots.\footnote{The project code follows a different convention than FRED-MD-style transformations: it estimates the BVAR in levels or log-levels and evaluates forecasts on cumulative growth rates constructed from those levels.}
    However, the BVAR literature typically favors estimating the model in \textbf{levels} or \textbf{log-levels} \cite{Sims1980,GLP2015}. The key reason is that Minnesota-style shrinkage can be interpreted as a structured way of regularizing persistent dynamics, including behavior close to a random walk, so that long-run comovement is not mechanically removed by differencing. If one differences the data mechanically, stationarity is ensured, but long-run equilibrium information may be attenuated.

    Accordingly, I adopt the following strategy. In the estimation stage, \texttt{INDPRO}, \texttt{CPIAUCSL}, and \texttt{SP500} enter in log-levels, $x_t=\ln(X_t)$, while \texttt{UNRATE}, \texttt{FEDFUNDS}, \texttt{GS10}, and \texttt{UMCSENT} enter in levels. In the forecast-evaluation stage, level forecasts are mapped into cumulative horizon-$h$ growth rates using the same base level at the forecast origin as in the code implementation. For log variables, the evaluation target is the annualized cumulative log change,
    \[
        z_{t,h} = \frac{1200}{h}\left(x_{t+h}-x_t\right),
    \]
    so that $h=12$ corresponds to year-over-year growth because $1200/12=100$. This definition ensures that forecast errors compare the realized and predicted \emph{cumulative} change from the same origin date and places all reported errors in percentage points at annual rates.

    For inflation based on \texttt{CPIAUCSL}, the evaluation target at horizon $h$ is constructed from the log CPI level $p_t=\ln(P_t)$ as
    \[
        \pi_{t,h}=\frac{1200}{h}\left(p_{t+h}-p_t\right),
    \]
    so that $h=12$ corresponds to year-over-year inflation. The same mapping is applied to industrial production growth from $\ln(\texttt{INDPRO})$. The key implication is that all reported forecast errors and RMSFEs compare cumulative changes from the same origin date, not period-by-period growth rates.


    \subsection{Implementation in \textsf{R}}

    I use the \textsf{R} package \texttt{BVAR} \cite{KuschnigVashold2021}, which implements hierarchical prior selection in the spirit of \citet{GLP2015}.

    \paragraph{Prior setup and calibration rationale}
    The prior is configured via \texttt{bv\_priors(hyper = "auto")} and combines a Minnesota prior with sum-of-coefficients and dummy-initial-observation components. The overall Minnesota tightness parameter $\lambda$ and lag-decay parameter $\alpha$ are specified via hyperpriors and optimized via marginal likelihood. Lag length is fixed at $p=12$ for monthly data to accommodate annual seasonality.

    \emph{Hyperparameter choices.} The $\lambda$ hyperprior is specified with mode 0.05, standard deviation 0.2, and bounds $[0.001, 2.0]$, tighter than conventional choices (e.g., mode 0.2 in \citet{BanburaGLP2010}). This reflects two \emph{forecasting regularization principles}. First, long-horizon inflation forecasting is notoriously difficult: naive random-walk models are hard to beat \citep{AtkesonOhanian2001,StockWatson2007}, and adding variables increases parameter uncertainty. Tighter baseline shrinkage guards against overfitting to sample-specific patterns that do not generalize pseudo out-of-sample. Second, in high-dimensional VARs (up to 8 variables $\times$ 12 lags = 96 coefficients per equation), aggressive shrinkage toward the random-walk benchmark prevents the posterior from chasing transient correlations. The hierarchical mechanism allows $\lambda$ to adapt: the data discipline shrinkage intensity via marginal likelihood, ensuring fair comparisons as model size varies (Section~\ref{sec:lambda_evolution} documents this adaptation).

    The lag-decay parameter is set to $\alpha=3.0$, accelerating prior-variance decay with lag length relative to the conventional $\alpha=2.0$ \citep{Litterman1986}. In monthly macroeconomic data, information content decays rapidly beyond 3--6 recent months: distant lags (9--12) contain limited incremental information once near lags are conditioned upon, especially for high-frequency cyclical variables. Setting $\alpha=3$ down-weights distant lags, concentrating prior mass on recent dynamics where signal-to-noise is highest. This choice reflects \emph{statistical regularization}, not a behavioral assumption.

    Cross-variable shrinkage is handled automatically via residual-variance ratios $\sigma_i^2/\sigma_j^2$ from univariate AR benchmarks, ensuring variables with different scales receive appropriately calibrated shrinkage. The recursive output (\texttt{results/forecasts/hyperparameters\_evolution.csv}) records poster ior means for $\lambda$ at each forecast origin, making regularization evolution transparent.

    \paragraph{Recursive pseudo out-of-sample forecasting}
    To approximate real-time forecasting, I use an expanding-window design with an initial estimation sample 1985M1--2000M12 and forecast origins running from 2001M1 through 2019M11. At each origin date, the model is re-estimated using data available up to that date, the hierarchical shrinkage parameters are updated within the \texttt{BVAR} framework, and multi-horizon forecasts are produced. Forecasts and auxiliary objects are saved to disk, including aligned forecast--actual datasets and a time series of hyperparameter summaries (\texttt{results/forecasts/hyperparameters\_evolution.csv}). Because the exercise uses the latest-available vintage of macro series, it is best interpreted as \emph{pseudo} out-of-sample rather than fully real-time.

\fi

% --- Section 3: Econometric Framework ---
\section{Empirical design}

\ifsubmission
    \subsection{Model specification and hierarchical shrinkage}
    For each information set, I estimate a reduced-form VAR($p=12$) with Minnesota-style prior and hierarchical prior selection \citep{KuschnigVashold2021,GLP2015}. The hierarchical approach treats shrinkage intensity $\lambda$ as a hyperparameter optimized via marginal likelihood rather than fixed \emph{a priori}. As the information set expands (Small $\to$ Medium $\to$ Full), the hierarchical procedure endogenously tightens shrinkage to control overfitting, ensuring fair cross-model comparisons. Section~\ref{sec:lambda_evolution} documents how posterior-mean $\lambda$ declines systematically with model size.

    \subsection{Pseudo out-of-sample evaluation and nested-model inference}
    I use an expanding-window pseudo out-of-sample design with forecast origins spanning 2001M1--2019M11. Accuracy is summarized by RMSFE on annualized cumulative growth targets; forecasts are compared to no-change and AR(1) benchmarks.

    \paragraph{Nested-model inference caveat.} Because the information sets are nested (Small $\subset$ Medium $\subset$ Full), standard Diebold--Mariano equal-accuracy tests exhibit size distortions under the null of equal forecast performance \citep{ClarkMcCracken2001}: the null distribution of loss differentials is non-standard when the larger model nests the smaller, and conventional critical values may over-reject. I therefore treat pairwise DM tests as \emph{descriptive evidence} and supplement them with Clark--West MSPE-adjusted tests \citep{ClarkWest2007}, which correct for the upward bias in nested-model loss differentials (Appendix Table~\ref{tab:clark_west}). In the main text, I emphasize the \emph{magnitude and stability} of RMSFE differences rather than sharp statistical rejection claims.

    \subsection{Forecast-revision diagnostics}
    To assess whether expanding the information set alters forecast-updating patterns, I estimate error-on-revision regressions following \citet{CG2015}:
    \begin{equation}
        \text{FE}_{t,h}^{(m)} = \alpha_h + \beta_h \, \text{FR}_{t,h}^{(m)} + \varepsilon_{t,h},
        \label{eq:cg}
    \end{equation}
    where $\text{FE}_{t,h}^{(m)} = z_{t+h} - \hat{z}_{t+h|t}^{(m)}$ is the forecast error for model $m$ and $\text{FR}_{t,h}^{(m)} = \hat{z}_{t+h|t}^{(m)} - \hat{z}_{t+h|t-1}^{(m)}$ is the forecast revision (the change in the $t+h$ forecast made one period apart). Standard errors are Newey--West HAC with lag truncation parameter $h$.

    Under rational expectations, $\beta_h=0$: forecast revisions should be orthogonal to ex-post errors. A positive $\beta_h$ indicates forecast revisions move in the correct direction but with insufficient magnitude relative to subsequent realizations (\emph{underreaction}). A negative $\beta_h$ suggests revisions systematically overshoot (\emph{overreaction}). In the context of \emph{model-based forecasts} (as opposed to survey expectations of economic agents), $\beta_h$ measures the \emph{internal consistency of the forecasting model's probability updates}. Systematic deviations from zero could reflect: (i) prior-induced conservatism (tight shrinkage mechanically attenuates revisions), (ii) model misspecification (omitted variables or functional-form error), (iii) structural instability (regime shifts during the sample), or (iv) dynamics \emph{consistent with} belief-distortion mechanisms such as diagnostic expectations \citep{BordaloGS2018,Bordalo2020}, whereby sentiment proxies households' overweighting of recent inflation signals. These diagnostics are \emph{suggestive} rather than causally identifying: we cannot distinguish which mechanism drives observed $\beta_h$ patterns without additional structure.
\else

    \subsection{Hierarchical Bayesian VAR: Regularization and Hyperparameter Learning}

    The core methodology rests on a reduced-form VAR estimated under a hierarchical Minnesota-style prior that makes shrinkage intensity data-driven rather than fixed by assumption. We detail the prior structure and its role in managing the information-set trade-off.

    For each of the three nested specifications, we estimate a BVAR with $p=12$ monthly lags:
    \begin{equation}
        y_t = c + \sum_{\ell=1}^{p} B_{\ell} y_{t-\ell} + u_t,\qquad u_t \sim \mathcal{N}(0, \Sigma),
        \label{eq:varest}
    \end{equation}
    where $y_t$ is the vector of observables. The Minnesota prior encodes a prior belief that macroeconomic variables follow near-unit-root processes (i.e., random walks), consistent with the persistence observed in many economic series.

    Stacking observations yields $Y = X\Phi + U$, where $\Phi$ collects $(c,B_1,\dots,B_p)$, and we impose a Gaussian prior on $\Phi$ conditional on $\Sigma$:
    \[
        \mathrm{vec}(\Phi)\mid \Sigma,\lambda \sim \mathcal{N}\!\left(\mathrm{vec}(\underline{\Phi}),\, \Sigma \otimes \underline{\Omega}(\lambda)\right),
        \qquad
        \Sigma \sim \mathcal{IW}(\underline{S},\underline{\nu}).
    \]

    The prior mean $\underline{\Phi}$ encodes a random-walk belief: each variable's first own lag receives a prior mean of 1, while other coefficients are centered at zero. The prior covariance matrix $\underline{\Omega}(\lambda)$ incorporates lag decay and cross-variable scaling:
    \[
        \mathbb{V} \left[(B_\ell)_{ij}\mid \lambda\right]=
        \begin{cases}
            \lambda^2/\ell^{\alpha},                                & i=j,     \\[2pt]
            (\lambda^2/\ell^{\alpha})\cdot (\sigma_i^2/\sigma_j^2), & i\neq j,
        \end{cases}
    \]
    where $\ell$ indexes the lag, $\alpha=3$ is the lag-decay parameter (calibrated to reflect rapid information decay and potential recency bias in monthly data; see implementation discussion in Section 2.2), $\sigma_i^2$ are univariate AR benchmark residual variances, and $\lambda$ is the overall tightness (shrinkage intensity) hyperparameter with mode 0.05 and hierarchical learning.

    \paragraph{Data-driven hyperparameter selection via hierarchical shrinkage.}
    The key departure from ad hoc prior calibration is that $\lambda$ is \emph{endogenized} as a hyperparameter with its own hyperprior. Rather than fixing $\lambda$ (e.g., at a conventional 0.1 or 0.2), we treat it as an unknown to be learned from the data's marginal likelihood:
    \[
        p(Y\mid \lambda) = \int p(Y\mid \Phi,\Sigma)\,p(\Phi,\Sigma\mid \lambda)\,d\Phi\,d\Sigma.
    \]
    We place a Gamma hyperprior on $\lambda$ and search over its posterior mode through Metropolis--Hastings steps embedded in the BVAR estimation routine (following the implementation in \citet{GLP2015}). This approach has three advantages. First, it eliminates the need for subjective prior calibration, making comparisons across models of different dimensions more fair---each model learns its own optimal shrinkage from the data. Second, it provides a transparent trace of how regularization intensity changes as the information set expands; we document this below. Third, the posterior draws for $\lambda$ allow us to quantify uncertainty in the optimal shrinkage level.

    The same hierarchical treatment is applied to additional shrinkage components (sum-of-coefficients and dummy-initial-observation priors), which further help the model accommodate potential unit-root behavior and nonstationarity while guarding against over-parameterization.

    \subsection{Empirical Implementation: Expanding-Window Pseudo Out-of-Sample Design}

    We conduct recursive forecasting with an expanding window from 1985M1 through 2019M12. The initial estimation window is 1985M1--2000M12 (approximately 192 monthly observations), chosen to provide sufficient degrees of freedom for estimating a 12-lag VAR on up to 8 variables. Beginning at forecast origin $T=2001\text{M}1$, we:

    \begin{enumerate}
        \item Re-estimate the BVAR using all data from 1985M1 through $T$;
        \item Jointly optimize $\lambda$ and other hyperparameters via the hierarchical prior's marginal likelihood;
        \item Generate $h$-step-ahead point forecasts (posterior predictive means) for $h\in\{1,3,12\}$;
        \item Expand the sample by one month to $T+1$ and repeat.
    \end{enumerate}

    This expanding-window design mimics a practitioner's real-time forecasting environment but uses the final-vintage data (pseudo out-of-sample rather than fully real-time). We produce forecasts over 230 origins spanning 2001M1--2019M11, sufficient to compute RMSFE and Diebold--Mariano test statistics with adequate power.

    \subsection{Forecast Evaluation and the Revision Diagnostic}

    \paragraph{Forecast accuracy.}
    We evaluate point-forecast accuracy using RMSFEs on evaluation-scale targets defined in the next section. Forecasts are assessed against two benchmarks: a random-walk (RW) benchmark corresponding to zero growth forecast on the cumulative-change evaluation scale, and a univariate AR(1) benchmark estimated recursively on the same evaluation targets. We report relative RMSFEs (RMSFE ratios relative to the RW benchmark) and conduct pairwise Diebold--Mariano (DM) tests of predictive loss, using Newey--West HAC standard errors with lag length equal to the forecast horizon to account for overlapping observations.

    \paragraph{Expectation updating diagnostics: The Coibion-Gorodnichenko regression.}
    To assess whether forecast revisions exhibit systematic biases, we estimate the regression
    \begin{equation}
        (z_{t,h} - \hat{z}_{t,h|t}^{(m)}) = \alpha_h + \beta_h r_{t,h}^{(m)} + \varepsilon_{t,h},
        \label{eq:cg}
    \end{equation}
    where $z_{t,h}$ is the realized value of the evaluation-scale target from origin $t$ to $t+h$, $\hat{z}_{t,h|t}^{(m)}$ is the model-implied forecast from model $m$, and $r_{t,h}^{(m)}=\hat{z}_{t,h|t}^{(m)} - \hat{z}_{t,h|t-1}^{(m)}$ is the forecast revision (the change in the forecast for the same target date made one period apart).

    Under rational expectations with no forecast bias, $\beta_h=0$. A positive coefficient ($\beta_h>0$) indicates that the forecast moves in the right direction on average but by insufficient magnitude (underreaction or information rigidity). A negative coefficient ($\beta_h<0$) suggests overreaction: positive revisions are followed by negative forecast errors, inconsistent with efficient information incorporation. In the context of a model-based forecasting system, this diagnostic measures the internal consistency of probability updates rather than structural beliefs; a systematic positive $\beta_h$ might indicate that the prior is too tight and revisions lack sufficient force, while negative $\beta_h$ might signal overfitting to low-frequency trends. We report estimates of $\beta_h$ with Newey--West HAC standard errors and compute differences $\Delta\beta_h = \beta_h^{(\text{Full})} - \beta_h^{(\text{Small})}$ to quantify sentiment's incremental effect on the revision pattern.

    Stacking observations yields $Y = X\Phi + U$, where $\Phi$ collects $(c,B_1,\dots,B_p)$.
    I impose a Minnesota-style Gaussian prior on $\Phi$ conditional on $\Sigma$:
    \[
        \mathrm{vec}(\Phi)\mid \Sigma,\lambda \sim \mathcal{N}\!\left(\mathrm{vec}(\underline{\Phi}),\, \Sigma \otimes \underline{\Omega}(\lambda)\right),
        \qquad
        \Sigma \sim \mathcal{IW}(\underline{S},\underline{\nu}),
    \]
    where $\underline{\Phi}$ encodes the random-walk / near-random-walk belief on own first lags, and $\underline{\Omega}(\lambda)$ implements lag decay and cross-variable shrinkage. In particular, for coefficient $(B_\ell)_{ij}$,
    \[
        \mathbb{V} \left[(B_\ell)_{ij}\mid \lambda\right]=
        \begin{cases}
            \lambda^2/\ell^{\alpha},                                & i=j,     \\[2pt]
            (\lambda^2/\ell^{\alpha})\cdot (\sigma_i^2/\sigma_j^2), & i\neq j,
        \end{cases}
    \]
    with lag-decay $\alpha$ fixed at 2 in the baseline implementation and $\sigma_i^2$ set from residual scales in univariate AR benchmarks.

    The key departure from ad hoc calibration is that the overall tightness $\lambda$ is \emph{endogenized}. Following \citet{GLP2015}, the code treats $\lambda$ (and additional shrinkage components) as hyperparameters with proper hyperpriors and explores them via a Metropolis--Hastings step implemented in \texttt{BVAR}. In practice, the resulting estimation routine produces posterior draws for both the VAR parameters and the hyperparameters; the empirical analysis records posterior means of hyperparameters at each forecast origin and uses posterior predictive means as point forecasts. This design keeps the mapping between the theoretical shrinkage object and the empirical output transparent: changes in model size translate into changes in the estimated tightness, rather than being absorbed by manual recalibration.


    \subsection{Pseudo out-of-sample forecasting and evaluation}

    I implement an expanding-window pseudo out-of-sample exercise. The initial estimation window is 1985M1--2000M12. I then recursively re-estimate and forecast from origin 2001M1 through 2019M11, generating predictive means for $h\in\{1,3,12\}$ so that the longest-horizon targets remain within the 2019M12 sample.

    \medskip
    \noindent\textbf{Forecast accuracy.} For target $i$ and horizon $h$, compute RMSFE,
    \[
        \mathrm{RMSFE}_{i,h}=\left(\frac{1}{P}\sum_{t=1}^{P} (y_{i,t+h}-\hat y_{i,t+h|t})^2\right)^{1/2},
    \]
    and report relative RMSFEs versus the no-change and AR(1) benchmarks. Differences in predictive loss are assessed using Diebold--Mariano tests \cite{DieboldMariano1995} with Newey--West standard errors \cite{NeweyWest1987}, following the implementation in the analysis code.

\fi

% --- Section 4: Interpretation ---
\section{Results}
\ifsubmission
    This section reports the core evidence on the incremental role of sentiment: a forecast-accuracy horse race and a forecast-revision diagnostic.

    \subsection{Forecast accuracy}
    Table~\ref{tab:rmsfe} summarizes RMSFEs for inflation and industrial production across the three nested information sets. The main pattern is that expanding the information set improves some short-horizon forecasts (especially for industrial production), but sentiment adds little incremental accuracy once financial variables are included; for inflation at longer horizons, the baseline macro specification attains the lowest RMSFE.

    \begin{table}[h!]
\centering
\small
\caption{Root Mean Squared Forecast Errors}
\label{tab:rmsfe}
\begin{tabular}{@{}lcccc@{}}
\toprule
model & variable & h1 & h3 & h12 \\
\midrule
Small & CPI & 3.482 & 2.671 & 1.341 \\
Small & INDPRO & 7.633 & 5.536 & 5.084 \\
Medium & CPI & 3.474 & 2.683 & 1.375 \\
Medium & INDPRO & 7.322 & 5.077 & 4.817 \\
Full & CPI & 3.542 & 2.670 & 1.312 \\
Full & INDPRO & 7.534 & 5.245 & 4.610 \\
\bottomrule
\end{tabular}
\vspace{0.5em}
\begin{minipage}{\textwidth}
\footnotesize
\textit{Notes:}
RMSFE (in percentage points for inflation and growth rates).\\
Sample period: 2001M1--2019M12 (230 forecast origins).\\
Forecast horizons: $h = \{1, 3, 12\}$ months ahead.\\
Models: Small (INDPRO, CPI, UNRATE, FEDFUNDS), Medium (+ GS10, SP500), Full (+ UMCSENT).\\
\end{minipage}
\end{table}


    \begin{figure}[H]
        \centering
        \includegraphics[width=0.92\textwidth]{fig1_rmsfe_comparison.png}
        \caption{Forecast accuracy by horizon (RMSFE; lower is better)}
        \label{fig:rmsfe_bar}
        \captionsetup{font=small}
        \vspace{-0.2cm}
        \begin{minipage}{0.92\textwidth}
            \footnotesize
            Notes: Bars report RMSFEs (evaluation scale) for each information set and horizon; values correspond to Table~\ref{tab:rmsfe}. Source: \texttt{results/tables/rmsfe\_results.csv}.
        \end{minipage}
    \end{figure}

    \subsection{Forecast revisions and the CG diagnostic}
    Table~\ref{tab:cg_regression} reports the \cite{CG2015} error-on-revision coefficients. For inflation, short-horizon coefficients are positive (underreaction) while long-horizon coefficients are near zero or slightly negative (overreaction), and richer information sets move long-horizon coefficients closer to the rational-expectations benchmark. For industrial production, coefficients are small and statistically weak across horizons.

    \begin{table}[h!]
\centering
\small
\caption{Coibion--Gorodnichenko Regression Results}
\label{tab:cg_regression}
\begin{tabular}{@{}lcccc@{}}
\toprule
term & estimate & std.error & statistic & p.value \\
\midrule
Small CPI h=1 & 2.4083 & 1.1820 & 2.0375 & 0.0428 \\
Small CPI h=3 & 0.6837 & 0.7595 & 0.9002 & 0.3690 \\
Small CPI h=12 & -0.5593 & 0.2998 & -1.8658 & 0.0634 \\
Small INDPRO h=1 & 0.7665 & 0.5953 & 1.2877 & 0.1992 \\
Small INDPRO h=3 & 0.8633 & 0.4804 & 1.7970 & 0.0737 \\
Small INDPRO h=12 & 0.1143 & 0.4971 & 0.2299 & 0.8184 \\
Medium CPI h=1 & 1.9173 & 0.6703 & 2.8606 & 0.0046 \\
Medium CPI h=3 & 0.5806 & 0.5064 & 1.1465 & 0.2529 \\
Medium CPI h=12 & -0.5482 & 0.2670 & -2.0532 & 0.0413 \\
Medium INDPRO h=1 & 0.6204 & 0.5064 & 1.2251 & 0.2219 \\
Medium INDPRO h=3 & 0.7987 & 0.4144 & 1.9273 & 0.0553 \\
Medium INDPRO h=12 & 0.3052 & 0.4824 & 0.6327 & 0.5276 \\
Full CPI h=1 & 0.8520 & 0.4452 & 1.9136 & 0.0570 \\
Full CPI h=3 & 0.2564 & 0.5871 & 0.4368 & 0.6627 \\
Full CPI h=12 & -0.6366 & 0.3195 & -1.9921 & 0.0476 \\
Full INDPRO h=1 & 0.3201 & 0.3793 & 0.8439 & 0.3997 \\
Full INDPRO h=3 & 0.2736 & 0.4115 & 0.6647 & 0.5069 \\
Full INDPRO h=12 & 0.2291 & 0.4400 & 0.5206 & 0.6032 \\
\bottomrule
\end{tabular}
\vspace{0.5em}
\begin{minipage}{\textwidth}
\footnotesize
\textit{Notes:}
OLS regression of forecast errors on forecast revisions: $(y_{t+h} - \hat{y}_{t+h|t}) = \alpha_h + \beta_h (\hat{y}_{t+h|t} - \hat{y}_{t+h|t-1}) + \varepsilon_{t+h}$.\\
Standard errors are Newey--West HAC-robust with lag truncation parameter equal to the forecast horizon.\\
Under rational expectations, $\beta_h = 0$. Positive values indicate under-reaction (sticky information),\\
while negative values suggest over-reaction consistent with diagnostic expectations.\\
Sample: 2001M1--2019M12. Variables: CPI (annualized inflation), INDPRO (industrial production growth).\\
\end{minipage}
\end{table}


    Because the information sets are nested (Small $\subset$ Medium $\subset$ Full), standard equal-accuracy tests can have nonstandard behavior under the null \cite{ClarkMcCracken2001}. I therefore treat Diebold--Mariano comparisons across nested models as suggestive and report Clark--West MSPE-adjusted tests as robustness \cite{ClarkWest2007} (Appendix Table~\ref{tab:clark_west}). The main interpretation emphasizes magnitudes and stability of RMSFE differences.

    \subsection{Economic Interpretation}

    The empirical patterns documented in Tables~\ref{tab:rmsfe} and \ref{tab:cg_regression} invite three economic interpretations that connect forecast performance to underlying information structures and expectation-formation mechanisms.

    Financial asset prices aggregate diverse market participants' expectations and embed real-time information about monetary policy, growth risks, and inflation via term spreads and equity valuations \citep{Estrella1998}. Sentiment indices, by contrast, are survey-based and reflect households' stated beliefs, which may lag or diverge from market pricing. If households' inflation expectations are already incorporated into wage-setting behavior and captured by realized unemployment dynamics (the Phillips curve channel), then sentiment provides limited \emph{independent signal} for forecasting models conditional on financial variables. This \emph{information-overlap hypothesis} is consistent with our RMSFE patterns: at short horizons, the Medium model (with financial variables) dominates, while adding sentiment (Full) yields no further accuracy gains. Financial markets' capacity to aggregate dispersed information efficiently may render household surveys redundant for the specific task of minimizing squared forecast errors.

    Long-horizon inflation forecasting is notoriously difficult: \citet{AtkesonOhanian2001} show a naive random-walk model is hard to beat, and \citet{StockWatson2007} document that inflation has become harder to forecast over time. At $h=12$, forecast accuracy is dominated by \emph{low-frequency trend inflation}, which is more stable and better captured by the Federal Reserve's policy stance (the federal funds rate in our Small model) than by high-frequency financial or sentiment fluctuations. Adding variables increases parameter uncertainty, and hierarchical shrinkage cannot fully offset this curse of dimensionality when the signal-to-noise ratio is low. Therefore, the Small model's h=12 dominance (Table~\ref{tab:rmsfe}) reflects a well-known \emph{parsimony advantage}: at long horizons, simpler models avoid overfitting to transient correlations and better anchor to persistent policy regimes. This finding aligns with the broader inflation-forecasting literature emphasizing the value of simple benchmarks.

    If sentiment captures \emph{diagnostic expectations}---whereby households overweight recent inflation experiences when forming beliefs \citep{BordaloGS2018,Bordalo2020}---then its primary contribution may lie in \emph{disciplining forecast revisions} rather than improving point accuracy. This interpretation is consistent with Table~\ref{tab:cg_regression}: sentiment's inclusion alters revision coefficients (moving long-horizon $\beta_h$ closer to zero, attenuating overreaction) even when RMSFE is unchanged (Table~\ref{tab:rmsfe}). Sentiment thus serves as a \emph{belief-distortion proxy}, informative for understanding expectation formation and forecast-updating dynamics but not necessarily for minimizing squared forecast errors. This distinction---between forecast \emph{accuracy} (RMSFE performance) and forecast \emph{discipline} (consistency of probability updates)---is central to the paper's contribution. Soft information may improve how models revise forecasts in response to new data, even if terminal forecast accuracy remains similar.

    \subsection{Limitations and Future Research}

    Our analysis has several notable limitations that motivate extensions.

    We use pseudo-out-of-sample forecasts with final-vintage data rather than real-time vintages that forecasters would have actually observed at each origin. This abstracts from nowcasting and data-revision challenges. A natural extension would re-implement the analysis using FRED real-time database vintages, testing whether sentiment's predictive content (or lack thereof) survives the revision process and whether sentiment indices themselves are robust to later revisions.

    While hierarchical selection endogenizes shrinkage intensity $\lambda$ via marginal likelihood, the hyperprior mode (0.05) and lag-decay parameter ($\alpha=3$) are pre-specified. A robustness exercise varying these hyperprior parameters would clarify whether findings are prior-driven artifacts or robust to reasonable prior perturbations. Documenting how RMSFE rankings and CG coefficients change across a grid of $\lambda$ modes and $\alpha$ values would strengthen inference.

    The sample spans 1985M1--2019M12 but does not explicitly model structural breaks. The Great Recession and subsequent low-inflation regime may have altered inflation dynamics and the information content of sentiment. Time-varying parameter BVARs or Markov-switching specifications could capture regime-dependent patterns, potentially revealing that sentiment matters more in high-uncertainty episodes or crisis periods.

    We use a single sentiment index (University of Michigan). Alternative measures (Conference Board consumer confidence, text-based sentiment from news media) or decomposing the Michigan index into sub-components (current conditions vs expectations) might reveal richer patterns. Exploring whether the expectations sub-component better predicts long-horizon inflation or whether text-based sentiment captures narrative-driven belief shifts would extend the analysis.

    We evaluate only \emph{point forecasts} (RMSFEs). Density forecast evaluation via log predictive scores or probability integral transforms could show sentiment improves forecast \emph{calibration}---e.g., better tail-risk assessment or sharper predictive distributions---even if RMSFE is unchanged. This would further distinguish sentiment's role in capturing uncertainty from its role in minimizing expected squared loss.

    These limitations suggest concrete next steps: real-time vintage forecasting, prior-sensitivity grids, time-varying parameter VARs, multivariate sentiment proxies, and density forecast assessment. Each would sharpen the economic interpretation and test the robustness of our central finding that sentiment adds limited point-forecast accuracy but may discipline forecast-revision dynamics.
\else
    This section interprets the empirical outputs produced by the forecasting pipeline. All numerical results cited below correspond to the CSV tables in \texttt{results/tables/} and figures in \texttt{results/figures/}.

    \subsection{Forecast accuracy and the role of the information set}

    Table~\ref{tab:rmsfe_main} summarizes forecast accuracy for CPI inflation and industrial production growth across the three information sets, along with two benchmarks. Figure~\ref{fig:rmsfe_bar} visualizes the same RMSFEs, while Figure~\ref{fig:relrmsfe} reports the corresponding relative performance against the no-change benchmark.

    \paragraph{Inflation forecasts: Substantial benchmark improvements; limited incremental gains from expanded information sets.}
    All BVAR specifications decisively outperform the random-walk benchmark at every horizon. At $h=1$, the Medium model achieves the lowest RMSFE (2.982 percentage points), beating the benchmark by 27\% (relative RMSFE 0.735). The Small and Full models perform comparably (3.468 and 3.128, respectively), yielding relative RMSFEs of 0.854 and 0.771. At $h=3$, performance remains strong: RMSFEs range from 2.500 (Medium) to 2.643 (Small), translating to 19--24\% improvements over the benchmark. The most pronounced gains emerge at the twelve-month horizon: all specifications achieve relative RMSFEs below 0.57, representing 43--45\% reductions in forecast error. Notably, the Small model delivers the lowest h=12 RMSFE (1.305 percentage points), marginally outperforming Full (1.330) and Medium (1.349).

    \emph{Interpreting the absence of information-set gains at long horizons.} The Small model's superior long-horizon performance is economically revealing. At $h=12$, inflation dynamics are dominated by low-frequency movements in trend inflation and inflation expectations. The baseline specification---industrial production, CPI, unemployment, and the federal funds rate---already encodes the key determinants of these trends through the Phillips curve (unemployment-inflation linkage) and monetary policy stance (federal funds rate). Adding financial variables (Medium) introduces signals primarily about near-term cyclical pressures, which contribute little to twelve-month inflation forecasting beyond what monetary aggregates already capture. Including sentiment (Full) likewise fails to improve accuracy: while consumer inflation expectations are theoretically relevant for wage/price setting, the Michigan sentiment index appears to provide no marginal information beyond that embedded in realized unemployment and policy rates.

    This null result does \emph{not} imply sentiment is uninformative. Rather, it suggests that, in this design, expanding the information set changes revision dynamics more than it changes point-forecast RMSFE. In the CG diagnostic, moving from Small to Medium markedly reduces the short-horizon underreaction coefficient for inflation, while the additional step from Medium to Full does not further reduce short-horizon underreaction. At the twelve-month horizon, the Full model's coefficient is closer to zero than Medium's, consistent with richer information sets reducing long-horizon overreaction, though these differences are estimated with substantial uncertainty. The distinction between forecast accuracy and forecast \emph{discipline} is therefore central: soft information may alter internal updating patterns even when incremental RMSFE gains are small.

    \emph{Prior calibration and short-horizon performance.} The aggressive shrinkage ($\lambda$ mode 0.05) and accelerated lag decay ($\alpha=3$) jointly discipline short-horizon forecasts. At $h=1$, the prior forces heavy reliance on the random-walk component, guarding against overfitting to transient price shocks. The Medium model's dominance at this horizon (RMSFE 2.982 vs. Small 3.468) reflects financial variables' capacity to capture imminent demand pressures: stock returns and yield spreads encode market expectations about monetary policy and cyclical shifts, which materialize over monthly intervals. Adding sentiment degrades $h=1$ performance relative to Medium (Full RMSFE 3.128), consistent with sentiment containing low-frequency trend information that is less diagnostic of month-to-month fluctuations. At $h=3$, the information-set ranking compresses (Medium 2.500, Full 2.538, Small 2.643), indicating that distinctions among specifications diminish as the forecast horizon extends toward the range where all models rely primarily on prior-induced persistence.

    \paragraph{Industrial production forecasts: Financial variables help at short horizons; long-horizon challenges persist.}
    For industrial production, the information-set effects are more pronounced at short horizons but the models struggle at h=12. At $h=1$, the Medium model delivers the lowest RMSFE (7.315 percentage points), beating Small (7.649) by 4.4\% and outperforming the random-walk benchmark (8.012) by 8.7\% (relative RMSFE 0.913). At $h=3$, Medium's advantage is even larger: RMSFE 4.966 versus Small's 5.558 (11\% improvement) and the benchmark's 5.680 (relative RMSFE 0.874). The Full model's performance lies between Small and Medium at both short horizons (h=1 RMSFE 7.424, h=3 RMSFE 5.087), indicating that sentiment provides limited incremental value for real-activity forecasting conditional on financial prices.

    At the twelve-month horizon, all BVAR specifications \emph{underperform} the random-walk benchmark: relative RMSFEs range from 1.018 (Full) to 1.185 (Small), meaning the models' forecast errors are 2--19\% \emph{larger} than simply projecting zero growth. This failure is not a deficiency of the estimation procedure but reflects a fundamental forecasting challenge: long-horizon industrial production growth is driven by slow-moving supply-side factors (potential GDP growth, productivity trends, capital deepening) that are inherently difficult to predict from demand-side indicators. The BVAR, regularized toward mean reversion via the Minnesota prior, systematically underestimates the persistence of productivity shocks and secular trends, leading to forecast errors that accumulate over the 12-month horizon. Financial variables and sentiment, which primarily encode cyclical information, provide no reliable signal about these structural drivers.

    \emph{Implications for prior calibration.} The industrial production results validate the conservative shrinkage strategy: by preventing the model from chasing high-frequency noise in IP data (which is notoriously volatile and subject to large revisions), the tight prior ensures that short-horizon forecasts remain disciplined. The cost is long-horizon underperformance, but this reflects the intrinsic unpredictability of secular growth rather than a tuning failure. Alternative prior specifications (e.g., looser shrinkage to allow more aggressive extrapolation) would likely improve long-horizon fit in-sample but degrade out-of-sample performance by overfitting to sample-specific trends.

    \begin{figure}[H]
        \centering
        \includegraphics[width=0.92\textwidth]{fig1_rmsfe_comparison.png}
        \caption{Forecast performance by horizon (RMSFE; lower is better)}
        \label{fig:rmsfe_bar}
        \captionsetup{font=small}
        \vspace{-0.2cm}
        \begin{minipage}{0.92\textwidth}
            \footnotesize
            Notes: Bars report RMSFEs on the evaluation scale for each BVAR information set and horizon. Values correspond to Table~\ref{tab:rmsfe_main}, Panel A, and are generated from \texttt{results/tables/}\allowbreak\texttt{rmsfe\_results.csv}.
        \end{minipage}
    \end{figure}

    \begin{table}[H]
        \centering
        \begin{threeparttable}
            \caption{Forecast accuracy across information sets}
            \label{tab:rmsfe_main}
            \begin{tabular}{ll S[table-format=1.3] S[table-format=1.3] S[table-format=1.3]}
                \toprule
                                &        & \multicolumn{1}{c}{$h=1$} & \multicolumn{1}{c}{$h=3$} & \multicolumn{1}{c}{$h=12$} \\
                \midrule
                \multicolumn{5}{l}{\textit{Panel A. RMSFE (percentage points, annualized)}}                                   \\
                Small           & CPI    & 3.468                     & 2.643                     & 1.305                      \\
                Medium          & CPI    & 2.982                     & 2.500                     & 1.349                      \\
                Full            & CPI    & 3.128                     & 2.538                     & 1.330                      \\
                \addlinespace
                Small           & INDPRO & 7.649                     & 5.558                     & 4.998                      \\
                Medium          & INDPRO & 7.315                     & 4.966                     & 4.371                      \\
                Full            & INDPRO & 7.424                     & 5.087                     & 4.387                      \\
                \addlinespace
                RW benchmark    & CPI    & 4.057                     & 3.267                     & 2.381                      \\
                AR(1) benchmark & CPI    & 3.226                     & 2.933                     & 1.535                      \\
                RW benchmark    & INDPRO & 8.012                     & 5.680                     & 4.294                      \\
                AR(1) benchmark & INDPRO & 8.117                     & 4.788                     & 6.678                      \\
                \addlinespace
                \multicolumn{5}{l}{\textit{Panel B. Relative RMSFE vs random-walk benchmark}}                                 \\
                Small           & CPI    & 0.854                     & 0.809                     & 0.548                      \\
                Medium          & CPI    & 0.735                     & 0.765                     & 0.567                      \\
                Full            & CPI    & 0.771                     & 0.777                     & 0.559                      \\
                \addlinespace
                Small           & INDPRO & 0.955                     & 0.979                     & 1.164                      \\
                Medium          & INDPRO & 0.913                     & 0.874                     & 1.018                      \\
                Full            & INDPRO & 0.927                     & 0.896                     & 1.022                      \\
                \bottomrule
            \end{tabular}
            \begin{tablenotes}[flushleft]
                \footnotesize
                \item Notes: Panel A reports RMSFEs computed from the expanding-window pseudo out-of-sample forecasts (\texttt{results/tables/}\allowbreak\texttt{rmsfe\_results.csv}) and benchmark RMSFEs (\texttt{results/tables/}\allowbreak\texttt{rw\_rmsfe\_benchmark.csv}, \texttt{results/tables/}\allowbreak\texttt{ar1\_rmsfe\_benchmark.csv}). The no-change benchmark corresponds to a random walk in levels (zero forecast on the cumulative-growth evaluation scale). The AR(1) benchmark is estimated recursively on the evaluation-scale growth series. Panel B reports RMSFEs relative to the no-change benchmark (\texttt{results/tables/}\allowbreak\texttt{relative\_rmsfe\_vs\_rw.csv}).
            \end{tablenotes}
        \end{threeparttable}
    \end{table}

    \begin{figure}[H]
        \centering
        \includegraphics[width=0.92\textwidth]{fig2_relative_rmsfe.png}
        \caption{Relative RMSFE versus no-change benchmark (random walk)}
        \label{fig:relrmsfe}
        \captionsetup{font=small}
        \vspace{-0.2cm}
        \begin{minipage}{0.92\textwidth}
            \footnotesize
            Notes: The figure plots RMSFE for each model divided by the RMSFE of the no-change benchmark at each horizon. Values below one indicate improvement over the benchmark. The plotted values correspond to \texttt{results/tables/}\allowbreak\texttt{relative\_rmsfe\_vs\_rw.csv}.
        \end{minipage}
    \end{figure}

    \subsection{Benchmarks, statistical uncertainty, and time variation}
    Because the information sets are nested (Small $\subset$ Medium $\subset$ Full), equal-accuracy tests based on loss differentials can be nonstandard under the null for nested model comparisons \cite{ClarkMcCracken2001}. I therefore interpret Diebold--Mariano tests primarily as descriptive checks and supplement them with Clark--West MSPE-adjusted tests for the nested comparisons \cite{ClarkWest2007} (Appendix Table~\ref{tab:clark_west}). In the main text, the emphasis is on the magnitude and stability of RMSFE differences rather than on sharp statistical dominance across closely related specifications.
    Formal comparisons of predictive accuracy use Diebold--Mariano tests on squared-error loss differentials with Newey--West standard errors. Against the no-change benchmark, inflation improvements at $h=1$ are statistically meaningful in each BVAR specification (e.g., the small model yields $t=-3.12$, $p=0.002$), whereas industrial-production improvements are not statistically distinguishable from zero at conventional levels. Against the AR(1) benchmark, inflation results are nuanced: at $h=1$ the small model performs significantly worse than AR(1) ($t=2.47$, $p=0.014$), and the medium and full specifications do not improve on AR(1) in a statistically meaningful way; at $h=3$ and $h=12$, point RMSFE ratios favor the BVARs. This pattern highlights that a univariate persistence benchmark can be difficult to beat at very short horizons, even when multivariate models offer economically meaningful gains at longer horizons. Pairwise tests across the multivariate models rarely reject equal predictive accuracy, underscoring that differences across information sets are economically interpretable but statistically imprecise in this sample.

    Rolling relative RMSFEs (Figure~\ref{fig:rolling_rw}) highlight time variation once a 60-month rolling window is available. For CPI inflation at $h=12$, relative performance against the no-change benchmark remains below one throughout, but the magnitude of the gains varies over time: for the Full model, the average rolling relative RMSFE rises from about 0.52 before 2013 to about 0.70 thereafter (still an improvement over the benchmark). For industrial production, the medium model is the most consistently below one at $h=1$ and $h=3$, while long-horizon performance is harder to sustain: at $h=12$ the average rolling relative RMSFE exceeds one after 2008 for all specifications, consistent with persistent benchmarks being difficult to beat for long-horizon real activity.

    \begin{figure}[H]
        \centering
        \includegraphics[width=0.96\textwidth]{fig_rolling_relative_rmsfe_rw.png}
        \caption{Rolling relative RMSFE versus no-change benchmark}
        \label{fig:rolling_rw}
        \captionsetup{font=small}
        \vspace{-0.2cm}
        \begin{minipage}{0.96\textwidth}
            \footnotesize
            Notes: Rolling-window relative RMSFEs (window length 60 months). Values below one indicate improvement over the no-change benchmark. The figure is generated from \texttt{results/tables/}\allowbreak\texttt{rolling\_relative\_rmsfe\_vs\_rw.csv}.
        \end{minipage}
    \end{figure}

    \subsection{Forecast-error decomposition and forecast-path diagnostics}
    Theil-type MSE decompositions (Figure~\ref{fig:error_decomp}) clarify what drives forecast errors across horizons. Decompose $\mathrm{MSE}=\mathbb{E}[(y-\hat y)^2]$ into a bias component (mean error), a variance component (dispersion mismatch), and a covariance component (imperfect co-movement), and report each as a share of total MSE. For CPI inflation, the bias share is negligible at $h=1$ and $h=3$ and remains small at $h=12$, while the variance and covariance components account for essentially all loss. Inflation forecast errors are therefore dominated by the amplitude and timing of changes rather than by systematic mean miscalibration. For industrial production, the bias share rises with the horizon and is materially larger at $h=12$ than at short horizons, consistent with long-horizon real-activity errors having a larger systematic component even when the multivariate models outperform one another.

    \begin{figure}[H]
        \centering
        \includegraphics[width=0.96\textwidth]{fig_error_decomposition.png}
        \caption{Forecast error decomposition (Theil MSE shares)}
        \label{fig:error_decomp}
        \captionsetup{font=small}
        \vspace{-0.2cm}
        \begin{minipage}{0.96\textwidth}
            \footnotesize
            Notes: Decomposition of mean squared forecast error into bias, variance, and covariance shares. Values correspond to \texttt{results/tables/}\allowbreak\texttt{error\_decomposition.csv}.
        \end{minipage}
    \end{figure}

    The forecast-path plots in Appendix~\ref{app:figures} provide a complementary view. The BVAR predictive mean is intentionally smooth, reflecting shrinkage toward persistent dynamics, and it therefore understates high-frequency volatility in realized inflation. Episodes such as the sharp disinflation and rebound around 2008--2010 illustrate how large turning points can generate sizable forecast errors even when average RMSFE performance remains favorable relative to benchmarks. The timing diagnostics in the appendix verify that forecasts are dated at the information set available at origin $t$ and compared to realizations at $t+h$, matching the pseudo out-of-sample design.

    \subsection{Forecast revisions and systematic expectation-updating patterns}

    Table~\ref{tab:cg_main} reports estimates of the forecast-error-on-revision regression (Equation~\ref{eq:cg}) for inflation and industrial production across the three information sets. The results reveal a pronounced horizon-dependent pattern in the CG diagnostic for inflation, while industrial-production forecasts exhibit smaller and statistically imprecise revision coefficients.

    \paragraph{Inflation: Underreaction at short horizons, overreaction at long horizons.}
    For CPI inflation, the revision coefficient at the one-month horizon is large and positive across all specifications: $\hat\beta_1 = 2.26$ in the Small model ($SE=1.19$, $p=0.060$), declining to $0.71$ in the Medium model ($SE=0.28$, $p=0.013$), and further to $0.93$ in the Full model ($SE=0.32$, $p=0.004$). Under the \citet{CG2015} interpretation, these positive coefficients indicate \emph{underreaction}: when the BVAR revises its inflation forecast upward at $t$, that revision moves in the correct direction on average but by insufficient magnitude to prevent a subsequent positive forecast error. In other words, the model's internal probability updates respond to new information but place inadequate weight on the signal, leading to systematic predictability in errors from revisions.

    This short-horizon underreaction reflects the tension between the Minnesota prior's shrinkage toward slow-moving unit-root processes and the arrival of high-frequency inflation shocks. The prior, calibrated with $\lambda$ mode at 0.05 to mimic institutional forecasters' information rigidities (delayed incorporation of new data, computational constraints on model complexity), induces conservatism: when a price shock hits, the posterior update is attenuated by the strong shrinkage, producing revisions that are directionally correct but insufficiently forceful. The decline in $\hat\beta_1$ from 2.26 (Small) to 0.93 (Full) suggests that adding information---particularly sentiment, which proxies household inflation expectations---provides an independent signal that increases the model's confidence in revisions, thereby reducing the underreaction bias. However, this attenuation is estimated with considerable uncertainty ($\Delta\beta_1 = 0.93 - 2.26 = -1.34$, $SE=1.24$, $p=0.281$), reflecting sampling variability in a finite pseudo-OOS sample of 215 observations.

    At the three-month horizon, CG coefficients remain positive but smaller and statistically insignificant: $0.69$ (Small, $p=0.387$), $0.56$ (Medium, $p=0.082$), and $0.69$ (Full, $p=0.066$). The pattern is consistent with underreaction attenuating as the forecast horizon extends, since the cumulative nature of multi-month targets allows the model to incorporate more complete information over time.

    At the twelve-month horizon, the sign reverses: CG coefficients are \emph{negative} across all specifications, indicating \emph{overreaction}. The Small model yields $\hat\beta_{12} = -0.52$ ($SE=0.32$, $p=0.109$), strengthening in magnitude to $-0.08$ in Medium ($p=0.684$) and $-0.03$ in Full ($p=0.897$). Negative $\beta_{12}$ means that upward forecast revisions are systematically followed by negative forecast errors, and vice versa---the hallmark of extrapolative forecasting or overfitting to low-frequency trends.

    \paragraph{Economic mechanisms underlying horizon-dependent biases.}
    The sign reversal from positive $\beta$ at $h=1$ to negative $\beta$ at $h=12$ reflects the interplay between prior-induced smoothness and trend persistence in the data. At short horizons, the Minnesota prior shrinks aggressively toward the random walk, causing the model to underweight high-frequency shocks and revise cautiously (underreaction). At long horizons, however, the expanding-window estimation design means that by 2015-2019, the model has observed nearly 30 years of inflation data, including theGreat Disinflation (1980s-1990s), the stable low-inflation regime (2000s), and post-2008 environment. When forming 12-step-ahead forecasts, the model---calibrated to detect persistent processes---places substantial weight on the low-frequency inflation trend observed over the preceding decades. If the model revises its 12-month forecast upward (e.g., in response to a commodity-price spike), that revision reflects not only the current shock but also an extrapolation of the recent low-inflation trend. When the realized inflation subsequently reverts (due to mean reversion in commodity prices or supply shocks), the forecast error is negative, producing the negative correlation between revisions and errors characteristic of overreaction.

    Adding sentiment does \emph{not} eliminate the long-horizon overreaction; in fact, the Full model's $\hat\beta_{12}$ is closer to zero ($-0.03$) than the Small model's ($-0.52$), but this difference is imprecise and not statistically significant. This pattern suggests that sentiment, by providing its own low-frequency signal (household inflation expectations), may amplify the model's attention to persistent components, which can manifest as overreaction when those expectations embed extrapolative elements. Importantly, this is not necessarily a flaw: if sentiment genuinely reflects households' inflation beliefs and those beliefs influence wage/price setting, the model \emph{should} incorporate them even if doing so sometimes leads to forecast errors when those beliefs prove overly pessimistic or optimistic. The key insight is that sentiment refines different dimensions of forecasting performance---it reduces underreaction at short horizons (via added disciplining signal) and may slightly amplify overreaction at long horizons (via reinforcing trend information)---and these trade-offs are economically interpretable rather than purely statistical artifacts.

    \paragraph{Magnitude and uncertainty of changes in short-horizon underreaction.}
    The difference $\Delta\beta_1$ reported in the project outputs compares the Full and Small specifications, i.e., the combined effect of expanding the information set from core macro variables to the full set that includes both financial variables and sentiment. The point estimate is negative, indicating that the short-horizon inflation underreaction coefficient is smaller in the Full specification than in the Small specification, but the estimate is imprecise (the associated standard error is large and the null cannot be rejected at conventional levels). This uncertainty highlights a general limitation of finite pseudo out-of-sample samples: revision-based diagnostics can detect large, stable biases, but are less powerful for isolating incremental changes across closely related specifications. When interpreting incremental effects of sentiment specifically, the Medium vs.\ Full comparison is the most relevant nested step and suggests only modest changes in short-horizon revision behavior.

    \paragraph{Industrial production: Absence of detectable revision biases.}
    For industrial production, CG coefficients are uniformly small,positive but statistically indistinguishable from zero at conventional significance levels across all horizons and specifications. For example, in the Small model at $h=1$, $\hat\beta=0.72$ ($p=0.202$); in the Full model, $\hat\beta=0.11$ ($p=0.756$). At $h=12$, coefficients range from $0.14$ (Small) to $0.22$ (Full), with $p$-values exceeding 0.6. This absence of systematic bias could reflect two mechanisms. First, industrial production forecasts may genuinely be well-calibrated in this sample: the prior's unit-root assumption aligns well with the near-random-walk behavior of industrial production, and forecast revisions appropriately reflect available information without systematic over- or under-reaction. Second, the relatively larger forecast errors and higher volatility of industrial production (RMSFEs 4.4--7.6 percentage points across horizons, compared to 1.3--3.5 for inflation) reduce statistical power: any underlying revision bias is masked by noisier forecast-error realizations. Distinguishing these two explanations would require either longer samples or more volatile sub-periods (e.g., recession-specific analysis), which we defer to future work.

    \begin{table}[H]
        \centering
        \begin{threeparttable}
            \caption{Forecast error on forecast revision (CG regression)}
            \label{tab:cg_main}
            \begin{tabular}{llc S[table-format=-1.3] S[table-format=1.3] S[table-format=-1.2] S[table-format=1.3] c}
                \toprule
                Model  & Target & Horizon & \multicolumn{1}{c}{$\hat\beta_h$} & \multicolumn{1}{c}{SE} & \multicolumn{1}{c}{$t$} & \multicolumn{1}{c}{$p$} & $N$ \\
                \midrule
                Small  & CPI    & $h=1$   & 2.261                             & 1.194                  & 1.89                    & 0.060                   & 215 \\
                Medium & CPI    & $h=1$   & 0.709                             & 0.284                  & 2.50                    & 0.013                   & 215 \\
                Full   & CPI    & $h=1$   & 0.926                             & 0.319                  & 2.90                    & 0.004                   & 215 \\
                \addlinespace
                Small  & CPI    & $h=3$   & 0.692                             & 0.799                  & 0.87                    & 0.387                   & 215 \\
                Medium & CPI    & $h=3$   & 0.560                             & 0.320                  & 1.75                    & 0.082                   & 215 \\
                Full   & CPI    & $h=3$   & 0.689                             & 0.373                  & 1.85                    & 0.066                   & 215 \\
                \addlinespace
                Small  & CPI    & $h=12$  & -0.518                            & 0.321                  & -1.61                   & 0.109                   & 215 \\
                Medium & CPI    & $h=12$  & -0.084                            & 0.207                  & -0.41                   & 0.684                   & 215 \\
                Full   & CPI    & $h=12$  & -0.027                            & 0.211                  & -0.13                   & 0.897                   & 215 \\
                \addlinespace
                Small  & INDPRO & $h=1$   & 0.718                             & 0.561                  & 1.28                    & 0.202                   & 215 \\
                Medium & INDPRO & $h=1$   & 0.266                             & 0.492                  & 0.54                    & 0.589                   & 215 \\
                Full   & INDPRO & $h=1$   & 0.108                             & 0.347                  & 0.31                    & 0.756                   & 215 \\
                \addlinespace
                Small  & INDPRO & $h=3$   & 0.892                             & 0.482                  & 1.85                    & 0.065                   & 215 \\
                Medium & INDPRO & $h=3$   & 0.598                             & 0.364                  & 1.64                    & 0.101                   & 215 \\
                Full   & INDPRO & $h=3$   & 0.189                             & 0.363                  & 0.52                    & 0.603                   & 215 \\
                \addlinespace
                Small  & INDPRO & $h=12$  & 0.145                             & 0.442                  & 0.33                    & 0.744                   & 215 \\
                Medium & INDPRO & $h=12$  & 0.318                             & 0.529                  & 0.60                    & 0.548                   & 215 \\
                Full   & INDPRO & $h=12$  & 0.224                             & 0.492                  & 0.45                    & 0.650                   & 215 \\
                \bottomrule
            \end{tabular}
            \begin{tablenotes}[flushleft]
                \footnotesize
                \item Notes: Coefficients from regressing forecast errors on forecast revisions, $FE_{t,h} = \alpha_h + \beta_h \times FR_{t,h} + \varepsilon_{t,h}$, where both forecast errors and revisions are constructed on the evaluation scale (annualized cumulative growth rates). Standard errors are Newey--West HAC with lag truncation parameter $h$. Under rational expectations, $\beta_h=0$. Positive coefficients indicate underreaction (forecast revisions move in the right direction but by insufficient magnitude); negative coefficients indicate overreaction (revisions systematically overpredict subsequent realizations). Values correspond to \texttt{results/tables/cg\_regression\_results.csv}. Sample: 215 forecast origins, 2001M1--2019M11.
            \end{tablenotes}
        \end{threeparttable}
    \end{table}

    \begin{figure}[H]
        \centering
        \includegraphics[width=0.92\textwidth]{fig3_cg_coefficients.png}
        \caption{CG regression coefficients with 95\% confidence intervals}
        \label{fig:cg_coeffs}
        \captionsetup{font=small}
        \vspace{-0.2cm}
        \begin{minipage}{0.92\textwidth}
            \footnotesize
            Notes: The figure plots $\hat\beta_h$ from Table~\ref{tab:cg_main} with normal-approximation 95\% confidence intervals based on Newey--West standard errors. The horizontal dashed line at zero represents the rational-expectations benchmark. Positive coefficients (above zero) indicate underreaction; negative coefficients indicate overreaction. Source: \texttt{results/tables/cg\_regression\_results.csv}.
        \end{minipage}
    \end{figure}

    To quantify the incremental effect of sentiment on revision patterns, define $\Delta\beta_h = \beta_h^{\textup{Full}} - \beta_h^{\textup{Small}}$. Table~\ref{tab:delta_beta} reports these differences along with standard errors (computed via the variance formula for linear combinations of correlated estimates). Figure~\ref{fig:delta_beta} visualizes the same differences with uncertainty bands.

    For CPI at $h=1$, $\Delta\beta_1 = -1.34$ ($SE=1.24$, $p=0.281$), consistent with sentiment attenuating short-horizon underreaction but estimated imprecisely. At $h=12$, $\Delta\beta_{12} = 0.49$ ($SE=0.38$, $p=0.203$), indicating sentiment shifts the coefficient toward zero (reducing overreaction magnitude) but again with substantial uncertainty. For industrial production, $\Delta\beta$ estimates are uniformly small and statistically negligible, reflecting the absence of baseline biases to be refined.

    \begin{figure}[H]
        \centering
        \includegraphics[width=0.92\textwidth]{fig4_delta_beta_overreaction.png}
        \caption{Incremental effect of sentiment on CG coefficients: $\Delta\beta_h = \beta_h^{\textup{Full}} - \beta_h^{\textup{Small}}$}
        \label{fig:delta_beta}
        \captionsetup{font=small}
        \vspace{-0.2cm}
        \begin{minipage}{0.92\textwidth}
            \footnotesize
            Notes: The figure reports $\Delta\beta_h$ with 95\% confidence intervals based on HAC-robust standard errors for the difference. Negative values indicate sentiment reduces $\beta_h$ (e.g., attenuating underreaction at $h=1$); positive values indicate sentiment increases $\beta_h$ (e.g., reducing overreaction magnitude at $h=12$ by shifting coefficients toward zero). The wide confidence bands reflect sampling uncertainty in finite pseudo-OOS samples. Source: \texttt{results/tables/delta\_beta\_overreaction\_test.csv}.
        \end{minipage}
    \end{figure}

    \subsection{Hyperparameter adaptation and data-driven regularization}

    A key virtue of the hierarchical prior approach is that it makes the shrinkage intensity $\lambda$ endogenous to model size, allowing us to observe how the data-generating process adjusts regularization as the information set expands. Table~\ref{tab:lambda_evolution} and Figure~\ref{fig:lambda} document this variation.

    \begin{table}[H]
        \centering
        \begin{threeparttable}
            \caption{Posterior mean of shrinkage parameter $\lambda$ by model and forecast origin (selected origins)}
            \label{tab:lambda_evolution}
            \begin{tabular}{lSSS}
                \toprule
                Period                       & \multicolumn{1}{c}{Small} & \multicolumn{1}{c}{Medium} & \multicolumn{1}{c}{Full} \\
                \midrule
                2001--2005 (average)         & 0.669                     & 0.597                      & 0.368                    \\
                2006--2008 (pre-crisis)      & 0.713                     & 0.643                      & 0.430                    \\
                2008--2010 (Great Recession) & 0.874                     & 0.754                      & 0.489                    \\
                2011--2015 (recovery)        & 0.976                     & 0.787                      & 0.503                    \\
                2016--2019 (late sample)     & 1.135                     & 0.816                      & 0.541                    \\
                Overall average              & 0.881                     & 0.721                      & 0.464                    \\
                \bottomrule
            \end{tabular}
            \begin{tablenotes}[flushleft]
                \footnotesize
                \item Notes: Values are posterior means of $\lambda$ from the hierarchical MCMC, averaged over forecast origins in each subperiod. Source: \texttt{results/forecasts/hyperparameters\_evolution.csv}.
            \end{tablenotes}
        \end{threeparttable}
    \end{table}

    \emph{Interpretation of model-size dependence of shrinkage.} The systematic pattern is striking: as the information set grows from Small (4 variables) to Medium (6 variables) to Full (7 variables), the posterior-mean $\lambda$ declines from 0.881 to 0.721 to 0.464. Because smaller $\lambda$ corresponds to tighter Minnesota shrinkage, this pattern implies that the hierarchical procedure automatically tightens the prior as the model becomes more parameter-rich, mitigating overfitting risk and improving comparability across specifications.

    The time variation is also informative. Posterior-mean $\lambda$ rises during the 2008--2010 crisis period and increases further in the late sample, consistent with the data favoring looser shrinkage when the linear VAR requires additional flexibility (either because volatility is elevated or because fit deteriorates under persistent regime changes). The key point is not a specific ``crisis spike'' date, but that the hierarchical procedure makes regularization time-varying and transparent, rather than fixed by assumption.

    \paragraph{Implications for forecast discipline.} These patterns have practical implications. Using a fixed $\lambda$ would mechanically impose the same tightness across specifications, despite very different parameter counts; hierarchical selection instead adjusts tightness by model size and over time, making comparisons across information sets less sensitive to arbitrary prior choices.

    \begin{figure}[H]
        \centering
        \includegraphics[width=0.92\textwidth]{fig5_lambda_evolution.png}
        \caption{Evolution of hierarchical tightness parameter $\lambda$ over forecast origins}
        \label{fig:lambda}
        \captionsetup{font=small}
        \vspace{-0.2cm}
        \begin{minipage}{0.92\textwidth}
            \footnotesize
            Notes: The figure plots posterior means of $\lambda$ at each recursive forecast origin from 2001M1 to 2019M11 for each model specification. $\lambda$ is elevated during the 2008--2010 crisis window and increases further in the late sample. Source: \texttt{results/forecasts/hyperparameters\_evolution.csv}.
        \end{minipage}
    \end{figure}

    \subsection{Robustness}
    Two robustness exercises vary (i) the lag length from $p=12$ to $p=6$ and (ii) the initial training window endpoint from 2000M12 to 1995M12, keeping the rest of the design unchanged. The qualitative implications are stable. The medium specification remains the strongest performer for industrial production at short horizons, and the full specification remains competitive for inflation. The alternative initial-window design yields the lowest twelve-month inflation RMSFE for the full model (1.286). Appendix Table~\ref{tab:robustness} and Appendix Figure~\ref{fig:robustness_rw} summarize these results.
\fi

\ifsubmission
    % (Omitted in submission build; limitations summarized in the conclusion.)
\else
    \section{Limitations and future research directions}

    \subsection{Data and identification boundaries}

    Our analysis uses pseudo out-of-sample forecasts constructed from final-vintage macroeconomic data, not real-time vintages that forecasters would have actually observed. This choice simplifies the analysis and focuses attention on the information content of various data sources, but it sidesteps the practical challenge of nowcasting and data revision that practitioners face. A natural extension would be to re-implement this analysis using FRED-RTDF real-time data, which would reveal whether sentiment's predictive value survives the revision process---i.e., whether sentiment indices themselves are robust to later revision.

    The paper is deliberately descriptive and does not attempt to identify causal relationships between sentiment and macro outcomes. Consumer sentiment and macroeconomic conditions are mutually endogenous: households' sentiment responds to current conditions (employment, inflation expectations, asset prices), and in turn, sentiment-driven changes in consumption and savings affect output and inflation. A structural VAR exercise (estimating causal impulse responses via sign or zero restrictions) is beyond the paper's scope, but it would be a valuable complement to clarify the direction of causality and the quantitative magnitude of sentiment's causal effect.

    \subsection{Model specification and functional form}

    The paper estimates a linear BVAR on all three information sets. Inflation and sentiment may be related through nonlinear channels: for instance, sentiment's predictive content might be stronger during crisis periods (high volatility, low sentiment) than during calm periods. A time-varying parameter VAR (TV-BVAR) or a model with regime-switching could capture this richer dynamic. Similarly, we do not explore whether sentiment is better measured by decomposing the Michigan index into sub-components (current vs. expected conditions) or by combining sentiment with alternative confidence measures (e.g., the Conference Board consumer confidence index).

    The evaluation-scale transformation (cumulative growth over $h$ periods from a fixed origin) is standard for forecast evaluation but may mask phenomena visible at other horizons. For instance, one-period-ahead growth-rate forecasts (as opposed to cumulative $h$-period changes) might reveal different roles for sentiment.

    \subsection{Limitations of the CG diagnostic for model-based forecasts}

    The Coibion-Gorodnichenko regression was originally developed to diagnose biases in survey expectations, where $\beta>0$ can be interpreted as information rigidity or rational inattention by households. When applied to a VAR forecasting model, the interpretation is less direct: $\beta$ measures the model's internal consistency in updating, not a structural behavioral phenomenon. A $\beta=2.4$ coefficient at $h=1$ for the small model means that the model tends to under-weight forecast revisions relative to the magnitude needed to eliminate subsequent errors, but this may reflect not an economic irrationality but a prior specification (the Minnesota prior may be too tight at short horizons) or a genuinely persistent signal that takes time to be incorporated.

    \subsection{Concrete proposals for extension}

    \begin{enumerate}
        \item \textbf{Real-time data and nowcasting.} Repeat the analysis using FRED-RTDF real-time data vintages at forecast origin $t$, incorporating realistic delays and revisions. Assess whether sentiment's predictive value is diminished by data uncertainty.

        \item \textbf{Time-varying and nonlinear structures.} Extend to TV-BVAR or Markov-switching BVAR to test whether sentiment's role varies across regimes (e.g., stronger during crisis periods or high-uncertainty environments).

        \item \textbf{Sentiment decomposition.} Decompose the Michigan sentiment index into its major sub-components (current conditions vs. expectations) and evaluate their independent predictive contributions. Explore whether the expectations sub-component better predicts long-horizon inflation.

        \item \textbf{Multivariate sentiment measures.} Combine the Michigan index with other sentiment indicators (Conference Board, stock market-based measures, news-based indices) and evaluate whether a factor model of sentiment improves predictions.

        \item \textbf{Structural identification.} Estimate sign-restricted VAR IRFs to identify the causal response of inflation and production to a structural sentiment shock, holding constant the responses to other shocks.

        \item \textbf{Comparative evaluation against production models.} Benchmark the BVAR's forecasts against professional forecasts from the Survey of Professional Forecasters (SPF) and other real-world prediction systems to assess practical competitive advantage.
    \end{enumerate}

\fi

\section{Conclusion}

\ifsubmission
    This paper evaluates the incremental role of consumer sentiment in macroeconomic forecasting via a transparent, nested-information-set BVAR horse race complemented by forecast-revision diagnostics. The core evidence is disciplined: sentiment adds little incremental \emph{point-forecast accuracy} (RMSFE) once financial variables are included, and for long-horizon inflation the baseline macro specification performs best. However, revision diagnostics (CG regressions) suggest richer information sets alter forecast-updating patterns---short-horizon underreaction attenuates and long-horizon overreaction coefficients move toward zero---consistent with (but not proving) sentiment's role in capturing belief distortion or diagnostic-expectations dynamics.

    These findings highlight a key distinction: soft information may \emph{discipline forecast revisions} and improve internal consistency of probability updates even when marginal RMSFE gains are negligible. If sentiment primarily reflects households' diagnostic beliefs rather than objective fundamentals, its value lies in understanding \emph{expectation formation} rather than optimizing point predictions. This interpretation is consistent with financial variables subsuming sentiment's informational content for point forecasting (via market aggregation) while sentiment retains independent value as a proxy for belief dynamics.

    Because our models are nested and our sample is pseudo-out-of-sample (final vintages), we emphasize magnitudes and stability of RMSFE differences and report nested-robust tests (Clark--West) as robustness, avoiding sharp statistical claims. The revision diagnostics are \emph{suggestive}: they measure the forecasting model's internal probability-update consistency and could reflect prior-induced conservatism, misspecification, structural instability, or belief distortion---we cannot distinguish these mechanisms without additional structure.

    Natural next steps include: (i) real-time vintage forecasting to assess whether sentiment survives data revisions; (ii) prior-sensitivity analysis to rule out hyperprior artifacts; (iii) time-varying parameter VARs to capture structural change and test whether sentiment matters more in crisis episodes; (iv) density forecast evaluation to test calibration; and (v) structural identification (sign-restricted VARs) to estimate causal sentiment effects on inflation dynamics.
\else
    This paper evaluates the incremental role of consumer sentiment in macro forecasting within a transparent, nested-information-set BVAR horse race, and complements accuracy comparisons with a forecast-revision diagnostic following \cite{CG2015}. The evidence is consistent with a disciplined message: adding financial variables helps some short-horizon forecasts (especially for industrial production), but sentiment adds little incremental improvement in point-forecast RMSFE once financial variables are included; for inflation at long horizons, the baseline macro specification performs best in this sample.

    The revision diagnostic shows that inflation forecasts exhibit short-horizon underreaction and long-horizon coefficients near zero or slightly negative, and that richer information sets move long-horizon coefficients toward the rational-expectations benchmark. Because the models are nested, I treat standard equal-accuracy tests as suggestive and report Clark--West MSPE-adjusted tests for nested comparisons \cite{ClarkWest2007}, emphasizing magnitudes and stability rather than sharp claims about statistical dominance \cite{ClarkMcCracken2001}. The main limitation is interpretational: both RMSFE differences and CG coefficients are descriptive summaries of forecast performance rather than causal effects of sentiment. Future work using real-time vintages, broader sentiment measures, and structural identification would sharpen the economic interpretation.
\fi

\label{LastMainTextPage}

\clearpage
\bibliographystyle{apacite}
\bibliography{ref}

\clearpage
\appendix
\section{Additional figures and robustness}\label{app:figures}

\paragraph{Nested-model forecast accuracy: Clark--West tests.}
Table~\ref{tab:clark_west} reports Clark--West MSPE-adjusted tests for nested model comparisons (Small vs.\ Medium; Medium vs.\ Full) at horizons $h\in\{1,3,12\}$; one-sided p-values correspond to the alternative that the larger model improves MSPE. This robustness addresses the nonstandard behavior of standard equal-accuracy tests under nesting.

\begin{table}[ht]
    \centering
    \small
    \caption{Clark--West (2007) MSPE-Adjusted Tests for Nested Models}
    \label{tab:clark_west}
    \begin{tabular}{@{}lccccccc@{}}
        \toprule
        Smaller & Larger & variable & horizon & t-stat   & p-value & N   & NW lag \\
        \midrule
        Small   & Medium & CPI      & $h=1$   & 3.411*** & 0.000   & 227 & 1      \\
        Small   & Medium & CPI      & $h=3$   & 2.343**  & 0.010   & 225 & 3      \\
        Small   & Medium & CPI      & $h=12$  & -0.175   & 0.569   & 216 & 12     \\
        Small   & Medium & INDPRO   & $h=1$   & 3.223*** & 0.001   & 227 & 1      \\
        Small   & Medium & INDPRO   & $h=3$   & 2.259**  & 0.012   & 225 & 3      \\
        Small   & Medium & INDPRO   & $h=12$  & 2.188**  & 0.015   & 216 & 12     \\
        Medium  & Full   & CPI      & $h=1$   & -1.185   & 0.881   & 227 & 1      \\
        Medium  & Full   & CPI      & $h=3$   & -0.281   & 0.610   & 225 & 3      \\
        Medium  & Full   & CPI      & $h=12$  & 0.822    & 0.206   & 216 & 12     \\
        Medium  & Full   & INDPRO   & $h=1$   & 0.160    & 0.436   & 227 & 1      \\
        Medium  & Full   & INDPRO   & $h=3$   & 0.080    & 0.468   & 225 & 3      \\
        Medium  & Full   & INDPRO   & $h=12$  & 0.288    & 0.387   & 216 & 12     \\
        \bottomrule
    \end{tabular}
    \vspace{0.5em}
    \begin{minipage}{\textwidth}
        \footnotesize
        \textit{Notes:}
        *** $p<0.01$, ** $p<0.05$, * $p<0.1$\\
        Clark--West (2007) MSPE-adjusted test for equal forecast accuracy in nested models.
        For smaller-model forecast error $e_{1t}=y_t-f_{1t}$ and larger-model error $e_{2t}=y_t-f_{2t}$, the adjusted loss differential is
        $d_t = e_{1t}^2 - \left(e_{2t}^2 - (f_{2t}-f_{1t})^2\right)$. The test regresses $d_t$ on a constant.
        Newey--West HAC standard errors use lag truncation equal to the forecast horizon (overlap adjustment).
        One-sided p-values reported for the alternative that the larger model improves MSPE.
    \end{minipage}
\end{table}


\begin{figure}[H]
    \centering
    \includegraphics[width=0.96\textwidth]{fig_rolling_relative_rmsfe_ar1.png}
    \caption{Rolling relative RMSFE versus AR(1) benchmark}
    \label{fig:rolling_ar1}
    \captionsetup{font=small}
    \vspace{-0.2cm}
    \begin{minipage}{0.96\textwidth}
        \footnotesize
        Notes: Rolling-window relative RMSFEs (window length 60 months). Values below one indicate improvement over the recursively estimated AR(1) benchmark.
    \end{minipage}
\end{figure}

\begin{figure}[H]
    \centering
    \includegraphics[width=0.96\textwidth]{fig_robustness_relative_rmsfe_rw.png}
    \caption{Robustness: relative RMSFE versus no-change benchmark}
    \label{fig:robustness_rw}
    \captionsetup{font=small}
    \vspace{-0.2cm}
    \begin{minipage}{0.96\textwidth}
        \footnotesize
        Notes: Relative RMSFEs under alternative lag length ($p=6$) and an earlier initial training window end date (1995M12). Values below one indicate improvement over the no-change benchmark.
    \end{minipage}
\end{figure}

\begin{figure}[H]
    \centering
    \includegraphics[width=0.96\textwidth]{fig_robustness_relative_rmsfe_ar1.png}
    \caption{Robustness: relative RMSFE versus AR(1) benchmark}
    \label{fig:robustness_ar1}
    \captionsetup{font=small}
    \vspace{-0.2cm}
    \begin{minipage}{0.96\textwidth}
        \footnotesize
        Notes: Relative RMSFEs under robustness scenarios, reported against the recursively estimated AR(1) benchmark.
    \end{minipage}
\end{figure}

\begin{figure}[H]
    \centering
    \includegraphics[width=0.96\textwidth]{fig6a_forecast_vs_actual_h1.png}
    \caption{CPI inflation: BVAR forecast versus realized ($h=1$)}
    \label{fig:cpi_forecast_h1}
    \captionsetup{font=small}
    \vspace{-0.2cm}
    \begin{minipage}{0.96\textwidth}
        \footnotesize
        Notes: The x-axis uses the target date ($t+h$). The plotted forecast is the model-implied predictive mean from the baseline specification.
    \end{minipage}
\end{figure}

\begin{figure}[H]
    \centering
    \includegraphics[width=0.96\textwidth]{fig6b_forecast_vs_actual_h3.png}
    \caption{CPI inflation: BVAR forecast versus realized ($h=3$)}
    \label{fig:cpi_forecast_h3}
    \captionsetup{font=small}
    \vspace{-0.2cm}
    \begin{minipage}{0.96\textwidth}
        \footnotesize
        Notes: The x-axis uses the target date ($t+h$). The plotted forecast is the model-implied predictive mean from the baseline specification.
    \end{minipage}
\end{figure}

\begin{figure}[H]
    \centering
    \includegraphics[width=0.96\textwidth]{fig6c_forecast_vs_actual_h12.png}
    \caption{CPI inflation: BVAR forecast versus realized ($h=12$)}
    \label{fig:cpi_forecast_h12}
    \captionsetup{font=small}
    \vspace{-0.2cm}
    \begin{minipage}{0.96\textwidth}
        \footnotesize
        Notes: The x-axis uses the target date ($t+h$). The plotted forecast is the model-implied predictive mean from the baseline specification.
    \end{minipage}
\end{figure}

\begin{figure}[H]
    \centering
    \includegraphics[width=0.96\textwidth]{fig6d_timing_diagnostic_h1.png}
    \caption{Forecast timing diagnostic ($h=1$)}
    \label{fig:timing_h1}
    \captionsetup{font=small}
    \vspace{-0.2cm}
    \begin{minipage}{0.96\textwidth}
        \footnotesize
        Notes: The black series is dated at the realization ($t+h$); the red forecast series is dated at the forecast origin ($t$).
    \end{minipage}
\end{figure}

\begin{figure}[H]
    \centering
    \includegraphics[width=0.96\textwidth]{fig6e_timing_diagnostic_h3.png}
    \caption{Forecast timing diagnostic ($h=3$)}
    \label{fig:timing_h3}
    \captionsetup{font=small}
    \vspace{-0.2cm}
    \begin{minipage}{0.96\textwidth}
        \footnotesize
        Notes: The black series is dated at the realization ($t+h$); the red forecast series is dated at the forecast origin ($t$).
    \end{minipage}
\end{figure}

\begin{figure}[H]
    \centering
    \includegraphics[width=0.96\textwidth]{fig6f_timing_diagnostic_h12.png}
    \caption{Forecast timing diagnostic ($h=12$)}
    \label{fig:timing_h12}
    \captionsetup{font=small}
    \vspace{-0.2cm}
    \begin{minipage}{0.96\textwidth}
        \footnotesize
        Notes: The black series is dated at the realization ($t+h$); the red forecast series is dated at the forecast origin ($t$).
    \end{minipage}
\end{figure}

\begin{figure}[H]
    \centering
    \includegraphics[width=0.96\textwidth]{fig7a_cg_scatter_h1.png}
    \caption{Revision diagnostic scatter: CPI ($h=1$)}
    \label{fig:cg_scatter_h1}
    \captionsetup{font=small}
    \vspace{-0.2cm}
    \begin{minipage}{0.96\textwidth}
        \footnotesize
        Notes: Scatter of forecast errors against forecast revisions for CPI inflation in the baseline design. The fitted line corresponds to the \citet{CG2015} regression.
    \end{minipage}
\end{figure}

\begin{figure}[H]
    \centering
    \includegraphics[width=0.96\textwidth]{fig7b_cg_scatter_h3.png}
    \caption{Revision diagnostic scatter: CPI ($h=3$)}
    \label{fig:cg_scatter_h3}
    \captionsetup{font=small}
    \vspace{-0.2cm}
    \begin{minipage}{0.96\textwidth}
        \footnotesize
        Notes: Scatter of forecast errors against forecast revisions for CPI inflation in the baseline design. The fitted line corresponds to the \citet{CG2015} regression.
    \end{minipage}
\end{figure}

\begin{figure}[H]
    \centering
    \includegraphics[width=0.96\textwidth]{fig7c_cg_scatter_h12.png}
    \caption{Revision diagnostic scatter: CPI ($h=12$)}
    \label{fig:cg_scatter_h12}
    \captionsetup{font=small}
    \vspace{-0.2cm}
    \begin{minipage}{0.96\textwidth}
        \footnotesize
        Notes: Scatter of forecast errors against forecast revisions for CPI inflation in the baseline design. The fitted line corresponds to the \citet{CG2015} regression.
    \end{minipage}
\end{figure}

\begin{table}[H]
    \centering
    \begin{threeparttable}
        \caption{Robustness: RMSFEs under alternative lag length and training window}
        \label{tab:robustness}
        \begin{tabular}{l ll S[table-format=1.3] S[table-format=1.3] S[table-format=1.3]}
            \toprule
            Scenario                    & Model  & Target & \multicolumn{1}{c}{$h=1$} & \multicolumn{1}{c}{$h=3$} & \multicolumn{1}{c}{$h=12$} \\
            \midrule
            $p=6$                       & Small  & CPI    & 3.451                     & 2.639                     & 1.291                      \\
            $p=6$                       & Medium & CPI    & 3.445                     & 2.641                     & 1.294                      \\
            $p=6$                       & Full   & CPI    & 3.529                     & 2.673                     & 1.342                      \\
            $p=6$                       & Small  & INDPRO & 7.582                     & 5.410                     & 4.825                      \\
            $p=6$                       & Medium & INDPRO & 7.336                     & 5.033                     & 4.621                      \\
            $p=6$                       & Full   & INDPRO & 7.494                     & 5.173                     & 4.572                      \\
            \addlinespace
            Initial window ends 1995M12 & Small  & CPI    & 3.231                     & 2.448                     & 1.323                      \\
            Initial window ends 1995M12 & Medium & CPI    & 3.216                     & 2.446                     & 1.325                      \\
            Initial window ends 1995M12 & Full   & CPI    & 3.267                     & 2.431                     & 1.286                      \\
            Initial window ends 1995M12 & Small  & INDPRO & 7.329                     & 5.204                     & 4.802                      \\
            Initial window ends 1995M12 & Medium & INDPRO & 7.077                     & 4.803                     & 4.590                      \\
            Initial window ends 1995M12 & Full   & INDPRO & 7.218                     & 4.930                     & 4.453                      \\
            \bottomrule
        \end{tabular}
        \begin{tablenotes}[flushleft]
            \footnotesize
            \item Notes: Values are taken from \texttt{results/robustness/lag6/tables/rmsfe\_results.csv} and \texttt{results/robustness/window1995/tables/rmsfe\_results.csv}.
        \end{tablenotes}
    \end{threeparttable}
\end{table}

\end{document}
