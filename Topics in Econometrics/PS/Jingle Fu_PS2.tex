\documentclass[a4paper,12pt]{article} % This defines the style of your paper

\usepackage[top = 2.5cm, bottom = 2.5cm, left = 2.5cm, right = 2.5cm]{geometry} 

% Unfortunately, LaTeX has a hard time interpreting German Umlaute. The following two lines and packages should help. If it doesn't work for you please let me know.
\usepackage[T1]{fontenc}
\usepackage[utf8]{inputenc}
\usepackage{pifont}
% \usepackage{ctex}
\usepackage{amsthm, amsmath, amssymb, mathrsfs,mathtools}

% Defining a new theorem style without italics
\newtheoremstyle{nonitalic}% name
  {\topsep}% Space above
  {\topsep}% Space below
  {\upshape}% Body font
  {}% Indent amount
  {\bfseries}% Theorem head font
  {.}% Punctuation after theorem head
  {.5em}% Space after theorem head
  {}% Theorem head spec (can be left empty, meaning ‘normal`)
  
\theoremstyle{nonitalic}
% Define new 'solution' environment
\newtheorem{innercustomsol}{Solution}
\newenvironment{solution}[1]
  {\renewcommand\theinnercustomsol{#1}\innercustomsol}
  {\endinnercustomsol}

% Custom counter for the solutions
\newcounter{solutionctr}
\renewcommand{\thesolutionctr}{(\alph{solutionctr})}

% Environment for auto-numbering with custom format
\newenvironment{autosolution}
  {\stepcounter{solutionctr}\begin{solution}{\thesolutionctr}}
  {\end{solution}}


\newtheorem{problem}{Problem}
\usepackage{color}

% The following two packages - multirow and booktabs - are needed to create nice looking tables.
\usepackage{multirow} % Multirow is for tables with multiple rows within one cell.
\usepackage{booktabs} % For even nicer tables.

% As we usually want to include some plots (.pdf files) we need a package for that.
\usepackage{graphicx} 
\usepackage{subfigure}


% The default setting of LaTeX is to indent new paragraphs. This is useful for articles. But not really nice for homework problem sets. The following command sets the indent to 0.
\usepackage{setspace}
\setlength{\parindent}{0in}
\usepackage{longtable}

% Package to place figures where you want them.
\usepackage{float}

% The fancyhdr package let's us create nice headers.
\usepackage{fancyhdr}

\usepackage{fancyvrb}

%Code environment 
\usepackage{listings} % Required for insertion of code
\usepackage{xcolor} % Required for custom colors

% Define colors for code listing
\definecolor{codegreen}{rgb}{0,0.6,0}
\definecolor{codegray}{rgb}{0.5,0.5,0.5}
\definecolor{codepurple}{rgb}{0.58,0,0.82}
\definecolor{backcolour}{rgb}{0.95,0.95,0.92}

% Code listing style named "mystyle"
\lstdefinestyle{mystyle}{
    backgroundcolor=\color{backcolour},   
    commentstyle=\color{codegreen},
    keywordstyle=\color{magenta},
    numberstyle=\tiny\color{codegray},
    stringstyle=\color{codepurple},
    basicstyle=\ttfamily\footnotesize, % Change to serif font
    breakatwhitespace=false,         
    breaklines=true,                 
    captionpos=b,                    
    keepspaces=true,                 
    numbers=left,                    
    numbersep=5pt,                  
    showspaces=false,                
    showstringspaces=false,
    showtabs=false,                  
    tabsize=2
}

\lstset{style=mystyle}

%%%%%%%%%%%%%%%%%%%%%%%%%%%%%%%%%%%%%%%%%%%%%%%%
% 3. Header (and Footer)
%%%%%%%%%%%%%%%%%%%%%%%%%%%%%%%%%%%%%%%%%%%%%%%%

% To make our document nice we want a header and number the pages in the footer.

\pagestyle{fancy} % With this command we can customize the header style.

\fancyhf{} % This makes sure we do not have other information in our header or footer.

\lhead{\footnotesize EI035 Econometrics II}% \lhead puts text in the top left corner. \footnotesize sets our font to a smaller size.

%\rhead works just like \lhead (you can also use \chead)
\rhead{\footnotesize Jingle Fu} %<---- Fill in your lastnames.

% Similar commands work for the footer (\lfoot, \cfoot and \rfoot).
% We want to put our page number in the center.
\cfoot{\footnotesize \thepage}
\IfFileExists{upquote.sty}{\usepackage{upquote}}{}
\begin{document}


\thispagestyle{empty} % This command disables the header on the first page. 

\begin{tabular}{p{15.5cm}} % This is a simple tabular environment to align your text nicely 
  {\large \bf EI035 Econometrics II}                 \\
  The Graduate Institute, Spring 2025, Marko Mlikota \\
  \hline % \hline produces horizontal lines.
  \\
\end{tabular} % Our tabular environment ends here.

\vspace*{0.3cm} % Now we want to add some vertical space in between the line and our title.

\begin{center} % Everything within the center environment is centered.
  {\Large \bf PS2 Solutions} % <---- Don't forget to put in the right number
  \vspace{2mm}

  % YOUR NAMES GO HERE
  {\bf Jingle Fu} % <---- Fill in your names here!

\end{center}

\vspace{0.4cm}
\setstretch{1.2}


\section*{Problem 1: Dynamic Panel Data with Correlated Random Effects}

\subsection*{Model}
\begin{equation*}
  y_{it} = \alpha_i + \rho y_{it-1} + u_{it}, \quad u_{it} \sim iid \mathcal{N}(0, 1)
\end{equation*}

\subsection*{CRE Distribution}
\begin{equation*}
  \alpha_i | (y_{i0}, \phi) \sim \mathcal{N}(\phi y_{i0}, 1)
\end{equation*}

\subsection*{(a) The Incidental Parameter Problem (IPP)}
The incidental parameter problem arises in panel data models when the number of parameters to be estimated grows with the sample size $N$.
Here, the unit-specific effects $\alpha_1, ..., \alpha_N$ are the incidental parameters.

In a dynamic panel (where $y_{it-1}$ is a regressor),
the standard Fixed Effects (Within) estimator or the naive MLE for $\alpha_i$ and $\rho$ yields inconsistent estimates for $\rho$ when $N \to \infty$ but $T$ remains fixed.

Treating $\{\alpha_i\}_{i=1}^n$ as fixed parameters in FE--ML yields
\[
  \ell_i(\alpha_i,\rho)
  \;=\;
  -\frac{T}{2}\log(2\pi)
  -\frac{1}{2}\sum_{t=1}^T\big(y_{it}-\alpha_i-\rho\,y_{i,t-1}\big)^2,
\]
and the first-order condition for $\alpha_i$ is
\[
  \frac{\partial \ell_i}{\partial \alpha_i}
  \;=\;
  \sum_{t=1}^T\big(y_{it}-\alpha_i-\rho\,y_{i,t-1}\big)
  \;=\;0
  \;=\;
  T\,\hat{\alpha}_i - \sum_{t=1}^T\big(y_{it}-\rho\,y_{i,t-1}\big),
\]
hence
\[
  \hat{\alpha}_i
  \;=\;
  \frac{1}{T}\sum_{t=1}^T\big(y_{it}-\rho\,y_{i,t-1}\big)
  \;=\;
  \alpha_i
  \;+\; \frac{1}{T}\sum_{t=1}^T u_{it}.
\]
Because $y_{i,t-1}$ embeds $\alpha_i$, the estimation error $\hat\alpha_i-\alpha_i$ remains correlated with $y_{i,t-1}$ when $T$ is fixed:
\[
  \mathrm{Cov}\!\big(y_{i,t-1},\,\hat{\alpha}_i-\alpha_i\big)
  \;\neq\; 0 \quad\text{for fixed $T$}.
\]
Thus, as $n\to\infty$ with $T$ fixed, the FE estimator $\hat\rho$ has an $O(1/T)$ Nickell bias.
In contrast, the CRE--Bayesian route treats $\alpha_i$ hierarchically so that inference on $(\phi,\rho)$ does not suffer from the IPP.

% This happens because the estimation error of $\alpha_i$ (which does not vanish as $N \to \infty$) is correlated with the regressor $y_{it-1}$ (since $y_{it-1}$ contains $\alpha_i$).
% This induces a downward bias in the estimate of $\rho$, commonly known as the \textbf{Nickell Bias} (of order $1/T$).

\subsection*{(b) Integrating out $\alpha_i$}

% \paragraph{Step 1: Cumulative multiplication of conditional densities}
Given $\tilde y_{it}=\alpha_i+u_{it}$ and $u_{it}\sim\mathcal{N} (0,1)$, we have:
\[
  p(\tilde y_{it}\mid \alpha_i)=\frac{1}{\sqrt{2\pi}}\exp\Big(-\tfrac12(\tilde y_{it}-\alpha_i)^2\Big).
\]
Due to independence across $t$ (conditional on $\alpha_i$):
\begin{align*}
  p(\tilde{\mathbf y}_i\mid \alpha_i) & = \prod_{t=1}^T \frac{1}{\sqrt{2\pi}}\exp\Big(-\tfrac12(\tilde y_{it}-\alpha_i)^2\Big) \\
                                      & = (2\pi)^{-T/2}\exp\Big(-\tfrac12\sum_{t=1}^T(\tilde y_{it}-\alpha_i)^2\Big).
\end{align*}

% \paragraph{Step 2: Multiplication with the prior to obtain the joint kernel}
The prior (CRE) is:
\[
  p(\alpha_i\mid y_{i0},\phi)=\frac{1}{\sqrt{2\pi}}\exp\Big(-\tfrac12(\alpha_i-\phi y_{i0})^2\Big).
\]
Thus:
\[
  \begin{aligned}
    p(\tilde{\mathbf y}_i,\alpha_i\mid y_{i0},\phi,\rho)
     & =p(\tilde{\mathbf y}_i\mid \alpha_i)\,p(\alpha_i\mid y_{i0},\phi)                                                                                           \\
     & =(2\pi)^{-(T+1)/2}\exp\Big(-\tfrac12\,\underbrace{\Big[\sum_{t=1}^T(\tilde y_{it}-\alpha_i)^2+(\alpha_i-\phi y_{i0})^2\Big]}_{\mathcal{Q} (\alpha_i)}\Big).
  \end{aligned}
\]

% \paragraph{Step 3: Expanding $\mathcal{Q} (\alpha_i)$ to ``complete the square''}
\[
  \sum_{t=1}^T(\tilde y_{it}-\alpha_i)^2
  =\sum_{t=1}^T\tilde y_{it}^2-2\alpha_i\sum_{t=1}^T\tilde y_{it}+T\alpha_i^2,
\]
\[
  (\alpha_i-\phi y_{i0})^2=\alpha_i^2-2\phi y_{i0}\alpha_i+\phi^2y_{i0}^2.
\]
Summing these yields:
\[
  \mathcal{Q} (\alpha_i)
  =(T+1)\alpha_i^2-2\alpha_i\Big(\sum_{t=1}^T\tilde y_{it}+\phi y_{i0}\Big)
  +\Big(\sum_{t=1}^T\tilde y_{it}^2+\phi^2y_{i0}^2\Big).
\]
Let:
\[
  c\equiv T+1,\quad b\equiv \sum_{t=1}^T\tilde y_{it}+\phi y_{i0},\quad a\equiv \sum_{t=1}^T\tilde y_{it}^2+\phi^2y_{i0}^2,
\]
then:
\[
  \mathcal{Q} (\alpha_i)=c\alpha_i^2-2b\alpha_i+a
  =c\Big(\alpha_i-\frac{b}{c}\Big)^2+\Big(a-\frac{b^2}{c}\Big).
\]

% \paragraph{Step 4: One-dimensional Gaussian integration with respect to $\alpha_i$}
\[
  \begin{aligned}
    p(\tilde{\mathbf y}_i\mid y_{i0},\phi,\rho)
     & =\int_{-\infty}^{\infty} p(\tilde{\mathbf y}_i,\alpha_i\mid\cdot)\,d\alpha_i                                                                                    \\
     & =(2\pi)^{-(T+1)/2}\exp\Big(-\tfrac12\Big(a-\tfrac{b^2}{c}\Big)\Big)\int_{-\infty}^{\infty}\exp\Big(-\tfrac12 c\big(\alpha_i-\tfrac{b}{c}\big)^2\Big)\,d\alpha_i \\
     & =(2\pi)^{-(T+1)/2}\exp\Big(-\tfrac12\Big(a-\tfrac{b^2}{c}\Big)\Big)\cdot\sqrt{\frac{2\pi}{c}}                                                                   \\
     & =(2\pi)^{-T/2}\,c^{-1/2}\,\exp\Big(-\tfrac12\Big[a-\tfrac{b^2}{c}\Big]\Big).
  \end{aligned}
\]
Substituting $c=T+1$ back:
\[
  p(\tilde{\mathbf y}_i\mid y_{i0},\phi,\rho)
  =(2\pi)^{-T/2}(T+1)^{-1/2} \exp\Big(-\tfrac12 \Big[\sum_{t=1}^T \tilde y_{it}^2 + \phi^2y_{i0}^2 - \frac{\big(\sum_{t=1}^T \tilde y_{it} + \phi y_{i0} \big)^2}{T+1}\Big]\Big).
\]

% \paragraph{Step 5: Rearranging the exponent into Multivariate Normal Mahalanobis form}
Note that:
\[
  \sum_{t=1}^T \tilde y_{it}^2 = \tilde{\mathbf y}_i^\top\tilde{\mathbf y}_i,\qquad
  \sum_{t=1}^T \tilde y_{it} = \mathbf 1^\top \tilde{\mathbf y}_i.
\]
Let $\boldsymbol \mu\equiv \phi y_{i0}\mathbf 1$ and
% . We write
% \[
%   \sum_{t=1}^T \tilde y_{it}^2 + \phi^2 y_{i0}^2 - \frac{\big(\sum_{t=1}^T \tilde y_{it} + \phi y_{i0}\big)^2}{T+1}
% \]
% in the form of $(\tilde{\mathbf y}_i-\boldsymbol \mu)^\top \boldsymbol \Omega^{-1} (\tilde{\mathbf y}_i- \boldsymbol \mu)$.
% Let:
\[
  \boldsymbol \Omega \equiv I_T + \mathbf 1 \mathbf 1^\top \Rightarrow\, \boldsymbol \Omega^{-1} = I_T - \frac{1}{T+1} \mathbf 1 \mathbf 1^\top, \quad
  |\boldsymbol \Omega| = (1+T)\cdot 1^{T-1} = T + 1.
\]
% We directly \textbf{verify} its inverse and determinant:

% \begin{itemize}
%   \item First, conjecture $\boldsymbol\Omega^{-1}=I_T-\alpha\,\mathbf 1\mathbf 1^\top$. Since $\boldsymbol\Omega$ is a rank-1 perturbation in the direction of $\mathbf 1$, this structure holds. Multiplying back:
%         \[
%           \begin{aligned}
%             \boldsymbol\Omega\boldsymbol\Omega^{-1}
%              & =(I_T+\mathbf 1\mathbf 1^\top)(I_T-\alpha\,\mathbf 1\mathbf 1^\top)                                                                   \\
%              & =I_T-\alpha\,\mathbf 1\mathbf 1^\top+\mathbf 1\mathbf 1^\top-\alpha\,\mathbf 1\underbrace{\mathbf 1^\top\mathbf 1}_{=T}\mathbf 1^\top \\
%              & =I_T+\big(1-\alpha-\alpha T\big)\mathbf 1\mathbf 1^\top.
%           \end{aligned}
%         \]
%         To equate this to $I_T$, we require $1-\alpha-\alpha T=0\Rightarrow \alpha=\frac{1}{T+1}$. Thus:

\begin{align*}
  \mathcal{Q} & = \sum_{t=1}^T \tilde y_{it}^2 + \phi^2y_{i0}^2 - \frac{\big(\sum_{t=1}^T \tilde y_{it} + \phi y_{i0} \big)^2}{T+1}                            \\
              & = \tilde{\mathbf y}_i^\top \tilde{\mathbf y}_i
  - \frac{(\sum \tilde y_{it})^2}{T+1}
  - \frac{2 \phi y_{i0}}{T+1}\sum \tilde y_{it}
  + \phi^2 y_{i0}^2 \Big(1-\frac{1}{T+1}\Big)                                                                                                                  \\
              & = \tilde{\mathbf y}_i^\top \tilde{\mathbf y}_i
  - \frac{( \sum \tilde y_{it})^2}{T+1}
  + \Big[(\phi y_{i0})^2T - \frac{(\phi y_{i0})^2T^2}{T+1}\Big]
  + \Big[-2 \phi y_{i0} \sum \tilde y_{it} + \frac{2\phi y_{i0}T}{T+1} \sum \tilde y_{it} \Big]                                                                \\
              & = \tilde{\mathbf y}_i^\top \tilde{\mathbf y}_i
  + (\phi y_{i0})^2 T
  - 2 \phi y_{i0} \sum \tilde y_{it}
  - \frac{1}{T+1} \Big[ ( \sum \tilde y_{it})^2 - 2 \phi y_{i0}T \sum \tilde y_{it} + (\phi y_{i0})^2T^2\Big]                                                  \\
              & = \tilde{\mathbf y}_i^\top \tilde{\mathbf y}_i-2(\phi y_{i0}) \mathbf 1^\top \tilde{\mathbf y}_i+(\phi y_{i0})^2 \mathbf 1^\top \mathbf 1
  - \frac{1}{T+1} \big( \mathbf 1^\top \tilde{\mathbf y}_i-\phi y_{i0} \mathbf 1^\top \mathbf 1 \big)^2                                                        \\
              & = \tilde{\mathbf y}_i^\top \tilde{\mathbf y}_i-2 \boldsymbol \mu^\top \tilde{\mathbf y}_i + \boldsymbol \mu^\top \boldsymbol \mu
  - \frac{1}{T+1} \big(\mathbf 1^\top \tilde{\mathbf y}_i - \mathbf 1^\top \boldsymbol \mu \big)^2                                                             \\
              & = (\tilde{\mathbf y}_i - \boldsymbol \mu)^\top (\tilde{\mathbf y}_i - \boldsymbol \mu)
  - \frac{1}{T+1} \big[\mathbf 1^\top (\tilde{\mathbf y}_i - \boldsymbol \mu)\big]^2                                                                           \\
              & = (\tilde{\mathbf y}_i - \boldsymbol \mu)^\top \Big(I_T - \tfrac{1}{T+1} \mathbf 1 \mathbf 1^\top \Big)(\tilde{\mathbf y}_i - \boldsymbol \mu) \\
              & = (\tilde{\mathbf y}_i - \boldsymbol \mu)^\top \boldsymbol \Omega^{-1} (\tilde{\mathbf y}_i - \boldsymbol \mu)
\end{align*}

Therefore:
\[
  p(\tilde{\mathbf y}_i\mid y_{i0},\phi,\rho)
  = (2\pi)^{-T/2} |\boldsymbol \Omega|^{-1/2}
  \exp \Big( -\tfrac12 (\tilde{\mathbf y}_i - \boldsymbol \mu)^\top \boldsymbol \Omega^{-1} (\tilde{\mathbf y}_i - \boldsymbol \mu) \Big).
\]
% where
% \[
%   \boldsymbol\Omega=I_T+\mathbf 1\mathbf 1^\top,\quad
%   \boldsymbol\Omega^{-1}=I_T-\frac{1}{T+1}\mathbf 1\mathbf 1^\top,\quad
%   |\boldsymbol\Omega|=T+1.
% \]

\subsection*{(c) Consistency of $(\phi, \rho)$}
The sample log-likelihood is:
\[
  \ell(\phi,\rho)=\sum_{i=1}^n\ell_i(\phi,\rho),
\]
where (derived directly from the above equation):
\[
  \ell_i=-\frac{T}{2}\log(2\pi)-\frac{1}{2}\log(T+1)-\frac{1}{2}\,
  (\tilde{\mathbf y}_i-\phi y_{i0}\mathbf 1)^\top
  \Big(I_T-\frac{1}{T+1}\mathbf 1\mathbf 1^\top\Big)
  (\tilde{\mathbf y}_i-\phi y_{i0}\mathbf 1).
\]

\paragraph{Score w.r.t. $\phi$:}
\[
  \begin{aligned}
    \frac{\partial \ell}{\partial \phi}
     & =\sum_{i=1}^n \frac{\partial \ell_i}{\partial \phi}
    =\sum_{i=1}^n \frac{1}{2}\cdot 2\,y_{i0}\mathbf 1^\top\Big(I-\tfrac{1}{T+1}\mathbf 1\mathbf 1^\top\Big)
    (\tilde{\mathbf y}_i-\phi y_{i0}\mathbf 1)\cdot(+1)                                      \\
     & =\sum_{i=1}^n y_{i0}\,\mathbf 1^\top\Big(I-\tfrac{1}{T+1}\mathbf 1\mathbf 1^\top\Big)
    (\tilde{\mathbf y}_i-\phi y_{i0}\mathbf 1).
  \end{aligned}
\]
Setting this score to 0 implies:
\[
  \sum_i y_{i0}\,\mathbf 1^\top\Big(I-\tfrac{1}{T+1}\mathbf 1\mathbf 1^\top\Big)\tilde{\mathbf y}_i
  =\phi\sum_i y_{i0}^2\,\mathbf 1^\top\Big(I-\tfrac{1}{T+1}\mathbf 1\mathbf 1^\top\Big)\mathbf 1.
\]
In the RHS, $\mathbf 1^\top(I-\frac{1}{T+1}\mathbf 1\mathbf 1^\top)\mathbf 1 = \mathbf 1^\top\mathbf 1-\frac{1}{T+1}\mathbf 1^\top\mathbf 1\,\mathbf 1^\top\mathbf 1 = T-\frac{T^2}{T+1}=\frac{T}{T+1}$.

\paragraph{Score w.r.t. $\rho$:}
\[
  \tilde{\mathbf y}_i=\mathbf y_i-\rho\,\mathbf L_i\mathbf y_i\quad\Rightarrow\quad
  \frac{\partial \tilde{\mathbf y}_i}{\partial \rho}=-(\mathbf L_i\mathbf y_i).
\]
Thus:
\[
  \frac{\partial \ell}{\partial \rho}
  =\sum_i \big[-(\mathbf L_i\mathbf y_i)\big]^\top\Big(I-\tfrac{1}{T+1}\mathbf 1\mathbf 1^\top\Big)
  (\tilde{\mathbf y}_i-\phi y_{i0}\mathbf 1)=0.
\]
These two \textbf{moment equations} have an \textbf{expectation} of 0 at the true values $(\phi_0,\rho_0)$. As $n\to\infty$ with fixed $T$,
the sample average score converges to its expectation,
and provided the information matrix is non-degenerate (positive definite),
the Maximum Likelihood estimator is consistent.

\subsection*{(d) Estimation of $\alpha_i$}
In a Bayesian (or Correlated Random Effects) framework, since we cannot estimate $\alpha_i$ consistently (it does not converge to a point),
we estimate its \textbf{conditional posterior distribution} or its \textbf{conditional expectation (BLUP)} given the data.

The posterior kernel is:
\begin{align*}
  p(\alpha_i\mid \tilde{\mathbf y}_i,y_{i0},\phi,\rho)
   & \propto p(\tilde{\mathbf y}_i\mid \alpha_i)\,p(\alpha_i\mid y_{i0},\phi)       \\
   & \propto \exp\Big(-\tfrac12\sum_{t=1}^T(\tilde y_{it}-\alpha_i)^2\Big)\cdot
  \exp\Big(-\tfrac12(\alpha_i-\phi y_{i0})^2\Big)                                   \\
   & = \exp\Big(-\tfrac12\big[c(\alpha_i-\tfrac{b}{c})^2+a-\tfrac{b^2}{c}\big]\Big)
  \quad(\text{see definitions of } c,b,a \text{ above}).
\end{align*}

Therefore:
\[
  \alpha_i\mid \tilde{\mathbf y}_i,y_{i0},\phi,\rho\ \sim\
  \mathcal{N} \Big( \frac{\sum_{t=1}^T\tilde y_{it}+\phi y_{i0}}{T+1}, \frac{1}{T+1} \Big)
\]
The posterior mean (Bayes estimator) is:
\begin{align*}
  \hat{\alpha }_i & = \mathbb{E}[\alpha_i\mid \tilde{\mathbf y}_i,y_{i0},\phi,\rho] \\
                  & = \frac{\sum_{t=1}^T\tilde y_{it}+\phi y_{i0}}{T+1}             \\
                  & = \frac{\sum_{t=1}^T(y_{it}-\rho y_{i,t-1})+\phi y_{i0}}{T+1}.
\end{align*}
we would first estimate $(\hat{\phi}, \hat{\rho})$ from the marginal likelihood,
then plug them into the above expression to get $\hat{\alpha}_i$.

\section*{Problem 2: State-Space Model}

\subsection*{Model}
\begin{align*}
  y_t & = \lambda s_t + u_t                                                                             \\
  s_t & = \phi s_{t-1} + \varepsilon_t                                                                  \\
  u_t & \sim \mathcal{N}(0,1), \quad \varepsilon_t \sim \mathcal{N}(0,1), \quad u_t \perp \varepsilon_t
\end{align*}

\subsection*{(a) Autocovariance Function for $y_t$}
Assuming stationarity ($|\phi| < 1$), the variance of the state $s_t$ is $\text{Var}(s_t) = \frac{1}{1-\phi^2}$.
The covariance of the state is $\gamma_k^s = E[s_t s_{t-k}] = \phi^k \frac{1}{1-\phi^2}$.

For $y_t$:

\textbf{Variance ($\gamma_0$):}
\begin{equation*}
  \gamma_0 = E[(\lambda s_t + u_t)^2] = \lambda^2 \text{Var}(s_t) + \text{Var}(u_t) = \frac{\lambda^2}{1-\phi^2} + 1
\end{equation*}

\textbf{Autocovariance ($\gamma_k, k \ge 1$):}
\begin{align*}
  \gamma_k & = E[y_t y_{t-k}] = E[(\lambda s_t + u_t)(\lambda s_{t-k} + u_{t-k})] \\
           & = \lambda^2 E[s_t s_{t-k}] = \lambda^2 \frac{\phi^k}{1-\phi^2}
\end{align*}
Since $u_t$ is independent of $s_{t-k}$, $u_{t-k}$, and $s_t$ (for $k \ge 1$).

\[
  s_t=\sum_{j=0}^\infty \phi^j\varepsilon_{t-j}
  \Rightarrow \mathbb E[s_t]=0,\
  \mathbb V[s_t]=\sum_{j\ge0}\phi^{2j}=\frac{1}{1-\phi^2},
\]
\[
  \mathrm{Cov}(s_t,s_{t-k})
  =\sum_{j\ge0}\sum_{\ell\ge0}\phi^{j+\ell}\,\mathrm{Cov}(\varepsilon_{t-j},\varepsilon_{t-k-\ell})
  =\sum_{j\ge0}\phi^{j+k+j}=\frac{\phi^k}{1-\phi^2}.
\]
\[
  y_t=\lambda s_t+u_t\ \perp\ s_t
  \Rightarrow
  \gamma_0=\mathbb V[y_t]=\lambda^2\mathbb V[s_t]+1
  =1+\frac{\lambda^2}{1-\phi^2},
\]
\[
  \gamma_k=\mathrm{Cov}(y_t,y_{t-k})
  =\lambda^2\mathrm{Cov}(s_t,s_{t-k})
  =\frac{\lambda^2\phi^k}{1-\phi^2}\quad(k\ge1).
\]

\subsection*{(b) Identification}
We have two unknown parameters $(\lambda, \phi)$ and we observe the autocovariances of $y$.

\begin{enumerate}
  \item From $\gamma_1 = \frac{\lambda^2 \phi}{1-\phi^2}$ and $\gamma_0 = \frac{\lambda^2}{1-\phi^2} + 1$, notice that $\gamma_0 - 1 = \frac{\lambda^2}{1-\phi^2}$.
  \item Thus, $\frac{\gamma_1}{\gamma_0 - 1} = \phi$.
  \item Once $\phi$ is identified, $\lambda^2 = (\gamma_0 - 1)(1-\phi^2)$.
\end{enumerate}

\textbf{Result:} The coefficients are identified (up to the sign of $\lambda$, as only $\lambda^2$ enters the second moments).

\subsection*{(c) ARMA Representation}
From the state equation: $(1 - \phi L)s_t = \varepsilon_t \implies s_t = \frac{\varepsilon_t}{1 - \phi L}$.
Substitute into measurement equation:
\begin{equation*}
  y_t = \lambda \frac{\varepsilon_t}{1 - \phi L} + u_t
\end{equation*}
Multiply by $(1 - \phi L)$:
\begin{align*}
  (1 - \phi L)y_t\;  & = \lambda \varepsilon_t + (1 - \phi L)u_t    \\
  y_t - \phi y_{t-1} & = \lambda \varepsilon_t + u_t - \phi u_{t-1}
\end{align*}
Let the RHS be $w_t$. Since $w_t$ is a sum of MA processes, it is an MA(1) process $w_t = \nu_t + \theta \nu_{t-1}$.
The LHS is AR(1). Thus, $y_t$ follows an \textbf{ARMA(1,1)} process.
Parameters $(\phi_{AR}, \theta_{MA}, \sigma^2_\nu)$ are functions of $(\lambda, \phi, 1, 1)$.
$\phi_{AR} = \phi$.

\[
  (1-\phi L)y_t=\lambda\varepsilon_t+u_t-\phi u_{t-1}=:w_t.
\]
Calculate the second moments of $w_t$:
\[
  \gamma_0^w=\mathbb V[w_t]=\lambda^2+\mathbb V[u_t]+\phi^2\mathbb V[u_{t-1}]
  =\lambda^2+1+\phi^2,
\]
\[
  \gamma_1^w=\mathrm{Cov}(w_t,w_{t-1})
  =\mathrm{Cov}(-\phi u_{t-1},u_{t-1})=-\phi.
\]
For an MA(1) process: $w_t=v_t+\theta v_{t-1}$, where $v_t\sim\mathcal N(0,\sigma_v^2)$:
\[
  \gamma_0^w=(1+\theta^2)\sigma_v^2,\quad \gamma_1^w=\theta\sigma_v^2.
\]
Matching the equations:
\[
  \theta\sigma_v^2=-\phi,\quad (1+\theta^2)\sigma_v^2=\lambda^2+1+\phi^2
\]
\[
  \Rightarrow -\phi(1+\theta^2)=\theta(\lambda^2+1+\phi^2)
  \Rightarrow \boxed{\ \phi\theta^2+(\lambda^2+1+\phi^2)\theta+\phi=0\ }.
\]
\[
  \sigma_v^2=-\frac{\phi}{\theta},\quad \text{choosing the root where }|\theta|<1.
\]

\hrulefill

\subsection*{(i) Allowing Correlation: $\mathrm{Cov}(u_t,\varepsilon_t)=\rho$}

\textbf{Decomposition Method} (Moving the correlation to the measure in one step):
\[
  u_t=\rho\,\varepsilon_t+\eta_t,\qquad \eta_t\sim\mathcal N(0,,1-\rho^2),\ \eta_t\perp \varepsilon_s\ \forall s.
\]
Thus:
\[
  y_t=\lambda s_t+\rho\varepsilon_t+\eta_t.
\]

\paragraph{ACF (Term-by-term)}
\[
  \mathrm{Cov}(s_t,u_t)=\mathrm{Cov}\Big(\sum_{j\ge0}\phi^j\varepsilon_{t-j},\ \rho\varepsilon_t+\eta_t\Big)=\rho,
\]
\[
  \gamma_0=\mathbb V[y_t]
  =\lambda^2\mathbb V[s_t]+\mathbb V[u_t]+2\lambda\,\mathrm{Cov}(s_t,u_t)
  =\frac{\lambda^2}{1-\phi^2}+1+2\lambda\rho,
\]
\[
  \gamma_1=\mathrm{Cov}(y_t,y_{t-1})
  =\lambda^2\frac{\phi}{1-\phi^2}+\lambda\,\phi\rho,
  \qquad
  \gamma_k=\frac{\lambda^2\phi^k}{1-\phi^2}\ (k\ge2).
\]

\paragraph{Equivalent ARMA(1,1) (Explicitly writing $w_t$)}
\[
  (1-\phi L)y_t
  =(\lambda+\rho)\varepsilon_t+\eta_t-\phi\rho\,\varepsilon_{t-1}-\phi\,\eta_{t-1}
  =:w_t.
\]
Second moments:
\[
  \gamma_0^w
  =\underbrace{(\lambda+\rho)^2+\phi^2\rho^2}_{\text{from }\varepsilon}
  +\underbrace{(1-\rho^2)(1+\phi^2)}_{\text{from }\eta}
  =\lambda^2+1+\phi^2+2\lambda\rho,
\]
\[
  \gamma_1^w
  =\mathrm{Cov}(-\phi\rho\,\varepsilon_{t-1}-\phi\eta_{t-1},\ (\lambda+\rho)\varepsilon_{t-1}+\eta_{t-1})
  =-\phi\big(\rho(\lambda+\rho)+(1-\rho^2)\big)
  =-\phi(1+\lambda\rho).
\]
Matching with MA(1) moments:
\[
  \theta\sigma_v^2=\gamma_1^w=-\phi(1+\lambda\rho),\quad
  (1+\theta^2)\sigma_v^2=\gamma_0^w=\lambda^2+1+\phi^2+2\lambda\rho,
\]
\[
  \Rightarrow\ -\phi(1+\lambda\rho)(1+\theta^2)=\theta(\lambda^2+1+\phi^2+2\lambda\rho),
\]
\[
  \Rightarrow\ \boxed{\ \phi(1+\lambda\rho)\theta^2+(\lambda^2+1+\phi^2+2\lambda\rho)\theta+\phi(1+\lambda\rho)=0\ },
\]
\[
  \sigma_v^2=-\frac{\phi(1+\lambda\rho)}{\theta},\quad |\theta|<1.
\]

\subsection*{(d) - (h) Code Implementation}
Below is the Python code for the Kalman Filter, plotting, and optimization. Following that are the R and Julia translations.

\begin{lstlisting}[language=R]
  set.seed(2025)
  
  simulate_ssm <- function(T=100, lam=1.0, phi=0.8, rho=0.0, seed=NULL){
    if(!is.null(seed)) set.seed(seed)
    
    Sigma <- matrix(c(1, rho, rho, 1), 2, 2)
    eps_u <- MASS::mvrnorm(T, mu=c(0,0), Sigma=Sigma)
    eps <- eps_u[,1]
    u <- eps_u[,2]
    
    s <- numeric(T)
    s[1] <- rnorm(1, 0, sqrt(1/(1-phi^2)))
    
    for(t in 2:T){
      s[t] <- phi * s[t-1] + eps[t]
    }
    
    y <- lam * s + u
    list(s=s, y=y, eps=eps, u=u)
  }
  
  kalman_filter <- function(y, lam, phi, rho=0.0){
    Tn <- length(y)

    m_pred <- P_pred <- m_filt <- P_filt <- loglik_t <- numeric(Tn)

    m_prev <- 0.0
    P_prev <- 1/(1-phi^2)
    
    for(t in 1:Tn){
      m_t_pred <- phi * m_prev
      P_t_pred <- phi^2 * P_prev + 1.0
      v_t <- y[t] - lam * m_t_pred
      F_t <- lam^2 * P_t_pred + 1.0 + 2 * lam * rho 

      K_t <- (lam * P_t_pred + rho) / F_t

      m_t_filt <- m_t_pred + K_t * v_t
      P_t_filt <- P_t_pred - K_t * (lam * P_t_pred + rho)

      m_pred[t] <- m_t_pred    # E[s_t|Y_{1:t-1}]
      P_pred[t] <- P_t_pred    # V[s_t|Y_{1:t-1}]
      m_filt[t] <- m_t_filt    # E[s_t|Y_{1:t}]
      P_filt[t] <- P_t_filt    # V[s_t|Y_{1:t}]

      loglik_t[t] <- -0.5*(log(2*pi) + log(F_t) + v_t^2/F_t)
      
      m_prev <- m_t_filt
      P_prev <- P_t_filt
    }
    
    list(m_pred=m_pred, P_pred=P_pred, 
         m_filt=m_filt, P_filt=P_filt, 
         loglik_t=loglik_t)
  }

  loglik_ssm <- function(y, lam, phi, rho=0.0){
    sum(kalman_filter(y, lam, phi, rho)$loglik_t)
  }
  
  # =========================
  # (d)-(e)
  # =========================

  lam_true <- 0.9
  phi_true <- 0.7
  T_val <- 100
  
  sim <- simulate_ssm(T=T_val, lam=lam_true, phi=phi_true, rho=0.0, seed=42)
  y <- sim$y
  s <- sim$s

  out_true <- kalman_filter(y, lam=lam_true, phi=phi_true, rho=0.0)

  out_wrong <- kalman_filter(y, lam=0.5, phi=0.3, rho=0.0)
 
  tgrid <- 1:T_val
  se_pred <- sqrt(out_true$P_pred)

  plot(tgrid, s, type="l", lwd=2, col=1,
       ylab="State", xlab="Time",
       main="True vs Filtered States",
       ylim=range(s, out_true$m_pred, out_wrong$m_pred))
  lines(tgrid, out_true$m_pred, col="blue", lwd=1.5)
  lines(tgrid, out_true$m_pred + 1.96*se_pred, col="blue", lty=2, lwd=1)
  lines(tgrid, out_true$m_pred - 1.96*se_pred, col="blue", lty=2, lwd=1)
  lines(tgrid, out_wrong$m_pred, col="red", lwd=1.5, lty=3)
  legend("topleft", 
         c("True state s_t", 
           "Filtered E[s_t|Y_{1:t-1}] (true params)",
           "95% prediction bands",
           "Filtered (wrong params)"),
         col=c(1, "blue", "blue", "red"),
         lty=c(1,1,2,3), lwd=c(2,1.5,1,1.5),
         bty="n", cex=0.8)

  plot(tgrid, exp(out_true$loglik_t), type="l", lwd=2, col="blue",
       ylab="Likelihood increment p(y_t|Y_{1:t-1})",
       xlab="Time", main="Likelihood Increments")
  lines(tgrid, exp(out_wrong$loglik_t), col="red", lwd=1.5, lty=2)
  legend("topright", c("True parameters", "Wrong parameters"),
         col=c("blue", "red"), lty=1:2, lwd=2:1.5, bty="n")
\end{lstlisting}

\begin{lstlisting}[language=R]
  
  # =========================
  # (f)
  # =========================

  phi_grid <- seq(0.01, 0.99, length.out=200)
  ll_grid <- sapply(phi_grid, function(p) loglik_ssm(y, lam=lam_true, phi=p))
  
  phi_hat_grid <- phi_grid[which.max(ll_grid)]
  cat(sprintf("Grid search MLE: \phi_hat = %.4f (true \phi = %.2f)\n", 
  phi_hat_grid, phi_true))
  
  plot(phi_grid, ll_grid, type="l", lwd=2,
  xlab="\phi", ylab="Log-likelihood",
  main="Profile Log-likelihood for \phi")
  abline(v=phi_true, col="green", lty=2, lwd=1.5)
  abline(v=phi_hat_grid, col="red", lty=3, lwd=1.5)
  legend("topright", c("Log-likelihood", "True \phi", "MLE \phi"),
  col=c(1, "green", "red"), lty=c(1,2,3), lwd=c(2,1.5,1.5), bty="n")
\end{lstlisting}

\begin{lstlisting}[language=R]
  
# =========================
# (g)
# =========================

sample_sizes <- c(50, 100, 500)
colors <- c("red", "blue", "green")

plot(phi_grid, ll_grid, type="n",
     xlab="\phi", ylab="Log-likelihood",
     main="Log-likelihood for Different Sample Sizes",
     ylim=c(-1000, 0))

for (i in 1:length(sample_sizes)) {
  T_i <- sample_sizes[i]
  sim_i <- simulate_ssm(T=T_i, lam=lam_true, phi=phi_true, seed=100+T_i)
  ll_i <- sapply(phi_grid, function(p) loglik_ssm(sim_i$y, lam=lam_true, phi=p))
  lines(phi_grid, ll_i, col=colors[i], lwd=1.5)
}

legend("bottomleft", legend=paste0("T=", sample_sizes),
       col=colors, lty=1, lwd=1.5, bty="n")
\end{lstlisting}

\begin{lstlisting}[language=R]
  # =========================
  # (h)
  # =========================
  
  neg_loglik <- function(phi) {
      if(phi <= 0 || phi >= 1) return(Inf)
      -loglik_ssm(y, lam=lam_true, phi=phi)
    }
  
  opt_result <- optimize(neg_loglik, interval=c(0.01, 0.99))
  phi_hat_opt <- opt_result$minimum
  
  cat(sprintf("Numerical optimization:\n"))
  cat(sprintf("  \phi_hat (optim) = %.6f\n", phi_hat_opt))
  cat(sprintf("  \phi_hat (grid)  = %.6f\n", phi_hat_grid))
  cat(sprintf("  Difference    = %.6f\n", abs(phi_hat_opt - phi_hat_grid)))
  cat(sprintf("  Log-lik at opt = %.4f\n", -opt_result$objective))
\end{lstlisting}

\begin{lstlisting}[language=R]
  
# =========================
# (i)-(j)
# =========================

rho_val <- 0.5
sim_corr <- simulate_ssm(T=100, lam=lam_true, phi=phi_true, rho=rho_val, seed=123)

acf_y <- acf(sim_corr$y, lag.max=10, plot=FALSE)
cat("\nSample autocovariances (with correlation):\n")
for(k in 0:3) {
  cat(sprintf("\gamma [%d] = %.4f\n", k, acf_y$acf[k+1]*var(sim_corr$y)))
}

kf_corr <- kalman_filter(sim_corr$y, lam=lam_true, phi=phi_true, rho=rho_val)

kf_wrong <- kalman_filter(sim_corr$y, lam=lam_true, phi=phi_true, rho=0.0)

plot(tgrid, sim_corr$s, type="l", lwd=2, col=1,
     ylab="State", xlab="Time",
     main="KF with Correlated Errors",
     ylim=range(sim_corr$s, kf_corr$m_pred, kf_wrong$m_pred))
lines(tgrid, kf_corr$m_pred, col="blue", lwd=1.5)
lines(tgrid, kf_wrong$m_pred, col="red", lwd=1.5, lty=2)
legend("topleft", 
       c("True state", "KF with correct rho", "KF assuming rho=0"),
       col=c(1, "blue", "red"), lty=c(1,1,2), lwd=c(2,1.5,1.5),
       bty="n", cex=0.8)

ll_corr <- sum(kf_corr$loglik_t)
ll_wrong <- sum(kf_wrong$loglik_t)
cat(sprintf("\nLog-likelihood comparison:\n"))
cat(sprintf("  With correct rho=%.2f: %.4f\n", rho_val, ll_corr))
cat(sprintf("  Assuming rho=0:        %.4f\n", ll_wrong))
cat(sprintf("  Difference:            %.4f\n", ll_corr - ll_wrong))

par(mfrow=c(1,1))
\end{lstlisting}


\subsection*{(i) Correlated Errors}
Suppose $\text{Cov}(u_t, \varepsilon_t) = \rho$.

\textbf{Autocovariance:}
$\gamma_0 = \lambda^2 Var(s_t) + Var(u_t) + 2\lambda Cov(s_t, u_t)$.
Since $s_t = \phi s_{t-1} + \varepsilon_t$, $Cov(s_t, u_t) = Cov(\varepsilon_t, u_t) = \rho$.
\begin{equation*}
  \gamma_0 = \frac{\lambda^2}{1-\phi^2} + 1 + 2\lambda\rho
\end{equation*}

$\gamma_1 = E[(\lambda s_t + u_t)(\lambda s_{t-1} + u_{t-1})]$.
$E[u_t s_{t-1}] = 0$, $E[u_t u_{t-1}] = 0$.
$E[s_t u_{t-1}] = E[(\phi s_{t-1} + \varepsilon_t)u_{t-1}] = \phi Cov(s_{t-1}, u_{t-1}) = \phi \rho$.
\begin{equation*}
  \gamma_1 = \lambda^2 \phi Var(s_t) + \lambda E[s_t u_{t-1}] = \lambda^2 \frac{\phi}{1-\phi^2} + \lambda \phi \rho
\end{equation*}

\textbf{Identification:} Yes, if moments differ, though the mapping is more complex.

\textbf{ARMA:} Still ARMA(1,1) because it is the sum of two correlated processes, one AR(1) and one White Noise. The spectral density will maintain the rational form.

\subsection*{(j) Generalized Kalman Filter with Correlation}
If $E[u_t \varepsilon_t] = \rho \neq 0$, the innovation in the measurement ($u_t$) contains information about the innovation in the state ($\varepsilon_t$).
Standard KF Prediction step ($s_{t|t-1} \to s_{t+1|t}$) must change.
The posterior of the state $s_t$ given $y_t$ updates as usual, but when predicting $s_{t+1} = \phi s_t + \varepsilon_t$, we must note that $\varepsilon_t$ is correlated with the measurement error $u_t$ contained in $y_t$.

\textbf{Modified Algorithm:}

\begin{enumerate}
  \item \textbf{State Prediction:} $s_{t|t-1}$ (Same)
  \item \textbf{Measurement Prediction:} $y_{t|t-1} = \lambda s_{t|t-1}$. Error $v_t = y_t - y_{t|t-1}$.
  \item \textbf{Covariance of Innovation:}
        \begin{equation*}
          Cov(s_{t+1}, y_t | t-1) = E[(\phi(s_t - s_{t|t-1}) + \varepsilon_t)(\lambda(s_t - s_{t|t-1}) + u_t)] = \phi \lambda P_{t|t-1} + \rho
        \end{equation*}
        \textit{(Note the addition of $\rho$)}.
  \item \textbf{Kalman Gain:}
        \begin{equation*}
          K_t = (\phi \lambda P_{t|t-1} + \rho) F_t^{-1}
        \end{equation*}
  \item \textbf{State Update (Predict next step directly):}
        \begin{align*}
          s_{t+1|t} & = \phi s_{t|t-1} + K_t v_t            \\
          P_{t+1|t} & = \phi^2 P_{t|t-1} + 1 - K_t F_t K_t'
        \end{align*}
        \textit{(The standard KF separates update $t|t$ and predict $t+1|t$, but with correlation it is often cleaner to write the one-step ahead recursion directly).}
\end{enumerate}


Given $Y_{1:t-1}$, denote:
\[
  m_{t|t-1}=\mathbb E[s_t\mid Y_{1:t-1}],\quad
  P_{t|t-1}=\mathbb V[s_t\mid Y_{1:t-1}].
\]
State and Observation:
\[
  s_t=\phi s_{t-1}+\varepsilon_t,\quad y_t=\lambda s_t+u_t,\quad \mathrm{Cov}(\varepsilon_t,u_t)=\rho.
\]

\paragraph{1) Prediction:}
\[
  m_{t|t-1}=\phi\,m_{t-1|t-1},\qquad
  P_{t|t-1}=\phi^2 P_{t-1|t-1}+1.
\]

\paragraph{2) Innovation and its Variance (Explicit term-wise covariance):}
\[
  v_t\equiv y_t-\lambda m_{t|t-1}=\lambda(s_t-m_{t|t-1})+u_t.
\]
Thus:
\[
  \begin{aligned}
    F_t & =\mathbb V[v_t\mid Y_{1:t-1}]
    =\mathbb V[\lambda(s_t-m_{t|t-1})+u_t]                                                              \\
        & =\lambda^2\,\mathbb V[s_t-m_{t|t-1}]+\mathbb V[u_t]+2\lambda\,\mathrm{Cov}(s_t-m_{t|t-1},u_t) \\
        & =\lambda^2 P_{t|t-1}+1+2\lambda\,\mathrm{Cov}(s_t,u_t).
  \end{aligned}
\]
Also:
\[
  \mathrm{Cov}(s_t,u_t)=\mathrm{Cov}(\phi s_{t-1}+\varepsilon_t,\ u_t)
  =\mathrm{Cov}(\varepsilon_t,u_t)=\rho,
\]
So:
\[
  \boxed{\ F_t=\lambda^2 P_{t|t-1}+1+2\lambda\rho\ }.
\]

\paragraph{3) Gain (Using ``Cross-Covariance / Innovation Variance''):}
\[
  \begin{aligned}
    K_t
     & =\frac{\mathrm{Cov}(s_t,v_t\mid Y_{1:t-1})}{F_t}
    =\frac{\mathrm{Cov}(s_t,\lambda(s_t-m_{t|t-1})+u_t)}{F_t}               \\
     & =\frac{\lambda\,\mathbb V[s_t-m_{t|t-1}]+\mathrm{Cov}(s_t,u_t)}{F_t}
    =\boxed{\ \frac{\lambda P_{t|t-1}+\rho}{F_t}\ }.
  \end{aligned}
\]

\paragraph{4) Update:}
\[
  m_{t|t}=m_{t|t-1}+K_t v_t,\qquad
  P_{t|t}=P_{t|t-1}-K_t\,\mathrm{Cov}(s_t,v_t\mid Y_{1:t-1})
  =P_{t|t-1}-K_t(\lambda P_{t|t-1}+\rho).
\]

\paragraph{5) Log-Likelihood Increment:}
\[
  \log p(y_t\mid Y_{1:t-1})
  =-\tfrac12\Big(\log 2\pi+\log F_t+\frac{v_t^2}{F_t}\Big).
\]

\end{document}

