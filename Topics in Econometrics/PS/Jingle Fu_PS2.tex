\documentclass[a4paper,12pt]{article} % This defines the style of your paper

\usepackage[top = 2.5cm, bottom = 2.5cm, left = 2.5cm, right = 2.5cm]{geometry} 

% Unfortunately, LaTeX has a hard time interpreting German Umlaute. The following two lines and packages should help. If it doesn't work for you please let me know.
\usepackage[T1]{fontenc}
\usepackage[utf8]{inputenc}
\usepackage{pifont}
% \usepackage{ctex}
\usepackage{amsthm, amsmath, amssymb, mathrsfs,mathtools}

% Defining a new theorem style without italics
\newtheoremstyle{nonitalic}% name
  {\topsep}% Space above
  {\topsep}% Space below
  {\upshape}% Body font
  {}% Indent amount
  {\bfseries}% Theorem head font
  {.}% Punctuation after theorem head
  {.5em}% Space after theorem head
  {}% Theorem head spec (can be left empty, meaning ‘normal`)
  
\theoremstyle{nonitalic}
% Define new 'solution' environment
\newtheorem{innercustomsol}{Solution}
\newenvironment{solution}[1]
  {\renewcommand\theinnercustomsol{#1}\innercustomsol}
  {\endinnercustomsol}

% Custom counter for the solutions
\newcounter{solutionctr}
\renewcommand{\thesolutionctr}{(\alph{solutionctr})}

% Environment for auto-numbering with custom format
\newenvironment{autosolution}
  {\stepcounter{solutionctr}\begin{solution}{\thesolutionctr}}
  {\end{solution}}


\newtheorem{problem}{Problem}
\usepackage{color}

% The following two packages - multirow and booktabs - are needed to create nice looking tables.
\usepackage{multirow} % Multirow is for tables with multiple rows within one cell.
\usepackage{booktabs} % For even nicer tables.

% As we usually want to include some plots (.pdf files) we need a package for that.
\usepackage{graphicx} 
\usepackage{subfigure}


% The default setting of LaTeX is to indent new paragraphs. This is useful for articles. But not really nice for homework problem sets. The following command sets the indent to 0.
\usepackage{setspace}
\setlength{\parindent}{0in}
\usepackage{longtable}

% Package to place figures where you want them.
\usepackage{float}

% The fancyhdr package let's us create nice headers.
\usepackage{fancyhdr}

\usepackage{fancyvrb}

%Code environment 
\usepackage{listings} % Required for insertion of code
\usepackage{xcolor} % Required for custom colors

% Define colors for code listing
\definecolor{codegreen}{rgb}{0,0.6,0}
\definecolor{codegray}{rgb}{0.5,0.5,0.5}
\definecolor{codepurple}{rgb}{0.58,0,0.82}
\definecolor{backcolour}{rgb}{0.95,0.95,0.92}

% Code listing style named "mystyle"
\lstdefinestyle{mystyle}{
    backgroundcolor=\color{backcolour},   
    commentstyle=\color{codegreen},
    keywordstyle=\color{magenta},
    numberstyle=\tiny\color{codegray},
    stringstyle=\color{codepurple},
    basicstyle=\ttfamily\footnotesize, % Change to serif font
    breakatwhitespace=false,         
    breaklines=true,                 
    captionpos=b,                    
    keepspaces=true,                 
    numbers=left,                    
    numbersep=5pt,                  
    showspaces=false,                
    showstringspaces=false,
    showtabs=false,                  
    tabsize=2
}

\lstset{style=mystyle}

%%%%%%%%%%%%%%%%%%%%%%%%%%%%%%%%%%%%%%%%%%%%%%%%
% 3. Header (and Footer)
%%%%%%%%%%%%%%%%%%%%%%%%%%%%%%%%%%%%%%%%%%%%%%%%

% To make our document nice we want a header and number the pages in the footer.

\pagestyle{fancy} % With this command we can customize the header style.

\fancyhf{} % This makes sure we do not have other information in our header or footer.

\lhead{\footnotesize EI035 Econometrics II}% \lhead puts text in the top left corner. \footnotesize sets our font to a smaller size.

%\rhead works just like \lhead (you can also use \chead)
\rhead{\footnotesize Jingle Fu} %<---- Fill in your lastnames.

% Similar commands work for the footer (\lfoot, \cfoot and \rfoot).
% We want to put our page number in the center.
\cfoot{\footnotesize \thepage}
\IfFileExists{upquote.sty}{\usepackage{upquote}}{}
\begin{document}


\thispagestyle{empty} % This command disables the header on the first page. 

\begin{tabular}{p{15.5cm}} % This is a simple tabular environment to align your text nicely 
  {\large \bf EI035 Econometrics II}                 \\
  The Graduate Institute, Spring 2025, Marko Mlikota \\
  \hline % \hline produces horizontal lines.
  \\
\end{tabular} % Our tabular environment ends here.

\vspace*{0.3cm} % Now we want to add some vertical space in between the line and our title.

\begin{center} % Everything within the center environment is centered.
  {\Large \bf PS2 Solutions} % <---- Don't forget to put in the right number
  \vspace{2mm}

  % YOUR NAMES GO HERE
  {\bf Jingle Fu} % <---- Fill in your names here!

\end{center}

\vspace{0.4cm}
\setstretch{1.2}


\section*{Problem 1: Dynamic Panel Data with Correlated Random Effects}

\subsection*{Model}
\begin{equation*}
  y_{it} = \alpha_i + \rho y_{it-1} + u_{it}, \quad u_{it} \sim iid \mathcal{N}(0, 1)
\end{equation*}

\subsection*{CRE Distribution}
\begin{equation*}
  \alpha_i | (y_{i0}, \phi) \sim \mathcal{N}(\phi y_{i0}, 1)
\end{equation*}

\subsection*{(a) The Incidental Parameter Problem (IPP)}
The incidental parameter problem arises in panel data models when the number of parameters to be estimated grows with the sample size $N$. Here, the unit-specific effects $\alpha_1, ..., \alpha_N$ are the incidental parameters.

\textbf{Manifestation:} In a dynamic panel (where $y_{it-1}$ is a regressor), the standard Fixed Effects (Within) estimator or the naive MLE for $\alpha_i$ and $\rho$ yields inconsistent estimates for $\rho$ when $N \to \infty$ but $T$ remains fixed.
This happens because the estimation error of $\alpha_i$ (which does not vanish as $N \to \infty$) is correlated with the regressor $y_{it-1}$ (since $y_{it-1}$ contains $\alpha_i$). This induces a downward bias in the estimate of $\rho$, commonly known as the \textbf{Nickell Bias} (of order $1/T$).

\subsection*{(b) Integrating out $\alpha_i$}
To integrate out $\alpha_i$, we view the model vector-wise for individual $i$.
Let $\tilde{y}_{it} = y_{it} - \rho y_{it-1}$. The structural equation becomes:
\begin{equation*}
  \tilde{y}_{it} = \alpha_i + u_{it}
\end{equation*}
In vector notation for $t=1:T$:
\begin{equation*}
  \tilde{y}_i = \mathbf{1}_T \alpha_i + u_i
\end{equation*}
where $\mathbf{1}_T$ is a column vector of ones.

We are given $\alpha_i = \phi y_{i0} + \eta_i$, where $\eta_i \sim \mathcal{N}(0, 1)$.
Substituting this into the vector equation:
\begin{align*}
  \tilde{y}_i & = \mathbf{1}_T (\phi y_{i0} + \eta_i) + u_i              \\
  \tilde{y}_i & = \mathbf{1}_T \phi y_{i0} + (\mathbf{1}_T \eta_i + u_i)
\end{align*}

The composite error term is $v_i = \mathbf{1}_T \eta_i + u_i$.
We compute the mean and variance of $\tilde{y}_i$ conditional on $y_{i0}$:

\begin{enumerate}
  \item \textbf{Mean:} $E[\tilde{y}_i | y_{i0}] = \mathbf{1}_T \phi y_{i0}$
  \item \textbf{Variance:} $\Omega = \text{Var}(v_i) = E[(\mathbf{1}_T \eta_i + u_i)(\mathbf{1}_T \eta_i + u_i)'] = \mathbf{1}_T \mathbf{1}_T' \text{Var}(\eta_i) + \text{Var}(u_i)$.
        Since $\text{Var}(\eta_i)=1$ and $\text{Var}(u_i) = I_T$:
        \begin{equation*}
          \Omega = \mathbf{1}_T \mathbf{1}_T' + I_T
        \end{equation*}
\end{enumerate}

The marginal likelihood function $p(y_{i1}, \dots, y_{iT} | y_{i0}, \phi, \rho)$ is the multivariate normal density of $\tilde{y}_i$ evaluated at the observed data (transformed by $\rho$), with mean $\mathbf{1}_T \phi y_{i0}$ and covariance $\Omega$:

\begin{equation*}
  p(y_i | y_{i0}, \phi, \rho) = (2\pi)^{-\frac{T}{2}} |\Omega|^{-\frac{1}{2}} \exp \left( -\frac{1}{2} (\tilde{y}_i - \mathbf{1}_T \phi y_{i0})' \Omega^{-1} (\tilde{y}_i - \mathbf{1}_T \phi y_{i0}) \right)
\end{equation*}
\textit{Note: $\tilde{y}_i$ depends on $\rho$.}

\subsection*{(c) Consistency of $(\phi, \rho)$}
\textbf{Yes, $(\phi, \rho)$ can be consistently estimated.}
By integrating out $\alpha_i$, we have removed the incidental parameters. We are left with a likelihood function that depends on a finite number of common parameters ($\phi, \rho$) and data vectors $y_i$.
Assuming cross-sectional independence (observations $i=1:N$ are i.i.d.), the log-likelihood for the whole sample scales with $N$:
\begin{equation*}
  \mathcal{L}(\phi, \rho) = \sum_{i=1}^N \ln p(y_i | y_{i0}, \phi, \rho)
\end{equation*}
Standard Maximum Likelihood theory applies: as $N \to \infty$ (even with fixed $T$), the estimator maximizing this marginal likelihood is consistent and asymptotically normal, provided identification conditions hold.

\subsection*{(d) Estimation of $\alpha_i$}
In a Bayesian (or Correlated Random Effects) framework, since we cannot estimate $\alpha_i$ consistently (it does not converge to a point), we estimate its \textbf{conditional posterior distribution} or its \textbf{conditional expectation (BLUP)} given the data.

Using Bayes' rule: $p(\alpha_i | y_i, y_{i0}) \propto p(y_i | \alpha_i, \dots) p(\alpha_i | y_{i0})$.
Given the normal-normal conjugacy, the estimator would be the posterior mean:
\begin{equation*}
  E[\alpha_i | y_i, y_{i0}] = \hat{\alpha}_i = w \bar{\tilde{y}}_i + (1-w) \phi y_{i0}
\end{equation*}
where $\bar{\tilde{y}}_i$ is the mean of residuals $y_{it} - \rho y_{it-1}$ and $w$ is a shrinkage factor depending on the relative precision of the signal $T/\sigma^2_u$ and the prior precision $1/\sigma^2_\alpha$.

\section*{Problem 2: State-Space Model}

\subsection*{Model}
\begin{align*}
  y_t & = \lambda s_t + u_t                                                                       \\
  s_t & = \phi s_{t-1} + \epsilon_t                                                               \\
  u_t & \sim \mathcal{N}(0,1), \quad \epsilon_t \sim \mathcal{N}(0,1), \quad u_t \perp \epsilon_t
\end{align*}

\subsection*{(a) Autocovariance Function for $y_t$}
Assuming stationarity ($|\phi| < 1$), the variance of the state $s_t$ is $\text{Var}(s_t) = \frac{1}{1-\phi^2}$.
The covariance of the state is $\gamma_k^s = E[s_t s_{t-k}] = \phi^k \frac{1}{1-\phi^2}$.

For $y_t$:

\textbf{Variance ($\gamma_0$):}
\begin{equation*}
  \gamma_0 = E[(\lambda s_t + u_t)^2] = \lambda^2 \text{Var}(s_t) + \text{Var}(u_t) = \frac{\lambda^2}{1-\phi^2} + 1
\end{equation*}

\textbf{Autocovariance ($\gamma_k, k \ge 1$):}
\begin{align*}
  \gamma_k & = E[y_t y_{t-k}] = E[(\lambda s_t + u_t)(\lambda s_{t-k} + u_{t-k})] \\
           & = \lambda^2 E[s_t s_{t-k}] = \lambda^2 \frac{\phi^k}{1-\phi^2}
\end{align*}
Since $u_t$ is independent of $s_{t-k}$, $u_{t-k}$, and $s_t$ (for $k \ge 1$).

\subsection*{(b) Identification}
We have two unknown parameters $(\lambda, \phi)$ and we observe the autocovariances of $y$.

\begin{enumerate}
  \item From $\gamma_1 = \frac{\lambda^2 \phi}{1-\phi^2}$ and $\gamma_0 = \frac{\lambda^2}{1-\phi^2} + 1$, notice that $\gamma_0 - 1 = \frac{\lambda^2}{1-\phi^2}$.
  \item Thus, $\frac{\gamma_1}{\gamma_0 - 1} = \phi$.
  \item Once $\phi$ is identified, $\lambda^2 = (\gamma_0 - 1)(1-\phi^2)$.
\end{enumerate}

\textbf{Result:} The coefficients are identified (up to the sign of $\lambda$, as only $\lambda^2$ enters the second moments).

\subsection*{(c) ARMA Representation}
From the state equation: $(1 - \phi L)s_t = \epsilon_t \implies s_t = \frac{\epsilon_t}{1 - \phi L}$.
Substitute into measurement equation:
\begin{equation*}
  y_t = \lambda \frac{\epsilon_t}{1 - \phi L} + u_t
\end{equation*}
Multiply by $(1 - \phi L)$:
\begin{align*}
  (1 - \phi L)y_t\;  & = \lambda \epsilon_t + (1 - \phi L)u_t    \\
  y_t - \phi y_{t-1} & = \lambda \epsilon_t + u_t - \phi u_{t-1}
\end{align*}
Let the RHS be $w_t$. Since $w_t$ is a sum of MA processes, it is an MA(1) process $w_t = \nu_t + \theta \nu_{t-1}$.
The LHS is AR(1). Thus, $y_t$ follows an \textbf{ARMA(1,1)} process.
Parameters $(\phi_{AR}, \theta_{MA}, \sigma^2_\nu)$ are functions of $(\lambda, \phi, 1, 1)$.
$\phi_{AR} = \phi$.

\subsection*{(d) - (h) Code Implementation}
Below is the Python code for the Kalman Filter, plotting, and optimization. Following that are the R and Julia translations.

\begin{lstlisting}[language=R]
set.seed(2025)

# Simulate Data
simulate_data <- function(T, lam, phi) {
  s <- numeric(T)
  y <- numeric(T)
  s[1] <- rnorm(1, 0, sqrt(1/(1-phi^2)))
  y[1] <- lam * s[1] + rnorm(1)
  
  eps <- rnorm(T)
  u <- rnorm(T)
  
  for(t in 2:T) {
    s[t] <- phi * s[t-1] + eps[t]
    y[t] <- lam * s[t] + u[t]
  }
  return(list(y=y, s=s))
}

# Kalman Filter
kalman_filter <- function(Y, lam, phi) {
  T <- length(Y)
  s_pred <- numeric(T)
  P_pred <- numeric(T)
  s_upd <- numeric(T)
  P_upd <- numeric(T)
  ll_contrib <- numeric(T)
  
  # Initialization
  s_pred[1] <- 0
  P_pred[1] <- 1 / (1 - phi^2)
  
  for(t in 1:T) {
    # Prediction error decomp
    y_pred <- lam * s_pred[t]
    F_t <- lam^2 * P_pred[t] + 1
    v_t <- Y[t] - y_pred
    
    ll_contrib[t] <- -0.5 * log(2 * pi) - 0.5 * log(F_t) - 0.5 * (v_t^2 / F_t)
    
    # Update
    K_t <- P_pred[t] * lam / F_t
    s_upd[t] <- s_pred[t] + K_t * v_t
    P_upd[t] <- P_pred[t] * (1 - K_t * lam)
    
    # Predict next
    if(t < T) {
      s_pred[t+1] <- phi * s_upd[t]
      P_pred[t+1] <- phi^2 * P_upd[t] + 1
    }
  }
  return(list(ll=ll_contrib, s_pred=s_pred, P_pred=P_pred))
}

# Analysis
T <- 100
true_lam <- 1.0
true_phi <- 0.8
data <- simulate_data(T, true_lam, true_phi)

kf_res <- kalman_filter(data$y, true_lam, true_phi)

# Optimization
neg_ll <- function(p) {
  if(abs(p) >= 0.999) return(Inf)
  -sum(kalman_filter(data$y, true_lam, p)$ll)
}

opt_res <- optim(0.5, neg_ll, method="L-BFGS-B", lower=-0.99, upper=0.99)
print(paste("Optimization Estimate:", opt_res$par))
\end{lstlisting}

\subsection*{(i) Correlated Errors}
Suppose $\text{Cov}(u_t, \epsilon_t) = \rho$.

\textbf{Autocovariance:}
$\gamma_0 = \lambda^2 Var(s_t) + Var(u_t) + 2\lambda Cov(s_t, u_t)$.
Since $s_t = \phi s_{t-1} + \epsilon_t$, $Cov(s_t, u_t) = Cov(\epsilon_t, u_t) = \rho$.
\begin{equation*}
  \gamma_0 = \frac{\lambda^2}{1-\phi^2} + 1 + 2\lambda\rho
\end{equation*}

$\gamma_1 = E[(\lambda s_t + u_t)(\lambda s_{t-1} + u_{t-1})]$.
$E[u_t s_{t-1}] = 0$, $E[u_t u_{t-1}] = 0$.
$E[s_t u_{t-1}] = E[(\phi s_{t-1} + \epsilon_t)u_{t-1}] = \phi Cov(s_{t-1}, u_{t-1}) = \phi \rho$.
\begin{equation*}
  \gamma_1 = \lambda^2 \phi Var(s_t) + \lambda E[s_t u_{t-1}] = \lambda^2 \frac{\phi}{1-\phi^2} + \lambda \phi \rho
\end{equation*}

\textbf{Identification:} Yes, if moments differ, though the mapping is more complex.

\textbf{ARMA:} Still ARMA(1,1) because it is the sum of two correlated processes, one AR(1) and one White Noise. The spectral density will maintain the rational form.

\subsection*{(j) Generalized Kalman Filter with Correlation}
If $E[u_t \epsilon_t] = \rho \neq 0$, the innovation in the measurement ($u_t$) contains information about the innovation in the state ($\epsilon_t$).
Standard KF Prediction step ($s_{t|t-1} \to s_{t+1|t}$) must change.
The posterior of the state $s_t$ given $y_t$ updates as usual, but when predicting $s_{t+1} = \phi s_t + \epsilon_t$, we must note that $\epsilon_t$ is correlated with the measurement error $u_t$ contained in $y_t$.

\textbf{Modified Algorithm:}

\begin{enumerate}
  \item \textbf{State Prediction:} $s_{t|t-1}$ (Same)
  \item \textbf{Measurement Prediction:} $y_{t|t-1} = \lambda s_{t|t-1}$. Error $v_t = y_t - y_{t|t-1}$.
  \item \textbf{Covariance of Innovation:}
        \begin{equation*}
          Cov(s_{t+1}, y_t | t-1) = E[(\phi(s_t - s_{t|t-1}) + \epsilon_t)(\lambda(s_t - s_{t|t-1}) + u_t)] = \phi \lambda P_{t|t-1} + \rho
        \end{equation*}
        \textit{(Note the addition of $\rho$)}.
  \item \textbf{Kalman Gain:}
        \begin{equation*}
          K_t = (\phi \lambda P_{t|t-1} + \rho) F_t^{-1}
        \end{equation*}
  \item \textbf{State Update (Predict next step directly):}
        \begin{align*}
          s_{t+1|t} & = \phi s_{t|t-1} + K_t v_t            \\
          P_{t+1|t} & = \phi^2 P_{t|t-1} + 1 - K_t F_t K_t'
        \end{align*}
        \textit{(The standard KF separates update $t|t$ and predict $t+1|t$, but with correlation it is often cleaner to write the one-step ahead recursion directly).}
\end{enumerate}


\end{document}

