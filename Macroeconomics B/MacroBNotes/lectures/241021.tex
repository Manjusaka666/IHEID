\section{A more General Model}
Assuming not full depreciation $0 < \delta < 1$. The household's problem is
\[ 
U = E_t \sum_{t=0}^{\infty} \beta^i U(C_{t+i} , 1 - L_{t+i} )
\]
where the utility function is:
\[
U(C_t, 1-L_t) = \ln C_t + \theta \frac{(1-L_t)^{1-\gamma_l}}{1-\gamma_l}
\]
log in consumption, CES is leisure. $\gamma_l = 1$ shows log in leisure as well. 

\subsection{The log-linearized version}
Small letters denote the log-deviations from the non-stochastic BGP:
$c_t = \ln C_t - \ln C_t^{\ast}$, $l_t = \ln L_t - \ln L_t^{\ast}.$


\textcolor{blue}{1. The production function}
The production function is:
\[ Y_t = K_t^{\alpha} (A_t L_t)^{1-\alpha} \]
rewrite as:
\[ Y_t^{*} e^{y_t} = K^{*^{\alpha}} _t e^{\alpha k_t}(A_t^* L_t^*)^{1-\alpha }e^{(a_t + l_t)(1-\alpha )}  \]
and use the production function to get:
\[ y_t = \alpha k_t + (a_t + l_t)(1-\alpha ).\]

\textcolor{blue}{2. Capital accumulation}
$$K_{t+1}=(1-\delta)K_t+Y_t-C_t$$
re-write as
$$K_{t+1}^*e^{k_{t+1}}=(1-\delta)K_t^*e^{k_t}+Y_t^*e^{y_t}-C_t^*e^{c_t}$$
Now use the fact that for $x$ small enough $e^x\approx1+x$ to get:
$$K_{t+1}^{*}+K_{t+1}^{*}k_{t+1}\approx(1-\delta)K_{t}^{*}+(1-\delta)K_{t}^{*}k_{t}+Y_{t}^{*}+Y_{t}^{*}y_{t}-C_{t}^{*}-C_{t}^{*}c_{t}$$
which simplifies to
$$\frac{K_{t+1}^*}{K_t^*}k_{t+1}\approx(1-\delta)k_t+\frac{Y_t^*}{K_t^*}y_t-\frac{C_t^*}{K_t^*}c_t.$$

\textcolor{blue}{3. The rental rate}
$$R_t=\alpha\left(\frac{A_tL_t}{K_t}\right)^{1-\alpha}+(1-\delta)$$
rewrite as
$$R_t^*e^{r_t}=\alpha\left(\frac{A_t^*L^*}{K_t^*}\right)^{1-\alpha}e^{(1-\alpha)(a_t+l_t-k_t)}+(1-\delta)$$
this is approximately
$$R_t^*r_t\approx\alpha(1-\alpha)\left(\frac{A_t^*L^*}{K_t^*}\right)^{1-\alpha}(a_t+I_t-k_t).$$

\textcolor{blue}{4. The wage rate}
$$W_t=(1-\alpha)A_t^{1-\alpha}\left(\frac{K_t}{L_t}\right)^\alpha $$
re-write as
$$W_t^*e^{w_t}=(1-\alpha)A_t^{*^{1-\alpha}}e^{(1-\alpha)a_t}\left(\frac{K_t^*}{L_t^*}\right)^\alpha e^{\alpha(k_t-I_t)}$$
which simplifies to
$$w_t=(1-\alpha)a_t+\alpha(k_t-I_t).$$

\textcolor{blue}{5. The intra-temporal FOC}
$$\frac{W_t}{C_t}=\theta(1-L_t)^{-\gamma_l}$$
re-write and approximate:
$$\frac{W_t^*e^{W_t}}{C_t^*e^{C_t}}=\theta(1-L^*e^{I_t})^{-\gamma_l}\Rightarrow\frac{W_t^*+W_t^*w_t}{C_t^*+C_t^*c_t}\approx\theta(1-L^*-L^*I_t)^{-\gamma_l}$$
we do a first order taylor approximation around $c_t=w_t=I_t=0$:
$$\frac{W_t^*}{C_t^*}+\frac{W_t^*}{C_t^*}w_t-\frac{W_t^*}{C_t^*}c_t\approx\theta(1-L^*)^{-\gamma_l}+\theta\gamma_l(1-L^*)^{-\gamma_l-1}L^*I_t\\\frac{W_t^*}{C_t^*}(w_t-c_t)\approx\theta\gamma_l(1-L^*)^{-\gamma_l-1}L^*I_t$$
which simplifies to
$$w_t-c_t\approx\gamma_l\frac{L^*}{1-L^*}I_t.$$

\textcolor{blue}{6. The inter-temporal FOC}
$${C_t}=\beta E_t\left(\frac{R_{t+1}}{C_{t+1}}\right) $$
re-write as
$$C_t^*e^{C_t}=\beta E_t\left(\frac{R^*e^{r_{t+1}}}{C_{t+1}^*e^{c_{t+1}}}\right)$$
which is
$$e^{c_t}=E_t\left(\frac{e^{r_{t+1}}}{e^{c_{t+1}}}\right)\Rightarrow\frac1{1+c_t}\approx E_t\left(\frac{1+r_{t+1}}{1+c_{t+1}}\right)$$
finally, do a first order Taylor approximation
$$-c_t\approx E_t(r_{t+1}-c_{t+1}). $$

Blanchard-Kahn conditions


\section{Summary of the RBC model}

\subsection{What basic RBC models cannot do}

\subsubsection{\textcolor{blue}{endogenous persistence}}
We were expecting persistence to come from capital
accumulation, but in RBC model, it goes up for a while, yet quantitatively not enough.

Output dynamics are essentially the same as the assumed
dynamics of the shocks.

\subsubsection{\textcolor{blue}{amplification problem}}
We were expecting this to come from the labor supply. But in the RBC model, 
only in extreme cases is the labor supply and/or capital
accumulation response strong enough to generate multiplier
like responses. 

This means that slower than normal, but still positive
technological growth can cause output to grow slower, but not
to fall

For output to fall, technology must decline, for strongly pro-cyclical labor supply, 
we need strongly pro-cyclical wages.

\subsubsection{\textcolor{blue}{possible solutions:} }
composition bias, implicit contracts,
indivisible labor