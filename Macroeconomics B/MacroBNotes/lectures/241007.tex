\section{Introduction}
Why do we have aggregate fluctuations (booms \& recessions)?

What can we do against them?

\subsection{Review of Undergrad Macro}
Economy divided into three markets: labor market, money market,
goods market.
\begin{itemize}
  \item IS: $Y = \underset{Demand for good}{\underbrace{A(Y, r, T, Y^{\prime e}, r^{\prime e}, T^{\prime e} )}} + G$
  \item LM: $M = P\cdot L(Y, i)$
  \item PC: $\pi = \pi^e + \underset{Deviation of output from MR level}{\underbrace{f(Y, z)}}$
\end{itemize}
where $Y^{\prime e}$ is the expected output of the next period, $r = i - \pi^e$ is the real expected interest rate.

\subsection{Assumptions / implications of this model}
In the short run the price level, $P$ is given, hence IS and LM determines output. Demand matters and many factors affect demand.

In the medium run, $\pi^e = \pi$, so $f(Y, z) = 0$, output at the natural level. 

money affects output in the short run, not in the medium run:
\begin{itemize}
  \item money growth affects inflation and the nominal rate
  one-for-one in the medium run
  \item a fiscal expansion increases output in the short run, may
  decrease it in the medium run
  \item expectations matter: an anticipated fiscal contraction can be
  expansionary
\end{itemize}

\subsection{Strength and weakness of these models}
\begin{itemize}
  \item Strengths:
    \begin{itemize}
      \item provides a simple way of thinking about general equilibrium
      (complicated)
      \item implicit micro-foundations: life cycle, q theory,...
      \item time-tested shortcuts (but often not based on empirical
      evidence)
    \end{itemize}
  \item Weakness:
    \begin{itemize}
      \item Static, not dynamic
      \item not really made for quantitative analysis (quantitative
      attempts very short-lived)
      \item need explicit micro-foundations: hard to do welfare analysis
      without that (Lucas critique: curves may not be invariant to
      policy)
    \end{itemize}
\end{itemize}

\subsection{Business Cycle Facts}

\textbf{Business Cycle Facts I}

We'll study the \textbf{detrended macro times series:}
\[ x_t = \log (X_t) - \log (X_t^{\ast}) \]
is the percentage deviation of variable $X$ from its trend $X^{\ast}$.

How's the trend?

\begin{enumerate}
  \item[1.] First linear,
  
  \item[2.] more sophisticated filters: Baxter-King(Bankpass) filter,
  Hodrick-Prescott(HP) filter
\end{enumerate}

Linear: just run OLS regression and use residuals.

\textbf{Business Cycle Facts II:}
Look at the highest correlation with GDP:
\[
\rho(X_t, Y_{t+k}) \quad k = -6, -5, \dots, 0, \dots, 5, 6
\]
\begin{itemize}
  \item if $\rho > 0$, then $x$ is pro-cyclical.
  \item if $\rho < 0$, then $x$ is counter-cyclical.
  \item if $k < 0$, then $x$ lags behind output.
  \item if $k > 0$, then $x$ leads output.
\end{itemize}
$\rho(X_t, X_{t+k})$ is called the autocorrelation function (as a function of $k$) of the stochastic process $X_t$.

\textbf{Business Cycle Facts III:}

Governmetn spending, not strongly correlated with level of output, main driver
of business cycles.

Real wage is mildly pro-cyclical: average wage evolves differently than the
wage of a continuously employed worker.

\textbf{\textbf{Business Cycle Facts IV:}}

Standard deviations (unit of measure is percentage deviation from
trend).

\begin{itemize}
  \item GDP is more volatile than consumption
  \item investment is much more volatile than GDP
  \item government spending is pretty volatile
  \item working hours is almost exactly as volatile as GDP
  \item vast majority of the volatility of working hours is explained by
  the employment volatility
  \item TFP is very volatile
\end{itemize}

\subsection{Towards the RBC Model}
First, it's important to state the assumptions of our economic model:
\begin{itemize}
  \item We will use a Walrasian model.
  \item The model assumes perfectly competitive markets.
  \item There are no externalities, asymmetric information, missing markets, or other imperfections.
\end{itemize}

\textcolor{blue}{\textbf{Choice of Model}}

All the neoclassical models reviewed so far share these assumptions. A familiar model that could be built upon is the Ramsey model:
\begin{itemize}
  \item The Ramsey model is known for converging to a balanced growth path in the absence of shocks, following which it exhibits smooth growth.
\end{itemize}

\textcolor{blue}{\textbf{Model Extensions}}

It appears sensible to extend the Ramsey model to incorporate:
\begin{itemize}
  \item Business cycle fluctuations.
  \item Real economic shocks.
\end{itemize}
These extensions will enhance the model's capability to reflect dynamic economic changes.

\textcolor{blue}{\textbf{Factors to Consider}}

In extending the model, the following factors should be considered:
\begin{itemize}
  \item Worker productivity.
  \item Government purchases.
\end{itemize}
Notably, since the Ramsey model does not incorporate monetary aspects, all introduced shocks will be in real terms, affecting the actual productive capacity of the economy.

\textcolor{blue}{\textbf{The Real Business Cycle (RBC) Model}}

The modified version of the Ramsey model, incorporating real shocks and business cycle fluctuations, is known as the Real Business Cycle (RBC) model.

\section{The Basic RBC Model}

\subsection{Preparation}
We take the Cobb-Douglas production function as the basis for our model:
\[ Y_t = K_t^{\alpha} (A_t L_t)^{1 - \alpha} \]
Define the technological progress as:
\[ A_t = A_t^* \tilde{A}_t \]
where $A_t^* = G^t \overline{A} $ is the long-run non-stochastic log-linear trend $G>1$.

shock process is an autoregressive process of order 1
\[\log \tilde{A}_t = \rho \log \tilde{A}_{t-1} + \varepsilon_{A, t} \]
where $\varepsilon_{A, t} \sim N(0, \sigma^2)$.

The capital accumulation equation is:
\[ K_{t+1} = (1 - \delta) K_t + I_t \]
where $I_t = Y_t - C_t$ is the investment.

Define the household's utility function as:
\[ U = \mathbb{E}_t \sum_{i=0}^{\infty} \beta^i U(C_{t+i}, 1-L_{t+i} )\]
where $\beta$ is the discount factor, $C$ is consumption, and $L$ is fraction of time spent working.

$U_C (C, 1 - L) > 0, U_{C C} (C, 1 - L) < 0$ and $U_L (C, 1 - L) > 0$, $U_{L
L} (C, 1 - L) < 0$

\subsection{The Household's Problem}

\begin{align*}
  \max_{\{C_t, L_t, K_t, S_t\}} & \quad \mathbb{E}_t \sum_{i=0}^{\infty} \beta^i U(C_{t+i}, 1 - L_{t+i}) \\
  \text{Subject to:} & \quad S_{t+i} + C_{t+i} = \tilde{R}_{t+i} K_{t+i} + W_{t+i} L_{t+i} \\
  & \quad K_{t+i+1} = (1 - \delta) K_{t+i} + S_{t+i} \\
  & \quad K_{t+i} \geq 0; \quad K_0 > 0
\end{align*}
Define the Lagrangian as:
\[
\mathcal{L} = \mathbb{E}_t \sum_{i=0}^{\infty} \beta^i U(C_{t+i}, 1 - L_{t+i}) - \mathbb{E}_t \sum_{i=0}^{\infty} \beta^i \lambda_{t+i} (K_{t+i+1} - (1 - \delta) K_{t+i} - \tilde{R}_{t+i} K_{t+i} - W_{t+i} L_{t+i} + C_{t+i})
\]
We want to take a snapshot of how the variables behave in the period in which we are optimising
in, $t$. Then, the period $t$ variables appear as:
\begin{align*}
  \mathcal{L}_t &= U(C_{t}, 1 - L_{t}) - \lambda_{t} (K_{t+1} - (1 - \delta) K_{t} - \tilde{R}_{t} K_{t} - W_{t} L_{t} + C_{t})
\end{align*}

\subsubsection{Household Behavior I}

The inter-temporal Euler equation is:
\[U_1(X_t, 1-L_t) = \beta \mathbb{E}_t \left(R_{t+1} U_1(C_{t+1}, 1-L_{t+1} ) \right)\]
where $R_{t+1} = 1 + \tilde{R}_{t+1} - \delta $.

\subsection{Effect of RBC}

What are the effects of a positive technological shock?

It increases current and future $R$ and $W$.
\subsubsection{Consumption}
\begin{itemize}
  \item \textbf{Income effect:} People feel richer which leads to an increase in consumption.
  \item \textbf{Substitution effect:} Saving is worth more, thus consumption might decrease.
  \item \textbf{Net effect:} Consumption is probably increased.
\end{itemize}

\subsubsection{Leisure}
\begin{itemize}
  \item \textbf{Income effect:} People feel richer, thus they want to enjoy more leisure, leading to an increase in leisure.
  \item \textbf{Substitution effect:} Higher wage leads to a decrease in leisure.
  \item \textbf{Net effect:} The overall impact on leisure depends on the relative strength of the two effects.
\end{itemize}

\subsubsection{Type of Shock}
\begin{itemize}
  \item \textbf{Transitory shock:} Leads to a smaller wealth effect and a stronger substitution effect.
  \item \textbf{Permanent shock:} It is possible that consumption goes up and employment goes down.
\end{itemize}

\subsection{Brief Summary}
\begin{itemize}
  \item Consumption is too pro-cyclical;
  \item investment $I_t = s_t Y_t = \alpha \beta Y_t$ also pro-cyclical;
  \item effects of R on saving?
  $r$ increase, then inter-temporal substitution effect led $C_t$ decreases, 
  then the income effect increases $C_t$(in this model they cancel out)
  \item $W_t$ is too pro-cyclical
\end{itemize}