\documentclass[a4paper,12pt]{article} % This defines the style of your paper

\usepackage[top = 2.5cm, bottom = 2.5cm, left = 1.5cm, right = 1.5cm]{geometry} 

% Unfortunately, LaTeX has a hard time interpreting German Umlaute. The following two lines and packages should help. If it doesn't work for you please let me know.
\usepackage[T1]{fontenc}
\usepackage[utf8]{inputenc}
\usepackage{pifont}
% \usepackage{ctex}
\usepackage{amsthm, amsmath, amssymb, mathrsfs,mathtools}

% Defining a new theorem style without italics
\newtheoremstyle{nonitalic}% name
  {\topsep}% Space above
  {\topsep}% Space below
  {\upshape}% Body font
  {}% Indent amount
  {\bfseries}% Theorem head font
  {.}% Punctuation after theorem head
  {.5em}% Space after theorem head
  {}% Theorem head spec (can be left empty, meaning ‘normal`)
  
\theoremstyle{nonitalic}
% Define new 'solution' environment
\newtheorem{innercustomsol}{Solution}
\newenvironment{solution}[1]
  {\renewcommand\theinnercustomsol{#1}\innercustomsol}
  {\endinnercustomsol}

% Custom counter for the solutions
\newcounter{solutionctr}
\renewcommand{\thesolutionctr}{(\alph{solutionctr})}

% Environment for auto-numbering with custom format
\newenvironment{autosolution}
  {\stepcounter{solutionctr}\begin{solution}{\thesolutionctr}}
  {\end{solution}}


\newtheorem{problem}{Problem}
\usepackage{color}

% The following two packages - multirow and booktabs - are needed to create nice looking tables.
\usepackage{multirow} % Multirow is for tables with multiple rows within one cell.
\usepackage{booktabs} % For even nicer tables.

% As we usually want to include some plots (.pdf files) we need a package for that.
\usepackage{graphicx} 
\usepackage{subfigure}


% The default setting of LaTeX is to indent new paragraphs. This is useful for articles. But not really nice for homework problem sets. The following command sets the indent to 0.
\usepackage{setspace}
\setlength{\parindent}{0in}
\usepackage{longtable}

% Package to place figures where you want them.
\usepackage{float}

% The fancyhdr package let's us create nice headers.
\usepackage{fancyhdr}

\usepackage{fancyvrb}

%Code environment 
\usepackage{listings} % Required for insertion of code
\usepackage{xcolor} % Required for custom colors

% Define colors for code listing
\definecolor{codegreen}{rgb}{0,0.6,0}
\definecolor{codegray}{rgb}{0.5,0.5,0.5}
\definecolor{codepurple}{rgb}{0.58,0,0.82}
\definecolor{backcolour}{rgb}{0.95,0.95,0.92}

% Code listing style named "mystyle"
\lstdefinestyle{mystyle}{
    backgroundcolor=\color{backcolour},   
    commentstyle=\color{codegreen},
    keywordstyle=\color{magenta},
    numberstyle=\tiny\color{codegray},
    stringstyle=\color{codepurple},
    basicstyle=\ttfamily\footnotesize, % Change to serif font
    breakatwhitespace=false,         
    breaklines=true,                 
    captionpos=b,                    
    keepspaces=true,                 
    numbers=left,                    
    numbersep=5pt,                  
    showspaces=false,                
    showstringspaces=false,
    showtabs=false,                  
    tabsize=2
}

\lstset{style=mystyle}

\pagestyle{fancy} % With this command we can customize the header style.

\fancyhf{} % This makes sure we do not have other information in our header or footer.

\lhead{\footnotesize EI060 Macroeconomics B}% \lhead puts text in the top left corner. \footnotesize sets our font to a smaller size.

%\rhead works just like \lhead (you can also use \chead)
\rhead{\footnotesize Jingle Fu} %<---- Fill in your lastnames.

% Similar commands work for the footer (\lfoot, \cfoot and \rfoot).
% We want to put our page number in the center.
\cfoot{\footnotesize \thepage}
\IfFileExists{upquote.sty}{\usepackage{upquote}}{}
\begin{document}


\thispagestyle{empty} % This command disables the header on the first page. 

\begin{tabular}{p{15.5cm}} % This is a simple tabular environment to align your text nicely 
{\large \bf EI060 Macroeconomics B} \\
The Graduate Institute, Spring 2025, Cedric Tille\\
\hline % \hline produces horizontal lines.
\\
\end{tabular} % Our tabular environment ends here.

\vspace*{0.3cm} % Now we want to add some vertical space in between the line and our title.

\begin{center} % Everything within the center environment is centered.
	{\Large \bf PS1 Solutions} % <---- Don't forget to put in the right number
	\vspace{2mm}
	
        % YOUR NAMES GO HERE
	{\bf Jingle Fu \\ Group members: Yingjie Zhang, Irene Licastro} % <---- Fill in your names here!
		
\end{center}  

\vspace{0.4cm}
\setstretch{1.1}

\section{First generation crisis model}

\subsection{Consumption under a peg}\label{sec:1.1}

We begin with the Euler equation. Under fixed exchange rate, the exchange depreciation rate $\varepsilon_t = 0$.
Thus the interest parity gives that $i = r$, so that $c_t^{-\frac{1}{\sigma}} = \lambda (1 + \theta + \alpha r)$,
as we know that $r=\beta$ and $\theta$ is the tax rate on consumption, which is constant,
consumption will be constant, we can write it as $\tilde{c}$.

We then identify the constant by the intertemporal budget constraint:
\begin{align*}
    \alpha_0 + \frac{y}{r} &= \int _0^{\infty} e^{-rt} \left(\tilde{c}(1+\theta) + r m_t \right) dt \\
    &= \int _0^{\infty} e^{-rt} \left(\tilde{c}(1+\theta) + r \alpha c_t \right) dt \\
    &= \int _0^{\infty} e^{-rt} \tilde{c}(1+\theta + r \alpha) dt \\
    &= \tilde{c} (1+\theta +\alpha r) \int _0^{\infty} e^{-rt} dt \\
    &= \tilde{c} (1+\theta +\alpha r) \frac{1}{r}
\end{align*}
Thus we have:
\begin{gather*}
    \tilde{c} = \frac{\alpha_0 r + y}{1 + \theta + \alpha r}
\end{gather*}



\subsection{Unsustainable peg}\label{sec:1.2}

As we know that the government spending is over a threshold:
\begin{gather*}
    g > r h_0 + \frac{\alpha_0 r + y}{1 + \theta + \alpha r}
\end{gather*}
From \ref{sec:1.1}, we know that the consumption is a constant $\tilde{c} = \frac{\alpha_0 r + y}{1 + \theta + \alpha r}$,
thus the government tax income would be $\theta \tilde{c} = \frac{\theta (\alpha_0 r + y)}{1 + \theta + \alpha r} $.

The government tax revenue is:
\begin{gather*}
    s^p = \theta \tilde{c} - g = \frac{\theta (\alpha_0 r + y)}{1 + \theta + \alpha r} - g < -r h_0 < 0
\end{gather*}
Under a fixed exchange rate, $\varepsilon_t = 0$ and at the steady state, the real balance is constant, $\dot{m}_t = 0$,
we know that the foreign reserves cahnge is:
\begin{gather*}
    \dot{h}_t = r h_t + s_t^p < r h_t - r h_0
\end{gather*}
at $t=0$, $h_0 < 0$. As time passes, the foreign reserves will be negative and the government will default on its debt.
Hence the government cannot continue to run a peg, and the peg is unsustainable.


\subsection{Consumption and depreciation pre- and post-break}\label{sec:1.3}
From the Euler equation, before the abandon of the peg, we have:
\begin{gather*}
    c_1^{-\frac{1}{\sigma}} = \lambda (1 + \theta + \alpha r)
\end{gather*}
and after the abandon, we have:
\begin{gather*}
    c_2^{-\frac{1}{\sigma}} = \lambda (1 + \theta + \alpha (r+\varepsilon)).
\end{gather*}
So,
\begin{gather*}
    \left(\frac{c_1}{c_2}\right) = \frac{1+\theta + \alpha (r+\varepsilon)}{1 + \theta + \alpha r} > 1
\end{gather*}
as $\varepsilon > 0$, giving that after the abando of the peg, the consumption decrceases.

Now we use that budget constraint and the cash in advance constraint, we have:

With $m$ constant and in steady state of reserves: $\dot{h}_t = 0$,
\begin{gather*}
    0 = r h_t + (\theta c_2 - g) + \varepsilon m_2 = \theta (c_2 - c_1 ) + \varepsilon \alpha c_2
\end{gather*}
since $\theta c_1 = g$, and we can solve:
\begin{gather*}
    \varepsilon  = \frac{\theta}{\alpha} \left( \frac{c_1}{c_2} - 1 \right)
\end{gather*}
At the equilibrium values, $\frac{c_1}{c_2} > 1$ and thus $\varepsilon > 0$.


\subsection{Dynamics of reserves and assets}\label{sec:1.4}
Before the break, $s_t^p = \theta c_1 - g = 0$, $\dot{m}_t = 0$ and $\varepsilon_t = 0$, so $\dot{h}_t = r h_t$.

For foreign assets, we have:
\begin{gather*}
    \dot{a}_t = r a_t + y - c_1(1+\theta) - r m_1 = r a_t + y - c_1 (1 + \theta + \alpha r)
\end{gather*}
and the uverall position is:
\begin{gather*}
    \dot{a}_t + \dot{h}_t = r(a_t + h_t) + y - c_1(1 + \theta + \alpha r) = r(a_t + h_t) + y - \frac{g (1 + \theta + \alpha r)}{\theta}
\end{gather*}
since $c_1 = \frac{g}{\theta}$.

Evaluated at $t=0$, we have:
\begin{gather*}
    \dot{h}_0 = r h_ 0, \quad \dot{a}_0 + \dot{h}_0 = r (a_0 + h_0) + y - \frac{g (1 + \theta + \alpha r)}{\theta}
\end{gather*}
which is exactly the current-account balance. Observing $\dot{a} + \dot{h} < 0$ ex-ante would signal an impending crisis.


\subsection{Timing of break}\label{sec:1.5}
Recall that the intertemporal budget constraint is:
\begin{gather*}
    \frac{g}{r} = h_0 + \int_0^{\infty} e^{-rt} \left[  \theta c_t + \dot{m}_t + \varepsilon_t m_t \right] dt + e^{-rT} \left[ m_T - m_{T-} \right].
\end{gather*}

Before the break fo the peg($0 \leq t \leq T$):
\begin{gather*}
    c_t = c_1, \quad \dot{m}_t = 0, \quad \varepsilon_t = 0
\end{gather*}
and after the break, ($t > T$):
\begin{gather*}
    c_t = c_2, \quad \dot{m}_t = 0, \quad \varepsilon_t = \varepsilon > 0
\end{gather*}
Bringing these conditions into the intertemporal budget constraint, we have:
\begin{gather*}
    \frac{g}{r} = h_0 + \int_{0}^{T} e^{-rt} \theta c_1 dt + \int_{T}^{\infty} e^{-rt} \left[ \theta c_2 + \varepsilon m_2 \right] dt + e^{-rT} \left[ m_T - m_{T-} \right].
\end{gather*}
As $\theta c_1 = g$, we can write:
\begin{gather*}
    \int_{0}^{T} e^{-rt} \theta c_1 dt = g \int_{0}^{T} e^{-rt} dt = \frac{g \left( 1 - e^{-rT} \right)}{r} 
\end{gather*}
and that the second term is:
\begin{gather*}
    \int_{T}^{\infty} e^{-rt} \left[ \theta c_2 + \varepsilon m_2 \right] dt = [\theta c_2 + \varepsilon m_2] \frac{e^{-rT}}{r} 
\end{gather*}
Thus,
\begin{gather*}
    \frac{g}{r} = h_0 + \frac{g \left( 1 - e^{-rT} \right)}{r} + \frac{[\theta c_2 + \varepsilon m_2] e^{-rT}}{r} + e^{-rT} \left[ m_T - m_{T-} \right].
\end{gather*}
As $m_2 = \alpha c_2$, $\varepsilon = \frac{\theta}{\alpha} \left( \frac{c_1}{c_2} -1 \right)$, we can further simplify that:
\begin{gather*}
    \theta c_2 + \varepsilon m_2 = \theta c_2 + \frac{\theta}{\alpha} \left( \frac{c_1}{c_2} -1 \right) \alpha c_2 = \theta c_2 + \theta (c_1 - c_2) = \theta c_1 = g
\end{gather*}
Thus the budget constraint is reduced to:
\begin{align*}
    \frac{g}{r} &= h_0 + \frac{g \left( 1 - e^{-rT} \right)}{r} + \frac{g e^{-rT}}{r} + e^{-rT} \left[ m_T - m_{T-} \right] \\
                &= h_0 + \frac{g}{r} + e^{-rT} [m_T - m_{T-} ] \\
    \Rightarrow e^{-rT} [m_T - m_{T-}] &= -h_0 \\
    \Rightarrow T &= \frac{1}{r} \ln \left( \frac{m_{T-} - m_T}{h_0} \right)
\end{align*}

\end{document}

