\documentclass[a4paper,12pt]{article} % This defines the style of your paper

\usepackage[top = 2.5cm, bottom = 2.5cm, left = 1.5cm, right = 1.5cm]{geometry} 

% Unfortunately, LaTeX has a hard time interpreting German Umlaute. The following two lines and packages should help. If it doesn't work for you please let me know.
\usepackage[T1]{fontenc}
\usepackage[utf8]{inputenc}
\usepackage{pifont}
% \usepackage{ctex}
\usepackage{amsthm, amsmath, amssymb, mathrsfs,mathtools}

% Defining a new theorem style without italics
\newtheoremstyle{nonitalic}% name
  {\topsep}% Space above
  {\topsep}% Space below
  {\upshape}% Body font
  {}% Indent amount
  {\bfseries}% Theorem head font
  {.}% Punctuation after theorem head
  {.5em}% Space after theorem head
  {}% Theorem head spec (can be left empty, meaning ‘normal`)
  
\theoremstyle{nonitalic}
% Define new 'solution' environment
\newtheorem{innercustomsol}{Solution}
\newenvironment{solution}[1]
  {\renewcommand\theinnercustomsol{#1}\innercustomsol}
  {\endinnercustomsol}

% Custom counter for the solutions
\newcounter{solutionctr}
\renewcommand{\thesolutionctr}{(\alph{solutionctr})}

% Environment for auto-numbering with custom format
\newenvironment{autosolution}
  {\stepcounter{solutionctr}\begin{solution}{\thesolutionctr}}
  {\end{solution}}


\newtheorem{problem}{Problem}
\usepackage{color}

% The following two packages - multirow and booktabs - are needed to create nice looking tables.
\usepackage{multirow} % Multirow is for tables with multiple rows within one cell.
\usepackage{booktabs} % For even nicer tables.

% As we usually want to include some plots (.pdf files) we need a package for that.
\usepackage{graphicx} 
\usepackage{subfigure}


% The default setting of LaTeX is to indent new paragraphs. This is useful for articles. But not really nice for homework problem sets. The following command sets the indent to 0.
\usepackage{setspace}
\setlength{\parindent}{0in}
\usepackage{longtable}

% Package to place figures where you want them.
\usepackage{float}

% The fancyhdr package let's us create nice headers.
\usepackage{fancyhdr}

\usepackage{fancyvrb}

%Code environment 
\usepackage{listings} % Required for insertion of code
\usepackage{xcolor} % Required for custom colors

% Define colors for code listing
\definecolor{codegreen}{rgb}{0,0.6,0}
\definecolor{codegray}{rgb}{0.5,0.5,0.5}
\definecolor{codepurple}{rgb}{0.58,0,0.82}
\definecolor{backcolour}{rgb}{0.95,0.95,0.92}

% Code listing style named "mystyle"
\lstdefinestyle{mystyle}{
    backgroundcolor=\color{backcolour},   
    commentstyle=\color{codegreen},
    keywordstyle=\color{magenta},
    numberstyle=\tiny\color{codegray},
    stringstyle=\color{codepurple},
    basicstyle=\ttfamily\footnotesize, % Change to serif font
    breakatwhitespace=false,         
    breaklines=true,                 
    captionpos=b,                    
    keepspaces=true,                 
    numbers=left,                    
    numbersep=5pt,                  
    showspaces=false,                
    showstringspaces=false,
    showtabs=false,                  
    tabsize=2
}

\lstset{style=mystyle}

\pagestyle{fancy} % With this command we can customize the header style.

\fancyhf{} % This makes sure we do not have other information in our header or footer.

\lhead{\footnotesize EI060 Macroeconomics B}% \lhead puts text in the top left corner. \footnotesize sets our font to a smaller size.

%\rhead works just like \lhead (you can also use \chead)
\rhead{\footnotesize Jingle Fu} %<---- Fill in your lastnames.

% Similar commands work for the footer (\lfoot, \cfoot and \rfoot).
% We want to put our page number in the center.
\cfoot{\footnotesize \thepage}
\IfFileExists{upquote.sty}{\usepackage{upquote}}{}
\begin{document}


\thispagestyle{empty} % This command disables the header on the first page. 

\begin{tabular}{p{15.5cm}} % This is a simple tabular environment to align your text nicely 
{\large \bf EI060 Macroeconomics B} \\
The Graduate Institute, Spring 2025, Cedric Tille\\
\hline % \hline produces horizontal lines.
\\
\end{tabular} % Our tabular environment ends here.

\vspace*{0.3cm} % Now we want to add some vertical space in between the line and our title.

\begin{center} % Everything within the center environment is centered.
	{\Large \bf PS2 Solutions} % <---- Don't forget to put in the right number
	\vspace{2mm}
	
        % YOUR NAMES GO HERE
	{\bf Jingle Fu \\ Group members: Yingjie Zhang, Irene Licastro} % <---- Fill in your names here!
		
\end{center}  

\vspace{0.4cm}
\setstretch{1.1}

\section{First generation crisis model}

\subsection{Consumption under a peg}\label{sec:1.1}

We begin with the Euler equation. Under fixed exchange rate, the exchange depreciation rate $\varepsilon_t = 0$.
Thus the interest parity gives that $i = r$, so that $c_t^{-\frac{1}{\sigma}} = \lambda (1 + \theta + \alpha r)$,
as we know that $r=\beta$ and $\theta$ is the tax rate on consumption, which is constant,
consumption will be constant, we can write it as $\tilde{c}$.

We then identify the constant by the intertemporal budget constraint:
\begin{align*}
    \alpha_0 + \frac{y}{r} &= \int _0^{\infty} e^{-rt} \left(\tilde{c}(1+\theta) + r m_t \right) dt \\
    &= \int _0^{\infty} e^{-rt} \left(\tilde{c}(1+\theta) + r \alpha c_t \right) dt \\
    &= \int _0^{\infty} e^{-rt} \tilde{c}(1+\theta + r \alpha) dt \\
    &= \tilde{c} (1+\theta +\alpha r) \int _0^{\infty} e^{-rt} dt \\
    &= \tilde{c} (1+\theta +\alpha r) \frac{1}{r}
\end{align*}
Thus we have:
\begin{gather*}
    \tilde{c} = \frac{\alpha_0 r + y}{1 + \theta + \alpha r}
\end{gather*}



\subsection{Unsustainable peg}\label{sec:1.2}

From Question 1.1, we know that the consumption is a constant $\tilde{c} = \frac{\alpha_0 r + y}{1 + \theta + \alpha r}$,
thus the government tax income would be $\theta \tilde{c} = \frac{\theta (\alpha_0 r + y)}{1 + \theta + \alpha r} $.

If the peg holds forever, then $\epsilon_t = 0$, $i = r$, $m_t = \alpha \tilde{c}$:
\begin{align*}
    \frac{g}{r} &= h_0+\int_0^\infty e^{-rt}\left[\theta c_t+\dot{m}_t+\varepsilon_tm_t\right]dt+e^{-rT}(m_T-m_{T-}) \\
    &= h_0+\theta\tilde{c}\int_0^\infty e^{-rt}dt \\
    &= h_0+\frac{\theta\tilde{c}}{r} \\
    \Rightarrow g &= r h_0 + \frac{\theta (\alpha_0 r + y)}{1 + \theta + \alpha r}
\end{align*}

If government spending $g$ exceeds the threshold, we will have:

The government tax revenue is:
\begin{gather*}
    s^p = \theta \tilde{c} - g = \frac{\theta (\alpha_0 r + y)}{1 + \theta + \alpha r} - g < -r h_0 < 0
\end{gather*}
Under a fixed exchange rate, $\varepsilon_t = 0$ and at the steady state, the real balance is constant, $\dot{m}_t = 0$,
we know that the foreign reserves change is:
\begin{gather*}
    \dot{h}_t = r h_t + s_t^p < r h_t - r h_0
\end{gather*}
at $t=0$, $h_0 < 0$. As time passes, the foreign reserves will be negative and the government will default on its debt.
Hence the government cannot continue to run a peg, and the peg is unsustainable.


\subsection{Consumption and depreciation pre- and post-break}\label{sec:1.3}
From the Euler equation, before the abandon of the peg, we have:
\begin{gather*}
    c_1^{-\frac{1}{\sigma}} = \lambda (1 + \theta + \alpha r)
\end{gather*}
and after the abandon, we have:
\begin{gather*}
    c_2^{-\frac{1}{\sigma}} = \lambda (1 + \theta + \alpha (r+\varepsilon)).
\end{gather*}
So,
\begin{gather*}
    \left(\frac{c_1}{c_2}\right)^{\frac{1}{\sigma}} = \frac{1+\theta + \alpha (r+\varepsilon)}{1 + \theta + \alpha r} > 1
\end{gather*}
as $\varepsilon > 0$, giving that after the abando of the peg, the consumption decrceases.

Now we use that budget constraint and the cash in advance constraint, we have:

With $m$ constant and in steady state of reserves: $\dot{h}_t = 0$,
\begin{gather*}
    0 = r h_t + (\theta c_2 - g) + \varepsilon m_2 = \theta (c_2 - c_1 ) + \varepsilon \alpha c_2
\end{gather*}
since $\theta c_1 = g$, and we can solve:
\begin{gather*}
    \varepsilon  = \frac{\theta}{\alpha} \left( \frac{c_1}{c_2} - 1 \right)
\end{gather*}
At the equilibrium values, $\frac{c_1}{c_2} > 1$ and thus $\varepsilon > 0$.


\subsection{Dynamics of reserves and assets}\label{sec:1.4}
Before the break, $s_t^p = \theta c_1 - g = 0$, $\dot{m}_t = 0$ and $\varepsilon_t = 0$, so $\dot{h}_t = r h_t$.

For foreign assets, we have:
\begin{gather*}
    \dot{a}_t = r a_t + y - c_1(1+\theta) - r m_1 = r a_t + y - c_1 (1 + \theta + \alpha r)
\end{gather*}
and the uverall position is:
\begin{gather*}
    \dot{a}_t + \dot{h}_t = r(a_t + h_t) + y - c_1(1 + \theta + \alpha r) = r(a_t + h_t) + y - \frac{g (1 + \theta + \alpha r)}{\theta}
\end{gather*}
since $c_1 = \frac{g}{\theta}$.

Evaluated at $t=0$, we have:
\begin{gather*}
    \dot{h}_0 = r h_ 0 \\
    \dot{a}_0 + \dot{h}_0 = r (a_0 + h_0) + y - \frac{g (1 + \theta + \alpha r)}{\theta}
\end{gather*}
which is exactly the current-account balance. 

If $g$ exceeds the threshold, we will have:
\begin{align*}
    CA_0 & \equiv \dot{a}_0 + \dot{h}_0 \\
    &< r(a_0 + h_0) + y -  \left[ r h_0 + \frac{\theta (r a_0 + y)}{1 + \theta + \alpha r} \right] \frac{(1 + \theta + \alpha r)}{\theta} \\
    &= r h_0 \left(1 - \frac{1 + \theta + \alpha r}{\theta}\right) \\
    &< 0
\end{align*}
This is a classic Krugman-style ``first-generation'' crisis: persistent fiscal deficits erode reserves,
and the peg becomes internally inconsistent with solvency.

Looking at the dynamics, we know tha tthe the reserves keeps growing, but the deterioration of the currrent account
indicates a deterioration of the net foreign asset position.

Once that inequality holds, the model guarantees that net foreign assets will start declining,
reserves will be gradually depleted (after private assets fall), and the peg will become unsustainable.
So a persistent current account deficit—uncovered by reserve gains—would be a clear,
quantitative early-warning of the pending crisis.

\subsection{Timing of break}\label{sec:1.5}
Recall that the intertemporal budget constraint is:
\begin{gather*}
    \frac{g}{r} = h_0 + \int_0^{\infty} e^{-rt} \left[  \theta c_t + \dot{m}_t + \varepsilon_t m_t \right] dt + e^{-rT} \left[ m_T - m_{T-} \right].
\end{gather*}

Before the break fo the peg($0 \leq t \leq T$):
\begin{gather*}
    c_t = c_1, \quad \dot{m}_t = 0, \quad \varepsilon_t = 0
\end{gather*}
and after the break, ($t > T$):
\begin{gather*}
    c_t = c_2, \quad \dot{m}_t = 0, \quad \varepsilon_t = \varepsilon > 0
\end{gather*}
Bringing these conditions into the intertemporal budget constraint, we have:
\begin{gather*}
    \frac{g}{r} = h_0 + \int_{0}^{T} e^{-rt} \theta c_1 dt + \int_{T}^{\infty} e^{-rt} \left[ \theta c_2 + \varepsilon m_2 \right] dt + e^{-rT} \left[ m_T - m_{T-} \right].
\end{gather*}
As $\theta c_1 = g$, we can write:
\begin{gather*}
    \int_{0}^{T} e^{-rt} \theta c_1 dt = g \int_{0}^{T} e^{-rt} dt = \frac{g \left( 1 - e^{-rT} \right)}{r} 
\end{gather*}
and that the second term is:
\begin{gather*}
    \int_{T}^{\infty} e^{-rt} \left[ \theta c_2 + \varepsilon m_2 \right] dt = [\theta c_2 + \varepsilon m_2] \frac{e^{-rT}}{r} 
\end{gather*}
Thus,
\begin{gather*}
    \frac{g}{r} = h_0 + \frac{g \left( 1 - e^{-rT} \right)}{r} + \frac{[\theta c_2 + \varepsilon m_2] e^{-rT}}{r} + e^{-rT} \left[ m_T - m_{T-} \right].
\end{gather*}
As $m_2 = \alpha c_2$, $\varepsilon = \frac{\theta}{\alpha} \left( \frac{c_1}{c_2} -1 \right)$, we can further simplify that:
\begin{gather*}
    \theta c_2 + \varepsilon m_2 = \theta c_2 + \frac{\theta}{\alpha} \left( \frac{c_1}{c_2} -1 \right) \alpha c_2 = \theta c_2 + \theta (c_1 - c_2) = \theta c_1 = g
\end{gather*}
Thus the budget constraint is reduced to:
\begin{align*}
    \frac{g}{r} &= h_0 + \frac{g \left( 1 - e^{-rT} \right)}{r} + \frac{g e^{-rT}}{r} + e^{-rT} \left[ m_T - m_{T-} \right] \\
                &= h_0 + \frac{g}{r} + e^{-rT} [m_T - m_{T-} ] \\
    \Rightarrow h_0 &= -e^{-rT} [m_T - m_{T-}] \\
    \Rightarrow T &= \frac{1}{r} \ln \left( \frac{m_{T-} - m_T}{h_0} \right)
\end{align*}



\section{Choice of policy regime}

\subsection{Constant money}

The shocks $\epsilon_t$ and $v_t$ have zero expected value, $\mathbb{E}_{t-1}[\varepsilon_t] = \mathbb{E}_{t-1}[v_t] = 0$.
Taking hte expectation of uncovered interest parity, aggregate supply, aggregate demand, and money demand at time $t-1$, we have:
\begin{align*}
    \mathbb{E}_{t-1}[i_{t+1}] &= \mathbb{E}_{t-1}[\mathbb{E}_t[e_{t+1}]] - \mathbb{E}_{t-1}[e_t] = \mathbb{E}_{t-1}[e_{t+1}] - \mathbb{E}_{t-1}[e_t] \\
    \mathbb{E}_{t-1}[y_t] &= \theta \left( \mathbb{E}_{t-1}[p_t] - \mathbb{E}_{t-1}[p_t] \right) = 0 \\
    \mathbb{E}_{t-1}[y_t] &= \delta (\mathbb{E}_{t-1} [e_t] - \mathbb{E}_{t-1}[p_t]) \\
    \overline{m} - \mathbb{E}_{t-1}[p_t] &= -\eta \mathbb{E}_{t-1}[i_{t+1}] + \phi \mathbb{E}_{t-1}[y_t]    
\end{align*}
Solve the four equations, we have
\begin{gather*}
    \mathbb{E}_{t-1}[e_t] = \mathbb{E}_{t-1}[p_t] = \overline{m}
\end{gather*}
Similarly, we take the expectations of the four equations at time $t$, we can obtain: $\mathbb{E}_t[e_{t+1}] = \overline{m}$.
So, we have $i_{t+1} = \overline{m} - e_t$, and we implement the total output function, we can get:
\begin{align*}
    \overline{m} - p_t &= -\eta (\overline{m} - e_t) + \phi y_t + v_t \\
                       &= -\eta \overline{m} + \eta e_t + \phi \left[\delta (e_t - p_t) + \varepsilon_t\right] + v_t \\
                       &= -\eta \overline{m} + \eta e_t + \phi \delta (e_t - p_t) + \phi \varepsilon_t + v_t \\
    \Rightarrow  (\phi \delta - 1)p_t &= (\eta + \phi \delta )e_t + \phi \varepsilon_t + v_t - (1 + \eta)\overline{m} \tag{2.1.1}
\end{align*}
We bring this back to the total supply function, with $\mathbb{E}_{t-1}[p_t] = \overline{m}$(expectation price is equal to the long-term equilibrium),
we have:
\begin{align*}
    y_t &= \theta (p_t - \overline{m}) \Rightarrow p_t = \frac{y_t}{\theta} + \overline{m} \\
    \Rightarrow y_t &= \delta \left( e_t - \overline{m} - \frac{y_t}{\theta} \right) + \varepsilon_t \\
    \Rightarrow y_t &= \frac{\theta\left[\delta (e_t - \overline{m}) + \varepsilon_t \right]}{\theta + \delta} \\
    \Rightarrow p_t &= \overline{m} + \frac{\delta (e_t - \overline{m}) + \varepsilon_t}{\theta + \delta}  \\
\end{align*}
Bring this back to equation 2.1.1, we have:
\begin{gather*}
    (\phi \delta - 1) \left[ \overline{m} + \frac{\delta (e_t - \overline{m}) + \varepsilon_t}{\theta + \delta} \right] = (\eta + \phi \delta )e_t + \phi \varepsilon_t + v_t - (1 + \eta)\overline{m} \\
    \Rightarrow (\phi \delta + \eta) \overline{m} + \frac{\phi \delta - 1}{\theta + \delta} \left[ \delta (e_t - \overline{m}) + \varepsilon_t \right] = (\eta + \phi \delta )e_t + \phi \varepsilon_t + v_t \\
    \Rightarrow \left[ \phi \delta + \eta - \frac{(\phi \delta - 1) \delta}{\theta + \delta} \right] \overline{m} + \left[ \frac{\phi \delta - 1}{\theta + \delta} - \phi \right]\varepsilon_t = \left[ \eta + \phi \delta - \frac{(\phi \delta - 1) \delta}{\theta + \delta} \right]e_t + v_t \\
    \Rightarrow \frac{\phi \delta \theta + \eta (\theta + \delta) + \delta}{\theta + \delta} \overline{m} - \frac{1 + \phi \theta}{\theta + \delta} \varepsilon_t = \frac{\phi \delta \theta + \eta (\theta + \delta) + \delta }{\theta + \delta} e_t + v_t \\
    \Rightarrow \left[ (\phi \theta + 1)\delta + \eta (\theta + \delta) \right] \overline{m} - (1 + \phi \theta) \varepsilon_t = \left[ (\phi \theta + 1)\delta + \eta (\theta + \delta) \right] e_t + v_t \\
    \Rightarrow e_t = \overline{m} - \frac{(1 + \phi \theta)\varepsilon_t + (\theta + \delta)v_t}{(\phi \theta + 1)\delta + \eta (\theta + \delta)} \tag{2.1.2}
\end{gather*}
We bring this result back to $p_t$, we have:
\begin{align*}
    p_t &= \overline{m} + \frac{\delta (e_t - \overline{m}) + \varepsilon_t}{\theta + \delta} \\
        &= \overline{m} + \frac{ \varepsilon_t - \frac{\delta (1 + \phi \theta) \varepsilon_t + \delta(\theta + \delta)v_t}{(\phi \theta + 1)\delta + \eta (\theta + \delta)} }{\theta + \delta} \\
        &= \overline{m} + \frac{\eta (\theta + \delta)\varepsilon_t - \delta (\theta + \delta)}{(\theta + \delta) \left[ (\phi \theta + 1)\delta + \eta (\theta + \delta) \right]} \\
        &= \overline{m} + \frac{\eta \varepsilon_t - \delta v_t}{(1 + \phi \theta)\delta + \eta (\theta + \delta)} \tag{2.1.3}
\end{align*}
and that
\begin{align*}
    y_t &= \frac{\theta \left[\delta (e_t - \overline{m}) + \varepsilon_t \right]}{\theta + \delta} \\
        &= \frac{\eta \theta \varepsilon_t - \delta \theta v_t}{(\phi \theta + 1)\delta + \eta (\theta + \delta)} \tag{2.1.4}
\end{align*}

Then, the variance of output should be:
\begin{gather*}
    \mathbb{E}_{t-1}[y_t^2] = \left( \frac{\eta \theta}{(\phi \theta + 1)\delta + \eta (\theta + \delta)}\right)^2 \sigma_{\varepsilon}^2 + \left( \frac{\delta \theta}{(\phi \theta + 1)\delta + \eta (\theta + \delta)} \right)^2 \sigma_v^2
\end{gather*}
denote $A = (\phi \theta + 1)\delta + \eta (\theta + \delta)$, we have:
\begin{gather*}
    \mathbb{E}_{t-1}[y_t^2] = \frac{\theta^2 \eta^2 \sigma_{\varepsilon}^2 + \theta^2 \delta^2 \sigma_v^2}{A^2}.
\end{gather*}

\subsection{Exchange rate peg}

Under a fixed exchange rate, $e_t = \overline{e} = \overline{m} $.

From the total demand function, we have:
\begin{gather*}
    y_t = \delta (e_t - p_t) + \varepsilon_t = \delta (\overline{m} - p_t) + \varepsilon_t
\end{gather*}
From the total supply function, we have:
\begin{gather*}
    y_t = \theta (p_t - \mathbb{E}_{t-1}[p_t]) = \theta (p_t - \overline{m}) 
\end{gather*}
Combining these two equations, we have:
\begin{align*}
    \theta (p_t - \overline{m}) &= \delta (\overline{m} - p_t) + \varepsilon_t \\
    \Rightarrow p_t &= \overline{m} + \frac{\varepsilon_t}{\theta + \delta}   
\end{align*}
Bring $p_t$ back to the total supply function, we have:
\begin{gather*}
    y_t = \theta (p_t - \overline{m}) = \theta \frac{\varepsilon_t}{\theta + \delta} = \frac{\theta}{\theta + \delta} \varepsilon_t 
\end{gather*}
As the exchange rate is pegged, $i_{t+1} = \mathbb{E}_t[e_{t+1}] - e_t = 0$.
From the money demand function, we could get:
\begin{align*}
    m_t - p_t &= \phi y_t + v_t = \phi \frac{\theta}{\theta + \delta} \varepsilon_t + v_t \\
    \Rightarrow m_t &= p_t + \phi \frac{\theta}{\theta + \delta} \varepsilon_t + v_t \\
    \Rightarrow m_t &= \overline{m} + \frac{\varepsilon_t}{\theta + \delta} + \phi \frac{\theta}{\theta + \delta} \varepsilon_t + v_t \\
    &= \overline{m} + \frac{(1 + \phi \theta) \varepsilon_t}{\theta + \delta} + v_t
\end{align*}
For the variance:
\begin{gather*}
    \mathbb{E}_{t-1}[y_t^2] = \left( \frac{\theta}{\theta + \delta} \right)^2 \sigma_{\varepsilon}^2
\end{gather*}

\subsection{Regime choice}

The relative volatility is given by the ratio of variance under two policies, given by:
\begin{gather*}
    \frac{\mathbb{V}[y_t]_{peg}}{\mathbb{V}[y_t]_{money}} = \frac{\left[ \eta (\theta + \delta) + \delta (1 + \phi \theta) \right]^2}{(\theta + \delta)^2} \frac{\sigma_{\varepsilon}^2}{\eta^2 \sigma_{\varepsilon}^2 + \delta^2 \sigma_v^2}
\end{gather*}
So, if $\sigma_{\varepsilon}^2 / \sigma_v^2 \to 0$, the relative volativity is determined by $\sigma_v^2 / \sigma_{\varepsilon}^2 \to  \infty $, thus the ratio is close to 0,
thus a peg policy is less volatile than a fixed money supply.

When the main source of volatility is a money demand shock ($v_t$) rather than a real economy shock ($\varepsilon_t$),
a fixed exchange rate regime stabilizes output by allowing the money supply to adjust automatically to offset money demand shocks.
Therefore, the output equation only relies on the real shock $\epsilon_t$.

In contrast, a fixed money supply regime is unable to adjust in the face of money demand shocks,
leading to higher output volatility.

\subsection{Optimal rule}
At steady state, we have:
\begin{gather*}
    i = \mathbb{E}_t \overline{e} - \overline{e} = 0 \\
    \overline{y} = \theta(\overline{p} - \overline{p}) = 0 \\
    \overline{y} = \delta(\overline{e} - \overline{p}) + 0 = 0 \implies \overline{e} = \overline{p} \\
    \overline{m} - \overline{p} = -\eta i + \phi \overline{y} + 0 \implies \overline{m} = \overline{p}
\end{gather*}
which means $\overline{e} = \overline{p} = \overline{m}$ and $i_{t+1} = \mathbb{E}_t e_{t+1} - e_t = \overline{e} - e_t = \overline{m} - e_t$. Therefore:
\begin{gather*}
    m_t - p_t = -\eta i_{t+1} + \phi y_t + v_t \\
    \overline{m} + \Phi(\overline{e} - e_t) - p_t = -\eta (\overline{e} - e_t) + \phi y_t + v_t \\
    \overline{m} - p_t = -(\eta + \Phi)(\overline{e} - e_t) + \phi y_t + v_t
\end{gather*}

Compare with section 2.1, we can see that the only difference is the $\Phi$ term, which is the Taylor rule coefficient.
Thus we only need to replace the $\eta$ with $(\eta + \Phi)$ in the previous equations, we can get:
\begin{align*}
    p_t &= \overline{m} + \frac{(\eta + \Phi ) \varepsilon_t - \delta v_t}{(1 + \phi \theta)\delta + (\eta + \Phi) (\theta + \delta)} \\
    y_t &= \frac{\theta (\eta + \Phi) \varepsilon_t - \theta \delta v_t}{\left[ (\phi \theta + 1)\delta + (\eta + \Phi) (\theta + \delta) \right]^2} \\
    e_t &= \overline{m} - \frac{(1 + \phi \theta)\varepsilon_t + (\theta + \delta)v_t}{(\phi \theta + 1)\delta + (\eta + \Phi) (\theta + \delta)}
\end{align*}

Then, the variance of output should be:
\begin{gather*}
    \mathbb{E}_{t-1} [y_t^2] = \frac{\theta^2 (\eta + \Phi)^2 \sigma_{\varepsilon}^2 + \theta^2 \delta^2 v_t^2}{\left[ (\phi \theta + 1)\delta + (\eta + \Phi) (\theta + \delta) \right]^2}
\end{gather*}

To solve the optimal value of $\Phi$, we aim to minimizes tha variance of output.
First we denote $A = \eta + \Phi$ and $D = A(\theta +\delta) + \delta (1 + \phi \theta)$,
then we know that $\frac{d A}{d \Phi} = 1$ and $\frac{d D}{d \Phi} = \theta + \delta$.
Take the FOC of $\mathbb{E}_{t-1}[y_t^2]$, we have:
\begin{gather*}
    \frac{\partial \mathbb{E}_{t-1}[y_t^2]}{\partial \Phi} = \theta^2 \frac{2A \sigma_{\varepsilon}^2 D^2 - \left[A^2 \sigma_{\varepsilon}^2 + \delta^2 \sigma_v^2\right] 2D D^{\prime}}{D^4}  = 0
\end{gather*}
Then, we have:
\begin{align*}
    & A \sigma_{\varepsilon}^2 D = \left( A^2 \sigma_{\varepsilon}^2 + \delta^2 \sigma_v^2 \right)(\theta + \delta) \\
    \Rightarrow & \left[ A^2(\theta + \delta) + A \delta (1 + \phi \theta) \right] \sigma_{\varepsilon}^2 = A^2 (\theta + \delta) \sigma_{\varepsilon}^2 + (\theta + \delta) \delta ^2 \sigma _v^2 \\
    \Rightarrow & A = \frac{(\theta + \delta) \delta \sigma_v^2}{(1 + \phi \theta) \sigma_{\varepsilon}^2}
\end{align*}
So, we have:
\begin{gather*}
    \Phi^* = \frac{(\theta + \delta) \delta \sigma_v^2}{(1 + \phi \theta) \sigma_{\varepsilon}^2} - \eta.
\end{gather*}

When $\sigma_\varepsilon^2/\sigma_v^2 \to 0$, $\Phi \to \infty$, 
this implies that the central bank should fix the exchange rate completely. 
This is consistent with our conclusion in 2.3.

When $\sigma_v^2/\sigma_\varepsilon^2 \to 0$, $\Phi \to -\eta$. 
This implies that the central bank should implement a fully floating exchange rate regime 
because at this point $\Phi + \eta \approx 0$, which is equivalent to not reacting to exchange rate fluctuations.

When money demand shocks dominate, 
it is better to fix the exchange rate to offset these shocks; 
when real economy shocks dominate, it is better to let the exchange rate float freely to absorb these shocks.


\section{Taxation of debt}

\subsection{Decentralized and centralized choice}

Under decentralized allocation, the household's budget constraint is:
\begin{gather*}
    C_2= Y_2 - (1 + r_s)D_1 = Y_2 - (1 + r + \alpha D_1)D_1 
\end{gather*}
while the household maximizes the utility function:
\begin{gather*}
    \max_{D_1} U(D_1) = \max_{D_1} \left\{ D_1 + \frac{1}{1 + \delta} \left[ Y_2 - (1 + r_s)D_1 \right] \right\} 
\end{gather*}
Take the first derivative w.r.t. $D_1$ and set it to 0, we have:
\begin{align*}
    & 1 - \frac{1 + r_s}{1+\delta} = 0 \\
    \Rightarrow & \delta = r_s = r + \alpha D_1 \\
    \Rightarrow & D_1^{decentralized} = \frac{\delta - r}{\alpha}
\end{align*}
$r_s^{decentralized} = \delta.$
For planner's decision, we have:
\begin{gather*}
    \max_{D_1} U(D_1) = \max_{D_1} \left\{ D_1 + \frac{1}{1 + \delta} \left[ Y_2 - (1 + r + \alpha D_1)D_1 \right] \right\} 
\end{gather*}
Take the first derivative w.r.t. $D_1$ and set it to 0, we have:
\begin{align*}
    & 1 - \frac{1 + r + \alpha D_1 + \alpha D_1}{1+\delta} = 0 \\
    \Rightarrow & \delta = r + 2 \alpha D_1 \\
    \Rightarrow & D_1^{centralized} = \frac{\delta - r}{2 \alpha}
\end{align*}
$r_s^{centralized} = r + \alpha D_1^{centralized} = \frac{\delta + r}{2}$.

The speaking order decentralized decision leads to overborrowing ($D_1^{decentralized} > D_1^{planner}$)
because individual households do not take into account the fact that their own borrowing behavior 
increases the cost of borrowing for the whole economy by raising interest rates. 
This is a classic externality problem.

The planner takes this externality into account and therefore chooses a lower level of borrowing, 
which reduces the overall cost of borrowing. 
Under the planner's allocation, the interest rate is lower than the decentralized allocation ($r_s^{planner} < r_s^{decentralized}$),
which reflects an internalization of the borrowing externality.

\subsection{Taxes}
Under the first tax regime, the interest rate with tax is:
\begin{gather*}
    r_s + \tau^{variable} = r_s + \gamma D_1
\end{gather*}
So, the household's utility maximization is:
\begin{gather*}
    \max_{D_1} U(D_1) = \max_{D_1} \left\{ D_1 + \frac{1}{1 + \delta} \left[ Y_2 - (1 + r_s + \tau^{variable})D_1 \right] \right\} 
\end{gather*}
where $r_s$ is still regarded as a constant.
Take the first order derivative w.r.t. $D_1$ and set it to 0, we have:
\begin{align*}
    & 1 - \frac{1 + r_s + 2 \tau^{variable} D_1}{1+\delta} = 0 \\
    \Rightarrow & \delta = r_s + \gamma D_1 
\end{align*}
If we want the household to choose $D_1^{centralized}$, we bring this back to the tax rate:
\begin{align*}
    \delta &= r + \frac{\delta - r}{2} +  \gamma \frac{\delta - r}{2 \alpha} \\
    \Rightarrow \gamma &= \alpha \\
    \Rightarrow \tau^{variable} &= \alpha D_1
\end{align*}

Under a flat tax rate, we have the household's utility maximization problem as:
\begin{gather*}
    \max_{D_1} U(D_1) = \max_{D_1} \left\{ D_1 + \frac{1}{1 + \delta} \left[ Y_2 - (1 + r_s + \tau^{flat})D_1 \right] \right\}
\end{gather*}
Take the first order derivative w.r.t. $D_1$ and set it to 0, we have:
\begin{align*}
    & 1 - \frac{1 + r_s + \tau^{flat}}{1 + \delta} = 0 \\
    \Rightarrow & \delta = r_s + \tau^{flat}
\end{align*}
As we still require $D_1 = D_1^{centralized}$, we bring it back:
\begin{align*}
    \tau^{flat} &= \delta - r_s \\
    &= \delta - (r + \alpha \frac{\delta - r}{2 \alpha}) \\
    &= \frac{\delta - r}{2}
\end{align*}
From a feasibility point of view, there are advantages and disadvantages to each of these two tax systems: 

Variable tax rate ($\tau_{variable} = \gamma D_1$ where $\gamma = \alpha$):
\begin{itemize}
    \item Advantages: adjusts with debt level, more accurate treatment of externalities
    \item Disadvantages: governments need to accurately estimate the $\alpha$ parameter, which can be challenging in practice
\end{itemize}

Fixed tax rate ($\tau^{flat} = \frac{\delta - r}{2}$):

\begin{itemize}
    \item Advantages: simple to implement, no need to monitor each household's debt level 
    \item Disadvantages: government needs to accurately estimate $\delta, r$ parameters
\end{itemize}

A flat tax rate may be easier to implement because 
it does not require constant monitoring of debt levels 
and adjusting the tax rate accordingly. 
The risk-free interest rate $r$ can be directly observed in financial markets.
However, it relies on an accurate estimate of the time preference rate $\delta$, 
which depends on the model used and can be challenging in heterogeneous household settings.

A debt-dependent tax rate may be more precise in theory, 
but is more difficult to implement because it requires the government to understand 
the precise nature of the debt-interest rate relationship
and would need frequent recalculation as debt levels change.

Overall, flat tax rates may be more feasible in practical policy settings, 
especially when governments face information constraints and administrative costs.

\end{document}

