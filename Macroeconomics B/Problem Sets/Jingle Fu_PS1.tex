\documentclass[a4paper,12pt]{article} % This defines the style of your paper

\usepackage[top = 2.5cm, bottom = 2.5cm, left = 2.5cm, right = 2.5cm]{geometry} 

% Unfortunately, LaTeX has a hard time interpreting German Umlaute. The following two lines and packages should help. If it doesn't work for you please let me know.
\usepackage[T1]{fontenc}
\usepackage[utf8]{inputenc}
\usepackage{pifont}
% \usepackage{ctex}
\usepackage{amsthm, amsmath, amssymb, mathrsfs,mathtools}

% Defining a new theorem style without italics
\newtheoremstyle{nonitalic}% name
  {\topsep}% Space above
  {\topsep}% Space below
  {\upshape}% Body font
  {}% Indent amount
  {\bfseries}% Theorem head font
  {.}% Punctuation after theorem head
  {.5em}% Space after theorem head
  {}% Theorem head spec (can be left empty, meaning ‘normal`)
  
\theoremstyle{nonitalic}
% Define new 'solution' environment
\newtheorem{innercustomsol}{Solution}
\newenvironment{solution}[1]
  {\renewcommand\theinnercustomsol{#1}\innercustomsol}
  {\endinnercustomsol}

% Custom counter for the solutions
\newcounter{solutionctr}
\renewcommand{\thesolutionctr}{(\alph{solutionctr})}

% Environment for auto-numbering with custom format
\newenvironment{autosolution}
  {\stepcounter{solutionctr}\begin{solution}{\thesolutionctr}}
  {\end{solution}}


\newtheorem{problem}{Problem}
\usepackage{color}

% The following two packages - multirow and booktabs - are needed to create nice looking tables.
\usepackage{multirow} % Multirow is for tables with multiple rows within one cell.
\usepackage{booktabs} % For even nicer tables.

% As we usually want to include some plots (.pdf files) we need a package for that.
\usepackage{graphicx} 
\usepackage{subfigure}


% The default setting of LaTeX is to indent new paragraphs. This is useful for articles. But not really nice for homework problem sets. The following command sets the indent to 0.
\usepackage{setspace}
\setlength{\parindent}{0in}
\usepackage{longtable}

% Package to place figures where you want them.
\usepackage{float}

% The fancyhdr package let's us create nice headers.
\usepackage{fancyhdr}

\usepackage{fancyvrb}

%Code environment 
\usepackage{listings} % Required for insertion of code
\usepackage{xcolor} % Required for custom colors

% Define colors for code listing
\definecolor{codegreen}{rgb}{0,0.6,0}
\definecolor{codegray}{rgb}{0.5,0.5,0.5}
\definecolor{codepurple}{rgb}{0.58,0,0.82}
\definecolor{backcolour}{rgb}{0.95,0.95,0.92}

% Code listing style named "mystyle"
\lstdefinestyle{mystyle}{
    backgroundcolor=\color{backcolour},   
    commentstyle=\color{codegreen},
    keywordstyle=\color{magenta},
    numberstyle=\tiny\color{codegray},
    stringstyle=\color{codepurple},
    basicstyle=\ttfamily\footnotesize, % Change to serif font
    breakatwhitespace=false,         
    breaklines=true,                 
    captionpos=b,                    
    keepspaces=true,                 
    numbers=left,                    
    numbersep=5pt,                  
    showspaces=false,                
    showstringspaces=false,
    showtabs=false,                  
    tabsize=2
}

\lstset{style=mystyle}

\pagestyle{fancy} % With this command we can customize the header style.

\fancyhf{} % This makes sure we do not have other information in our header or footer.

\lhead{\footnotesize EI060 Macroeconomics B}% \lhead puts text in the top left corner. \footnotesize sets our font to a smaller size.

%\rhead works just like \lhead (you can also use \chead)
\rhead{\footnotesize Jingle Fu} %<---- Fill in your lastnames.

% Similar commands work for the footer (\lfoot, \cfoot and \rfoot).
% We want to put our page number in the center.
\cfoot{\footnotesize \thepage}
\IfFileExists{upquote.sty}{\usepackage{upquote}}{}
\begin{document}


\thispagestyle{empty} % This command disables the header on the first page. 

\begin{tabular}{p{15.5cm}} % This is a simple tabular environment to align your text nicely 
{\large \bf EI060 Macroeconomics B} \\
The Graduate Institute, Spring 2025, Cedric Tille\\
\hline % \hline produces horizontal lines.
\\
\end{tabular} % Our tabular environment ends here.

\vspace*{0.3cm} % Now we want to add some vertical space in between the line and our title.

\begin{center} % Everything within the center environment is centered.
	{\Large \bf PS1 Solutions} % <---- Don't forget to put in the right number
	\vspace{2mm}
	
        % YOUR NAMES GO HERE
	{\bf Jingle Fu} % <---- Fill in your names here!
		
\end{center}  

\vspace{0.4cm}
\setstretch{1.2}

\section{Consumption Allocation}

\subsection*{Problem Setup}
The Home agent's consumption basket is given by
\[
C_t = \left(\frac{C_{T,t}}{\gamma } \right)^{\gamma}\,\left(\frac{C_{N,t}}{1 - \gamma} \right)^{\,1-\gamma},
\]
where:
\begin{itemize}
    \item $C_{T,t}$ is the quantity of the traded good (its price is normalized to 1),
    \item $C_{N,t}$ is the quantity of the domestic non-traded good (price $P_{N,t}$),
    \item $\gamma$ is the expenditure share on the traded good.
\end{itemize}

The consumer minimizes total expenditure subject to attaining a given consumption level $C_t$. The problem is
\begin{align*}
    \min_{C_{T,t},\,C_{N,t}} & P_t C_t = C_{T,t} + P_{N,t}\,C_{N,t} \\
    \text{s.t.} \quad &C_t = \left(\frac{C_{T,t}}{\gamma } \right)^{\gamma}\,\left(\frac{C_{N,t}}{1 - \gamma} \right)^{\,1-\gamma}.
\end{align*}
Define the Lagrangian function:
\[
\mathcal{L} = C_{T,t} + P_{N,t}\,C_{N,t} + \lambda \Bigl(C_t - \Bigl(\frac{C_{T,t}}{\gamma } \Bigr)^{\gamma}\,\Bigl(\frac{C_{N,t}}{1 - \gamma} \Bigr)^{\,1-\gamma}\Bigr).
\]
The FOCs with respect to $C_{T,t}$ and $C_{N,t}$ are:
\begin{align*}
    \mathcal{L}_{C_{T,t}} & = 1 - \lambda\,\gamma\,\Bigl(\frac{C_{T,t}}{\gamma } \Bigr)^{\gamma-1}\,\frac{1}{\gamma}\,\Bigl(\frac{C_{N,t}}{1 - \gamma} \Bigr)^{\,1-\gamma} = 0, \\
    \mathcal{L}_{C_{N,t}} & = P_{N,t} - \lambda\,(1-\gamma)\,\Bigl(\frac{C_{T,t}}{\gamma } \Bigr)^{\gamma}\,\frac{1}{1-\gamma}\,\Bigl(\frac{C_{N,t}}{1 - \gamma} \Bigr)^{\,-\gamma} = 0 \\
    \Rightarrow \quad \frac{1}{P_{N,t}} & = \frac{\gamma}{1-\gamma}\,\frac{C_{N,t}}{C_{T,t}}.
\end{align*}

The dual (expenditure) minimization problem yields the unit cost function (composite price index) for the consumption bundle:
\[
P_t\,C_t = \min \left\{ C_{T,t} + P_{N,t}\,C_{N,t} : \; C_t = \left(\frac{C_{T,t}}{\gamma } \right)^{\gamma}\,\left(\frac{C_{N,t}}{1 - \gamma} \right)^{\,1-\gamma} \right\}.
\]
So, we have:
\begin{align*}
    P_t C_T & = C_{T,t} + P_{N,t}\,C_{N,t} \\
    &= C_{T,t} + \frac{1-\gamma}{\gamma} C_{T,t} \\
    \Rightarrow C_{T,t} &= \gamma\,P_t\,C_t \\
    \Rightarrow C_{N,t} &= \frac{1-\gamma}{\gamma} \frac{C_{T,t}}{P_{N,t}}\\
    &= (1-\gamma)\,P_t\,C_t.
\end{align*}
As $P_t$ is the minimum expenditure required to attain the given consumption level $C_t=1$, we have:
\begin{align*}
    & \left(\frac{C_{T,t}}{\gamma } \right)^{\gamma}\,\left(\frac{C_{N,t}}{1 - \gamma} \right)^{\,1-\gamma} = 1 \\
    \Rightarrow \quad & (P_t C_t)^{\gamma}\,\left(\frac{P_t C_t}{P_N,t}\right)^{1-\gamma} = 1 \\
    \Rightarrow \quad & P_t = (P_{N,t})^{1-\gamma}. 
\end{align*}
Analogously, for the Foreign agent, we have 
\begin{align*}
    C^*_{T,t} &= \gamma\,P^*_t\,C^*_t \\
    C^*_{N,t} &= (1-\gamma)\,P^*_t\,C^*_t \\
    P^*_t &= (P^*_{N,t})^{1-\gamma}.
\end{align*}

\subsection*{Economic Intuition}
\begin{itemize}
    \item The parameter $\gamma$ reflects the expenditure share on the traded good.
    \item Since the traded good is the num\'eraire (price normalized to 1), its cost enters directly, while the cost of the non-traded good is weighted by its price $P_{N,t}$.
    \item The composite price index $P_t$ is a weighted geometric mean of the individual prices. With the traded good’s price equal to 1, we have $P_t = (P_{N,t})^{1-\gamma}$.
    \item The optimal consumption choices allocate expenditure in a way that equates the marginal rate of substitution to the ratio of prices.
\end{itemize}

\section{Market Clearing}
Under market clearing, the quantities of traded and non-traded goods produced must equal the quantities consumed. We have:
\subsection*{Non-Traded Goods Market}
For Home, market clearing in non-traded goods is:
\[
n\,C_{N,t} = A_{N,t}\,(L_{N,t})^{\,1-\alpha}.
\]
Substituting $ C_{N,t} = (1-\gamma)\frac{P_t C_t}{P_{N,t}}$ with $ P_t = (P_{N,t})^{1-\gamma} $, we obtain:
\[
\boxed{n(1-\gamma)\,(P_{N,t})^{-\gamma}\,C_t = A_{N,t}\,(L_{N,t})^{\,1-\alpha}.}
\]
For Foreign, the market clearing condition is: $(1-n)C_{N,t}^* = A_{N,t}^*(L_{N,t}^*)^{(1-\alpha)}$
\[
\boxed{(1-n)(1-\gamma)\,(P^*_{N,t})^{-\gamma}\,C^*_t = A^*_{N,t}\,(L^*_{N,t})^{\,1-\alpha}.}
\]

\subsection*{Traded Goods Market}
Global market clearing for traded goods is:
\[
n\,C_{T,t} + (1-n)\,C^*_{T,t} = A_{T,t}\,(n-L_{N,t})^{\,1-\alpha} + A^*_{T,t}\,\big((1-n)-L^*_{N,t}\big)^{\,1-\alpha}.
\]
Substituting $ C_{T,t}=\gamma\,P_t\,C_t $ with $ P_t=(P_{N,t})^{1-\gamma} $ (and similarly for Foreign), we have:
\[
\boxed{n\gamma\,(P_{N,t})^{1-\gamma}\,C_t + (1-n)\gamma\,(P^*_{N,t})^{1-\gamma}\,C^*_t = A_{T,t}\,(n-L_{N,t})^{\,1-\alpha} + A^*_{T,t}\,\big((1-n)-L^*_{N,t}\big)^{\,1-\alpha}.}
\]
\textbf{Intuition:} Non-traded goods are produced and consumed domestically, while traded goods are balanced across countries.

\section{Intertemporal Allocation}

\subsection*{Home Optimization}
The Home agent maximizes:
\[
U_t = \sum_{s=0}^{\infty} (\beta_{H,t+s})^s \ln C_{t+s},
\]
subject to the period-$ t $ budget constraint:
\[
n\,P_t\,C_t + n\,B_{t+1} = A_{T,t}\,(n-L_{N,t})^{\,1-\alpha} + P_{N,t}\,A_{N,t}\,(L_{N,t})^{\,1-\alpha} + n(1+r_t)B_t.
\]
Focusing on a single period (the infinite-horizon structure is standard), we write:
\[
\mathcal{L} = \ln C_t + \beta_{H,t+1}\ln C_{t+1} - \lambda_t \Bigl[A_{T,t}(n-L_{N,t})^{1-\alpha} + P_{N,t}A_{N,t}(L_{N,t})^{1-\alpha} + n(1+r_t)B_t - n\,P_t\,C_t - n\,B_{t+1}\Bigr].
\]
Take the FOCs:
\begin{align*}
    \frac{\partial \mathcal{L}}{\partial C_t} &= \frac{1}{C_t} - \lambda_t nP_t = 0 \quad \Rightarrow \quad \lambda_t = \frac{1}{nP_t\,C_t} \\
    \frac{\partial \mathcal{L}}{\partial B_{t+1}} &= -\lambda_t + \beta_{H,t+1}(1+r_{t+1})\,\lambda_{t+1} = 0.
\end{align*}

Substitute the expressions for \(\lambda_t\) and \(\lambda_{t+1}\):
\[
\frac{1}{nP_t\,C_t} = \beta_{H,t+1}(1+r_{t+1})\frac{1}{nP_{t+1}\,C_{t+1}}.
\]
Cancel \(n\) and rearrange:
\[
\frac{1}{C_t} = \beta_{H,t+1}(1+r_{t+1})\frac{P_t}{P_{t+1}}\frac{1}{C_{t+1}}.
\]
Set:
\[
1 + r_{t+1}^C \equiv (1+r_{t+1})\frac{P_t}{P_{t+1}},
\]
so the Euler equation becomes:
\[
\boxed{C_{t+1} = \beta_{H,t+1}(1+r_{t+1}^C)\, C_t.}
\]
From Question (1), we know that $P_t = (P_{N,t})^{(1-\gamma)}$,
so, we have:
\[
1 + r_{t+1}^C = (1+r_{t+1})\frac{P_t}{P_{t+1}} = (1+r_{t+1})\Bigl(\frac{P_{N,t}}{P_{N,t+1}}\Bigr)^{1-\gamma}.
\]
Since
\[
C_{T,t} = \gamma\,P_t\,C_t,
\]
it follows that
\[
C_{T,t+1} = \gamma\,P_{t+1}\,C_{t+1} = \gamma\,P_{t+1}\,\beta_{H,t+1}(1+r_{t+1}^C)\,C_t.
\]
As we note that
\[
\beta_{H,t+1}(1+r_{t+1}^C) = \beta_{H,t+1}(1+r_{t+1})\frac{P_t}{P_{t+1}},
\]
so simplifying, we obtain:
\[
\boxed{C_{T,t+1} = \beta_{H,t+1}(1+r_{t+1})\, C_{T,t}.}
\]

\paragraph{Remark.}  
The Foreign agent's optimization yields analogous Euler equations:
\[
C^*_{t+1} = \beta_{F,t+1}(1+r^*_{C,t+1})\, C^*_t,\quad
C^*_{T,t+1} = \beta_{F,t+1}(1+r_{t+1})\, C^*_{T,t},
\]
with
\[
1+r_{t+1}^{*C} = (1+r_{t+1})\frac{P^*_t}{P^*_{t+1}}.
\]
\subsection*{Foreign Optimization}
Similarly, for the Foreign country:
\[
\boxed{C^*_{t+1} = \beta_{F,t+1}\,(1+r^*_{C,t+1})\,C^*_t, \qquad C^*_{T,t+1} = \beta_{F,t+1}\,(1+r_{t+1})\,C^*_{T,t},}
\]
with
\[
1+r_{t+1}^{*C} = (1+r_{t+1})\,\left(\frac{P^*_{N,t}}{P^*_{N,t+1}}\right)^{1-\gamma}.
\]
\textbf{Intuition:} Households equate the marginal utility cost of consuming today versus tomorrow, with intertemporal decisions affected by relative price changes.

\section{Labor Allocation}

The production functions are given by:
\[
Y_{T,t} = A_{T,t}(n-L_{N,t})^{1-\alpha}, \quad Y_{N,t} = A_{N,t}(L_{N,t})^{1-\alpha}.
\]
Compute the marginal product of labor in each sector:
\[
\frac{\partial Y_{T,t}}{\partial (n-L_{N,t})} = (1-\alpha) A_{T,t}(n-L_{N,t})^{-\alpha},
\]
\[
\frac{\partial Y_{N,t}}{\partial L_{N,t}} = (1-\alpha) A_{N,t}(L_{N,t})^{-\alpha}.
\]
The Home agent allocates labor so that the marginal value product in the traded sector equals the marginal value product (adjusted by the non-traded price) in the non-traded sector:
\[
(1-\alpha)A_{T,t}(n-L_{N,t})^{-\alpha} = P_{N,t}\,(1-\alpha)A_{N,t}(L_{N,t})^{-\alpha}.
\]
Cancel the common factor \(1-\alpha\) and rearrange:
\[
\boxed{A_{T,t}(n-L_{N,t})^{-\alpha} = P_{N,t}\,A_{N,t}(L_{N,t})^{-\alpha}.}
\]
The analogous condition for the Foreign country is:
\[
\boxed{A^*_{T,t}\big((1-n)-L^*_{N,t}\big)^{-\alpha} = P^*_{N,t}\,A^*_{N,t}(L^*_{N,t})^{-\alpha}.}
\]

\section{Resource Constraints and the Real Exchange Rate}

\subsection*{Resource Constraints}
Recall the Home budget constraint:
\[
n\,P_t\,C_t + n\,B_{t+1} = A_{T,t}(n-L_{N,t})^{1-\alpha} + P_{N,t}A_{N,t}(L_{N,t})^{1-\alpha} + n(1+r_t)B_t.
\]
From Question 1, we have:
\[
C_{N,t} = (1-\gamma)\,\frac{P_t}{P_{N,t}}\,C_t.
\]
Since \(P_t=(P_{N,t})^{1-\gamma}\), then:
\[
C_{N,t} = (1-\gamma)(P_{N,t})^{-\gamma}C_t.
\]
Given that non-traded goods are produced solely for domestic consumption, we also have the production identity (from Question 2):
\[
n\,(1-\gamma)(P_{N,t})^{-\gamma}C_t = A_{N,t}(L_{N,t})^{1-\alpha}.
\]
Thus, the expenditure on traded goods (which uses the consumption price index) plus net asset accumulation must equal traded output plus bond returns:
\[
\boxed{n\gamma\,(P_{N,t})^{1-\gamma}C_t + n\,B_{t+1} = A_{T,t}(n-L_{N,t})^{1-\alpha} + n(1+r_t)B_t.}
\]
Similarly, for Foreign we obtain:
\[
\boxed{(1-n)\gamma\,(P^*_{N,t})^{1-\gamma}C^*_t - n\,B_{t+1} = A^*_{T,t}\big((1-n)-L^*_{N,t}\big)^{1-\alpha} - n(1+r_t)B_t.}
\]
Then, we define the real exchange rate as the ratio of Foreign to Home consumption price indices:
\[
Q_t \equiv \frac{P^*_t}{P_t}.
\]
Since
\[
P_t = (P_{N,t})^{1-\gamma} \quad \text{and} \quad P^*_t = (P^*_{N,t})^{1-\gamma},
\]
we have:
\[
\boxed{Q_t = \left(\frac{P^*_{N,t}}{P_{N,t}}\right)^{1-\gamma}.}
\]

\section{Steady State}

In steady state, consumption is constant so that $C_{t+1}=C_t$. The Euler equation for the Home agent is
\[
C_{t+1} = \beta_0(1+r_{t+1}^C)\,C_t.
\]
But by definition the real return in consumption units is
\[
1+r_{t+1}^C = (1+r_{t+1})\left(\frac{P_t}{P_{t+1}}\right)^{1-\gamma}.
\]
In steady state prices do not change ($P_t=P_{t+1}$) so that
\[
1+r_{t+1}^C = 1+r_{t+1}\quad\Rightarrow\quad C_t = \beta_0(1+r_0)C_t.
\]
Dividing by $C_t>0$ yields:
\[
1=\beta_0(1+r_0).
\]

\subsection*{Step 2. Price Normalization}
By convention we normalize the steady state price of non-tradables to one:
\[
P_{N,0}=1,\quad \text{and similarly}\quad P^*_{N,0}=1.
\]
Then the consumption price indices become
\[
P_0=(P_{N,0})^{1-\gamma}=1,\quad P^*_0=1.
\]

\subsection*{Step 3. Labor Allocation in the Home Country}
The Home intratemporal labor allocation condition is:
\[
A_{T,0}\,(n-L_{N,0})^{-\alpha} = P_{N,0}\,A_{N,0}\,(L_{N,0})^{-\alpha}.
\]
Since $P_{N,0}=1$, this simplifies to:
\[
A_{T,0}\,(n-L_{N,0})^{-\alpha} = A_{N,0}\,(L_{N,0})^{-\alpha}.
\]
Rearrange by dividing both sides by $A_{T,0}$ and by $(L_{N,0})^{-\alpha}$:
\[
\left(\frac{n-L_{N,0}}{L_{N,0}}\right)^{-\alpha} = \frac{A_{N,0}}{A_{T,0}}.
\]
Taking the reciprocal and then the $1/\alpha$-th root,
\[
\frac{n-L_{N,0}}{L_{N,0}} = \left(\frac{A_{T,0}}{A_{N,0}}\right)^{1/\alpha}.
\]
Now, using the calibration 
\[
A_{N,0} = A_{T,0}\left(\frac{1-\gamma}{\gamma}\right)^{\alpha},
\]
we have
\[
\frac{A_{T,0}}{A_{N,0}} = \left(\frac{\gamma}{1-\gamma}\right)^{\alpha}.
\]
Then,
\[
\frac{n-L_{N,0}}{L_{N,0}} = \left[\left(\frac{\gamma}{1-\gamma}\right)^{\alpha}\right]^{1/\alpha} = \frac{\gamma}{1-\gamma}.
\]
Thus,
\[
\boxed{L_{N,0} = n(1-\gamma).}
\]
By symmetry, the corresponding condition for Foreign is:
\[
A^*_{T,0}\Bigl((1-n)-L^*_{N,0}\Bigr)^{-\alpha} = P^*_{N,0}\,A^*_{N,0}\,(L^*_{N,0})^{-\alpha}.
\]
Since $P^*_{N,0}=1$, the same steps lead to:
\[
\frac{(1-n)-L^*_{N,0}}{L^*_{N,0}} = \frac{\gamma}{1-\gamma},
\]
so that
\[
\boxed{L^*_{N,0} = (1-n)(1-\gamma).}
\]
We derive the steady-state consumption from the market clearing conditions for non-traded and traded goods.
The non-traded goods clearing condition is
\[
n\,C_{N,0} = A_{N,0}\,(L_{N,0})^{1-\alpha}.
\]
From Question 1 we have
\[
C_{N,0} = (1-\gamma)\,\frac{P_0}{P_{N,0}}\,C_0.
\]
Since $P_0=1$ and $P_{N,0}=1$, it follows that
\[
C_{N,0} = (1-\gamma)\,C_0.
\]
Substitute into the clearing condition:
\[
n\,(1-\gamma)\,C_0 = A_{N,0}\,(L_{N,0})^{1-\alpha}.
\]
Recall that $L_{N,0} = n(1-\gamma)$, so
\[
n\,(1-\gamma)\,C_0 = A_{N,0}\,\bigl[n(1-\gamma)\bigr]^{1-\alpha}.
\]
Solve for $C_0$:
\[
C_0^N = A_{N,0}\, \bigl[n(1-\gamma)\bigr]^{-\alpha}.
\]
We then derive from the traded goods clearing condition:
\begin{align*}
    & n\gamma\,(P_{N,t})^{1-\gamma}\,C_t + (1-n)\gamma\,(P^*_{N,t})^{1-\gamma}\,C^*_t = A_{T,t}\,(n-L_{N,t})^{\,1-\alpha} + A^*_{T,t}\,\big((1-n)-L^*_{N,t}\big)^{\,1-\alpha} \\
    \Rightarrow & \quad n\gamma\,C_0 + \frac{\gamma}{1-\gamma} A_{N,0}^*(L_{N,0}^*)^{1-\alpha} = A_{T,0}\,(n - L_{N,0})^{1-\alpha} + A^*_{T,0}\,(1-n - L_{N,0}^*)^{1-\alpha} \\
    \Rightarrow & \quad n\gamma\,C_0 + \frac{\gamma}{1-\gamma} A_{N,0}^*(1-n)^{1-\alpha} (1-\gamma)^{1-\alpha} = A_{T,0}\,(n - n(1-\gamma))^{1-\alpha} + \\
    & A^*_{T,0}\,(1-n - (1-n)(1-\gamma))^{1-\alpha} \\
    \Rightarrow & \quad n\gamma\,C_0 + \frac{\gamma}{1-\gamma} A_{N,0}^*(1-n)^{1-\alpha} (1-\gamma)^{1-\alpha} = A_{T,0}\,(n\gamma)^{1-\alpha} + A^*_{T,0}\,\Bigl[(1-n)\gamma\Bigr]^{1-\alpha} \\
    \Rightarrow & \quad n\gamma\,C_0 + \frac{\gamma}{1-\gamma} A_{T,0} \Bigl(\frac{1-n}{n}\Bigr)^{\alpha} \Bigl(\frac{1-\gamma}{\gamma}\Bigr)^{\alpha} (1-n)^{1-\alpha} (1-\gamma)^{1-\alpha} = A_{T,0}\,(n\gamma)^{1-\alpha} + \\
    & A_{T,0} \Bigl(\frac{1-n}{n}\Bigr)^{\alpha} \Bigl[(1-n)\gamma\Bigr]^{1-\alpha} \\
    \Rightarrow & \quad n\gamma\,C_0 + A_{T,0}(1-n) n^{-\alpha} \gamma^{1-\alpha} = A_{T,0}\,(n\gamma)^{1-\alpha} + A_{T,0} (1-n) n^{-\alpha} \gamma^{1-\alpha} \\
    \Rightarrow & \quad C_0^T = A_{T,0}\,(n\gamma)^{-\alpha}.
\end{align*}
We take a weighted geometric mean with weights $\gamma$ and $1-\gamma$. That is,
\[
C_0 = (C_{0}^N)^{1-\gamma} \cdot (C_{0}^T)^{\gamma},
\]
so that
\[
C_0 = \left[A_{N,0}\, n^{-\alpha}(1-\gamma)^{-\alpha}\right]^{1-\gamma}
\left[A_{T,0}\, n^{-\alpha}\,\gamma^{-\alpha}\right]^{\gamma}.
\]
We obtain:
\[
\boxed{
C_0 = (A_{T,0})^{\gamma}(A_{N,0})^{1-\gamma}\, \left[n \gamma^{\gamma}(1-\gamma)^{1-\gamma}\right]^{-\alpha}.
}
\]
A similar derivation for Foreign (noting that the population is $1-n$) gives:
\[
\boxed{
C^*_0 = (A^*_{T,0})^{\gamma}(A^*_{N,0})^{1-\gamma}\,\left[(1-n)\gamma^{\gamma}(1-\gamma)^{1-\gamma}\right]^{-\alpha}.
}
\]
Because of the calibration (and the fact that the relative productivities satisfy
\[
\frac{A_{N,0}}{A_{T,0}} = \left(\frac{1-\gamma}{\gamma}\right)^{\alpha} \quad \text{and} \quad \frac{A^*_{N,0}}{A^*_{T,0}} = \left(\frac{1-n}{n}\right)^{\alpha}\left(\frac{1-\gamma}{\gamma}\right)^{\alpha}),
\]
we can check that indeed
\[
\frac{C_0}{C^*_0}=1.
\]
\section{Log-Linear Approximation}
We linearize the equilibrium conditions around the steady state. Denote for any variable $x_t$ its deviation from steady state by
\[
\hat{x}_t = \frac{x_t-x_0}{x_0}.
\]
We also define cross-country differences later but for now we linearize the Home equations.

\subsection*{A. Non-Traded Goods Market}
The Home non-traded goods market clearing condition is:
\[
n(1-\gamma)(P_{N,t})^{-\gamma}C_t = A_{N,t}(L_{N,t})^{1-\alpha}.
\]
Taking logarithms, we have
\[
\ln n + \ln(1-\gamma) - \gamma\ln P_{N,t} + \ln C_t = \ln A_{N,t} + (1-\alpha)\ln L_{N,t}.
\]
Linearizing around the steady state (and noting that constants vanish in the difference), we obtain:
\[
-\gamma\,\hat{P}_{N,t} + \hat{C}_t = \hat{A}_{N,t} + (1-\alpha)\,\hat{L}_{N,t}.
\]

\subsection*{B. Resource Constraint}
The Home resource constraint is:
\[
n\gamma\,(P_{N,t})^{1-\gamma}C_t + nB_{t+1} = A_{T,t}(n-L_{N,t})^{1-\alpha}+ n(1+r_t)B_t.
\]
Taking logs and linearizing (and assuming that in steady state $B_t=0$, so only percentage deviations matter), we have:
\[
(1-\gamma)\,\hat{P}_{N,t} + \hat{C}_t + \hat{B}_{t+1} = \hat{A}_{T,t} - \frac{(1-\alpha)(n-L_{N,0})}{n-L_{N,0}}\,\hat{L}_{N,t} + \frac{1}{\beta_0}\,\hat{B}_t.
\]
Since in steady state $n-L_{N,0}= n\gamma$, the coefficient on $\hat{L}_{N,t}$ becomes $(1-\alpha)$. Thus,
\[
\boxed{(1-\gamma)\,\hat{P}_{N,t} + \hat{C}_t + \hat{B}_{t+1} = \hat{A}_{T,t} - (1-\alpha)\,\hat{L}_{N,t} + \frac{1}{\beta_0}\,\hat{B}_t. \tag{7b}}
\]

\subsection*{C. Euler Equation}
The Home Euler equation is:
\[
C_{t+1} = \beta_0(1+r_{t+1}^C)\,C_t.
\]
Taking logs,
\[
\ln C_{t+1} - \ln C_t = \ln \beta_0 + \ln(1+r_{t+1}^C).
\]
Linearizing (and noting $\ln\beta_0$ is constant), we have
\[
\hat{C}_{t+1}-\hat{C}_t = \hat{\beta}_{H,t+1} + \beta_0\,\hat{r}_{C,t+1}.
\]
Recall that the real rate in consumption terms is related to the non-traded price by
\[
1+r_{t+1}^C = (1+r_{t+1})\left(\frac{P_{N,t}}{P_{N,t+1}}\right)^{1-\gamma}.
\]
Taking logs and linearizing,
\[
\ln(1+r_{t+1}^C) \approx \ln(1+r_{t+1}) + (1-\gamma)(\ln P_{N,t}-\ln P_{N,t+1}),
\]
so that
\[
\hat{r}_{C,t+1} \approx \hat{r}_{t+1} + (1-\gamma)(\hat{P}_{N,t}-\hat{P}_{N,t+1}).
\]
Thus, we can write the Euler equation as:
\[
\boxed{\hat{C}_{t+1}-\hat{C}_t = (1-\gamma)(\hat{P}_{N,t}-\hat{P}_{N,t+1}) + \hat{\beta}_{H,t+1} + \beta_0\,\hat{r}_{t+1}. \tag{7c}}
\]

\subsection*{D. Labor Allocation}
The intratemporal condition is:
\[
A_{T,t}(n-L_{N,t})^{-\alpha} = P_{N,t}\,A_{N,t}(L_{N,t})^{-\alpha}.
\]
Taking logarithms:
\[
\ln A_{T,t} - \alpha \ln (n-L_{N,t}) = \ln P_{N,t} + \ln A_{N,t} - \alpha \ln L_{N,t}.
\]
Linearize around steady state:
\[
\hat{A}_{T,t} - \alpha\,\frac{n-L_{N,t}}{n-L_{N,0}}\hat{(n-L_{N,t})} = \hat{P}_{N,t} + \hat{A}_{N,t} - \alpha\,\hat{L}_{N,t}.
\]
Using the fact that in steady state $n-L_{N,0}=n\gamma$, a careful linearization (which involves the derivative of $\ln (n-L_{N,t})$) yields:
\[
\boxed{\hat{A}_{T,t} + \frac{\alpha}{\gamma}\,\hat{L}_{N,t} = \hat{P}_{N,t} + \hat{A}_{N,t}. \tag{7d}}
\]

\subsection*{E. Real Exchange Rate}
Since
\[
P_t=(P_{N,t})^{1-\gamma}, \quad P^*_t=(P^*_{N,t})^{1-\gamma},
\]
taking logs gives
\[
\hat{P}_t=(1-\gamma)\,\hat{P}_{N,t},\quad \hat{P}^*_t=(1-\gamma)\,\hat{P}^*_{N,t}.
\]
Thus, the log deviation of the real exchange rate defined by
\[
Q_t = \frac{P^*_t}{P_t}
\]
is
\[
\hat{Q}_t = \hat{P}^*_t-\hat{P}_t = (1-\gamma)(\hat{P}^*_{N,t}-\hat{P}_{N,t}).
\]

\section{Worldwide Equilibrium}

Define world aggregates as population-weighted averages (e.g., $ \hat{C}^W_t = n\,\hat{C}_t + (1-n)\,\hat{C}^*_t $). Then, from the above log-linearized equations one can show:
\begin{itemize}
    \item $\hat{L}^W_{N,t} = 0$, i.e., aggregate non-traded labor remains fixed.
    \item The world non-traded price satisfies
    \[
    \hat{P}^W_{N,t} = \hat{A}^W_{T,t} - \hat{A}^W_{N,t}.
    \]
    \item World consumption is given by
    \[
    \hat{C}^W_t = \gamma\,\hat{A}^W_{T,t} + (1-\gamma)\,\hat{A}^W_{N,t}.
    \]
    \item The world Euler equation becomes
    \[
    \beta_0\,\hat{r}_{t+1} = -\hat{\beta}^W_{t+1} + \left(\hat{A}^W_{T,t+1}-\hat{A}^W_{T,t}\right).
    \]
\end{itemize}

\textbf{Intuition:} World aggregates respond only to symmetric shocks, with the real interest rate driven by global productivity changes and aggregate patience.

\section{Cross-Country Differences}

Define differences as (Home minus Foreign) for a variable $ x $ by $ \tilde{x}_t = \hat{x}_t - \hat{x}^*_t $. Then:

\textbf{Non-Traded Goods Market (Difference):}
\[
\frac{\gamma}{1-\gamma}\,\hat{Q}_t + \tilde{C}_t = \tilde{A}_{N,t} + (1-\alpha)\,\tilde{L}_{N,t}. \tag{9a}
\]

\textbf{Resource Constraints (Difference):}
\[
-\,\hat{Q}_t + \tilde{C}_t + \frac{\hat{B}_{t+1}}{1-n} = \tilde{A}_{T,t} - \frac{(1-\alpha)(1-\gamma)}{\gamma}\,\tilde{L}_{N,t}. \tag{9b}
\]

\textbf{Euler Equation (Difference):}
\[
\tilde{C}_{t+1}-\tilde{C}_t = (1-\gamma)\Big[(\hat{P}_{N,t}-\hat{P}^*_{N,t}) - (\hat{P}_{N,t+1}-\hat{P}^*_{N,t+1})\Big] + (\hat{\beta}_{H,t+1}-\hat{\beta}_{F,t+1}) + \beta_0\Big(\hat{r}_{C,t+1}-\hat{r}^*_{C,t+1}\Big). \tag{9c}
\]

\textbf{Labor Allocation (Difference):}
\[
\frac{\alpha}{\gamma}\,\tilde{L}_{N,t} = -\frac{1}{1-\gamma}\,\hat{Q}_t - \tilde{A}_{N,t}. \tag{9d}
\]

\textbf{Intuition:} These equations link cross-country differences in consumption, labor, and the real exchange rate to differences in productivity and intertemporal preferences.

\section{Long-Run Allocation (Period $t+1$)}

Assume that from $ t+1 $ onward the economy reaches a new steady state with no further discount factor shocks ($ \hat{\beta}_{H,t+2} = \hat{\beta}_{F,t+2} = 0 $). Taking the cross-country asset position $ \hat{B}_{t+1} $ as given, one can show:
\[
\hat{Q}_{t+1} = - (1-\gamma)\left[(\hat{A}_{T,t+1}-\hat{A}^*_{T,t+1}) - (\hat{A}_{N,t+1}-\hat{A}^*_{N,t+1})\right] - \frac{\alpha(1-\gamma)}{1-\beta_0}\frac{1}{\beta_0}\frac{\hat{B}_{t+1}}{1-n},
\]
\[
\tilde{L}_{N,t+1} = \frac{\gamma}{1-\beta_0}\frac{1}{\beta_0}\frac{\hat{B}_{t+1}}{1-n},
\]
\[
\tilde{C}_{t+1} = \gamma\big(\hat{A}_{T,t+1}-\hat{A}^*_{T,t+1}\big) + (1-\gamma)\big(\hat{A}_{N,t+1}-\hat{A}^*_{N,t+1}\big) + \frac{\gamma}{1-\beta_0}\frac{1}{\beta_0}\frac{\hat{B}_{t+1}}{1-n}.
\]
\textbf{Interpretation:}
\begin{itemize}
    \item A positive $ \hat{B}_{t+1} $ (Home wealthier) implies higher relative consumption and a lower $ Q_{t+1} $ (Home’s goods become relatively more expensive).
    \item Permanent productivity differences affect steady state consumption and prices directly.
\end{itemize}

\section{Short-Run Allocation (Period $t$)}

Assume initially $ \hat{B}_t=0 $. Solving the system (starting with the labor and non-traded market conditions, then using the resource constraint and Euler equations) yields:
\[
\frac{\hat{B}_{t+1}}{1-n} = \frac{\beta_0}{\gamma + \alpha(1-\gamma)}\Big[(\hat{\beta}_{H,t+1}-\hat{\beta}_{F,t+1}) - (\hat{A}_{T,t+1}-\hat{A}^*_{T,t+1}) + (\hat{A}_{T,t}-\hat{A}^*_{T,t})\Big], \tag{11a}
\]
\[
\tilde{C}_t = \gamma(\hat{A}_{T,t}-\hat{A}^*_{T,t}) + (1-\gamma)(\hat{A}_{N,t}-\hat{A}^*_{N,t}) - \frac{\gamma\,\beta_0}{\gamma + \alpha(1-\gamma)}\Big[(\hat{\beta}_{H,t+1}-\hat{\beta}_{F,t+1}) - (\hat{A}_{T,t+1}-\hat{A}^*_{T,t+1}) + (\hat{A}_{T,t}-\hat{A}^*_{T,t})\Big], \tag{11b}
\]
\[
\hat{Q}_t = - (1-\gamma)(\hat{A}_{T,t}-\hat{A}^*_{T,t}) + (1-\gamma)(\hat{A}_{N,t}-\hat{A}^*_{N,t}) + \frac{(1-\gamma)\,\alpha\,\beta_0}{\gamma + \alpha(1-\gamma)}\Big[(\hat{\beta}_{H,t+1}-\hat{\beta}_{F,t+1}) - (\hat{A}_{T,t+1}-\hat{A}^*_{T,t+1}) + (\hat{A}_{T,t}-\hat{A}^*_{T,t})\Big], \tag{11c}
\]
\[
\tilde{L}_{N,t} = -\frac{\gamma\,\beta_0}{\gamma + \alpha(1-\gamma)}\Big[(\hat{\beta}_{H,t+1}-\hat{\beta}_{F,t+1}) - (\hat{A}_{T,t+1}-\hat{A}^*_{T,t+1}) + (\hat{A}_{T,t}-\hat{A}^*_{T,t})\Big]. \tag{11d}
\]

\textbf{Interpretation:}
\begin{itemize}
    \item A temporary increase in Home patience (i.e. $ \hat{\beta}_{H,t+1}-\hat{\beta}_{F,t+1} > 0 $) leads to $ \hat{B}_{t+1}>0 $ (Home runs a current account surplus), lower relative consumption, and a real exchange rate movement consistent with a trade surplus.
    \item Temporary traded or non-traded productivity shocks affect the current account and labor allocation differently.
    \item For permanent shocks ($ \hat{A}_{T,t}=\hat{A}_{T,t+1} $), intertemporal balance is restored with $ \hat{B}_{t+1}=0 $ and immediate adjustment to the new steady state.
\end{itemize}

\section{Summary of Key Economic Insights}
\begin{itemize}
    \item \textbf{Consumption and Prices:} The structure of the consumption basket implies that a rise in the non‐traded good price $ P_{N,t} $ increases the overall consumption price $ P_t $ and shifts the consumption mix.
    \item \textbf{Market Clearing:} Non-traded goods are produced and consumed domestically, whereas traded goods are allocated internationally, linking domestic production choices to the real exchange rate.
    \item \textbf{Intertemporal Choices:} The Euler equations show that higher patience or higher real returns lead to deferred consumption. Cross-country differences in patience drive current account imbalances.
    \item \textbf{Labor Allocation:} Labor is reallocated across sectors until the marginal value products are equalized; productivity shifts affect both output composition and relative prices.
    \item \textbf{Steady State and Log-Linearization:} In a symmetric steady state, relative prices and allocations are balanced. Log-linearization permits analysis of small shocks and their propagation.
    \item \textbf{Short-Run vs. Long-Run Dynamics:} Temporary shocks generate current account imbalances and short-run reallocation, while permanent shocks adjust consumption and prices directly with no asset accumulation.
    \item \textbf{Wealth Effects:} A positive net asset position (Home wealthier) implies higher steady-state consumption and a relatively stronger (appreciated) currency.
\end{itemize}

\end{document}

