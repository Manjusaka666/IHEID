\documentclass[a4paper,12pt]{article} % This defines the style of your paper

\usepackage[top = 2.5cm, bottom = 2.5cm, left = 1.5cm, right = 1.5cm]{geometry} 

% Unfortunately, LaTeX has a hard time interpreting German Umlaute. The following two lines and packages should help. If it doesn't work for you please let me know.
\usepackage[T1]{fontenc}
\usepackage[utf8]{inputenc}
\usepackage{pifont}
% \usepackage{ctex}
\usepackage{amsthm, amsmath, amssymb, mathrsfs,mathtools}

% Defining a new theorem style without italics
\newtheoremstyle{nonitalic}% name
  {\topsep}% Space above
  {\topsep}% Space below
  {\upshape}% Body font
  {}% Indent amount
  {\bfseries}% Theorem head font
  {.}% Punctuation after theorem head
  {.5em}% Space after theorem head
  {}% Theorem head spec (can be left empty, meaning ‘normal`)
  
\theoremstyle{nonitalic}
% Define new 'solution' environment
\newtheorem{innercustomsol}{Solution}
\newenvironment{solution}[1]
  {\renewcommand\theinnercustomsol{#1}\innercustomsol}
  {\endinnercustomsol}

% Custom counter for the solutions
\newcounter{solutionctr}
\renewcommand{\thesolutionctr}{(\alph{solutionctr})}

% Environment for auto-numbering with custom format
\newenvironment{autosolution}
  {\stepcounter{solutionctr}\begin{solution}{\thesolutionctr}}
  {\end{solution}}


\newtheorem{problem}{Problem}
\usepackage{color}

% The following two packages - multirow and booktabs - are needed to create nice looking tables.
\usepackage{multirow} % Multirow is for tables with multiple rows within one cell.
\usepackage{booktabs} % For even nicer tables.

% As we usually want to include some plots (.pdf files) we need a package for that.
\usepackage{graphicx} 
\usepackage{subfigure}


% The default setting of LaTeX is to indent new paragraphs. This is useful for articles. But not really nice for homework problem sets. The following command sets the indent to 0.
\usepackage{setspace}
\setlength{\parindent}{0in}
\usepackage{longtable}

% Package to place figures where you want them.
\usepackage{float}

% The fancyhdr package let's us create nice headers.
\usepackage{fancyhdr}

\usepackage{fancyvrb}

%Code environment 
\usepackage{listings} % Required for insertion of code
\usepackage{xcolor} % Required for custom colors

% Define colors for code listing
\definecolor{codegreen}{rgb}{0,0.6,0}
\definecolor{codegray}{rgb}{0.5,0.5,0.5}
\definecolor{codepurple}{rgb}{0.58,0,0.82}
\definecolor{backcolour}{rgb}{0.95,0.95,0.92}

% Code listing style named "mystyle"
\lstdefinestyle{mystyle}{
    backgroundcolor=\color{backcolour},   
    commentstyle=\color{codegreen},
    keywordstyle=\color{magenta},
    numberstyle=\tiny\color{codegray},
    stringstyle=\color{codepurple},
    basicstyle=\ttfamily\footnotesize, % Change to serif font
    breakatwhitespace=false,         
    breaklines=true,                 
    captionpos=b,                    
    keepspaces=true,                 
    numbers=left,                    
    numbersep=5pt,                  
    showspaces=false,                
    showstringspaces=false,
    showtabs=false,                  
    tabsize=2
}

\lstset{style=mystyle}

\pagestyle{fancy} % With this command we can customize the header style.

\fancyhf{} % This makes sure we do not have other information in our header or footer.

\lhead{\footnotesize EI060 Macroeconomics B}% \lhead puts text in the top left corner. \footnotesize sets our font to a smaller size.

%\rhead works just like \lhead (you can also use \chead)
\rhead{\footnotesize Jingle Fu} %<---- Fill in your lastnames.

% Similar commands work for the footer (\lfoot, \cfoot and \rfoot).
% We want to put our page number in the center.
\cfoot{\footnotesize \thepage}
\IfFileExists{upquote.sty}{\usepackage{upquote}}{}
\begin{document}


\thispagestyle{empty} % This command disables the header on the first page. 

\begin{tabular}{p{15.5cm}} % This is a simple tabular environment to align your text nicely 
{\large \bf EI060 Macroeconomics B} \\
The Graduate Institute, Spring 2025, Cedric Tille\\
\hline % \hline produces horizontal lines.
\\
\end{tabular} % Our tabular environment ends here.

\vspace*{0.3cm} % Now we want to add some vertical space in between the line and our title.

\begin{center} % Everything within the center environment is centered.
	{\Large \bf PS1 Solutions} % <---- Don't forget to put in the right number
	\vspace{2mm}
	
        % YOUR NAMES GO HERE
	{\bf Jingle Fu} % <---- Fill in your names here!
		
\end{center}  

\vspace{0.4cm}
\setstretch{1.1}

\section{Consumption Allocation}

\subsection*{Problem Setup}
The Home agent's consumption basket is given by
\[
C_t = \left(\frac{C_{T,t}}{\gamma } \right)^{\gamma} \left(\frac{C_{N,t}}{1 - \gamma} \right)^{ 1-\gamma},
\]
where:
\begin{itemize}
    \item $C_{T,t}$ is the quantity of the traded good (its price is normalized to 1),
    \item $C_{N,t}$ is the quantity of the domestic non-traded good (price $P_{N,t}$),
    \item $\gamma$ is the expenditure share on the traded good.
\end{itemize}

The consumer minimizes total expenditure subject to attaining a given consumption level $C_t$. The problem is
\begin{align*}
    \min_{C_{T,t}, C_{N,t}} & P_t C_t = C_{T,t} + P_{N,t} C_{N,t} \\
    \text{s.t.} \quad &C_t = \left(\frac{C_{T,t}}{\gamma } \right)^{\gamma} \left(\frac{C_{N,t}}{1 - \gamma} \right)^{ 1-\gamma}.
\end{align*}
Define the Lagrangian function:
\[
\mathcal{L} = C_{T,t} + P_{N,t} C_{N,t} + \lambda \Bigl(C_t - \Bigl(\frac{C_{T,t}}{\gamma } \Bigr)^{\gamma} \Bigl(\frac{C_{N,t}}{1 - \gamma} \Bigr)^{ 1-\gamma}\Bigr).
\]
The FOCs with respect to $C_{T,t}$ and $C_{N,t}$ are:
\begin{align*}
    \mathcal{L}_{C_{T,t}} & = 1 - \lambda \gamma \Bigl(\frac{C_{T,t}}{\gamma } \Bigr)^{\gamma-1} \frac{1}{\gamma} \Bigl(\frac{C_{N,t}}{1 - \gamma} \Bigr)^{ 1-\gamma} = 0, \\
    \mathcal{L}_{C_{N,t}} & = P_{N,t} - \lambda (1-\gamma) \Bigl(\frac{C_{T,t}}{\gamma } \Bigr)^{\gamma} \frac{1}{1-\gamma} \Bigl(\frac{C_{N,t}}{1 - \gamma} \Bigr)^{ -\gamma} = 0 \\
    \Rightarrow \quad \frac{1}{P_{N,t}} & = \frac{\gamma}{1-\gamma} \frac{C_{N,t}}{C_{T,t}}.
\end{align*}

The dual (expenditure) minimization problem yields the unit cost function (composite price index) for the consumption bundle:
\[
P_t C_t = \min \left\{ C_{T,t} + P_{N,t} C_{N,t} : \; C_t = \left(\frac{C_{T,t}}{\gamma } \right)^{\gamma} \left(\frac{C_{N,t}}{1 - \gamma} \right)^{ 1-\gamma} \right\}.
\]
So, we have:
\begin{align*}
    P_t C_t & = C_{T,t} + P_{N,t} C_{N,t} \\
    &= C_{T,t} + \frac{1-\gamma}{\gamma} C_{T,t} \\
    \Rightarrow C_{T,t} &= \gamma P_t C_t \tag{1a}\\
    \Rightarrow C_{N,t} &= \frac{1-\gamma}{\gamma} \frac{C_{T,t}}{P_{N,t}}\\
    &= (1-\gamma) P_t C_t. \tag{1b}
\end{align*}
\begin{align*}
    & \left(\frac{C_{T,t}}{\gamma } \right)^{\gamma} \left(\frac{C_{N,t}}{1 - \gamma} \right)^{ 1-\gamma} = C_t \\
    \Rightarrow \quad & (P_t C_t)^{\gamma} \left(\frac{P_t C_t}{P_N,t}\right)^{1-\gamma} = C_t \\
    \Rightarrow \quad & P_t = (P_{N,t})^{1-\gamma}. \tag{1c} 
\end{align*}
Analogously, for the Foreign agent, we have 
\begin{align*}
    C^*_{T,t} &= \gamma P^*_t C^*_t \tag{1d}\\
    C^*_{N,t} &= (1-\gamma) P^*_t C^*_t \tag{1e}\\
    P^*_t &= (P^*_{N,t})^{1-\gamma}. \tag{1f}
\end{align*}

% \subsection*{Economic Intuition}
% \begin{itemize}
%     \item The parameter $\gamma$ reflects the expenditure share on the traded good.
%     \item Since the traded good is the num\'eraire (price normalized to 1), its cost enters directly, while the cost of the non-traded good is weighted by its price $P_{N,t}$.
%     \item The composite price index $P_t$ is a weighted geometric mean of the individual prices. With the traded good's price equal to 1, we have $P_t = (P_{N,t})^{1-\gamma}$.
%     \item The optimal consumption choices allocate expenditure in a way that equates the marginal rate of substitution to the ratio of prices.
% \end{itemize}

\section{Market Clearing}
Under market clearing, the quantities of traded and non-traded goods produced must equal the quantities consumed. We have:
\subsection*{Non-Traded Goods Market}
For Home, market clearing in non-traded goods is:
\[
n C_{N,t} = A_{N,t} (L_{N,t})^{ 1-\alpha}.
\]
Substituting $ C_{N,t} = (1-\gamma)\frac{P_t C_t}{P_{N,t}}$ with $ P_t = (P_{N,t})^{1-\gamma} $, we obtain:
\[
n(1-\gamma) (P_{N,t})^{-\gamma} C_t = A_{N,t} (L_{N,t})^{ 1-\alpha}. \tag{2a}
\]
For Foreign, the market clearing condition is: $(1-n)C_{N,t}^* = A_{N,t}^*(L_{N,t}^*)^{1-\alpha}$. Following the same method, we have:
\[
(1-n)(1-\gamma) (P^*_{N,t})^{-\gamma} C^*_t = A^*_{N,t} (L^*_{N,t})^{ 1-\alpha}. \tag{2b}
\]

\subsection*{Traded Goods Market}
Global market clearing for traded goods is:
\[
n C_{T,t} + (1-n) C^*_{T,t} = A_{T,t} (n-L_{N,t})^{ 1-\alpha} + A^*_{T,t} \big((1-n)-L^*_{N,t}\big)^{ 1-\alpha}.
\]
Substituting $ C_{T,t}=\gamma P_t C_t $ with $ P_t=(P_{N,t})^{1-\gamma} $ (and similarly for Foreign), we have:
\[
n\gamma (P_{N,t})^{1-\gamma} C_t + (1-n)\gamma (P^*_{N,t})^{1-\gamma} C^*_t = A_{T,t} (n-L_{N,t})^{ 1-\alpha} + A^*_{T,t} \big((1-n)-L^*_{N,t}\big)^{ 1-\alpha}. \tag{2c}
\]
% \textbf{Intuition:} Non-traded goods are produced and consumed domestically, while traded goods are balanced across countries.

\section{Intertemporal Allocation}

\subsection*{Home Optimization}
The Home agent maximizes:
\[
U_t = \sum_{s=0}^{\infty} (\beta_{H,t+s})^s \ln C_{t+s},
\]
subject to the period-$ t $ budget constraint:
\[
n P_t C_t + n B_{t+1} = A_{T,t} (n-L_{N,t})^{ 1-\alpha} + P_{N,t} A_{N,t} (L_{N,t})^{ 1-\alpha} + n(1+r_t)B_t.
\]
Focusing on a single period (the infinite-horizon structure is standard), we write:
\[
\mathcal{L}_t = \ln C_t + \beta_{H,t+1}\ln C_{t+1} - \lambda_t \Bigl[A_{T,t}(n-L_{N,t})^{1-\alpha} + P_{N,t}A_{N,t}(L_{N,t})^{1-\alpha} + n(1+r_t)B_t - n P_t C_t - n B_{t+1}\Bigr].
\]
Take the FOCs:
\begin{align*}
    \frac{\partial \mathcal{L}_t}{\partial C_t} &= \frac{1}{C_t} - \lambda_t nP_t = 0 \quad \Rightarrow \quad \lambda_t = \frac{1}{nP_t C_t} \\
    \frac{\partial \mathcal{L}_t}{\partial B_{t+1}} &= -\lambda_t + \beta_{H,t+1}(1+r_{t+1}) \lambda_{t+1} = 0.
\end{align*}

Substitute the expressions for \(\lambda_t\) and \(\lambda_{t+1}\):
\[
\frac{1}{nP_t C_t} = \beta_{H,t+1}(1+r_{t+1})\frac{1}{nP_{t+1} C_{t+1}}.
\]
Cancel \(n\) and rearrange:
\[
\frac{1}{C_t} = \beta_{H,t+1}(1+r_{t+1})\frac{P_t}{P_{t+1}}\frac{1}{C_{t+1}}.
\]
Set:
\[
1 + r_{t+1}^C \equiv (1+r_{t+1})\frac{P_t}{P_{t+1}},
\]
so the Euler equation becomes:
\[
C_{t+1} = \beta_{H,t+1}(1+r_{t+1}^C)  C_t. \tag{3a}
\]
From Question (1), we know that $P_t = (P_{N,t})^{(1-\gamma)}$,
so, we have:
\[
1 + r_{t+1}^C = (1+r_{t+1})\frac{P_t}{P_{t+1}} = (1+r_{t+1})\Bigl(\frac{P_{N,t}}{P_{N,t+1}}\Bigr)^{1-\gamma}. \tag{3b}
\]
Since
\[
C_{T,t} = \gamma P_t C_t,
\]
it follows that
\[
C_{T,t+1} = \gamma P_{t+1} C_{t+1} = \gamma P_{t+1} \beta_{H,t+1}(1+r_{t+1}^C) C_t.
\]
As we note that
\[
\beta_{H,t+1}(1+r_{t+1}^C) = \beta_{H,t+1}(1+r_{t+1})\frac{P_t}{P_{t+1}},
\]
so simplifying, we obtain:
\[
C_{T,t+1} = \beta_{H,t+1}(1+r_{t+1})  C_{T,t}. \tag{3c}
\]

\paragraph{Remark.}  
The Foreign agent's optimization yields analogous Euler equations:
\[
C_{t+1}^* = \beta_{F,t+1}(1+r_{C,t+1}^*)  C_t^*,\quad
C_{T,t+1}^* = \beta_{F,t+1}(1+r_{t+1}) C_{T,t}^*,
\]
with
\[
1+r_{t+1}^{*C} = (1+r_{t+1})\frac{P_t^*}{P_{t+1}^*}.
\]
\subsection*{Foreign Optimization}
Similarly, for the Foreign country:
\begin{align*}
    C_{t+1}^* &= \beta_{F,t+1} (1+r_{C,t+1}^*) C_t^* \tag{3d} \\
    C^*_{T,t+1} &= \beta_{F,t+1} (1+r_{t+1}) C^*_{T,t} \tag{3e}
\end{align*}
with
\[
1+r_{t+1}^{*C} = (1+r_{t+1}) \left(\frac{P^*_{N,t}}{P^*_{N,t+1}}\right)^{1-\gamma}. \tag{3f}
\]
% \textbf{Intuition:} Households equate the marginal utility cost of consuming today versus tomorrow, with intertemporal decisions affected by relative price changes.

\section{Labor Allocation}

The production functions are given by:
\[
Y_{T,t} = A_{T,t}(n-L_{N,t})^{1-\alpha}, \quad Y_{N,t} = A_{N,t}(L_{N,t})^{1-\alpha}.
\]
Compute the marginal product of labor in each sector:
\[
\frac{\partial Y_{T,t}}{\partial (n-L_{N,t})} = (1-\alpha) A_{T,t}(n-L_{N,t})^{-\alpha},
\]
\[
\frac{\partial Y_{N,t}}{\partial L_{N,t}} = (1-\alpha) A_{N,t}(L_{N,t})^{-\alpha}.
\]
The Home agent allocates labor so that the marginal value product in the traded sector equals the marginal value product (adjusted by the non-traded price) in the non-traded sector:
\[
(1-\alpha)A_{T,t}(n-L_{N,t})^{-\alpha} = P_{N,t} (1-\alpha)A_{N,t}(L_{N,t})^{-\alpha}.
\]
Cancel the common factor \(1-\alpha\) and rearrange:
\[
A_{T,t}(n-L_{N,t})^{-\alpha} = P_{N,t} A_{N,t}(L_{N,t})^{-\alpha}. \tag{4a}
\]
As $Y_{T,t}^* = A_{T,t}^* \Bigl(1-n-L_{N,t}^* \Bigr)^{-\alpha}$, the analogous condition for the Foreign country is:
\[
A_{T,t}^*\big((1-n)-L_{N,t}^*\big)^{-\alpha} = P_{N,t}^* A_{N,t}^*(L_{N,t}^*)^{-\alpha}. \tag{4b}
\]

\section{Resource Constraints and the Real Exchange Rate}

\subsection*{Resource Constraints}
Recall the Home budget constraint:
\[
n P_t C_t + n B_{t+1} = A_{T,t}(n-L_{N,t})^{1-\alpha} + P_{N,t}A_{N,t}(L_{N,t})^{1-\alpha} + n(1+r_t)B_t.
\]
Given (2a),
\[
P_{N,t} A_{N,t} \bigl(L_{N,t}\bigr)^{1-\alpha} = n (1-\gamma) \bigl(P_{N,t}\bigr)^{1-\gamma} C_t \tag{5.1}
\]
By (1a) and (5.1), we have:
\[
n P_t C_t - P_{N,t} A_{N,t} \bigl(L_{N,t}\bigr)^{1-\alpha} = n \bigl(P_{N,t}\bigr)^{1-\gamma} C_t \bigl(1 - (1-\gamma)\bigr) = n \gamma (P_{N,t})^{1-\gamma} C_t \tag{5.2}
\]
Bring (5.2) back to the budget constraint, we have:
\[
n \gamma \bigl(P_{N,t}\bigr)^{1-\gamma} C_t + n B_{t+1} = A_{T,t}(n-L_{N,t})^{1-\alpha} + n(1+r_t)B_t. \tag{5a} 
\]
Similarly, for Foreign we obtain:
\[
(1-n)\gamma (P^*_{N,t})^{1-\gamma}C^*_t - n B_{t+1} = A^*_{T,t}\big((1-n)-L^*_{N,t}\big)^{1-\alpha} - n(1+r_t)B_t. \tag{5b}
\]
Then, we define the real exchange rate as the ratio of Foreign to Home consumption price indices:
\[
Q_t \equiv \frac{P^*_t}{P_t}.
\]
Since
\[
P_t = (P_{N,t})^{1-\gamma} \quad \text{and} \quad P^*_t = (P^*_{N,t})^{1-\gamma},
\]
we have:
\[
Q_t = \left(\frac{P^*_{N,t}}{P_{N,t}}\right)^{1-\gamma}. \tag{5c}
\]

\section{Steady State}

In steady state, consumption is constant so that $C_{t+1}=C_t$. The Euler equation for the Home agent is
\[
C_{t+1} = \beta_{H,t+1}(1+r_{t+1}^C) C_t. \tag{6.1}
\]
But by definition the real return in consumption units is
\[
1+r_{t+1}^C = (1+r_{t+1})\left(\frac{P_t}{P_{t+1}}\right)^{1-\gamma}. \tag{6.2}
\]
In steady state prices do not change ($P_t=P_{t+1}$) so that
\[
1+r_{t+1}^C = 1+r_{t+1}\quad\Rightarrow\quad C_t = \beta_t(1+r_t)C_t.
\]
Dividing by $C_t>0$ and take $t=0$, yields:
\[
1=\beta_0(1+r_0). \tag{6a}
\]

\subsection*{Step 2. Price Normalization}
By convention we normalize the steady state price of non-tradables to one:
\[
P_{N,0}=1,\quad \text{and similarly}\quad P_{N,0}^*=1.
\]
Then the consumption price indices become
\[
P_0=(P_{N,0})^{1-\gamma}=1,\quad _0P^*=1. \tag{6b}
\]

\subsection*{Step 3. Labor Allocation in the Home Country}
The Home intratemporal labor allocation condition is:
\[
A_{T,0} (n-L_{N,0})^{-\alpha} = P_{N,0} A_{N,0} (L_{N,0})^{-\alpha}.
\]
Since $P_{N,0}=1$, this simplifies to:
\[
A_{T,0} (n-L_{N,0})^{-\alpha} = A_{N,0} (L_{N,0})^{-\alpha}.
\]
Rearrange by dividing both sides by $A_{T,0}$ and by $(L_{N,0})^{-\alpha}$:
\[
\left(\frac{n-L_{N,0}}{L_{N,0}}\right)^{-\alpha} = \frac{A_{N,0}}{A_{T,0}}.
\]
Taking the reciprocal and then the $1/\alpha$-th root,
\[
\frac{n-L_{N,0}}{L_{N,0}} = \left(\frac{A_{T,0}}{A_{N,0}}\right)^{1/\alpha}.
\]
Now, using the calibration 
\[
A_{N,0} = A_{T,0}\left(\frac{1-\gamma}{\gamma}\right)^{\alpha},
\]
we have
\[
\frac{A_{T,0}}{A_{N,0}} = \left(\frac{\gamma}{1-\gamma}\right)^{\alpha}.
\]
Then,
\[
\frac{n-L_{N,0}}{L_{N,0}} = \left[\left(\frac{\gamma}{1-\gamma}\right)^{\alpha}\right]^{1/\alpha} = \frac{\gamma}{1-\gamma}.
\]
Thus,
\[
L_{N,0} = n(1-\gamma). \tag{6c}
\]
By symmetry, the corresponding condition for Foreign is:
\[
A^*_{T,0}\Bigl((1-n)-L^*_{N,0}\Bigr)^{-\alpha} = P^*_{N,0} A^*_{N,0} (L^*_{N,0})^{-\alpha}.
\]
Since $P^*_{N,0}=1$, the same steps lead to:
\[
\frac{(1-n)-L^*_{N,0}}{L^*_{N,0}} = \frac{\gamma}{1-\gamma},
\]
so that
\[
L^*_{N,0} = (1-n)(1-\gamma). \tag{6d}
\]
We derive the steady-state consumption from the market clearing conditions for non-traded and traded goods.
The non-traded goods clearing condition is
\[
n C_{N,0} = A_{N,0} (L_{N,0})^{1-\alpha}.
\]
From Question 1 we have
\[
C_{N,0} = (1-\gamma) \frac{P_0}{P_{N,0}} C_0.
\]
Since $P_0=1$ and $P_{N,0}=1$, it follows that
\[
C_{N,0} = (1-\gamma) C_0.
\]
Substitute into the clearing condition:
\[
n (1-\gamma) C_0 = A_{N,0} (L_{N,0})^{1-\alpha}.
\]
Recall that $L_{N,0} = n(1-\gamma)$, so
\[
n (1-\gamma) C_0 = A_{N,0} \bigl[n(1-\gamma)\bigr]^{1-\alpha}.
\]
Solve for $C_0$(as it is an expression of $A_{N,0}$, we denote by $C_0^N$):
\[
C_0^N = A_{N,0}  \bigl[n(1-\gamma)\bigr]^{-\alpha}.
\]
We then derive from the traded goods clearing condition:
\begin{align*}
    & n\gamma (P_{N,t})^{1-\gamma} C_t + (1-n)\gamma (P^*_{N,t})^{1-\gamma} C^*_t = A_{T,t} (n-L_{N,t})^{ 1-\alpha} + A^*_{T,t} \big((1-n)-L^*_{N,t}\big)^{ 1-\alpha} \\
    \Rightarrow & \quad n\gamma C_0 + \frac{\gamma}{1-\gamma} A_{N,0}^*(L_{N,0}^*)^{1-\alpha} = A_{T,0} (n - L_{N,0})^{1-\alpha} + A^*_{T,0} (1-n - L_{N,0}^*)^{1-\alpha} \\
    \Rightarrow & \quad n\gamma C_0 + \frac{\gamma}{1-\gamma} A_{N,0}^*(1-n)^{1-\alpha} (1-\gamma)^{1-\alpha} = A_{T,0} (n - n(1-\gamma))^{1-\alpha} + \\
    & A^*_{T,0} (1-n - (1-n)(1-\gamma))^{1-\alpha} \\
    \Rightarrow & \quad n\gamma C_0 + \frac{\gamma}{1-\gamma} A_{N,0}^*(1-n)^{1-\alpha} (1-\gamma)^{1-\alpha} = A_{T,0} (n\gamma)^{1-\alpha} + A^*_{T,0} \Bigl[(1-n)\gamma\Bigr]^{1-\alpha} \\
    \Rightarrow & \quad n\gamma C_0 + \frac{\gamma}{1-\gamma} A_{T,0} \Bigl(\frac{1-n}{n}\Bigr)^{\alpha} \Bigl(\frac{1-\gamma}{\gamma}\Bigr)^{\alpha} (1-n)^{1-\alpha} (1-\gamma)^{1-\alpha} = A_{T,0} (n\gamma)^{1-\alpha} + \\
    & A_{T,0} \Bigl(\frac{1-n}{n}\Bigr)^{\alpha} \Bigl[(1-n)\gamma\Bigr]^{1-\alpha} \\
    \Rightarrow & \quad n\gamma C_0 + A_{T,0}(1-n) n^{-\alpha} \gamma^{1-\alpha} = A_{T,0} (n\gamma)^{1-\alpha} + A_{T,0} (1-n) n^{-\alpha} \gamma^{1-\alpha} \\
    \Rightarrow & \quad C_0^T = A_{T,0} (n\gamma)^{-\alpha}.
\end{align*}
We take a weighted geometric mean with weights $\gamma$ and $1-\gamma$. That is,
\[
C_0 = (C_{0}^N)^{1-\gamma} \cdot (C_{0}^T)^{\gamma},
\]
so that
\[
C_0 = \left[A_{N,0}  n^{-\alpha}(1-\gamma)^{-\alpha}\right]^{1-\gamma}
\left[A_{T,0}  n^{-\alpha} \gamma^{-\alpha}\right]^{\gamma}.
\]
We obtain:
\[
C_0 = (A_{T,0})^{\gamma}(A_{N,0})^{1-\gamma}  \left[n \gamma^{\gamma}(1-\gamma)^{1-\gamma}\right]^{-\alpha}. \tag{6e}
\]
A similar derivation for Foreign (noting that the population is $1-n$) gives:
\[
C^*_0 = (A^*_{T,0})^{\gamma}(A^*_{N,0})^{1-\gamma} \left[(1-n)\gamma^{\gamma}(1-\gamma)^{1-\gamma}\right]^{-\alpha}. \tag{6f}
\]
Because of the calibration and the fact that the relative productivities satisfy
\[
\frac{A_{N,0}}{A_{T,0}} = \left(\frac{1-\gamma}{\gamma}\right)^{\alpha} \quad \text{and} \quad \frac{A^*_{N,0}}{A^*_{T,0}} = \left(\frac{1-n}{n}\right)^{\alpha}\left(\frac{1-\gamma}{\gamma}\right)^{\alpha},
\]
we can check that indeed
\[
\frac{C_0}{C^*_0}=1. \tag{6g}
\]
\section{Log-Linear Approximation}
\subsection*{A. Non-Traded Goods Market}
The Home non-traded goods market clearing condition is:
\[
n(1-\gamma)(P_{N,t})^{-\gamma}C_t = A_{N,t}(L_{N,t})^{1-\alpha}.
\]
Taking logarithms, we have
\[
\ln n + \ln(1-\gamma) - \gamma\ln P_{N,t} + \ln C_t = \ln A_{N,t} + (1-\alpha)\ln L_{N,t}.
\]
Linearizing around the steady state (and noting that constants vanish in the difference), we obtain:
\[
-\gamma  \widehat{P_{N,t}} + \widehat{C_t} =  \widehat{A_{N,t}} + (1-\alpha)  \widehat{L_{N,t}}. \tag{7a}
\]
For foreign non-traded goods, we have:
\[
\ln (1-n) + \ln(1-\gamma) - \gamma\ln P^*_{N,t} + \ln C^*_t = \ln A^*_{N,t} + (1-\alpha)\ln L^*_{N,t}.
\]
Linearizing around the steady state, we obtain:
\[
-\gamma  \widehat{P^*_{N,t}} + \widehat{C^*_t} =  \widehat{A^*_{N,t}} + (1-\alpha)  \widehat{L^*_{N,t}}. \tag{7b}
\]

\subsection*{B. Resource Constraint}
The Home resource constraint is:
\[
n\gamma (P_{N,t})^{1-\gamma}C_t + nB_{t+1} = A_{T,t}(n-L_{N,t})^{1-\alpha}+ n(1+r_t)B_t.
\]
Divide both sides by $n \gamma C_0$, we get:
\[
\frac{(P_{N,t})^{1-\gamma}C_t}{C_0} + \widehat{B_{t+1}} = \frac{A_{T,t}(n-L_{N,t})^{1-\alpha}}{n\gamma C_0} + (1+r_t)\widehat{B_t}.
\]

Taking logs and linearizing, we have:
\[
(1-\gamma)  \widehat{P_{N,t}} + \widehat{C_t} + \widehat{B_{t+1}} = \widehat{A_{T,t}} - \frac{(1-\alpha)(L_{N,t} - L_{N,0})}{n-L_{N,0}}  \widehat{L_{N,t}} + \frac{1}{\beta_0} \widehat{B_t}.
\]
Since in steady state $n-L_{N,0}= n\gamma$, and that $\widehat{L_{N,t}} = \frac{L_{N,t} - L_{N,0}}{L_{N,0}}$. Thus,
\[
(1-\gamma)  \widehat{P_{N,t}} + \widehat{C_t} + \widehat{B_{t+1}} = \widehat{A_{T,t}} - (1-\alpha) \frac{1 - \gamma}{\gamma} \widehat{L_{N,t}} + \frac{1}{\beta_0} \widehat{B_t}. \tag{7c}
\]
Recall foreign budget constraint (5b):
\[
(1-n)\gamma (P^*_{N,t})^{1-\gamma}C^*_t - n B_{t+1} = A^*_{T,t}\big((1-n)-L^*_{N,t}\big)^{1-\alpha} - n(1+r_t)B_t.
\]
Following similar steps, we have:
\[
(1-n) (1-\gamma) \bigl(\widehat{P^*_{N,t}} + \widehat{C^*_t} \bigr) - n \widehat{B_{t+1}} = \widehat{A^*_{T,t}} - (1-\alpha) \frac{1 - \gamma}{\gamma} \widehat{L^*_{N,t}} - \frac{n}{\beta_0} \widehat{B_t}.
\]
Divide both sides by $1-n$, we get:
\[
(1-\gamma)  \widehat{P^*_{N,t}} + \widehat{C^*_t} - \frac{n}{1-n} \widehat{B_{t+1}} = \widehat{A^*_{T,t}} - (1-\alpha) \frac{1 - \gamma}{\gamma} \widehat{L^*_{N,t}} - \frac{1}{\beta_0} \frac{n}{1-n} \widehat{B_t}. \tag{7d}
\]

\subsection*{C. Euler Equation}
Recall that:
\[
C_{t+1} = C_t \beta_{H, t+1}(1+r_{t+1})\Bigl(\frac{P_{N,t}}{P_{N,t+1}}\Bigr)^{1-\gamma}
\]
Taking logs and linearizing, we have:
\[
\widehat{C_{t+1}} = \widehat{C_t} + \widehat{\beta_{H,t+1}} + \frac{r_{t+1}-r_0}{1+r_0} + (1-\gamma)( \widehat{P_{N,t}}- \widehat{P_{N,t+1}}).
\]
As $\beta_0(1+r_0) = 1$,
\begin{align*}
    \widehat{r_t} &= r_t - \frac{1-\beta_0}{\beta_0} \\
    &= r_t - \frac{1-\frac{1}{1+r_0}}{\frac{1}{1+r_0}} \\
    &= r_t - r_0
\end{align*}
So, the Euler equation becomes:
\[
\widehat{C_{t+1}} = \widehat{C_t} + \widehat{\beta_{H,t+1}} + \beta_0  \widehat{r_{t+1}} + (1-\gamma)( \widehat{P_{N,t}}- \widehat{P_{N,t+1}}). \tag{7e}
\]
Similarly for Foreign:
\[
\widehat{C_{t+1}^*} = \widehat{C_t^*} + \widehat{\beta_{F,t+1}} + \beta_0  \widehat{r_{t+1}}^* + (1-\gamma)( \widehat{P_{N,t}^*}- \widehat{P_{N,t+1}}^*). \tag{7f}
\]

\subsection*{D. Labor Allocation}
The intratemporal condition is:
\[
A_{T,t}(n-L_{N,t})^{-\alpha} = P_{N,t} A_{N,t}(L_{N,t})^{-\alpha}.
\]
Taking logarithms:
\[
\ln A_{T,t} - \alpha \ln (n-L_{N,t}) = \ln P_{N,t} + \ln A_{N,t} - \alpha \ln L_{N,t}.
\]
Linearize around steady state:
\[
\widehat{A_{T,t}} - \alpha \frac{L_{N,t} - L_{N,0}}{n-L_{N,0}} =  \widehat{P_{N,t}} +  \widehat{A_{N,t}} - \alpha  \widehat{L_{N,t}}.
\]
Using the fact that in steady state $n-L_{N,0}=n\gamma$:
\[
\widehat{A_{T,t}} - \alpha \frac{ \widehat{L_{N,t}} n(1-\gamma)}{n \gamma} =  \widehat{P_{N,t}} +  \widehat{A_{N,t}} - \alpha  \widehat{L_{N,t}}.
\]
\[
\widehat{A_{T,t}} + \frac{\alpha}{\gamma}  \widehat{L_{N,t}} =  \widehat{P_{N,t}} +  \widehat{A_{N,t}}. \tag{7g}
\]
Similarly for Foreign:
\[
\widehat{A_{T,t}^*} + \frac{\alpha}{1-\gamma}  \widehat{L_{N,t}^*} =  \widehat{P_{N,t}^*} +  \widehat{A_{N,t}^*}. \tag{7h}
\]

\subsection*{E. Real Exchange Rate}
As we know:
\[
1+r_{t+1}^{C} = (1+r_{t+1}) \Bigl(\frac{P_{N,t}}{P_{N,t+1}}\Bigr)^{1-\gamma},
\]
take logs and linearize both sides, we have:
\[
\frac{r_{t+1}^C - r_0^C }{1+r_0^C} = \frac{r_{t+1} - r_0}{1+r_0} + (1-\gamma)( \widehat{P_{N,t}} -  \widehat{P_{N,t+1}}).
\]
\[
\Rightarrow \beta_0  \widehat{r_{t+1}^C} = \beta_0  \widehat{r_{t+1}} + (1-\gamma)( \widehat{P_{N,t}} -  \widehat{P_{N,t+1}}).
\]
\[
\widehat{r_{t+1}^C} =  \widehat{r_{t+1}} + (1-\gamma) \frac{1}{\beta_0} ( \widehat{P_{N,t}} -  \widehat{P_{N,t+1}}). \tag{7i}
\]
Similarly for Foreign:
\[
\widehat{r_{t+1}^{*C}} =  \widehat{r_{t+1}^*} + (1-\gamma) \frac{1}{\beta_0} ( \widehat{P_{N,t}^*} -  \widehat{P_{N,t+1}^*}). \tag{7j}
\]


\section{Worldwide Equilibrium}

Define world aggregates as population-weighted averages (e.g., $  \widehat{C_t^W} = n \widehat{C_t} + (1-n) \widehat{C_t^*}$).
Then, from the above log-linearized equations one can show:
\begin{itemize}
    \item \textbf{Non-Traded Goods Market:}
    \begin{align*}
        n \times (7a) + (1-n) \times (7b) &\Rightarrow -\gamma  \widehat{P_{N,t}^W} +  \widehat{C_t^W} =  \widehat{A_{N,t}^W} + (1-\alpha)  \widehat{L_{N,t}^W} \tag{8a}.
    \end{align*}
    \item \textbf{Resource Constraint:}
    \begin{align*}
        n \times (7c) + (1-n) \times (7d) &\Rightarrow (1-\gamma)  \widehat{P_{N,t}^W} +  \widehat{C_t^W} =  \widehat{A_{T,t}^W} - (1-\alpha) \frac{1-\gamma}{\gamma}  \widehat{L_{N,t}^W} \tag{8b}.
    \end{align*}
    \item \textbf{Euler Equation:}
    \begin{align*}
        n \times (7e) + (1-n) \times (7f) &\Rightarrow  \widehat{C_{t+1}^W} =  \widehat{C_t^W} + (1-\gamma) ( \widehat{P_{N,t}^W} -  \widehat{P_{N,t+1}^W}) +  \widehat{\beta_{H,t+1}^W} + \beta_0 \widehat{r_{t+1}} \tag{8c}.
    \end{align*}
    \item \textbf{Labor Allocation:}
    \begin{align*}
        n \times (7g) + (1-n) \times (7h) &\Rightarrow  \widehat{A_{T,t}^W} + \frac{\alpha}{\gamma}  \widehat{L_{N,t}^W} =  \widehat{P_{N,t}^W} +  \widehat{A_{N,t}^W} \tag{8d}.
    \end{align*}
    \item \textbf{Real Exchange Rate:}
    \begin{align*}
        n \times (7i) + (1-n) \times (7j) &\Rightarrow \beta_0  \widehat{r_{t+1}^{C W}} = (1-\gamma) ( \widehat{P_{N,t}^W} -  \widehat{P_{N,t+1}^W}) + \beta_0  \widehat{r_{t+1}} \tag{8e}.
    \end{align*}
\end{itemize}
Let (8a)-(8d), we get:
\[
(1-\gamma)  \widehat{P_{N,t}^W} +  \widehat{C_t^W} =  \widehat{A_{T,t}^W} - \Bigl(1-\alpha + \frac{\alpha}{\gamma} \Bigr)  \widehat{L_{N,t}^W}.
\]
Combine with (8b), we have:
\begin{align*}
    \Bigl(1 - \alpha + \frac{\alpha}{\gamma} \Bigr)  \widehat{L_{N,t}^W} &= -(1-\alpha) \frac{1-\gamma}{\gamma}  \widehat{L_{N,t}^W} \\
    &= (1-\alpha)  \widehat{L_{N,t}^W} - \frac{1-\alpha}{\gamma}  \widehat{L_{N,t}^W} \\
    \Rightarrow \frac{1}{\gamma}  \widehat{L_{N,t}^W} &= 0. \tag{8f}
\end{align*}
Let (8b)-(8a), we have:
\begin{align*}
     \widehat{P_{N,t}^W} &=  \widehat{A_{T,t}^W} -  \widehat{A_{N,t}^W} - (1-\alpha)\Bigl(\frac{1-\gamma}{\gamma} + 1\Bigr)  \widehat{L_{N,t}^W} \\
    &=  \widehat{A_{T,t}^W} -  \widehat{A_{N,t}^W}. \tag{8g}
\end{align*}
Take $(1-\gamma) \times \text{(8a)} + \gamma \text{(8b)}$, we have:
\[
(1-\gamma)  \widehat{C_t^W} + \gamma  \widehat{C_t^W} =  \widehat{C_t^W} = (1-\gamma)  \widehat{A_{N,t}^W} + \gamma  \widehat{A_{T,t}^W}. \tag{8h}
\]
Finally, from (8c), we know that:
\begin{align*}
    \beta_0  \widehat{r_{t+1}} + \widehat{\beta_{H, t+1}^W} &=  \widehat{C_{t+1}^W} -  \widehat{C_t^W} - (1-\gamma)\Bigl( \widehat{P_{N,t}^W} - \widehat{P_{N, t+1}^W}\Bigr) \\
    &= \gamma \hat{A}_{t+1}^W + (1-\gamma)  \widehat{A_{N,t+1}^W} - \gamma  \widehat{A_{T,t+1}^W} - (1-\gamma)  \widehat{A_{N,t+1}^W} \\
    \quad \quad &- (1-\gamma)\Bigl( \widehat{A_{T,t}^W} -  \widehat{A_{N,t}^W}\Bigr) + (1-\gamma) \Bigl( \widehat{A_{T,t+1}^W} - \widehat{A_{N, t+1}^W} \Bigr) \\
    &= \widehat{A_{T, t+1}^W} -  \widehat{A_{T,t}^W}. \tag{8i}
\end{align*}

\textbf{Relative Price of Goods:}\\
The relative price of goods in our model is captured by the real exchange rate
\[
Q_t = \left(\frac{P_{N,t^*}}{P_{N,t}}\right)^{1-\gamma}.
\]
Its log-deviation is
\[
\widehat{Q_t} = (1-\gamma)(\widehat{P_{N,t}^*}-\widehat{P_{N,t}}).
\]
This relative price is driven by:
\begin{itemize}
    \item \textbf{Productivity Differences:} Differences in sector-specific productivity. An increase in Home's traded productivity (a higher \( \widehat{A_{T,t}} \)) relative to non-traded productivity (or relative to Foreign's productivity) lowers the relative cost of producing tradables. In our derivations, the term \( \widehat{A}_{T,t} - \widehat{A}^*_{T,t} \) appears with a negative sign in \( \widehat{Q}_t \), indicating that higher Home traded productivity tends to lower Home's non-tradable price relative to Foreign's.
    \item \textbf{Relative Demand Effects:} Shocks that affect consumption (e.g. through changes in discount factors) alter demand for non-traded goods, hence influencing \( P_{N,t} \).
\end{itemize}

\textbf{Consumption:} \\
Aggregate consumption in each country is determined by both traded and non-traded output. In log-linear terms, world consumption is given by
\[
\widehat{C_t^W} = \gamma\,\widehat{A_{T,t}^W} + (1-\gamma)\,\widehat{A_{N,t}^W},
\]
where \( \widehat{A_{T,t}^W} \) and \( \widehat{A_{N,t}^W} \) are the population-weighted productivity shocks in the traded and non-traded sectors. Thus, consumption is driven by overall improvements in productivity—improvements in traded productivity increase income (and hence consumption) by \( \gamma \) while improvements in non-traded productivity raise consumption by \( 1-\gamma \).

\textbf{Real Interest Rate:}  \\
The consumption-based real interest rate is determined by the Euler equation
\[
1 = \beta_0 (1+r_0),
\]
and its log-linear deviations are influenced by both the intertemporal preference parameter \( \beta \) and by the relative price changes in non-tradables (through \( \widehat{P}_{N,t} \)). Specifically, we have
\[
\widehat{r_{t+1}^C} = \widehat{r_{t+1}} + \frac{1-\gamma}{\beta_0}\left(\widehat{P_{N,t}} - \widehat{P_{N,t+1}}\right).
\]
Thus, the real interest rate responds to:
\begin{itemize}
    \item \textbf{Household Patience:} A higher \( \beta_0 \) (or a more favorable \( \widehat{\beta_{H,t}} \)) implies lower required returns to equate current and future consumption.
    \item \textbf{Expected Price Changes:} Changes in \( \widehat{P_{N,t}} \) affect the real return by altering the consumption price index.
\end{itemize}

\underline{What Drives Relative Price:} \\
This relative price is driven by the productivity gap between the traded and non-traded sectors. 
When traded-goods productivity rises relative to non-traded-goods productivity, 
the price of non-traded goods increases (since traded goods become relatively cheaper to produce). 
Conversely, higher productivity in the non-traded sector (relative to traded) makes non-traded goods relatively cheaper, 
lowering their relative price.

\underline{What Drives Consumption:} \\
Aggregate consumption is driven by productivity improvements in both sectors, 
weighted by their share in consumption. 
In particular, higher productivity in the traded sector raises consumption (through more output/income of tradables), 
and higher productivity in the non-traded sector also raises consumption (more domestic goods), 
each in proportion to their expenditure share ($\gamma$ for traded, $1-\gamma$ for non-traded).

\underline{What Drives the Real Exchange Rate:} \\
The real interest rate is determined by households' patience and expected productivity growth. 
An increase in the Home discount factor (greater patience) lowers the real interest rate, 
since more patient consumers save more (reducing the return needed to equilibrate)
In addition, if future traded-sector productivity is expected to grow faster than present 
(a positive productivity growth outlook), 
the real interest rate tends to rise (reflecting higher expected returns to saving/investing when future output is higher). 
In summary, a shock that makes consumers more patient drives down the consumption-based real interest rate, 
while faster productivity growth prospects push it up.

\section{Cross-Country Differences}
As we know that $Q_t = \Bigl(\frac{P_{N,t}^*}{P_{N,t}}\Bigr)^{1-\gamma}$, log-linearize the equation, we have:
\[\widehat{Q_t} = (1-\gamma)(\widehat{P_{N,t}^*}) - \widehat{P_{N,t}}.\]
Use (7a) - (7b), we get:
\begin{align*}
    \widehat{C_t} - \widehat{C_t^*} - \gamma(\widehat{P_{N,t}} - \widehat{P_{N,t}^*}) &= (\widehat{A_{N,t}} - \widehat{A_{N,t}^*}) + (1-\alpha)(\widehat{L_{N,t}} - \widehat{L_{N,t}^*}) \\
    \Rightarrow \widehat{C_t} - \widehat{C_t^*} + \frac{\gamma}{1-\gamma}\widehat{Q_t} &= (\widehat{A_{N,t}} - \widehat{A_{N,t}^*}) + (1-\alpha)(\widehat{L_{N,t}} - \widehat{L_{N,t}^*}) \tag{9a}.
\end{align*}
Use (7c) - (7d), we get:
\begin{align*}
    \text{LHS} &= (1-\gamma)(\widehat{P_{N,t}} - \widehat{P_{N,t}^*}) + \widehat{C_t} - \widehat{C_t^*} + \frac{\widehat{B_{t+1}}}{1-n} \\
    \text{RHS} &= (\widehat{A_{T,t}} - \widehat{A_{T,t}^*}) - (1-\alpha)\frac{1-\gamma}{\gamma}(\widehat{L_{N,t}} - \widehat{L_{N,t}^*}) + \frac{1}{\beta_0} \frac{\widehat{B_t}}{1-n} \\
    \text{As } \widehat{Q_t} &= (1-\gamma)(\widehat{P_{N,t}^*} - \widehat{P_{N,t}}), \text{ we have} \\ 
    \text{LHS} &= - \widehat{Q_t} + (\widehat{C_t} - \widehat{C_t^*}) + \frac{\widehat{B_{t+1}}}{1-n} = \text{RHS} \tag{9b}
\end{align*}
Use (7e) - (7f), we get:
\begin{align*}
    (\widehat{C_{t+1}} - \widehat{C_{t+1}^*}) &= (\widehat{C_t} - \widehat{C_t^*}) + (1-\gamma)(\widehat{P_{N,t}} - \widehat{P_{N,t}^*}) + (1-\gamma)(\widehat{P_{N,t+1}^*} - \widehat{P_{N,t+1}}) + (\widehat{\beta_{H, t+1}} - \widehat{\beta_{F, t+1}}) \\
    &= (\widehat{C_t} - \widehat{C_t^*}) - \widehat{Q_t} + \widehat{Q_{t+1}} + \widehat{\beta_{H, t+1}} - \widehat{\beta_{F, t+1}} \tag{9c}.
\end{align*}
Use (7g) - (7h), we get:
\begin{align*}
    (\widehat{A_{T,t}} - \widehat{A_{T,t}^*}) + \frac{\alpha}{\gamma}(\widehat{L_{N,t}} - \widehat{L_{N,t}^*}) &= (\widehat{P_{N,t}} - \widehat{P_{N,t}^*}) + (\widehat{A_{N,t}} - \widehat{A_{N,t}^*}) \\
    &= - \frac{1}{1-\gamma} \widehat{Q_t} + (\widehat{A_{N,t}} - \widehat{A_{N,t}^*}) \tag{9d}.
\end{align*}
Use (7i) - (7j), we get:
\begin{align*}
    \beta_0(\widehat{r_{t+1}^C} - \widehat{r_{t+1}^{C*}}) &= (1-\gamma)(\widehat{P_{N,t}} - \widehat{P_{N,t}^*}) + (1-\gamma)(\widehat{P_{N,t}^*} - \widehat{P_{N,t}}) \\
    &= \widehat{Q_{t+1}} - \widehat{Q_t} \tag{9e}.
\end{align*}


% \textbf{Intuition:} These equations link cross-country differences in consumption, labor, and the real exchange rate to differences in productivity and intertemporal preferences.

\section{Long-Run Allocation (Period $t+1$)}

Assume that from $ t+1 $ onward the economy reaches a new steady state with no further discount factor shocks ($ \widehat{\beta_{H,t+2}} = \widehat{\beta_{F,t+2}} = 0 $).
In the steady state, the consumption growth rate is zero, the asset position is fixed and the real exchange rate is stable. 

Using the labor allocation equation at $ t+1 $, we have:
\[
\frac{\alpha}{\gamma} (\widehat{L_{N,t}} - \widehat{L_{N,t}^*}) = -\frac{1}{1-\gamma} \widehat{Q_{t+1}} + \Bigl[ (\widehat{A_{N,t}} - \widehat{A_{N,t}^*}) - (\widehat{A_{T,t}} - \widehat{A_{T,t}^*}) \Bigr] \tag{10.1}.
\]
Then, we use the market clearing condition for non-traded goods and the resource allocation constraints at $ t+1 $:
\begin{align*}
    \frac{\gamma}{1-\gamma} \widehat{Q_{t+1}} + (\widehat{C_{t+1}} - \widehat{C_{t+1}^*}) &= (\widehat{A_{N,t+1}} - \widehat{A_{N,t+1}^*}) + (1-\alpha )(\widehat{L_{N,t+1}} - \widehat{L_{N,t+1}^*}) \tag{10.2} \\
    -\widehat{Q_{t+1}} + (\widehat{C_{t+1}} - \widehat{C_{t+1}}^*) + \frac{\widehat{B_{t+2}}}{1-n} &= (\widehat{A_{T,t+1}} - \widehat{A_{T,t+1}^*}) - (1-\alpha) \frac{1-\gamma}{\gamma} (\widehat{L_{N,t+1}} - \widehat{L_{N,t+1}^*}) + \frac{1}{\beta_0} \frac{\widehat{B_{t+1}}}{1-n}. \tag{10.3}
\end{align*}
As $\widehat{B_{t+2}} = \widehat{B_{t+1}}$, using (10.2)-(10.3), we get:
\begin{align*}
    & \quad \Bigl(\frac{1-\gamma}{\gamma} + 1 \Bigr) \widehat{Q_{t+1}} = \Bigl[ (\widehat{A_{N,t+1}} - \widehat{A_{N,t+1}^*}) - (\widehat{A_{T,t+1}} - \widehat{A_{T,t+1}^*}) \Bigr] \\
    &\quad\quad\quad\quad\quad\quad\quad\quad\quad\quad + (1-\alpha) \Bigl(1 + \frac{1-\gamma}{\gamma}\Bigr) (\widehat{L_{N,t+1}} - \widehat{L_{N,t+1}^*}) - \frac{1-\beta_0}{\beta_0} \frac{\widehat{B_{t+1}}}{1-n} \\
    \Rightarrow & \quad \widehat{Q_{t+1}} = (1-\gamma)\Bigl[ (\widehat{A_{N,t+1}} - \widehat{A_{N,t+1}^*}) - (\widehat{A_{T,t+1}} - \widehat{A_{T,t+1}^*}) \Bigr] + (1-\alpha) \frac{1-\gamma}{\gamma} (\widehat{L_{N,t+1}} - \widehat{L_{N,t+1}^*}) \\ 
    &\quad\quad\quad\quad- (1-\gamma) \frac{1-\beta_0}{\beta_0} \frac{\widehat{B_{t+1}}}{1-n} \tag{10.4}.
\end{align*}
Replacing $\widehat{L_{N,t+1}} - \widehat{L_{N,t+1}^*}$ using (10.1), we get:
\begin{align*}
    & \quad \widehat{Q_{t+1}} = (1-\gamma) \Bigl[ (\widehat{A_{N,t+1}} - \widehat{A_{N,t+1}^*}) - (\widehat{A_{T,t+1}} - \widehat{A_{T,t+1}^*}) \Bigr] - \frac{1-\alpha}{\alpha} \widehat{Q_{t+1}} \\
    &\quad\quad\quad\quad+ \frac{1-\alpha}{\alpha} (1-\gamma) \Bigl[ (\widehat{A_{N,t+1}} - \widehat{A_{N,t+1}^*}) - (\widehat{A_{T,t+1}} - \widehat{A_{T,t+1}^*}) \Bigr] - (1-\gamma) \frac{1-\beta_0}{\beta_0} \frac{\widehat{B_{t+1}}}{1-n} \\
    \Rightarrow & \quad \Bigl( 1 + \frac{1-\alpha}{\alpha} \Bigr) \widehat{Q_{t+1}} = (1-\gamma) \Bigl(1 + \frac{1-\alpha}{\alpha} \Bigr) \Bigl[ (\widehat{A_{N,t+1}} - \widehat{A_{N,t+1}^*}) - (\widehat{A_{T,t+1}} - \widehat{A_{T,t+1}^*}) \Bigr] \\
    &\quad\quad\quad\quad\quad\quad\quad\quad\quad\quad - (1-\gamma) \frac{1-\beta_0}{\beta_0} \frac{\widehat{B_{t+1}}}{1-n} \\
    \Rightarrow & \quad \widehat{Q_{t+1}} = -(1-\gamma) \Bigl[ (\widehat{A_{T,t+1}} - \widehat{A_{T,t+1}^*}) - (\widehat{A_{N,t+1}} - \widehat{A_{N,t+1}^*}) \Bigr] - \alpha (1-\gamma) \frac{1-\beta_0}{\beta_0} \frac{\widehat{B_{t+1}}}{1-n} \tag{10a}.
\end{align*}
Comparing (10.4) and (10a), we have:
\begin{align*}
    & \quad (1-\alpha) \frac{1-\gamma}{\gamma} (\widehat{L_{N,t+1}} - \widehat{L_{N,t+1}^*}) - (1-\gamma) \frac{1-\beta_0}{\beta_0} \frac{\widehat{B_{t+1}}}{1-n} = -\alpha (1-\gamma) \frac{1-\beta_0}{\beta_0} \frac{\widehat{B_{t+1}}}{1-n} \\
    \Rightarrow & \quad \widehat{L_{N,t+1}} - \widehat{L_{N,t+1}^*} = \gamma \frac{1-\beta_0}{\beta_0} \frac{\widehat{B_{t+1}}}{1-n} \tag{10b}.
\end{align*}
Implementing (10a) and (10b) back into (10.2), we get: 
\begin{align*}
    \widehat{C_{t+1}} - \widehat{C_{t+1}^*} &= \widehat{Q_{t+1}} + (\widehat{A_{T,t+1}} - \widehat{A_{T,t+1}^*}) - \frac{(1-\alpha)(1-\gamma)}{\gamma} \frac{1-\beta_0}{\beta_0} \frac{\widehat{B_{t+1}}}{1-n} + \frac{1-\beta_0}{\beta_0} \frac{\widehat{B_{t+1}}}{1-n} \\
    &= -(1-\gamma) (\widehat{A_{T,t+1}} - \widehat{A_{T,t+1}^*}) + (1-\gamma) (\widehat{A_{N,t+1}} - \widehat{A_{N,t+1}^*}) + (\widehat{A_{T,t+1}} - \widehat{A_{T,t+1}^*}) \\
    & \quad - \alpha (1-\gamma) \frac{1-\beta_0}{\beta_0} \frac{\widehat{B_{t+1}}}{1-n} - (1-\alpha) (1-\gamma) \frac{1-\beta_0}{\beta_0} \frac{\widehat{B_{t+1}}}{1-n} + \frac{1-\beta_0}{\beta_0} \frac{\widehat{B_{t+1}}}{1-n} \\
    &= \gamma(\widehat{A_{T,t+1}} - \widehat{A_{T,t+1}^*}) + (1-\gamma) (\widehat{A_{N,t+1}} - \widehat{A_{N,t+1}^*}) + \gamma \frac{1-\beta_0}{\beta_0} \frac{\widehat{B_{t+1}}}{1-n} \tag{10c}
\end{align*}

\textbf{(1) Impact of Home Being Wealthier (\( \widehat{B}_{t+1} > 0 \)):}\\
When \( \widehat{B}_{t+1} > 0 \), Home has accumulated net foreign assets. In the long run, this wealth effect leads to:
\begin{itemize}
    \item A higher steady-state consumption level in Home relative to Foreign, since net asset income augments Home's overall resources.
    \item An upward pressure on domestic (Home) demand, which, via the resource constraints, increases the domestic price of non-tradables. This results in a real appreciation of Home's currency (i.e. a lower \( Q_{t+1} \), 
    meaning Home's goods become relatively more expensive).
\end{itemize}
Thus, being wealthier (with \( \widehat{B}_{t+1} > 0 \)) directly boosts Home's consumption while causing a real appreciation.

\textbf{(2) Long-Run Impact of Productivity (Excluding the Asset Channel):}\\
\begin{itemize}
    \item An increase in Home's traded productivity (\( \widehat{A}_{T,t} > 0 \)) raises the output in the traded sector. This increases Home's income and consumption by the factor \( \gamma \) and, via the Balassa--Samuelson mechanism, raises wages and non-traded prices, leading to a real appreciation.
    \item Conversely, an increase in Home's non-traded productivity (\( \widehat{A}_{N,t} > 0 \)) improves domestic output in non-tradables, boosting consumption by the factor \( 1-\gamma \) and reducing the relative price of non-tradables, which tends to depreciate Home's currency.
\end{itemize}
In either case, productivity improvements raise long-run consumption directly, and the real exchange rate adjusts according to the sector in which the productivity shock occurs.

\underline{Impact of home country wealthier}: \\
If Home has a positive net foreign asset position (Home is wealthier), 
it can afford higher consumption relative to Foreign. 
Home's steady-state consumption will be higher than Foreign's, 
and this higher demand for Home goods makes Home's goods relatively more expensive. 
In other words, a wealthier Home leads to a real appreciation of Home's goods (a lower $Q_{t+1}$, 
meaning Foreign goods become cheap relative to Home goods). 
Home runs a trade balance surplus (earning income from its assets), 
which supports its higher consumption.

\underline{Longrun impact of productivity:} \\
Permanent productivity differences translate directly into differences in consumption and relative prices in steady state. 
A country with higher productivity (Home, in this case) 
will enjoy higher long-run consumption levels and a stronger relative price for its goods independent of net foreign assets.
In effect, Home's higher productivity raises its real income and consumption, 
and causes a direct real exchange rate adjustment: 
for example, higher traded-sector productivity in Home leads to higher Home wages and non-traded prices 
(Balassa-Samuelson effect), 
appreciating Home's real exchange rate; 
higher non-traded productivity would raise Home consumption and lower Home's non-traded prices, 
depreciating the real exchange rate. 
In all cases, the more productive economy has a higher standard of living 
(consumption) and its price levels adjust accordingly in the long run, 
aside from any asset accumulation.

\section{Short-Run Allocation (Period $t$)}
To simplify notation, we denote $ \widehat{A_{N,t}} - \widehat{A_{N,t}^*}  = \widetilde{A_{N,t}}$, $ \widehat{A_{T,t}} - \widehat{A_{T,t}^*}  = \widetilde{A_{T,t}}$, $ \widehat{L_{N,t}} - \widehat{L_{N,t}^*}  = \widetilde{L_{N,t}}$. 

Combining (9a) and (9d), we can get:
\begin{gather*}
    \widetilde{C_t} + \frac{\gamma}{1-\gamma} \widehat{Q_t} - \gamma \widetilde{A_{T,t}} - \alpha \widetilde{L_{N,t}} = \widetilde{A_{N,t}}(1-\gamma) + (1-\alpha)\widetilde{L_{N,t}} + \frac{\gamma}{1-\gamma}\widehat{Q_t} \\
    \Rightarrow \widetilde{L_{N,t}} = \widetilde{C_t} - \gamma\widetilde{A_{T,t}} - (1-\gamma) \widetilde{A_{N,t}} \tag{11.1}
\end{gather*}
and that
\begin{gather*}
    (1-\alpha) \gamma \widetilde{A_{T,t}} + \alpha (1-\alpha) \widetilde{L_{N,t}} + \frac{\alpha \gamma}{1-\gamma} \widehat{Q_t} + \alpha \widetilde{C_t} = -\frac{\gamma(1-\alpha)}{1-\gamma}\widehat{Q_t} + (1-\alpha) \gamma \widetilde{A_{N,t}} + \alpha \widetilde{A_{N,t}} + \alpha(1-\alpha)\widetilde{L_{N,t}} \\
    \Rightarrow \widehat{Q_t} = \Bigl(\frac{1-\gamma}{\gamma}\Bigr) [(1-\alpha)\gamma + \alpha] \widetilde{A_{N,t}} - (1-\alpha)(1-\gamma)\widetilde{A_{T,t}} - \frac{\alpha(1-\gamma)}{\gamma} \widetilde{C_t} \tag{11.2}
\end{gather*}
Implementing (11.1) and (11.2) back into (9b), we get:
\begin{gather*}
    - \Bigl(\frac{1-\gamma}{\gamma}\Bigr) [(1-\alpha)\gamma + \alpha] \widetilde{A_{N,t}} + (1-\alpha)(1-\gamma)\widetilde{A_{T,t}} + \frac{\alpha(1-\gamma)}{\gamma} \widetilde{C_t} + \widetilde{C_t} + \frac{\widehat{B_{t+1}}}{1-n} \\
        = \widehat{A_{T,t}} - \frac{(1-\alpha)(1-\gamma)}{\gamma} \widetilde{C_t} + (1-\alpha)(1-\gamma)\widetilde{A_{T,t}} + \frac{(1-\alpha)(1-\gamma)^2}{\gamma} \widetilde{ A_{N,t}} \\
        \Rightarrow \frac{\widehat{B_{t+1}}}{1-n} = \widetilde{A_{T,t}} + \frac{1-\gamma}{\gamma} \widetilde{A_{N,t}} - \frac{1}{\gamma} \widetilde{C_t}\\
        \Rightarrow \widetilde{C_t} = \gamma \widetilde{A_{T,t}} + (1-\gamma) \widetilde{A_{N,t}} - \gamma \frac{\widehat{B_{t+1}}}{1-n} \tag{11.3}
\end{gather*}
Bring (11.3) back to (11.2), we get:
\begin{gather*}
    \widehat{Q_t} = (1-\gamma) (\widetilde{A_{N,t}} - \widetilde{A_{T,t}}) + \gamma \frac{\widehat{B_{t+1}}}{1-n} \tag{11.4}
\end{gather*}
Subtracting (11.4) from (10a), we get:
\begin{gather*}
    \widehat{Q_{t+1}} - \widehat{Q_t} = (1-\gamma) (\widetilde{A_{N,t+1}} - \widetilde{A_{N,t}}) - (1-\gamma) (\widetilde{A_{T,t+1}} - \widetilde{A_{T,t}}) - \alpha (1-\gamma) \frac{1}{\beta_0} \frac{\widehat{B_{t+1}}}{1-n} \tag{11.5}
\end{gather*}
Subtracting (11.3) from (10c), we get:
\begin{gather*}
    \widetilde{C_{t+1}} - \widetilde{C_t} = \gamma(\widetilde{A_{T,t+1}} - \widetilde{A_{T,t}}) + (1-\gamma) (\widetilde{A_{N,t+1}} - \widetilde{A_{N,t}}) + \frac{\gamma}{\beta_0} \frac{\widehat{B_{t+1}}}{1-n} \tag{11.6}
\end{gather*} 
Bring (11.5) and (11.6) back to (9c), we get:
\begin{gather*}
    \gamma (\widetilde{A_{T,t+1}} - \widetilde{A_{T,t}}) + (1-\gamma) (\widetilde{A_{N,t+1}} - \widetilde{A_{N,t}}) + \frac{\gamma}{\beta_0} \frac{\widehat{B_{t+1}}}{1-n} \\
    = (1-\gamma) (\widetilde{A_{N,t+1}} - \widetilde{A_{N,t}}) - (1-\gamma) (\widetilde{A_{T,t+1}} - \widetilde{A_{T,t}}) - \alpha(1-\gamma) \frac{1}{\beta_0} \frac{\widehat{B_{t+1}}}{1-n} + \widehat{\beta_{H,t+1}} - \widehat{\beta_{F,t+1}} \\
    \Rightarrow \Bigl[\gamma + \alpha(1-\gamma)\Bigr] \frac{1}{\beta_0} \frac{\widehat{B_{t+1}}}{1-n} = \widehat{\beta_{H,t+1}} - \widehat{\beta_{F,t+1}} - (\widetilde{A_{T,t+1}} - \widetilde{A_{T,t}}) \\
    \Rightarrow \frac{\widehat{B_{t+1}}}{1-n} = \frac{\beta_0}{\gamma + \alpha(1-\gamma)} \Bigl[ \widehat{\beta_{H,t+1}} - \widehat{\beta_{F,t+1}} - (\widehat{A_{T,t+1}} - \widehat{A_{T,t+1}^*}) + (\widehat{A_{T,t}} - \widehat{A_{T,t}^*}) \Bigr] \tag{11a}
\end{gather*}
Bring (11a) back to (11.3), we get:
\begin{align*}
    \widehat{C_{t+1}} - \widehat{C_t} &= \gamma (\widehat{A_{T,t}} - \widehat{A_{T,t}^*}) + (1-\gamma) (\widehat{A_{N,t}} - \widehat{A_{N,t}^*}) \\
    &- \frac{\gamma \beta_0}{\gamma + \alpha(1-\gamma)} \Bigl[ \widehat{\beta_{H,t+1}} - \widehat{\beta_{F,t+1}} - (\widehat{A_{T,t+1}} - \widehat{A_{T,t+1}^*}) + (\widehat{A_{T,t}} - \widehat{A_{T,t}^*}) \Bigr] \tag{11b}
\end{align*}
Bring (11a) back to (11.4), we get:
\begin{align*}
    \widehat{Q_t} &= -(1-\gamma) (\widehat{A_{T,t}} - \widehat{A_{T,t}^*}) + (1-\gamma) (\widehat{A_{N,t}} - \widehat{A_{N,t}^*}) \\
    &+ \frac{\alpha (1-\gamma) \beta_0}{\gamma + \alpha(1-\gamma)} \Bigl[ \widehat{\beta_{H,t+1}} - \widehat{\beta_{F,t+1}} - (\widehat{A_{T,t+1}} - \widehat{A_{T,t+1}^*}) + (\widehat{A_{T,t}} - \widehat{A_{T,t}^*}) \Bigr] \tag{11c}
\end{align*}
Bring (11b) and (11c) back to (11.1), we get:
\begin{gather*}
    \widehat{L_{N,t}} - \widehat{L_{N,t}^*} = - \frac{\gamma \beta_0}{\gamma + \alpha(1-\gamma)} \Bigl[ \widehat{\beta_{H,t+1}} - \widehat{\beta_{F,t+1}} - (\widehat{A_{T,t+1}} - \widehat{A_{T,t+1}^*}) + (\widehat{A_{T,t}} - \widehat{A_{T,t}^*}) \Bigr] \tag{11d}
\end{gather*}
By (9c), we know that:
\begin{gather*}
    \beta_0 (\widehat{r_{t+1}^C} - \widehat{r_{t+1}^{C*}}) = \widehat{Q_{t+1}} - \widehat{Q_t}
\end{gather*}
Implementing (10.4) and (11c) back to the equation, we get:
\begin{align*}
    \beta_0 (\widehat{r_{t+1}^C} - \widehat{r_{t+1}^{C*}}) &= -(1-\gamma) \Bigl[ (\widehat{A_{T,t+1}} - \widehat{A_{T,t+1}^*}) - (\widehat{A_{N,t+1}} - \widehat{A_{N,t+1}^*}) \Bigr] - \alpha (1-\gamma) \frac{1-\beta_0}{\beta_0} \frac{\widehat{B_{t+1}}}{1-n} \\
    & \quad - \bigl\{ -(1-\gamma) (\widehat{A_{T,t}} - \widehat{A_{T,t}^*}) + (1-\gamma) (\widehat{A_{N,t}} - \widehat{A_{N,t}^*}) + \alpha (1-\gamma) \frac{\widehat{B_{t+1}}}{1-n} \bigr\}
\end{align*}
Bring (11a) back to the equation, we get:
\begin{align*}
    \beta_0 (\widehat{r_{t+1}^C} - \widehat{r_{t+1}^{C*}}) &= -(1-\gamma) \Bigl[ (\widehat{A_{T,t+1}} - \widehat{A_{T,t+1}^*}) - (\widehat{A_{N,t+1}} - \widehat{A_{N,t+1}^*}) \Bigr] - \alpha (1-\gamma) \frac{1-\beta_0}{\beta_0} \frac{\widehat{B_{t+1}}}{1-n} \\
    & \quad + (1-\gamma) \Bigl[(\widehat{A_{T,t}} - \widehat{A_{T,t}^*}) - (\widehat{A_{N,t}} - \widehat{A_{N,t}^*}) \Bigr] \\
    & \quad - \frac{\alpha (1-\gamma)}{\gamma + \alpha(1-\gamma)} \Bigl[ \widehat{\beta_{H,t+1}} - \widehat{\beta_{F,t+1}} - (\widehat{A_{T,t+1}} - \widehat{A_{T,t+1}^*}) + (\widehat{A_{T,t}} - \widehat{A_{T,t}^*}) \Bigr] \tag{11e}
\end{align*}

\textbf{(1) Impact of a Temporary Increase in Home Patience (\( \widehat{\beta}_{H,t+1} - \widehat{\beta}_{F,t+1} > 0 \)):}\\[1mm]
A temporary increase in Home's patience means that \( \widehat{\beta}_{H,t+1} > \widehat{\beta}_{F,t+1} \). The consequences are:
\begin{itemize}
    \item Home households postpone current consumption and save more, leading to a positive net asset accumulation, i.e., \( \widehat{B}_{t+1} > 0 \).
    \item The current account surplus (positive \( \widehat{B}_{t+1} \)) reduces Home's current consumption relative to Foreign, so that the cross-country consumption gap \( \widetilde{C}_t \) becomes negative.
    \item As Home saves more, there is a reallocation of labor away from the non-traded sector (since non-tradables are largely consumed domestically) and the real exchange rate adjusts accordingly. Typically, increased saving puts downward pressure on domestic non-tradable prices, leading to a real depreciation, which helps balance the external sector.
\end{itemize}

\textbf{(2) Impact of a Temporary Shock in Home Traded Productivity (\( \widehat{A}_{T,t} > 0 \)) versus Non-Traded Productivity (\( \widehat{A}_{N,t} > 0 \)):}\\[1mm]
\textbf{Temporary Traded Productivity Shock (\( \widehat{A}_{T,t} > 0 \)):}
\begin{itemize}
    \item A positive temporary shock in \( \widehat{A}_{T,t} \) increases Home's output of tradable goods.
    \item Because tradables are internationally traded, Home can export the surplus. However, households do not instantaneously increase their consumption by the full amount of the productivity gain; instead, some of the extra income is saved.
    \item As a result, Home runs a current account surplus (\( \widehat{B}_{t+1} > 0 \)) and, by the mechanism of reallocation, its non-traded goods sector experiences higher demand—leading to higher domestic prices and a real appreciation of Home's currency.
\end{itemize}

\textbf{Temporary Non-Traded Productivity Shock (\( \widehat{A}_{N,t} > 0 \)):}
\begin{itemize}
    \item A temporary shock in \( \widehat{A}_{N,t} \) boosts the production of non-tradable goods, which are consumed domestically.
    \item Since non-tradables are not exported, the entire output effect is absorbed by domestic consumption, raising Home's real consumption without significantly altering external trade.
    \item The increased supply of non-tradables tends to lower their relative price, so the domestic consumption price index falls. This produces a real depreciation of Home's currency.
    \item Additionally, with most of the income effect confined to domestic consumption, there is little pressure for a current account surplus or deficit.
\end{itemize}

\textbf{(3) Impact of Permanent Productivity Shocks:}\\[1mm]
For permanent shocks (i.e. when \( \widehat{A}_{T,t+1} = \widehat{A}_{T,t} \) or \( \widehat{A}_{N,t+1} = \widehat{A}_{N,t} \)), the following occurs:
\begin{itemize}
    \item The economy adjusts immediately to the new steady state, and by the intertemporal Euler condition no net asset accumulation is required (\( \widehat{B}_{t+1} = 0 \)).
    \item Home's consumption jumps to a higher level that reflects the permanent increase in productivity.
    \item The relative price adjustments occur immediately so that the real exchange rate settles at its new steady-state value.
\end{itemize}
In essence, permanent productivity shocks affect the long-run levels of consumption and relative prices without generating a temporary current account imbalance.

\end{document}

