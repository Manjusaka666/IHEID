\documentclass[12pt]{article}
\usepackage{amsmath,amssymb,amsthm}
\usepackage{hyperref}
\usepackage{enumitem}
\usepackage{ctex}
\usepackage[
    margin=2.5cm, 
    % top=2.8cm, bottom=2.8cm,
    % left=1in, right=1in, 
    % headheight=14.5pt
    ]{geometry}
\usepackage{setspace}
\setstretch{1.15}

\setlist[itemize]{noitemsep, topsep=2pt}
\setlist[enumerate]{noitemsep, topsep=2pt}

\begin{document}

\section*{宏观经济学B:理论体系框架、发展逻辑与经济学解释}

\subsection*{零、课程导论与宏观经济学基础 (Class 1)}

\textbf{核心内容:}
\begin{itemize}
    \item \textbf{国际收支平衡表 (BoP)}:
    \begin{itemize}
        \item \textbf{经济学解释}:BoP是衡量一国在特定时期内与世界其他地区所有经济交易的系统记录。它不仅仅是一张会计报表,更是分析一国对外经济健康状况、汇率压力、外债可持续性以及国际资本流动影响的关键工具。BoP的核心原则是复式记账法,每笔交易都同时产生借方和贷方记录,理论上总差额为零。
        \item \textbf{经常账户 (CA)}:记录了货物贸易、服务贸易、初次收入(如跨境投资的利润、利息、股息和雇员报酬)和二次收入(如侨汇、国际援助等单方面转移)。CA顺差意味着一国是净贷出者,增加了对外国的净债权;逆差则相反。其构成
        \[
        CA_t = NX_t + BPI_t + BSI_t
        \]
        反映了一国通过贸易和要素服务获得的净收入。
        \item \textbf{金融账户 (FA)}:记录了跨境金融资产交易,包括直接投资 (FDI)、证券投资、金融衍生品、其他投资(如跨境贷款和存款)以及储备资产的变动。FA反映了一国如何为CA赤字融资,或如何运用CA盈余进行对外投资。FA的净额(通常定义为净资本流出,即本国居民购买外国资产减去外国居民购买本国资产)在符号上与CA(忽略资本账户和统计误差)相反。
        \item \textbf{资本账户 (KA)}:规模通常较小,主要记录非生产性非金融资产的转移(如债务豁免、移民转移)和知识产权相关的交易。
        \item \textbf{BoP恒等式}:
        \[
        CA_t + KA_t - FA_t + NEO_t = 0
        \]
        \textbf{经济学解释}:这个恒等式是BoP的核心。它表明,一国的国际收支在事后总是平衡的。如果一国经常账户出现赤字(支出大于收入),它必须通过金融账户的净资本流入(即出售国内资产给外国人或借款)来弥补,或者动用其官方储备资产。反之,经常账户盈余意味着该国正在积累外国资产或偿还外债。
    \end{itemize}
    \item \textbf{国际投资头寸 (IIP)}:
    \begin{itemize}
        \item \textbf{经济学解释}:IIP衡量一国在特定时点上对外金融资产存量与负债存量的差额,即净国际投资头寸 (NIIP)。NIIP为正,说明该国是净债权国;为负,则是净债务国。IIP的变动不仅受当期BoP中金融账户流量的影响,还受到\textbf{估值效应}的显著影响。估值效应源于存量资产和负债的市场价格变动(如股价、债券价格变动)以及汇率变动(导致外币计价资产和负债的本币价值发生变化)。例如,美元贬值会增加美国以外币计价的海外资产的美元价值,从而改善其NIIP,即使其经常账户为赤字。
        \item \textbf{数学联系}:
        \[
        NIIP_t = NIIP_{t-1} + CA_t + \text{资本账户净额}_t + \text{估值调整}_t
        \]
        这里的$CA_t$可以被看作是当期新增的净对外投资(或融资需求)。
    \end{itemize}
    \item \textbf{国民账户体系与BoP的联系}:
    \begin{itemize}
        \item \textbf{经济学解释}:一国的经常账户差额与其国内储蓄和投资之间的差额密切相关。国民收入核算恒等式 $Y = C + I + G + NX$(其中$NX$是净出口,即货物和服务贸易差额)。将GNI(国民总收入)定义为 $GNI = GDP + BPI + BSI$,则 $CA = NX + BPI + BSI = GNI - (C+G) - I$。由于国民储蓄 $S = GNI - (C+G)$(私人储蓄+政府储蓄),因此
        \[
        CA_t = S^{total}_t - I^{total}_t
        \]
    \end{itemize}
    \item \textbf{开放经济的一些典型事实}:
    \begin{itemize}
        \item \textbf{资本流动的演变}:观察全球资本流动的规模、构成(如银行贷款、FDI、证券投资的相对重要性)及其随时间的变化,特别是在金融全球化和金融危机前后的变化。
        \item \textbf{全球失衡}:指一些国家(如美国)长期存在大规模经常账户赤字,而另一些国家(如中国、德国、石油出口国)则长期存在大规模盈余的现象。这种失衡的可持续性及其对全球经济的影响是重要的研究课题。
        \item \textbf{“过度特权” (Exorbitant Privilege)}:特指美国作为国际储备货币发行国,能够以较低成本从国外融资,并且其对外资产回报率系统性高于其对外负债成本的现象。这部分源于其金融市场的深度和流动性,以及美元的特殊地位。然而,这种特权也可能在危机时期面临挑战。
        \item \textbf{汇率动态}:名义汇率和实际汇率的短期大幅波动和长期趋势。长期内名义汇率变动往往反映通胀差异(相对PPP),但短期内波动较大。
        \item \textbf{危机}:开放经济体更容易受到外部冲击的影响,可能经历货币危机、债务危机或银行危机。
    \end{itemize}
\end{itemize}

\textbf{经济学发展逻辑 (起点):}
\begin{itemize}
    \item \textbf{为何学习BoP和国民账户?}
    \begin{itemize}
        \item \textbf{解释}:这是进行任何有意义的开放经济分析的基石。如同医生需要病人的体检报告才能诊断病情,经济学家需要BoP和国民账户数据来理解一国经济的“体征”。它们提供了描述一国与外部世界经济互动的标准化语言和量化指标,是后续理论模型校准和检验的基础。例如,理解美国为何能长期维持经常账户赤字并吸引大量资本流入,首先就需要分析其BoP的结构,特别是金融账户下资本的构成(是FDI、证券投资还是其他形式)以及IIP中“过度特权”的具体体现。
    \end{itemize}
\end{itemize}

\textbf{练习/思考方向:}
\begin{itemize}
    \item \textbf{分析特定国家的BoP数据,解读其经常账户和金融账户的构成及含义。}
    \begin{itemize}
        \item \textbf{解释}:这个练习旨在培养学生从原始数据中提取经济信息的能力。例如,如果一个发展中国家金融账户中FDI占比较高,可能表明外国投资者看好其长期增长潜力;如果短期证券投资(“热钱”)占比较高,则可能面临更大的资本流动波动风险。经常账户中,贸易差额、初次收入差额和二次收入差额的相对大小也能反映该国的经济结构和发展阶段。
    \end{itemize}
    \item \textbf{理解估值效应如何影响IIP的变动,而不完全由经常账户决定。} 
    \begin{itemize}
        \item \textbf{解释}:这是一个重要的澄清,有助于避免简单地将NIIP的恶化归咎于持续的经常账户赤字。例如,一个国家即使有CA赤字,但如果其海外资产(以外币计价)因汇率贬值而大幅升值,其NIIP仍可能改善。这提示我们关注资产组合的货币构成和资产类别的风险收益特征。
    \end{itemize}
    \item \textbf{运用储蓄-投资恒等式分析一国经常账户失衡的国内根源。} 
    \begin{itemize}
        \item \textbf{解释}:这个练习强调用国内视角理解国际收支问题。经常账户逆差本质上是国内支出(消费+投资)大于国内产出(国民可支配收入)的结果,也等同于国内投资大于国内总储蓄。因此,分析CA失衡不能仅仅看贸易政策,还需要考察国内的储蓄行为(私人储蓄、政府财政状况)和投资需求。例如,“双赤字”假说就认为政府财政赤字扩大(储蓄减少)会导致经常账户赤字扩大。
    \end{itemize}
\end{itemize}

\subsection*{一、跨期视角下的开放经济:消费平滑与经常账户 (Class 2)}

\textbf{核心内容:}
\begin{itemize}
    \item \textbf{两期模型:消费者跨期效用最大化与跨期预算约束}:
    \begin{itemize}
        \item \textbf{经济学解释}:该模型是分析跨期决策的基本框架。消费者(或国家)的目标是在不同时期分配资源(消费)以最大化其总效用。效用函数通常假定边际效用递减,$\beta$ 是时间偏好因子,反映未来的效用在今天看来有多重要($\beta < 1$ 表示未来效用会被打折)。跨期预算约束表明,所有时期消费的现值不能超过所有时期收入的现值(加上初始净资产)。
        \item \textbf{关键问题}:面对收入的波动,国家应如何安排各期的消费和储蓄/借贷才能达到最优?
    \end{itemize}
    \item \textbf{欧拉方程}:
    \[
    u'(C_1) = \beta(1+r)u'(C_2)
    \]
    \textbf{经济学解释}:这是跨期最优决策的一阶条件。它指出,在最优路径上,今天消费一单位带来的边际效用,应该等于将其储蓄起来并在明天消费所带来的期望边际效用(经过时间偏好和利率的调整)。如果左边大于右边,意味着今天消费的价值更高,应增加当期消费(减少储蓄);反之亦然。这个方程是理解消费平滑行为的关键。
    \item \textbf{金融一体化与消费平滑}:
    \begin{itemize}
        \item \textbf{经济学解释}:在一个封闭经济或金融市场不发达的经济中,消费往往高度依赖于当期收入。金融一体化(即国家可以自由地在国际金融市场上借贷)打破了这种刚性联系。当面临暂时的负面收入冲击时,国家可以通过借款来维持消费水平,避免消费大幅下降;当面临暂时的正面收入冲击时,可以将一部分额外收入储蓄起来(贷给外国)用于未来消费。这就是所谓的\textbf{消费平滑 (consumption smoothing)}。
        \item \textbf{图示}:通常用无差异曲线和预算线来表示。金融一体化相当于给了国家一条更优的(或至少不差于自给自足时的)预算线,使其能够达到更高的无差异曲线。
    \end{itemize}
    \item \textbf{引入政府与李嘉图等价}:
    \begin{itemize}
        \item \textbf{经济学解释}:李嘉图等价定理指出,在某些严格假设下(如政府和私人部门面临相同的利率、具有相同的无限期界、存在完全的资本市场、税收是一次性总额税等),政府融资方式(发行债券还是征税)的改变不会影响总需求和利率。因为理性的私人部门会预期到政府债务最终需要通过未来税收来偿还,因此政府减税并发行债券,私人部门会增加储蓄以应对未来的税收增加,总储蓄和总需求不变。在开放经济背景下,这意味着政府的财政赤字(如果符合李嘉图等价条件)不一定会直接导致经常账户恶化。然而,现实中这些假设往往不成立,因此李嘉图等价的经验证据并不充分。
    \end{itemize}
    \item \textbf{内生生产与投资决策}:
    \begin{itemize}
        \item \textbf{经济学解释}:模型可以扩展到包含生产和投资。企业会根据资本的边际产出 ($MPK$) 与资本的使用成本(实际利率$r$加折旧率$\delta$)进行比较来决定最优投资水平,直到 $A_t F'(K_t) = r + \delta$ 。这意味着,当预期投资回报率高于成本时,国家会增加投资,这可能需要通过经常账户赤字从国外融资。
    \end{itemize}
    \item \textbf{无限期界模型与永久性收入假说}:
    \begin{itemize}
        \item \textbf{经济学解释}:将两期模型推广到无限期界,其核心结论与米尔顿·弗里德曼的永久性收入假说类似:消费并非取决于当期收入,而是取决于\textbf{永久性收入}(或财富,即所有未来预期收入的现值)。因此,只有对永久性收入产生影响的冲击(永久性冲击)才会导致消费水平的持久性改变。暂时性收入冲击主要通过储蓄/借贷(即经常账户)来平滑,对当期消费影响较小。
        \item \textbf{数学表达}:
        \[
        CA_t \approx (Y_t - \tilde{Y}_t) - (I_t - \tilde{I}_t)
        \]
        其中 $\tilde{X}$ 表示变量X的永久性水平。这意味着,如果当期产出高于其永久性水平(暂时性利好),国家会储蓄(CA改善);如果当期投资高于其永久性水平(如暂时性的高投资机会),国家会借款(CA恶化)。
    \end{itemize}
    \item \textbf{两国一般均衡与世界利率决定}:
    \begin{itemize}
        \item \textbf{经济学解释}:在一个由多个国家组成的世界经济中,全球实际利率由全球总储蓄供给和全球总投资需求共同决定。一个国家如果储蓄倾向较高(储蓄曲线右移)或投资需求较低(投资曲线左移),会倾向于成为资本净输出国,压低世界利率,并可能导致其他国家出现经常账户赤字。例如,“储蓄过剩” (savings glut) 假说就认为某些国家(如东亚国家)的高储蓄是导致全球低利率和某些国家(如美国)经常账户赤字的原因之一。
    \end{itemize}
    \item \textbf{外部债务可持续性}:
    \begin{itemize}
        \item \textbf{经济学解释}:一国持续的经常账户赤字会导致外债积累。债务可持续性分析关注的是一国在不违约的情况下,能够维持其外债水平(通常以占GDP的比例衡量)的能力。关键在于未来能否产生足够的贸易盈余来支付利息和本金。债务国需要满足跨期预算约束,即其初始净债务加上未来所有贸易逆差的现值,不能超过未来所有贸易顺差的现值。国家的增长率($g$)、实际利率($r$)、以及未来产生贸易顺差的能力是决定债务可持续性的核心因素。如果$r > g$,债务占GDP的比例在没有足够贸易顺差的情况下会趋于发散。
    \end{itemize}
\end{itemize}

\textbf{经济学发展逻辑 (从账户到行为):}
\begin{itemize}
    \item \textbf{为何从BoP到跨期模型?}
    \begin{itemize}
        \item \textbf{解释}:Class 1  中对BoP的介绍只是一个事后核算的框架,它告诉我们发生了什么(例如,一个国家出现了经常账户赤字),但它并没有解释这些交易背后的经济动因。跨期模型  通过引入理性经济主体的跨期优化行为,为经常账户的决定提供了一个基于福利最大化的理论基础。它帮助我们理解国家(或其代表性居民)是如何在面临不同时期的收入和投资机会时,通过国际借贷来平滑消费路径,从而最大化其总体福利。这种从描述性核算到行为模型的转变,是经济学分析深化的体现,使得我们不仅能观察现象,还能解释现象背后的逻辑,并评估其合意性。
    \end{itemize}
\end{itemize}

\textbf{练习/思考方向:}
\begin{itemize}
    \item \textbf{分析暂时性与永久性产出冲击对一国消费、储蓄和经常账户的影响。}
    \begin{itemize}
        \item \textbf{解释}:这个练习的核心在于检验学生对消费平滑和永久性收入假说的理解。一个暂时性的负面产出冲击(如自然灾害导致当年农业减产),根据理论,国家应通过借款(CA恶化)来维持消费水平,因为其永久性收入并未显著改变。而一个永久性的负面产出冲击(如一种主要出口资源永久性枯竭),则会导致永久性收入下降,消费水平也必须相应下调,经常账户可能不会出现大幅度的持续恶化(除非初始消费调整不足)。
    \end{itemize}
    \item \textbf{运用欧拉方程解释利率变化如何影响消费路径。}
    \begin{itemize}
        \item \textbf{解释}:欧拉方程 $u'(C_1) = \beta(1+r)u'(C_2)$  表明,实际利率 $r$ 的上升,使得未来的消费相对更便宜(或者说,今天的储蓄回报更高)。这会产生替代效应(鼓励减少当期消费,增加储蓄)和收入效应(如果国家是净债权人,利息收入增加,可能增加消费;如果是净债务人,利息支出增加,可能减少消费)。这个练习要求学生分析这些效应的相对大小,以及它们如何共同作用于消费的时间路径。
    \end{itemize}
    \item \textbf{讨论李嘉图等价在何种条件下成立或不成立。}
    \begin{itemize}
        \item \textbf{解释}:这个问题的目的是让学生理解理论模型的假设及其局限性。李嘉图等价的成立条件非常苛刻,例如,如果存在流动性约束(某些消费者无法借款)、短视行为(消费者不完全理性地预期未来税收)、税收扭曲(税收不是一次性总额税而是对劳动收入或资本收入征税)、或者政府与私人部门面临的利率不同,则李嘉图等价很可能不成立。理解这些条件有助于判断在具体情境下,财政政策是否会对经常账户产生影响。
    \end{itemize}
    \item \textbf{评估一国外部债务水平的可持续性。}
    \begin{itemize}
        \item \textbf{解释}:这需要学生综合运用跨期预算约束、国家增长前景、实际利率水平、以及未来调整贸易账户以产生顺差的能力等因素进行分析。没有一个简单的阈值可以判断债务是否可持续,关键在于国家是否有能力和意愿在未来履行其偿债义务。例如,一个增长迅速且能够有效利用借入资金进行生产性投资的国家,其债务可持续性可能较强。
    \end{itemize}
\end{itemize}

\subsection*{二、相对价格的重要性:实际汇率与贸易条件 (Class 3)}

\textbf{核心内容:}
\begin{itemize}
    \item \textbf{实际汇率 (RER, Q) 定义与计算}:
    \begin{itemize}
        \item \textbf{经济学解释}:实际汇率衡量的是两国产品和服务的相对价格,或者说两国货币的实际购买力之比。它是决定一国国际竞争力的关键变量。当一国实际汇率升值($Q$下降,假设$Q$定义为 $EP^*/P$ ),其产品相对于外国产品变得更贵,出口竞争力下降,进口品相对便宜,可能导致贸易逆差扩大。实际汇率的变动反映了名义汇率变动和国内外通货膨胀差异的综合影响。
        \item \textbf{思考}:为何仅仅关注名义汇率不足以判断一国的国际竞争力?
    \end{itemize}
    \item \textbf{可贸易品与不可贸易品模型}:
    \begin{itemize}
        \item \textbf{经济学解释}:许多产品和服务(如理发、餐饮、住房)是不可在国际间贸易的。这些不可贸易品的价格主要由国内供求决定。实际汇率水平很大程度上受到不可贸易品相对价格(相对于可贸易品价格,或相对于外国不可贸易品价格)的影响。如果一国不可贸易品价格相对于可贸易品价格上涨,其整体价格水平$P$会上升,导致实际汇率升值($Q$下降)。
        \item \textbf{数学表达}:消费篮子
        \[
        C_t = [(\gamma)^{\eta}(C_t^T)^{1-\eta}+(1-\gamma)^{\eta}(C_t^N)^{1-\eta}]^{\frac{1}{1-\eta}}
        \]
        整体价格指数 $P_t$ 是 $P_t^T$ (通常设为1或由国际市场决定) 和 $P_t^N$ 的函数。实际汇率 $Q_t = P_t^*/P_t$ (若外国价格水平 $P_t^*$ 主要由可贸易品决定)。
    \end{itemize}
    \item \textbf{RER动态对经常账户的影响}:
    \begin{itemize}
        \item \textbf{经济学解释}:实际汇率的\textbf{预期变动路径}(而非仅仅是其当前水平)会通过影响跨期消费决策来影响经常账户。例如,如果预期未来本币会实际升值(即未来本国商品相对更贵),可能会激励当期增加对本国商品的消费(特别是可贸易品,因为未来购买成本更高),从而可能导致当期经常账户恶化。这涉及到复杂的替代效应和收入效应。
    \end{itemize}
    \item \textbf{贸易条件 (ToT) 定义与影响}:
    \begin{itemize}
        \item \textbf{经济学解释}:贸易条件是一国出口产品价格指数与进口产品价格指数的比率。ToT的改善(如出口品价格上涨或进口品价格下跌)意味着该国可以用同样数量的出口换取更多的进口,其实际收入增加。这会影响消费、储蓄和投资决策,进而影响经常账户。例如,ToT的暂时性改善可能会被视为暂时性收入增加,导致储蓄增加和CA改善;而永久性改善则可能导致消费和投资都增加。
        \item \textbf{关键问题}:贸易条件冲击(如石油价格上涨对石油出口国和进口国)如何影响各国的经常账户和福利?
    \end{itemize}
    \item \textbf{购买力平价 (PPP)}:
    \begin{itemize}
        \item \textbf{经济学解释}:
        \begin{itemize}
            \item \textbf{绝对PPP}:
            \[
            P_t = E_t P_t^*
            \]
            意味着一价定律对所有商品都成立,并且两国消费篮子结构相同。如果绝对PPP成立,则实际汇率 $Q=1$。现实中,由于存在不可贸易品、贸易成本、关税、以及消费篮子差异,绝对PPP很少成立。
            \item \textbf{相对PPP}:
            \[
            \Delta \% e_t \approx \Delta \% p_t - \Delta \% p_t^*
            \]
            即名义汇率的变动率约等于两国通货膨胀率之差。这意味着实际汇率是平稳的,围绕某个均值波动,即使其不等于1。相对PPP在长期内比绝对PPP有更强的经验支持。
        \end{itemize}
    \end{itemize}
    \item \textbf{Balassa-Samuelson效应 (实际汇率与生产率)}:
    \begin{itemize}
        \item \textbf{经济学解释}:该效应解释了为何富裕国家(通常其可贸易品部门生产率相对较高)的实际汇率往往倾向于升值,或者说其物价水平(换算成同种货币后)往往高于贫穷国家。逻辑如下:
        \begin{enumerate}
            \item 可贸易品价格由国际市场决定(一价定律近似成立)。
            \item 富裕国家可贸易品部门的生产率 ($A_T$) 增长较快,导致该部门工资上涨。
            \item 由于劳动力可以在部门间流动,不可贸易品部门 ($A_N$) 的工资也随之上涨。
            \item 不可贸易品部门的生产率增长通常慢于可贸易品部门,因此工资上涨会导致不可贸易品价格 ($P_N$) 上涨。
            \item 由于整体价格水平$P$是 $P_T$ 和 $P_N$ 的加权平均,$P_N$ 的上涨导致$P$上涨,从而在名义汇率不变或变动不大的情况下,实际汇率升值($Q = EP^*/P$ 下降)。
        \end{enumerate}
        \item \textbf{数学表达}(简化,两国生产率差异与不可贸易品相对价格差异的关系):
        \[
        \ln\left(\frac{P_N}{P_N^*}\right) \approx \ln\left(\frac{A_T}{A_T^*}\right) - \ln\left(\frac{A_N}{A_N^*}\right)
        \]
        (假设工资在两国可贸易品部门被拉平,且劳动份额相似)。
    \end{itemize}
\end{itemize}

\textbf{经济学发展逻辑 (从单一商品到多商品世界):}
\begin{itemize}
    \item \textbf{为何从跨期模型到RER与ToT?}
    \begin{itemize}
        \item \textbf{解释}:Class 2  的跨期模型在分析消费平滑和经常账户决定时,往往为了简化,假设了一个单一的同质化商品,或者说所有商品的相对价格保持不变。然而,现实世界的国际贸易和资本流动是在多种不同商品和服务的背景下发生的。这些商品之间的相对价格——即\textbf{实际汇率}(衡量国内外商品篮子的相对价格)和\textbf{贸易条件}(衡量出口商品与进口商品的相对价格)——对一国的贸易竞争力、国民收入、资源配置(例如在可贸易品部门和不可贸易品部门之间的配置)以及最终的福利都有至关重要的影响。引入RER和ToT,使得模型能够更细致地刻画这些相对价格变动如何通过影响贸易量、收入效应和替代效应来改变经常账户的动态,并解释诸如荷兰病、Balassa-Samuelson效应等重要的开放经济现象。这标志着从一个高度抽象的理论框架向一个更具经验相关性的框架的过渡。
    \end{itemize}
\end{itemize}

\textbf{练习/思考方向:}
\begin{itemize}
    \item \textbf{分析一国不可贸易品部门生产率提高对RER的影响。}
    \begin{itemize}
        \item \textbf{解释}:根据Balassa-Samuelson效应的逻辑 ,如果一国不可贸易品部门生产率 ($A_N$) 提高,在可贸易品部门生产率 ($A_T$) 和工资不变的情况下,不可贸易品的价格 ($P_N$) 倾向于下降(因为可以用更少的投入生产同样多的不可贸易品)。这将导致整体价格水平$P$下降,在名义汇率$E$和外国价格水平 $P^*$ 不变时,实际汇率$Q$会贬值($Q=EP^*/P$ 上升)。这与 $A_T$ 提高导致RER升值的情况相反,突出了生产率冲击发生的部门的重要性。
    \end{itemize}
    \item \textbf{讨论贸易条件冲击(如进口原材料价格上涨)如何通过影响国民收入和相对价格来改变经常账户。}
    \begin{itemize}
        \item \textbf{解释}:进口原材料价格上涨意味着贸易条件恶化。这首先会直接减少国民收入(因为需要用更多的出口去交换同样数量的进口原材料)。如果这种冲击被认为是暂时性的,国家可能会通过借款(CA恶化)来平滑消费。如果被认为是永久性的,则消费水平必须下调。此外,原材料价格上涨也会改变国内生产成本和最终产品的相对价格,可能影响出口竞争力和进口需求,进一步作用于经常账户。
    \end{itemize}
    \item \textbf{解释为何PPP在短期内往往不成立,但在长期内可能是衡量汇率均衡水平的基准。}
    \begin{itemize}
        \item \textbf{解释}:短期内,由于存在价格粘性(商品价格调整缓慢)、不可贸易品、贸易壁垒、市场分割以及资产市场扰动等因素,一价定律难以对所有商品成立,导致PPP偏离。然而,在长期内,如果贸易壁垒较低且资本可以自由流动,套利活动会趋向于拉平各国同类可贸易品的价格(经汇率调整后)。同时,长期的货币政策差异最终会反映在通货膨胀差异上,名义汇率的调整会部分抵消这种通胀差异,使得实际汇率趋向于一个相对稳定的水平,或者说向其PPP均值回归。因此,PPP常被用作判断名义汇率是否被高估或低估的长期基准之一。
    \end{itemize}
    \item \textbf{运用Balassa-Samuelson效应解释为何富裕国家的物价水平通常更高。}
    \begin{itemize}
        \item \textbf{解释}:如前所述,富裕国家通常在可贸易品部门有更高的生产率。这推高了整个经济的工资水平。由于不可贸易品部门的生产率增长相对较慢,更高的工资成本就转化为更高的不可贸易品价格。由于不可贸易品在整体消费篮子中占有相当大的比重,这就使得富裕国家的整体物价水平(用同一种货币衡量)显得更高。例如,在纽约理发比在新德里理发要贵得多,即使两国理发师的技能可能相差无几,这很大程度上就是因为纽约的总体工资水平和不可贸易品(如房租、本地服务)价格更高。
    \end{itemize}
\end{itemize}

\subsection*{三、深化金融市场分析:不确定性与摩擦 (Class 4 \& 5)}

\textbf{核心内容 (Class 4 - 不确定性下的跨期配置):}
\begin{itemize}
    \item \textbf{预防性储蓄 (Precautionary Savings)}:
    \begin{itemize}
        \item \textbf{经济学解释}:当经济主体面临未来收入或其他经济变量的不确定性,并且其效用函数表现出“审慎”特性(即三阶导数大于零,如CRRA效用函数),他们会倾向于增加当前的储蓄,以应对未来可能出现的不利冲击。这种额外的储蓄被称为预防性储蓄。其直觉是,在不确定性下,未来消费的边际效用的期望值会因为消费的潜在波动而增加(由于边际效用曲线是凸的,即 $u''' > 0$),这使得当期储蓄(即未来消费)更具吸引力。
        \item \textbf{关键问题}:不确定性的存在如何改变标准的跨期消费-储蓄决策?
    \end{itemize}
    \item \textbf{完全资产市场 (Complete Asset Markets) 与风险分担}:
    \begin{itemize}
        \item \textbf{经济学解释}:完全资产市场是指存在足够多的金融工具(理论上是对应未来每一种可能状态的Arrow-Debreu或有证券),使得经济主体可以针对任何特定的未来风险进行精确的保险或对冲。在这样的市场中,理想情况下,所有个体特质性风险 (idiosyncratic risk) 都可以被完全分散掉,经济主体只共同承担不可分散的系统性风险 (aggregate risk)。这意味着,在均衡状态下,不同国家(或个人)的消费增长率应该高度相关,甚至趋同(调整初始财富差异后,其边际效用之比应等于实际汇率或状态价格之比)。
        \item \textbf{数学表达}:关键结论是
        \[
        \frac{u'(C_H(k))}{u'(C_F(k))} = \text{常数}
        \]
        或
        \[
        \frac{u'(C_t^H)/P_t^H}{u'(C_t^F)/P_t^F} S_t = \Gamma_t
        \]
        这表明各国消费的边际效用(经价格和汇率调整)的比率是固定的,实现了有效的风险分担。
        \item \textbf{重要启示}:完全市场下的风险分担并不意味着消费水平的均等化,而是消费对冲击反应的同步化。富国依然富,穷国依然穷,但他们面临的消费增长路径的相对波动性会减小。
    \end{itemize}
    \item \textbf{国际证券组合选择与本土偏好之谜 (Home Bias in Portfolio)}:
    \begin{itemize}
        \item \textbf{经济学解释}:标准的国际资产组合理论(如CAPM的国际版本)预测,投资者为了分散风险,应该持有按市值加权的全球市场组合。然而,实证研究反复发现,各国投资者持有的本国资产(股票、债券)的比例远远超过其本国市场在全球市场中的份额,这种现象被称为“本土偏好”。
        \item \textbf{可能解释}:信息不对称(国内投资者更了解本国资产)、交易成本(跨境投资成本更高)、对冲本国特有风险的需求(如对冲本国非贸易品价格通胀风险、或与本国劳动力收入负相关的资产)、规制障碍、以及行为金融因素等。课程中提到,如果国内资产能在投资者消费较低(边际效用较高)时提供较高回报,或者能有效对冲劳动力收入风险(例如,当生产率提高导致股息增加而工资收入下降时,持有本国股票就是好的对冲),则可能产生本土偏好。然而,标准模型要内生地生成显著的本土偏好并非易事。
    \end{itemize}
\end{itemize}

\textbf{核心内容 (Class 5 - 金融摩擦):}
\begin{itemize}
    \item \textbf{主权债务违约 (Sovereign Default)}:
    \begin{itemize}
        \item \textbf{经济学解释}:主权国家(政府)无法像私人部门那样被强制执行债务偿还。因此,主权债务违约是一个基于“偿付意愿”而非仅仅是“偿付能力”的决策。国家会权衡按时偿债的效用与违约的效用。违约的好处是暂时摆脱了债务负担,可以将资源用于国内消费或投资。违约的成本可能包括:未来在国际市场上融资变得更加困难和昂贵(声誉损失、市场禁入)、遭受贸易制裁、国内金融体系因持有违约债券而遭受冲击、甚至国内产出因信贷中断而下降。
        \item \textbf{违约决策}:当债务负担过重,经济状况恶劣,且违约的预期成本低于继续偿债的痛苦时,国家可能选择违约。违约概率是内生的,受债务水平、产出波动、全球利率、以及违约惩罚机制等因素影响。
    \end{itemize}
    \item \textbf{风险溢价 (Risk Premium) 的内生决定}:
    \begin{itemize}
        \item \textbf{经济学解释}:由于存在违约风险,贷款给主权国家的投资者会要求一个风险溢价,即该国发行的债券利率会高于同期限的无风险利率(如美国国债利率)。风险溢价的大小反映了市场对该国违约概率和违约时预期损失(即回收率的倒数)的评估。违约概率、债务水平和风险溢价是相互影响、内生决定的。更高的债务水平可能提高违约概率,从而推高风险溢价,使得再融资成本上升,进一步恶化债务状况,形成恶性循环。
        \item \textbf{数学关系}(简化):
        \[
        1+r = (1-\pi)(1+r_s) + \pi z (1+r_s)
        \]
        其中 $r$ 是无风险利率, $r_s$ 是主权债券利率,$\pi$ 是违约概率,$z$ 是违约时的回收率。风险溢价可以看作是 $r_s - r$。
    \end{itemize}
    \item \textbf{或有资产下的违约与不完全保险}:
    \begin{itemize}
        \item \textbf{经济学解释}:即使存在或有资产(如与某种商品价格或GDP挂钩的债券),如果合约规定的支付在某些状态下超过了国家愿意或能够支付的上限(例如,超过了其可被债权人扣押的资产或产出份额),国家仍可能选择部分或全部违约。这导致即使在理论上可以通过或有合约实现完全保险的情况下,由于强制执行的限制,实际的保险程度也是不完全的。通常表现为,在极端不利的状态下,国家无法获得足额的保险赔付。
    \end{itemize}
    \item \textbf{国际投资中的道德风险 (Moral Hazard)}:
    \begin{itemize}
        \item \textbf{经济学解释}:当贷款方(如国际投资者)无法完全观察或控制借款方(如一国政府或企业)在获得资金后的行为时,就可能产生道德风险。例如,借款方可能将资金用于风险过高或回报过低的项目,或者努力程度不足,因为如果项目成功,收益主要归自己,如果失败,损失则部分或全部由贷款方承担。这种激励不相容会导致投资效率低下(投资不足或过度冒险)和信贷配给。
        \item \textbf{例子}:一国从国际组织获得贷款用于特定发展项目,但政府可能将部分资金挪作他用,或在项目执行中不尽力。为减轻道德风险,贷款合约中通常会包含一些监督条款、激励机制或要求提供抵押品。
    \end{itemize}
\end{itemize}

\textbf{经济学发展逻辑 (从完美市场到现实复杂性):}
\begin{itemize}
    \item \textbf{为何从无摩擦市场到引入不确定性与摩擦?}
    \begin{itemize}
        \item \textbf{解释}:前几讲建立的理论模型(如跨期消费模型、RER模型)大多是在一个理想化的、确定性的、且金融市场运转完美的框架下进行的。然而,现实世界充满了不确定性(如收入波动、政策突变、自然灾害),并且金融市场远非完美(存在信息不对称、合约不完全、强制执行困难等问题)。Class 4 首先引入不确定性,探讨其如何从根本上改变经济主体的储蓄行为(产生预防性动机)和资产配置决策(需要进行风险管理和分散化)。这使得模型能够解释为何人们会储蓄超过简单生命周期模型所预测的水平,以及为何国际资本流动的一个重要功能是风险分担。Class 5 则更进一步,直面金融市场中的各种“摩擦”。这些摩擦是理解许多重要国际金融现象(如主权债务危机为何频繁发生、国际信贷为何会受到限制、为何发展中国家难以获得充足的外部融资)的关键。如果不考虑这些摩擦,我们将无法解释为何国际资本流动并不总是流向资本最稀缺(即边际回报率最高)的地方(卢卡斯之谜),也无法理解为何金融危机会对实体经济产生如此巨大的破坏。因此,引入不确定性和摩擦,是使宏观经济模型从象牙塔走向现实的关键一步,极大地增强了模型的解释力和政策含义。
    \end{itemize}
\end{itemize}

\textbf{练习/思考方向:}
\begin{itemize}
    \item \textbf{比较有无完全资产市场下,各国如何应对特质性冲击和系统性冲击。}
    \begin{itemize}
        \item \textbf{解释}:这个练习旨在深化对风险分担机制的理解。在完全资产市场下,各国可以通过买卖针对特定状态的或有证券来完全对冲掉本国特有的(特质性)冲击,从而使得本国消费不受这些冲击的影响(或者说,所有国家共同承担这些冲击的平均影响)。而对于系统性冲击(影响所有国家的冲击),则无法通过国家间的资产交易来分散,各国消费都会受到影响。在没有完全资产市场(或市场不完善)的情况下,国家对特质性冲击的缓冲能力会大大减弱,消费波动性会更高。
    \end{itemize}
    \item \textbf{分析导致“本土偏好”的可能原因。}
    \begin{itemize}
        \item \textbf{解释}:“本土偏好”是国际金融领域一个著名的“谜题”。这个练习促使学生思考标准金融理论的局限性,并探索可能导致这种偏离的现实因素。除了幻灯片中提到的对冲劳动力收入风险和粘性价格下的需求对冲,还可以考虑交易成本(跨境投资费用更高)、信息不对称(对本国公司更了解)、规制限制(如养老基金对海外投资比例的限制)、甚至国民情感等。
    \end{itemize}
    \item \textbf{构建一个简单的模型,说明一国在何种条件下会选择主权债务违约。}
    \begin{itemize}
        \item \textbf{解释}:这要求学生将违约决策形式化为一个成本收益分析。例如,可以设想一个两期模型,国家在第一期借债,在第二期面临偿还或违约的选择。违约的收益是免除债务,成本是未来无法再借款(或产出下降)。学生需要设定效用函数和概率分布(如果产出不确定),然后找出使得违约效用大于偿债效用的临界债务水平或产出水平。
    \end{itemize}
    \item \textbf{讨论道德风险如何导致次优的国际投资水平。}
    \begin{itemize}
        \item \textbf{解释}:道德风险源于信息不对称和激励不相容。在国际投资中,如果投资者无法监督借款国的资金使用效率或努力程度,借款国可能会采取机会主义行为。这会导致投资者在事前就会预期到这种风险,从而要求更高的回报率,或者干脆减少投资额度。结果是,一些本应获得融资的、具有正净现值的项目可能无法获得融资,或者融资成本过高,导致全球资本配置效率低下。
    \end{itemize}
\end{itemize}

\subsection*{四、名义变量与短期波动:汇率的决定与政策含义 (Class 6, 7, 8)}

\textbf{核心内容 (Class 6 - 弹性价格下的汇率):}
\begin{itemize}
    \item \textbf{名义汇率市场特征}:
    \begin{itemize}
        \item \textbf{经济学解释}:外汇市场是全球规模最大、流动性最高的金融市场。其特点包括:巨额的日交易量(数万亿美元),以美元为主导的交易货币,以及多种交易工具(即期、远期、掉期)。理解这些市场特征有助于把握汇率决定的现实背景。
    \end{itemize}
    \item \textbf{利率平价 (Interest Rate Parity)}:
    \begin{itemize}
        \item \textbf{有抵补利率平价 (CIP)}:
        \[
        1+i_t^H = (1+i_t^F) \frac{F_t}{E_t}
        \]
        \textbf{经济学解释}:CIP指出,在没有交易成本和风险的情况下,通过远期合约锁定未来汇率的本国投资回报率应等于外国投资回报率。如果存在偏离,则存在无风险套利机会。近年来,由于银行监管加强和融资约束等因素,CIP出现持续偏离,这本身也成为重要的研究课题。
        \item \textbf{无抵补利率平价 (UIP)}:
        \[
        1+i_t^H = (1+i_t^F) \frac{E[E_{t+1}]}{E_t}
        \]
        \textbf{经济学解释}:UIP指出,如果投资者是风险中性的,那么本国投资的预期回报率(考虑预期汇率变动)应等于外国投资的预期回报率。这意味着,高利率货币预期会贬值,低利率货币预期会升值。UIP是开放经济宏观模型中一个核心的(尽管在经验上经常被拒绝的)资产市场均衡条件。其偏离通常归因于风险溢价的存在。
    \end{itemize}
    \item \textbf{货币需求与货币市场均衡对汇率的影响}:
    \begin{itemize}
        \item \textbf{经济学解释}:名义汇率作为两国货币的相对价格,其决定必然与各自的货币供求状况相关。货币需求通常取决于交易需求(与收入或消费正相关)和资产需求(与持有货币的机会成本——名义利率负相关)。在货币主义的汇率模型中(假设PPP持续成立且价格弹性),一国货币供给的相对增加(相对于货币需求)会导致其货币贬值。
        \item \textbf{数学表达}(简化,对数形式):
        \[
        m_t - p_t = \phi y_t - \lambda i_t
        \]
        结合PPP ($p_t = e_t + p_t^*$) 和UIP ($i_t = i_t^* + E_t[e_{t+1}] - e_t$) 可以推导出汇率的动态方程。
    \end{itemize}
    \item \textbf{汇率的远期资产价格性质:预期驱动}:
    \begin{itemize}
        \item \textbf{经济学解释}:由于UIP的存在,当前的汇率不仅取决于当前的经济基本面(如货币供给、利率、产出),还取决于对未来汇率的预期。这意味着汇率具有典型的资产价格特征:它是高度前瞻性的 (forward-looking),并且容易受到“消息” (news) ——即关于未来基本面预期的改变——的影响而发生跳跃。任何影响未来货币政策、通货膨胀或经济增长预期的信息,都会通过改变 $E_t[e_{t+1}]$ 而立即反映在当前汇率 $e_t$ 上。
        \item \textbf{数学结果}:
        \[
        e_t = \frac{1}{1+\lambda} \sum_{s=t}^{\infty} \left(\frac{\lambda}{1+\lambda}\right)^{s-t} E_t[m_s - (\phi y_s - \lambda i_s^* - p_s^* )]
        \]
        即汇率是未来所有“基本面驱动因素”(括号内项)现值的线性组合。
    \end{itemize}
\end{itemize}

\textbf{核心内容 (Class 7 - 货币危机与外汇干预):}
\begin{itemize}
    \item \textbf{第一代危机模型 (Krugman, 1979)}:
    \begin{itemize}
        \item \textbf{经济学解释}:这类模型强调货币危机源于政府\textbf{不可持续的宏观经济政策}(通常是持续的财政赤字,迫使央行进行货币化融资,即国内信贷扩张)与\textbf{固定汇率制度}之间的根本矛盾。在固定汇率下,国内信贷扩张会导致外汇储备持续流失。当储备下降到某个临界水平时,理性的投机者会预期到汇率制度即将崩溃,从而发起集中抛售本币、抢购外汇的投机性攻击,耗尽剩余储备,迫使央行放弃固定汇率。危机的爆发时点是可预测的(尽管对个体投机者而言可能存在协调问题),危机本身是政策失当的必然结果。
    \end{itemize}
    \item \textbf{第二代危机模型 (Obstfeld, 1994, 1996)}:
    \begin{itemize}
        \item \textbf{经济学解释}:这类模型引入了\textbf{政府在维持汇率承诺与追求其他国内政策目标(如降低失业率、赢得选举)之间的权衡},并强调了\textbf{自我实现预期 (self-fulfilling prophecies)} 和\textbf{多重均衡}的可能性。即使政府的宏观政策本身是可持续的,如果市场参与者普遍预期政府会因无法承受维持固定汇率的成本(如高利率导致的经济衰退)而放弃汇率承诺,这种预期本身就可能通过迫使政府提高利率来捍卫汇率,从而真的使维持汇率的成本变得过高,最终导致政府放弃。因此,危机的爆发可能并非必然,而是取决于市场情绪和预期的“传染”。基本面(如失业率、政府的政治意愿)决定了是否存在多重均衡的“脆弱区域”。
    \end{itemize}
    \item \textbf{第三代危机模型}:
    \begin{itemize}
        \item \textbf{经济学解释}:这类模型关注的是\textbf{金融体系的脆弱性}(特别是银行部门的资产负债表错配,如期限错配、货币错配)与货币危机之间的联系,常被称为“孪生危机”(即银行危机和货币危机同时或相继发生)。例如,如果国内银行大量借入外币短期债务,并投资于本币长期资产,一旦发生资本外流或本币贬值预期,银行可能面临流动性危机和偿付能力危机。政府为救助银行体系而进行的扩张性货币政策或财政支出,可能耗尽外汇储备或破坏对固定汇率的信心,从而触发货币危机。这类模型很好地解释了1990年代末的亚洲金融危机。
    \end{itemize}
    \item \textbf{外汇干预 (Foreign Exchange Intervention)}:
    \begin{itemize}
        \item \textbf{经济学解释}:中央银行通过买卖外汇来影响汇率的行为。\textbf{冲销式干预}指央行在买入(卖出)外汇的同时,通过卖出(买入)本币债券来抵消其对国内基础货币的影响,目的是在不改变货币政策立场的情况下影响汇率。其有效性存在争议。理论上,如果资产具有不完全替代性(UIP不成立),冲销式干预可以通过改变国内外债券的相对供给来影响风险溢价,进而影响汇率(\textbf{投资组合平衡渠道})。此外,干预也可能传递央行未来货币政策意图的信号(\textbf{信号渠道})。实证证据表明,大规模、公开宣布、且与其他政策协调一致的干预,在特定情况下(如市场失序时)可能有效,但其效果通常是短暂的。
    \end{itemize}
\end{itemize}

\textbf{核心内容 (Class 8 - 粘性价格下的汇率:蒙代尔-弗莱明与超调):}
\begin{itemize}
    \item \textbf{蒙代尔-弗莱明 (MF) 模型}:
    \begin{itemize}
        \item \textbf{经济学解释}:MF模型是分析开放经济短期宏观经济政策效果的经典IS-LM-BP框架。其核心假设是\textbf{价格粘性}(短期内价格水平固定不变,产出由总需求决定)和不同程度的\textbf{资本流动性}。
        \begin{itemize}
            \item \textbf{IS曲线}:代表产品市场均衡,描述了利率和产出之间的负相关关系(高利率抑制投资和消费,从而降低总需求和产出)。开放经济的IS曲线还受到净出口的影响,而净出口又取决于实际汇率(或名义汇率,在价格粘性下)和国内外收入水平。
            \item \textbf{LM曲线}:代表货币市场均衡,描述了利率和产出之间的正相关关系(高产出增加货币交易需求,在货币供给不变时需要更高的利率来平衡)。
            \item \textbf{IFM/BP曲线}:代表国际收支平衡(或在资本完全流动下,国内外利率通过UIP联系起来)。在资本完全流动和静态预期(或远期汇率钉住即期汇率)的简化下,IFM曲线是一条在国际利率水平上的水平线($i=i^*$)。
        \end{itemize}
        \item \textbf{政策有效性结论}:
        \begin{itemize}
            \item \textbf{固定汇率制度}:货币政策完全无效(任何试图改变货币供给的努力都会被央行为维持汇率而进行的外汇市场干预所抵消);财政政策非常有效(财政扩张导致利率上升压力,央行为维持汇率必须购入外汇、增加货币供给,从而强化财政扩张的效果)。
            \item \textbf{浮动汇率制度}:货币政策非常有效(货币扩张导致利率下降和本币贬值,贬值刺激净出口,从而增加产出);财政政策完全无效(财政扩张导致利率上升和本币升值,升值完全抵消了净出口,产出不变,即“完全挤出”)。
        \end{itemize}
    \end{itemize}
    \item \textbf{汇率超调 (Overshooting, Dornbusch, 1976)}:
    \begin{itemize}
        \item \textbf{经济学解释}:Dornbusch模型解释了在\textbf{价格粘性}和\textbf{理性预期}以及\textbf{UIP成立}的条件下,为何在受到货币冲击(如未预期的货币供给增加)后,名义汇率的短期反应会超过其长期均衡反应。
        \item \textbf{直觉逻辑}:
        \begin{enumerate}
            \item 货币供给永久性增加,长期内会导致价格水平同比例上升,名义汇率同比例贬值(根据相对PPP或货币数量论)。
            \item 短期内,由于商品价格具有粘性,价格水平不会立即调整。
            \item 货币供给增加导致货币市场超额供给,为恢复均衡,名义利率必须下降。
            \item 根据UIP,如果本国利率低于外国利率($i < i^*$),则市场必须预期本币未来会升值($E_t[e_{t+1}] < e_t$)。
            \item 为了使得汇率在经历了一个初始的、幅度超过长期贬值水平的“跳贬”之后,还能在后续过程中逐渐升值回到新的、但低于初始跳贬程度的长期均衡水平,这个初始的“跳贬”就必须“过度”。
        \end{enumerate}
        \item \textbf{相图分析}:通常用汇率水平和价格水平(或实际汇率)构成的动态系统来分析,存在唯一的鞍点稳定路径。货币冲击会导致系统瞬间跳到新的鞍点路径上,然后沿着该路径向新的长期均衡收敛。
    \end{itemize}
\end{itemize}

\textbf{经济学发展逻辑 (从实际变量到名义变量,从长期到短期):}
\begin{itemize}
    \item \textbf{为何从实际模型转向名义模型和短期分析?}
    \begin{itemize}
        \item \textbf{解释}:Class 2-5 主要构建了开放经济的实际模型,侧重于分析消费、储蓄、投资、实际汇率和贸易条件等实际变量的长期决定和跨期动态,价格通常被假定为完全弹性的,或者模型关注的是剔除通货膨胀影响后的实际量。然而,现实经济中,名义变量(如名义汇率、名义利率、货币供给、总体价格水平)的波动是宏观经济分析不可或缺的一部分,尤其是在短期。政策制定者主要通过调控名义工具(如利率、货币供给)来影响经济。Class 6 开始引入名义汇率的决定,但仍主要在弹性价格的框架下。Class 7 讨论了名义汇率制度选择带来的挑战(货币危机)。Class 8 则迈出了关键一步,明确引入了\textbf{价格粘性}这一核心的凯恩斯主义假设。价格粘性意味着短期内名义冲击(如货币政策的改变)会对实际产出和就业产生影响,这与弹性价格模型(货币中性)的结论显著不同。MF模型和Dornbusch的超调模型正是试图解释在这种名义刚性背景下,开放经济的短期动态以及政策的传导机制。这种从长期实际分析向短期名义分析的扩展,使得理论能够更好地解释商业周期波动和宏观政策的短期效应,是宏观经济学发展的重要方向。
    \end{itemize}
\end{itemize}

\textbf{练习/思考方向:}
\begin{itemize}
    \item \textbf{运用UIP解释为何一国利率上升(在其他条件不变时)会导致本币升值。}
    \begin{itemize}
        \item \textbf{解释}:根据UIP ($i_t^H \approx i_t^F + E_t[e_{t+1}] - e_t$),如果本国利率 $i_t^H$ 上升,而外国利率 $i_t^F$ 和对未来汇率的预期 $E_t[e_{t+1}]$ 不变,那么为了维持等式成立,当前汇率 $e_t$ 必须下降(即本币升值)。直觉上,本国利率上升使得本国资产相对更具吸引力,资本流入增加,导致对本币的需求上升,从而推动本币升值。
    \end{itemize}
    \item \textbf{比较第一代和第二代货币危机模型的关键区别。}
    \begin{itemize}
        \item \textbf{解释}:第一代模型将危机归咎于基本面(如财政赤字)与固定汇率之间的不一致,危机是“必然”的。第二代模型则强调即使基本面尚可,由于政府在维持汇率和国内目标之间的权衡,以及市场预期的自我实现,危机也可能爆发,存在“多重均衡”。前者是“基本面危机”,后者是“预期危机”或“逃逸条款危机”。
    \end{itemize}
    \item \textbf{分析在MF模型中,为何浮动汇率下财政政策无效而货币政策有效。}
    \begin{itemize}
        \item \textbf{解释}:在资本完全流动的浮动汇率MF模型中: (1) \textbf{财政政策无效}:财政扩张($G$增加)使IS曲线右移,在LM曲线不变时,有提高利率的压力。利率上升吸引资本流入,导致本币升值。本币升值使得净出口减少,IS曲线左移,完全抵消了最初财政扩张对总需求的刺激作用,产出回到初始水平。 (2) \textbf{货币政策有效}:货币扩张($M$增加)使LM曲线右移,利率下降。利率下降导致资本外流,本币贬值。本币贬值刺激净出口,IS曲线右移,产出增加。因此,货币政策通过利率和汇率两个渠道影响总需求。
    \end{itemize}
    \item \textbf{解释汇率超调发生的直觉原因:为何汇率短期反应会大于长期反应。}
    \begin{itemize}
        \item \textbf{解释}:当发生一次性未预期的货币供给永久性增加时:长期来看,价格会同比例上涨,实际货币余额不变,利率回到初始水平,汇率贬值到一个新的稳定水平。短期来看,价格是粘性的。货币供给增加导致利率需要下降以平衡货币市场。根据UIP,利率下降意味着本币未来预期会升值。要使得一个最终会贬值的货币在短期内预期会升值,它在冲击发生时必须立即“过度”贬值,贬值幅度超过其长期的稳定贬值幅度,这样它才有空间在之后逐步升值回到新的长期均衡点。这个“过度”的初始反应就是超调。金融市场(汇率)的调整速度远快于商品市场(价格)的调整速度是导致超调的关键。
    \end{itemize}
\end{itemize}

\subsection*{五、现代开放经济宏观:微观基础、福利与政策优化 (Class 9 \& 10)}

\textbf{核心内容 (Class 9 - 新开放经济宏观经济学 NOEM):}
\begin{itemize}
    \item \textbf{基于家庭和企业优化行为的微观基础模型}:
    \begin{itemize}
        \item \textbf{经济学解释}:NOEM模型试图为开放经济的短期波动和政策分析提供严格的微观基础。与MF模型依赖于特设的行为方程不同,NOEM模型中的总需求、总供给、消费、投资、劳动供给等关系都源于理性的、追求效用最大化(对家庭而言)或利润最大化(对企业而言)的经济主体的最优决策。这使得模型内部逻辑一致,参数具有更清晰的经济含义,并且能够进行规范性的福利分析。
        \item \textbf{典型结构}:通常包含两国,每国由代表性家庭和一系列具有垄断势力(因为产品存在差异化)的企业构成。家庭消费国内外产品构成的CES篮子,提供劳动,持有货币和债券。企业利用劳动生产差异化产品,并在面临需求曲线和成本约束的情况下设定价格。
    \end{itemize}
    \item \textbf{名义刚性 (Nominal Rigidities) 与不完全竞争 (Imperfect Competition)}:
    \begin{itemize}
        \item \textbf{经济学解释}:这是NOEM模型中产生短期实际效应的两个核心要素。名义刚性通常指企业调整价格的成本较高,或者存在菜单成本、信息不完全、长期合约等因素,导致价格在短期内是粘性的(预设的),不能对冲击做出即时调整。不完全竞争(通常是垄断竞争)意味着企业拥有一定的定价权,其产品价格会高于边际成本(存在加成markup),导致稳态产出水平低于社会最优(完全竞争)水平。这两个特征使得货币政策在短期内可以通过影响总需求来改变产出和就业,并可能通过纠正垄断扭曲来改善福利。
    \end{itemize}
    \item \textbf{汇率传递 (Exchange Rate Pass-through, ERPT) 的程度及其影响}:
    \begin{itemize}
        \item \textbf{经济学解释}:ERPT衡量的是名义汇率变动在多大程度上反映到进口商品价格上。
        \begin{itemize}
            \item \textbf{生产者货币定价 (PCP)}:出口商以本国货币设定出口价格,则汇率变动会完全传递到进口国的进口价格上(ERPT=1)。例如,美元贬值10\%,美国出口到欧洲的商品(以美元定价)的欧元价格会下降10\%。
            \item \textbf{当地货币定价 (LCP)}:出口商以进口国的货币设定出口价格,并根据当地市场情况调整,则汇率变动可能不会传递到进口价格上(ERPT=0),而是由出口商的利润边际吸收。例如,美元贬值10\%,但日本汽车在美国的美元售价可能保持不变(丰田公司承受了利润下降)。
            \item \textbf{不完全传递 (Incomplete Pass-through)}:现实中ERPT往往介于0和1之间。传递程度影响汇率变动对贸易量、贸易条件、国内通胀以及福利的实际效果。例如,如果传递不完全,本币贬值对刺激出口的作用会减弱,但对国内通胀的压力也较小。
        \end{itemize}
    \end{itemize}
    \item \textbf{短期与长期效应,经常账户的动态}:
    \begin{itemize}
        \item \textbf{经济学解释}:NOEM模型能够同时分析短期(价格粘性起主导作用)和长期(价格充分调整,实际因素更重要)的经济动态。短期内,货币冲击通过影响实际利率、实际汇率和总需求来改变产出、消费和就业。这些短期的实际效应会导致国家之间财富的重新分配(通过经常账户的变动,即净资产的积累或减少)。例如,一国货币扩张导致本币贬值,如果Marshall-Lerner条件成立且需求弹性较高(如课程中$\lambda > 1$),可能会导致贸易顺差和CA改善,从而在长期内积累更多的外国净资产。这会影响其长期的消费和福利水平。
    \end{itemize}
    \item \textbf{福利分析:“以邻为壑”或“以邻为壑自身”的可能性}:
    \begin{itemize}
        \item \textbf{经济学解释}:由于NOEM模型基于微观主体的效用函数,可以直接评估政策的福利效应。
        \begin{itemize}
            \item \textbf{“以邻为壑” (Beggar-thy-neighbor)}:指一国采取的政策(如竞争性贬值)以损害他国福利为代价来增进本国福利。例如,在不完全传递和特定参数条件下,本币贬值可能通过改善本国贸易条件(相对于产出成本而言)或抢夺他国需求来提升本国福利。
            \item \textbf{“以邻为壑自身” (Beggar-thyself)}:指一国政策虽然可能损害他国,但由于某些负面反馈效应(如本国贸易条件过度恶化导致实际收入下降),最终也可能损害本国自身的福利。
            \item \textbf{全球福利}:货币扩张通常可以通过缓解不完全竞争造成的产出过低问题来提高全球总福利,但其在国家间的分配则取决于汇率传递机制、需求弹性等多种因素。
        \end{itemize}
    \end{itemize}
\end{itemize}

\textbf{核心内容 (Class 10 - 随机模型中的最优政策):}
\begin{itemize}
    \item \textbf{在随机冲击环境下,如何设计最优货币政策以最大化社会福利}:
    \begin{itemize}
        \item \textbf{经济学解释}:这一部分从规范经济学的角度出发,探讨中央银行在面临外生冲击(如生产率冲击、需求冲击)时,应如何设定其政策工具(通常是货币供给或利率)以最大化代表性家庭的期望效用。核心思想是,最优政策应尽可能地抵消由市场不完美(如价格粘性、垄断竞争)和外生冲击所造成的扭曲,使经济运行尽可能接近其“有效”状态(即在没有这些扭曲和冲击的情况下的状态)。
    \end{itemize}
    \item \textbf{封闭经济与开放经济中的最优政策规则}:
    \begin{itemize}
        \item \textbf{封闭经济}:在价格粘性和垄断竞争下,最优货币政策通常旨在\textbf{稳定企业的边际成本},或者说\textbf{完全抵消生产率冲击对价格加成的影响},使得价格粘性变得无关紧要。例如,如果生产率$Z$提高,货币政策应相应调整名义总支出$\mu=PC$(如$\mu_k=Z_k$),以维持产出在“自然率”(无扭曲下的最优水平)附近,并消除价格设定的扭曲(即预设价格等于事后最优价格)。
        \item \textbf{开放经济}:情况更为复杂,因为还需要考虑汇率的波动以及国际间的溢出效应。最优政策规则取决于汇率传递机制(PCP, LCP, DCP - 主导货币定价)以及各国消费篮子中对本国和外国产品的偏好程度。
    \end{itemize}
    \item \textbf{汇率传递机制对最优政策和国际政策协调需求的影响}:
    \begin{itemize}
        \item \textbf{PCP (生产者货币定价)}:汇率变动完全传递到进口价格。各国独立采取类似封闭经济的最优政策(稳定经生产率调整的货币立场,如$\ln(\mu_k) = \ln(Z_k)$ \cite{1082})即可实现全球最优,无需明确的政策协调,因为汇率的灵活调整可以有效地隔离和吸收冲击,实现有效的资源配置。
        \item \textbf{LCP (当地货币定价)}:汇率变动不影响进口价格,也就是消费价格(以当地货币计价),只影响出口商的边际利润。此时,汇率失去了调节相对价格和贸易条件的作用。最优政策通常要求各国协调其货币政策,以稳定一个共同的全球货币立场(如$\ln(\mu_k) = \ln(\mu_k^*) = 0.5[\ln(Z_k) + \ln(Z_k^*)]$),这意味着名义汇率倾向于保持稳定(如果名义汇率无用,就不要让它波动)。
        \item \textbf{DCP (主导货币定价,如美元定价)}:例如,所有国际贸易都以美元定价。这种情况下,主导货币国(如美国)的货币政策对全球价格和产出有更大的影响。非主导货币国的汇率变动对其贸易条件的影响减弱。最优政策可能涉及非对称的反应,并且政策协调(或主导货币国承担更大的稳定责任)可能带来显著的福利改进。例如,幻灯片中的DCP(本币主导)下的最优政策是非对称的:$\ln(\mu_k) = 0.5[\ln(Z_k) + \ln(Z_k^*)]$ 而$\mu_k^* = Z_k^*$。合作下的政策则可能调整为$\ln(\mu_k) = \frac{2}{3}\ln(Z_k) + \frac{1}{3}\ln(Z_k^*)$。
        \item \textbf{政策协调的需求}:当一国政策对外国产生显著的溢出效应,且这种溢出效应与国内效应不同(即存在外部性),并且各国独立决策时不会充分考虑这种外部性,那么国际政策协调就可能改善全球福利。
    \end{itemize}
\end{itemize}

\textbf{经济学发展逻辑 (从特设关系到坚实基础,从实证描述到规范分析):}
\begin{itemize}
    \item \textbf{为何从MF模型到NOEM和最优政策?}
    \begin{itemize}
        \item \textbf{解释}:MF模型及其扩展(如Dornbusch超调模型)虽然为理解开放经济的短期动态和政策效应提供了有力的直觉和框架,但它们通常建立在一些\textbf{特设的行为方程}之上(例如,消费函数、投资函数、货币需求函数不是从微观主体的效用或利润最大化中严格推导出来的)。这使得这些模型在进行\textbf{福利分析}时缺乏坚实的基础(因为没有明确的社会福利函数或代表性家庭的效用函数),并且容易受到\textbf{卢卡斯批判}的质疑(即当政策规则改变时,人们的预期和行为方式也会改变,从而可能导致模型参数不稳定)。
        NOEM模型通过引入\textbf{微观基础}(即明确设定家庭的效用函数和企业的生产与定价行为),克服了这些局限性。这使得:
        \begin{enumerate}
            \item 模型的参数具有更清晰的结构性含义;
            \item 模型的预测和政策含义在面对政策规则变化时更为稳健;
            \item 最重要的是,可以直接从家庭的效用出发来评估不同政策的\textbf{福利后果}。
        \end{enumerate}
        Class 9 构建了这类NOEM模型的基本框架,分析了货币冲击在不同汇率传递机制下的短期和长期影响,并初步探讨了福利问题。Class 10 则更进一步,将NOEM模型置于\textbf{随机环境}中,并从\textbf{规范经济学 (normative economics)} 的角度出发,探讨在面临各种外生冲击(如生产率冲击)时,中央银行应该如何\textbf{主动地、前瞻性地}设定其货币政策,以期最大化社会福利。这种从解释经济现象(实证分析)到探寻理想政策(规范分析)的转变,是现代宏观经济学研究的一个核心特征,旨在为政策制定提供更为科学和有力的理论支持。
    \end{itemize}
\end{itemize}

\textbf{练习/思考方向:}
\begin{itemize}
    \item \textbf{比较PCP和LCP假设下,货币冲击对国内价格、进口价格和贸易条件的影响有何不同。}
    \begin{itemize}
        \item \textbf{解释}:这个练习旨在检验学生对不同汇率传递机制核心差异的理解。在PCP下,本币贬值会使本国出口商品的外币价格下降,进口商品的本币价格上升,贸易条件(出口价/进口价,以本币计)可能改善或恶化取决于具体情况,但相对价格变化显著。在LCP下,本币贬值主要影响出口商的利润边际,进口商品的当地货币价格不变,本国出口商品的外币价格也不变(由出口商承担汇率变动),贸易条件(以当地货币计)可能不变或变化很小,汇率对贸易量的调节作用减弱。
    \end{itemize}
    \item \textbf{在NOEM框架下,分析一国货币扩张的福利效应,并讨论是否存在国际溢出效应。}
    \begin{itemize}
        \item \textbf{解释}:一国货币扩张,在价格粘性和垄断竞争下,通常能刺激本国需求,提高产出,缓解产出过低的扭曲,从而提高本国福利。然而,其对外国的溢出效应则不确定:如果导致本币大幅贬值并改善了本国贸易条件(以牺牲他国为代价),则可能是“以邻为壑”;如果也刺激了对外国产品的需求,或者通过改善全球总需求使他国受益,则可能是正溢出。溢出效应的大小和方向取决于模型参数,如商品替代弹性、汇率传递程度等。
    \end{itemize}
    \item \textbf{解释为何在某些汇率传递机制下(如LCP),即使各国独立制定最优货币政策,结果也可能接近于合作解,而在另一些机制下(如DCP)则不然。}
    \begin{itemize}
        \item \textbf{解释}:国际政策协调的必要性取决于是否存在显著的、未被各国独立决策充分内化的\textbf{跨国外部性}。在LCP下,由于汇率变动不影响贸易商品的当地货币价格,汇率对贸易条件的调节作用非常有限。各国货币政策主要影响本国名义总需求和价格水平。如果各国的目标是稳定一个共同的全球实际变量(如经生产率调整的产出缺口),且其政策工具(货币供给)对本国经济的影响远大于对外国的影响(即溢出效应相对较小或对称),那么各国独立追求本国目标的结果可能与合作下的全球最优目标相差不大,名义汇率也倾向于保持稳定。而在DCP下(如美元定价),主导货币国的政策对全球(包括非主导国)的贸易价格和条件有很大影响,而非主导国的政策影响力则较小。这种非对称性导致显著的外部性,使得独立决策可能偏离全球最优,从而产生了政策协调(或主导国承担更大责任)的潜在收益。
    \end{itemize}
\end{itemize}

\subsection*{总结:理论发展的内在驱动力与课程的“练习”逻辑}

宏观经济学B的课程结构体现了经济学理论为解释日益复杂的开放经济现象并提供政策指导而不断演进的内在逻辑。每一个新主题的引入,通常是为了解决前一主题留下的未解问题,或放松过于简化的假设,或引入新的分析维度。

课程中隐含的“练习”逻辑,不仅仅是计算题或推导题,更重要的是培养学生运用这些理论工具进行\textbf{经济学推理 (economic reasoning)} 的能力。例如:

\begin{itemize}
    \item \textbf{识别冲击与传导机制}:能够判断一个外部事件(如外国利率上升、石油价格上涨、技术进步)属于何种类型的冲击(需求/供给,暂时/永久,实际/名义),并通过所学模型分析其在一国经济中的传导路径。
    \item \textbf{比较不同制度安排的后果}:能够分析在不同汇率制度(固定vs浮动)、不同市场结构(完全竞争vs垄断竞争)、不同信息条件(完全信息vs信息不对称)、不同汇率传递机制(PCP vs LCP)下,经济运行的特点以及政策效果的差异。
    \item \textbf{评估政策的有效性与合意性}:不仅能描述政策如何影响经济变量,还能运用福利经济学的原理(即使是初步的),来评估政策的净效应以及可能存在的取舍。
    \item \textbf{联系理论与现实}:能够将抽象的理论模型与观察到的实际经济现象(如全球失衡、货币危机、本土偏好)联系起来,尝试用理论去解释现实,并认识到理论的局限性。
\end{itemize}

\end{document}
