\section{Motivation}

Countries in the periphery of the European Union, such as Ireland, Portugal,
Greece, and a number of small eastern European countries adopted a fixed
exchange rate regime by joining the euro area

Most of these countries experienced an initial transition into the euro characterized
by low inflation, low interest rates, and economic expansion

The inception of the euro in 1999 was followed by massive capital inflows into the
region, possibly driven by expectations of quick convergence of peripheral and core
Europe

Large current account deficits and large increases in nominal hourly wages, with
declining rates of unemployment between 2000 and 2008

When the global crisis of 2008 starts, capital inflows dry up abruptly. Peripheral
Europe suffers a severe sudden stop (sharp reductions in current account deficits)

In spite of the collapse in aggregate demand and the lack of a devaluation,
nominal hourly wages remain as high as at the peak of the boom

Massive unemployment affects all countries in the region

\section{Challenge with Currency Pegs}

ountries adopt currency pegs for a variety of reasons. Often to fight high
inflation and to anchor expectations. (Example: Argentina's 1991
convertibility plan.)

However, fixed exchange rate arrangements can be difficult to maintain and
can have costs.

The Achilles' heel of currency pegs is that they hinder the efficient
adjustment of the economy to negative external shocks, such as drops in the
terms of trade or hikes in the interest-rate.

Such shocks produce a contraction in aggregate demand that requires a
decrease in the relative price of nontradables, that is, a real depreciation of
the domestic currency, in order to bring about an expenditure switch away
from tradables and toward nontradables.

In turn, the required real depreciation may come about via a nominal
devaluation of the domestic currency or via a fall in nominal prices or both.

he currency peg rules out a devaluation. Thus, the only way the necessary
real depreciation can occur is through a decline in the nominal price of
nontradables.

However, when nominal wages are downwardly rigid, producers of
nontradables are reluctant to lower prices, for doing so might render their
enterprises no longer profitable.

As a result, the necessary real depreciation takes place too slowly, causing
recession and unemployment along the way.

This narrative goes back at least to Keynes (1925) who argued that
Britain's 1925 decision to return to the gold standard at the 1913 parity
despite the significant increase in the aggregate price level that took place
during World War I would force deflation in nominal wages with deleterious
consequences for unemployment and economic activity.

Similarly, Friedman's (1953) seminal essay points at downward nominal wage
rigidity as the central argument against fixed exchange rates.

\section{Key Variables and Assumptions for Open Economy with Downward Nominal Regidity}

Stochastic and exogenous endowment of tradable goods (this
assumption can be relaxed): $y_t^T.$

Movements in $y_t^T$ can be interpreted as shocks to the availability of
traded goods or shocks to ToT.

Stochastic and exogenous country interest rate: $r_t.$

Nontraded goods, $y_t^{N}$,produced with labor, $h_t{:}y_t^{N}=F(h_t).$

Law of one price holds for tradables: $P_t^T=\mathcal{E}_tP_t^*.$

$P_{t}^{T}$,nominal price of tradable goods.

$\varepsilon_t$,nominal exchange rate, domestic-currency price of one unit of foreign currency($\varepsilon_{t}\uparrow$ depreciation of domestic currency).

$P_{t}^{* }$,foreign currency price of tradable goods.

Assume that $P_t^*=1$,so that $P_t^T=\varepsilon_t$

FOR Households:
\begin{align}
    \max_{\{C_t^T, C_t^N, d_{t+1} \}} & \mathbb{E}_0 \sum_{t=0}^{\infty}\beta^{t}U(c_t)\\
    s.t. c_t &= A(C_t^T, C_t^N)\\
    P_t^T c_t^T + P_t^N c_t^N + \varepsilon_t d_t &= P_t^T y_t^T + W_t h_t + \varepsilon \frac{d_{t+1}}{1+r_t}+\Phi_t\\
    d_{t+1} & \leq \bar{d}\\
    h_t & \leq \bar{h} 
\end{align}

The optimal condition is:
\[\frac{A_2(c_t^T, c_t^N)}{A_1(c_t^T, c_t^N)} = p_t.\]

If very little demand, wage go down but stuck at yesterday's wage,
full employment + yesterday's wage
Higher labor supply , must have yesterday's wage, no higher

Wages not rigid in both directions are better than wages rigid in one direction.

