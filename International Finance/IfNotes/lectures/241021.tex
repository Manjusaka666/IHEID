\section{From Advanced Economies (AE) to EMDEs}

We saw that a calibrated version of the SOE-RBC model captures well
key empirical regularities of a developed SOE like Canada (chapter 4).
\begin{question}
    Can the OE-RBC model also explain business cycles in EMDEs?
\end{question}

\begin{note}
    A quick reminder of the SOE-RBC Model:
    \begin{align*}
        \max \quad & E_0 \sum_{t=0}^{\infty}\beta ^t U(c_t, h_t) \\
        s.t. \quad & d_t + A_t F(k_t, h_t) = (1+r_{t-1} )d_{t-1} +c_t + k_{t+1} - (1-\delta)k_t + \Phi(k_{t+1} - k_t) \\
    \end{align*}
    and a non-Ponzi game constraint.
\end{note}

In theory, we can get higher volatility of every component by increasing $\epsilon$, or we can make consumption more volatile than output if we increasr
$\rho $.

The more persistent the shock is, the longer income stays above the steady state.
If the permanent income increase greatly, the consumption will respondingly increase.

\begin{intuition}
    \ 

    \begin{itemize}
        \item In principle, the SOE-RBC model can easily handle this difference.
        \item Simply jack up (by a factor of around 2) the standard deviation of the productivity shock. After all, in the SOE-RBC model, $\sigma _a$ was calibrated to match the standard deviation of Canadian GDP.
        \item Since not only output but also all of the components of aggregate demand (consumption, investment, net exports) are more volatile in
        EMDEs than in AEs, increasing $\sigma _a$ would help along more than one dimension.
        \item But in models with just one exogenous shocks, up to first order, the
        ratio of two standard deviations is independent of the standard
        deviation of the exogenous shocks
    \end{itemize}
\end{intuition}

\subsection{The SOE-RBC Model With Nonstationary Technology Shocks: Consumption}

The HH problem:
\[
\max \quad E_0\sum_{t=0}^\infty\beta^t\frac{[C_t^\gamma(1-h_t)^{1-\gamma}]^{1-\sigma}-1}{1-\sigma}
\]
subject to the standard NPG $$\lim_{j\to\infty}E_t\frac{D_{t+j+1}}{\Pi_{s=0}^j(1+r_{t+s})}\leq0$$ and

$$\frac{D_{t+1}}{1+r_t}+Y_t=D_t+C_t+K_{t+1}-(1-\delta)K_t+\frac\phi2\left(\frac{K_{t+1}}{K_t}-g\right)^2K_t,$$

\begin{itemize}
    \item Most variables and parameters as before.
    \item We now have a Cobb-Douglas period utility index instead of GHH. 
    \item Debt takes the form of one-period discount bonds. The HH receives $\frac{D_{t+1}}{(1+r_t)}$ units ot
    good in period $t$ and needs to repay $D_t+1$ in period $t+1.$ (for the HH, there is no difference between this formulation and the one introduced before, to see this set $D_t^{\prime}=D_t/(1+r_{t-1}).$
    \item Also: $\beta$, and $\delta$ are between 0 and 1 and $\gamma,\sigma,\phi$,and $g(g$ is the gross growth rate
    of $X$ in a non-stochastic equilibrium path; $X$ will be defined next page) are
positive
\end{itemize}

Set up the Lagrangean and get:

\begin{align*}
    & \frac{1-\gamma}\gamma\frac{C_t}{1-h_t}=(1-\alpha)a_tX_t\left(\frac{K_t}{X_th_t}\right)^\alpha \\
    & \gamma C_t^{\gamma(1-\sigma)-1}(1-h_t)^{(1-\gamma)(1-\sigma)}=\Lambda_t\\
    & \Lambda_t=\beta(1+r_t)E_t\Lambda_{t+1}
\end{align*}
\[
\Lambda_{t}\left(1+\phi\left(\frac{K_{t+1}}{K_{t}}-g\right)\right)=\beta E_{t}\Lambda_{t+1}\left[1-\delta+\alpha a_{t+1}\left(\frac{K_{t+1}}{X_{t+1}h_{t+1}}\right)^{\alpha-1}+\phi\frac{K_{t+2}}{K_{t+1}}\left(\frac{K_{t+2}}{K_{t+1}-g}\right)-\frac{\phi}{2}\left(\frac{K_{t+2}}{K_{t+1}-g}\right)^{2}\right]
\]

$\Lambda_t$ is the Lagrange multiplier associated with the sequential budget
constraint of the household.

