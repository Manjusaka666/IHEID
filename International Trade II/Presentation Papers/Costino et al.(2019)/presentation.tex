\documentclass{beamer}
\usetheme{Boadilla}
\usecolortheme{dolphin}
\usefonttheme{professionalfonts}
\usepackage[utf8]{inputenc}
\usepackage{amsmath}

\title[The More We Die, The More We Sell?]{The More We Die, The More We Sell? A Simple Test of the Home-Market Effect}
\author{Arnaud Costinot, Dave Donaldson, Margaret Kyle, Heidi Williams}
\institute{The Quarterly Journal of Economics (2019)}
\date{\today}

\begin{document}

% --- TITLE SLIDE ---
\begin{frame}
    \titlepage
    \begin{center}
        \vspace{1cm}
        \small{Presentation and Critical Analysis by: An International Trade Researcher}
    \end{center}
\end{frame}

% --- MOTIVATION ---
\begin{frame}{Motivation \& Research Question}
    \textbf{A Foundational Idea in International Trade:}
    \begin{itemize}
        \item The \textbf{Home-Market Effect} (Linder, 1961; Krugman, 1980): Countries with large domestic demand for a product tend to become net exporters of that product.
        \item This is a core prediction of \textbf{New Trade Theory}, which emphasizes increasing returns to scale.
    \end{itemize}

    \vspace{0.5cm}
    \textbf{The Key Empirical Challenge:}
    \begin{itemize}
        \item \textbf{Endogeneity.} How to isolate the \textit{causal} effect of demand?
        \item Previous tests used expenditure shares, but expenditure is an equilibrium outcome affected by both supply and demand. This leads to biased and inconclusive results.
    \end{itemize}

    \vspace{0.5cm}
    \textbf{This Paper's Research Question:}
    \begin{block}{Question}
        Can we find a clean, exogenous source of variation in home demand to provide a definitive test of the home-market effect?
    \end{block}
\end{frame}

% --- CONTRIBUTION & STRATEGY ---
\begin{frame}{Contribution: A Novel Identification Strategy}
    The paper's core innovation is its empirical strategy, applied to the global pharmaceutical industry.

    \begin{columns}[T]
        \begin{column}{0.5\textwidth}
            \textbf{Step 1: Predict Disease}
            \begin{itemize}
                \item Use a country's exogenous \textbf{demographic composition} (age, gender).
                \item Combine with global disease prevalence rates for each demographic group.
            \end{itemize}
        \end{column}
        \begin{column}{0.5\textwidth}
            \textbf{Step 2: Predict Demand}
            \begin{itemize}
                \item This creates a \textbf{“Predicted Disease Burden” (PDB)} for each country and disease.
                \item PDB serves as a plausibly \textbf{exogenous demand shifter} for drugs treating that disease.
            \end{itemize}
        \end{column}
    \end{columns}

    \begin{alertblock}{The Logic}
        Demographics $\rightarrow$ Predicted Disease $\rightarrow$ Exogenous Home Demand for Drugs
    \end{alertblock}

    \vspace{0.5cm}
    This allows for the first clean test of the home-market effect's causal claim.
\end{frame}

% --- THEORETICAL FRAMEWORK ---
\begin{frame}{Theoretical Framework: Weak vs. Strong Effects}
    The authors build a model to derive two precise, testable hypotheses. Bilateral sales from exporter \textit{i} to importer \textit{j} are modeled as:
    
    \begin{center}
    $ \ln(x_{ij}^n) = \text{Fixed Effects} + \beta_M \ln(\text{Demand}_j^n) + \beta_x \ln(\text{Demand}_i^n) + \varepsilon_{ij}^n $
    \end{center}

    \begin{enumerate}
        \item \textbf{Weak Home-Market Effect: $ \beta_x > 0 $}
        \begin{itemize}
            \item A larger home demand in the exporting country (\textit{i}) increases its foreign sales.
            \item \textbf{Crucial Insight:} This effect is only possible with \textbf{economies of scale} (a downward-sloping supply curve). Without them, higher home demand would raise prices and \textit{reduce} exports.
        \end{itemize}
        
        \item \textbf{Strong Home-Market Effect: $ \beta_x > \beta_M $}
        \begin{itemize}
            \item The boost to exports from home demand is larger than the boost to imports from foreign demand.
            \item This implies the country becomes a \textbf{net exporter}.
            \item Requires \textit{sufficiently strong} economies of scale.
        \end{itemize}
    \end{enumerate}
\end{frame}

% --- DATA ---
\begin{frame}{Data Sources}
    The analysis links two main datasets for a 2012 cross-section:

    \begin{block}{Bilateral Sales Data}
        \textbf{IMS MIDAS Dataset}
        \begin{itemize}
            \item Comprehensive data on pharmaceutical sales for 56 countries.
            \item Covers $>$20,000 molecules from $~$2,650 firms.
            \item Authors match firms to headquarter countries to get origin-destination sales flow.
        \end{itemize}
    \end{block}

    \begin{block}{Demand Shifter Data}
        \begin{itemize}
            \item \textbf{WHO Global Burden of Disease (GBD):} Disease incidence by demographic group (age/gender).
            \item \textbf{U.S. Census Bureau:} Population data for each demographic group in each country.
        \end{itemize}
        $\rightarrow$ Together, these are used to construct the \textbf{Predicted Disease Burden (PDB)}.
    \end{block}
\end{frame}

% --- BASELINE RESULTS ---
\begin{frame}{Baseline Results (Table III)}
    The regression estimates confirm the theory's predictions.
    
    \begin{center}
    \textbf{Dependent Variable: Log(Bilateral Sales)}
    \begin{tabular}{lc}
    \hline
    \textbf{Variable} & \textbf{Coefficient (Std. Err.)} \\
    \hline
    Log(PDB, destination) ($\beta_M$) & 0.545 (0.107) \\
    Log(PDB, origin) ($\beta_x$)      & \textbf{0.928 (0.123)} \\
    \hline
    Observations                     & 19,150 \\
    \hline
    \end{tabular}
    \end{center}
    
    \begin{alertblock}{Test 1: Weak Home-Market Effect ($\beta_x > 0$)}
        The coefficient on origin PDB is 0.928 and highly significant. \\ \textbf{Result: The weak HME is strongly supported.}
    \end{alertblock}
    
    \begin{exampleblock}{Test 2: Strong Home-Market Effect ($\beta_x > \beta_M$)}
        An F-test rejects the null hypothesis that $\beta_x \le \beta_M$ (p-value = 0.018). \\ \textbf{Result: The strong HME is also supported.}
    \end{exampleblock}
    
    These results are robust to dozens of sensitivity checks.
\end{frame}

% --- STRUCTURAL RESULTS ---
\begin{frame}{Mechanism: Disentangling Supply \& Demand}
    Is the effect driven by economies of scale (as theory predicts) or something else (e.g., inelastic demand)? The authors use an IV strategy to estimate the structural elasticities.
    
    \begin{columns}[T]
        \begin{column}{0.5\textwidth}
            \textbf{Demand Elasticity ($ \varepsilon^x $)}
            \begin{itemize}
                \item Estimated using trade distance and price data.
                \item Result: $ \varepsilon^x \approx 6.2 $
                \item \textbf{Conclusion:} Demand is highly elastic. This is \textit{not} driving the result.
            \end{itemize}
        \end{column}
        \begin{column}{0.5\textwidth}
            \textbf{Supply Elasticity ($ \varepsilon^s $)}
            \begin{itemize}
                \item Estimated using PDB as an instrument for total sales.
                \item Result: $ \varepsilon^s \approx -7.8 $
                \item \textbf{Conclusion:} The supply curve is strongly \textbf{downward-sloping}.
            \end{itemize}
        \end{column}
    \end{columns}
    
    \begin{alertblock}{The Verdict}
        The paper provides direct evidence that the home-market effect is driven by substantial \textbf{industry-level economies of scale}.
    \end{alertblock}
\end{frame}

% --- CONCLUSION ---
\begin{frame}{Conclusion}
    \begin{itemize}
        \item This paper provides the most convincing empirical evidence to date in favor of the \textbf{home-market effect}.
        
        \item Its novel identification strategy, using demographic structure to predict exogenous demand, solves a long-standing challenge in the field.
        
        \item The findings confirm the central mechanism of \textbf{New Trade Theory}: a large home market fosters economies of scale, which in turn drives export performance.
        
        \item In short: \textbf{The more we (are predicted to) die at home, the more we sell abroad.}
    \end{itemize}
\end{frame}

% --- CRITICAL ANALYSIS ---
\begin{frame}{Critical Analysis \& Discussion}
    This is an outstanding paper, but there are areas for future research.

    \begin{block}{Generalizability}
        The findings are from a single, unique industry (pharmaceuticals). Would the effect be as strong in sectors with lower fixed costs or different regulatory structures?
    \end{block}
    
    \begin{block}{Exports vs. FDI}
        The sales data combine exports and sales by foreign affiliates. The underlying economies of scale may differ (production vs. R\&D). Future work could try to disentangle these channels.
    \end{block}
    
    \begin{block}{Static Analysis}
        The cross-sectional design captures the effect at one point in time. A panel analysis, though difficult, could reveal the dynamic evolution of the home-market effect as demographics and industries change.
    \end{block}
\end{frame}

%------------------------------------------------

\begin{frame}{Motivation: The Home-Market Effect Puzzle}
\begin{block}{A Core Idea in ``New Trade Theory''}
The Home-Market Effect (HME), hypothesized by Linder (1961) and formalized by Krugman (1980), posits that countries with large domestic demand for a product tend to become net exporters of it.
\end{block}

\begin{alertblock}{A Major Empirical Challenge}
Testing this is notoriously difficult due to endogeneity.
\begin{itemize}
    \item Standard demand proxies (e.g., national expenditure) are equilibrium outcomes.
    \item A positive supply shock can increase both domestic expenditure and exports, creating a spurious correlation.
    \item \textbf{Key Question:} How can we isolate an exogenous shock to home demand?
\end{itemize}
\end{alertblock}
\end{frame}

%------------------------------------------------

\begin{frame}{This Paper's Contribution: A Novel Identification Strategy}
\begin{block}{The Core Idea}
Use a country's exogenous demographic structure as a predictor for its demand for specific pharmaceuticals.
\begin{itemize}
    \item This is a spatial analogue to the time-series strategy of Acemoglu \& Linn (2004).
    \item It creates a plausibly exogenous demand shifter: the \textbf{Predicted Disease Burden (PDB)}.
\end{itemize}
\end{block}

\begin{exampleblock}{The PDB Instrument}
For each country $i$ and disease $n$, the PDB is constructed as:
$$ (PDB)_{i}^{n}=\sum_{a,g}\left[\text{pop}_{iag}\times\left(\frac{\sum_{k\ne i}\text{burden}_{kag}^{n}}{\sum_{k\ne i}\text{pop}_{kag}}\right)\right] $$
\begin{itemize}
    \item $\text{pop}_{iag}$: Population of age-gender group $(a,g)$ in country $i$.
    \item The ratio is the average disease burden for group $(a,g)$ in the \textit{rest of the world} ($k \neq i$).
    \item This "leave-one-out" construction enhances exogeneity.
\end{itemize}
\end{exampleblock}
\end{frame}

%------------------------------------------------

\begin{frame}{Sources of Variation for the PDB Instrument}
\begin{columns}[c]
\column{0.5\textwidth}
\textbf{Variation in Demographics Across Countries}
\begin{figure}
    \includegraphics[width=\textwidth]{example-image-a}
    \caption{Share of population under age 60 varies dramatically (e.g., Japan vs. UAE). Based on Figure IV in Costinot et al. (2019).}
\end{figure}

\column{0.5\textwidth}
\textbf{Variation in Disease Profile Across Demographics}
\begin{figure}
    \includegraphics[width=\textwidth]{example-image-b}
    \caption{Share of global disease burden borne by population under 60 varies by disease (e.g., Alzheimer's vs. Whooping Cough). Based on Figure V.}
\end{figure}
\end{columns}
\end{frame}

%------------------------------------------------

\begin{frame}{The Theoretical Mechanism: Why Scale Economies are Key}
\begin{columns}[c]
\column{0.33\textwidth}
\textbf{No HME (Neoclassical)}
\begin{figure}
    \includegraphics[width=\textwidth]{example-image-c}
    \caption{Higher home demand $\rightarrow$ Higher Price $\rightarrow$ Lower Exports. Based on Figure I.}
\end{figure}

\column{0.33\textwidth}
\textbf{Weak HME}
\begin{figure}
    \includegraphics[width=\textwidth]{example-image-a}
    \caption{With scale economies, higher demand $\rightarrow$ Lower Price $\rightarrow$ Higher Exports ($\beta_X > 0$). Based on Figure II.}
\end{figure}

\column{0.33\textwidth}
\textbf{Strong HME}
\begin{figure}
    \includegraphics[width=\textwidth]{example-image-b}
    \caption{With strong scale economies, Exports rise more than Imports ($\beta_X > \beta_M$). Based on Figure III.}
\end{figure}
\end{columns}
\begin{alertblock}{}
A home-market effect requires a downward-sloping industry supply curve, i.e., increasing returns to scale.
\end{alertblock}
\end{frame}

%------------------------------------------------

\begin{frame}{Empirical Strategy: Baseline Specification}
The paper tests the HME by estimating a gravity-style equation:
\begin{block}{Equation (16)}
$$ \ln x_{ij}^{n}=\delta_{ij}+\delta^{n}+\tilde{\beta}_{M}\ln(PDB)_{j}^{n}+\tilde{\beta}_{X}\ln(PDB)_{i}^{n}+\tilde{\epsilon}_{ij}^{n} $$
\end{block}
\begin{itemize}
    \item $x_{ij}^{n}$: Bilateral sales from origin $i$ to destination $j$ for disease $n$.
    \item $(PDB)_{i}^{n}$: Predicted Disease Burden in the origin (exporter).
    \item $(PDB)_{j}^{n}$: Predicted Disease Burden in the destination (importer).
    \item $\delta_{ij}$: Origin-Destination fixed effects (absorbs distance, etc.).
    \item $\delta^{n}$: Disease fixed effects (absorbs global disease size).
\end{itemize}
\begin{exampleblock}{Hypothesis Tests}
\begin{itemize}
    \item \textbf{Weak HME}: Test if $\tilde{\beta}_X > 0$.
    \item \textbf{Strong HME}: Test if $\tilde{\beta}_X > \tilde{\beta}_M$.
\end{itemize}
\end{exampleblock}
\end{frame}

%------------------------------------------------

\begin{frame}{Main Result: Strong Evidence for HME}
\begin{block}{Baseline Results (Table III, Column 3)}
\begin{tabular}{lc}
\hline
\textbf{Variable} & \textbf{Coefficient (Std. Err.)} \\
\hline
$\ln(\text{PDB, destination})$ \quad ($\tilde{\beta}_M$) & 0.545 \\
& (0.107) \\
$\ln(\text{PDB, origin})$ \quad ($\tilde{\beta}_X$) & 0.928 \\
& (0.123) \\
\hline
Observations & 19,150 \\
Adjusted $R^2$ & 0.540 \\
\hline
\end{tabular}
\end{block}

\begin{alertblock}{Hypothesis Test Results}
\begin{itemize}
    \item \textbf{Weak HME ($H_0: \tilde{\beta}_X \le 0$):} p-value = 0.000. \textbf{Resoundingly rejected.}
    \item \textbf{Strong HME ($H_0: \tilde{\beta}_X \le \tilde{\beta}_M$):} p-value = 0.018. \textbf{Rejected at 5\% level.}
\end{itemize}
\end{alertblock}
Conclusion: Countries with higher exogenous demand for a drug export more of it and become net exporters.
\end{frame}

%------------------------------------------------

\begin{frame}{Robustness of the Main Finding}
The main result is robust to a wide array of alternative explanations and specifications.
\begin{itemize}
    \item \textbf{Controlling for Confounders (Table IV):} Results hold after controlling for interactions between disease characteristics and country characteristics like GDP per capita.
    \item \textbf{Supply-Side Stories (Table V):}
    \begin{itemize}
        \item The effect remains when controlling for US NIH subsidies.
        \item The weak HME is present even when looking only at \textbf{generic drugs}, where R\&D-related scale economies should be weaker.
    \end{itemize}
    \item \textbf{Spatial Correlation of Demand (Table VI):} Results are not driven by demand in neighboring countries.
    \item \textbf{Pricing-to-Market (Table VII):} The weak HME holds within the EU, where parallel trade limits price discrimination.
    \item \textbf{Zero Trade Flows (Table VIII):} Results are robust to using PPML estimation, which includes observations with zero sales.
\end{itemize}
\end{frame}

%------------------------------------------------

\begin{frame}{Structural Results: Quantifying Economies of Scale}
The paper goes beyond the reduced-form test to estimate the structural supply elasticity ($\epsilon^s$).
\begin{block}{IV Strategy}
\begin{itemize}
    \item \textbf{Goal:} Estimate the elasticity of the industry supply curve.
    \item \textbf{Problem:} Total sales (scale) is endogenous to supply shocks.
    \item \textbf{Solution:} Instrument for a country's total sales in a disease category ($\ln r_i^n$) with the exogenous demand shifter ($\ln(PDB)_i^n$).
\end{itemize}
\end{block}

\begin{exampleblock}{Key Structural Finding (Table X)}
The IV estimation yields a supply elasticity of:
$$\epsilon^s = -7.833$$
\begin{itemize}
    \item The negative sign provides direct evidence of a \textbf{downward-sloping supply curve}, confirming the existence of significant industry-level economies of scale.
    \item This magnitude is about 25\% smaller than the benchmark prediction from a standard Krugman (1980) model.
\end{itemize}
\end{exampleblock}
\end{frame}

%------------------------------------------------

\begin{frame}{Conclusion \& Critical Assessment}
\begin{block}{Summary of Contributions}
\begin{itemize}
    \item Provides a simple, powerful, and credible test of the home-market effect.
    \item Solves a major identification problem using a novel demographically-driven instrument (PDB).
    \item Finds strong evidence for both weak and strong HME in the global pharmaceutical industry.
    \item Quantifies the underlying economies of scale, finding $\epsilon^s = -7.833$.
\end{itemize}
\end{block}

\begin{alertblock}{Points for Discussion \& Future Research}
\begin{itemize}
    \item \textbf{External Validity:} The pharmaceutical industry is an ideal setting for HME. Would these results generalize to industries with weaker scale economies?
    \item \textbf{FDI vs. Exports:} The data combine exports and sales from local affiliates. Is this a "production HME" or a "headquarters HME" (driven by R\&D/marketing)?
    \item \textbf{The "Black Box" of Scale:} The test confirms the existence of scale economies but not their source (e.g., R\&D spillovers, love-of-variety). The welfare implications depend on the source.
\end{itemize}
\end{alertblock}
\end{frame}

%------------------------------------------------

\begin{frame}
\begin{center}
    \Huge Thank You \\ \vspace{1cm} Questions?
\end{center}
\end{frame}

\end{document}