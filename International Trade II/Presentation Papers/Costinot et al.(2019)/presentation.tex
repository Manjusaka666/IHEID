% \documentclass{beamer}
% \usetheme{Boadilla}
% \usecolortheme{dolphin}
% \usefonttheme{professionalfonts}
% \usepackage[utf8]{inputenc}
% \usepackage{amsmath}

% \title[The More We Die, The More We Sell?]{The More We Die, The More We Sell? A Simple Test of the Home-Market Effect}
% \author{Arnaud Costinot, Dave Donaldson, Margaret Kyle, Heidi Williams}
% \institute{The Quarterly Journal of Economics (2019)}
% \date{\today}

% \begin{document}

% % --- TITLE SLIDE ---
% \begin{frame}
%     \titlepage
%     \begin{center}
%         \vspace{1cm}
%         \small{Presentation and Critical Analysis by: An International Trade Researcher}
%     \end{center}
% \end{frame}

% % --- MOTIVATION ---
% \begin{frame}{Motivation \& Research Question}
%     \textbf{A Foundational Idea in International Trade:}
%     \begin{itemize}
%         \item The \textbf{Home-Market Effect} (Linder, 1961; Krugman, 1980): Countries with large domestic demand for a product tend to become net exporters of that product.
%         \item This is a core prediction of \textbf{New Trade Theory}, which emphasizes increasing returns to scale.
%     \end{itemize}
%     \textbf{The Key Empirical Challenge:}
%     \begin{itemize}
%         \item \textbf{Endogeneity.} How to isolate the \textit{causal} effect of demand?
%         \item Previous tests used expenditure shares, but expenditure is an equilibrium outcome affected by both supply and demand. This leads to biased and inconclusive results.
%     \end{itemize}
%     \textbf{This Paper's Research Question:}
%     \begin{block}{Question}
%         Can we find a clean, exogenous source of variation in home demand to provide a definitive test of the home-market effect?
%     \end{block}
% \end{frame}

% % --- CONTRIBUTION & STRATEGY ---
% \begin{frame}{Contribution: A Novel Identification Strategy}
%     The paper's core innovation is its empirical strategy, applied to the global pharmaceutical industry.

%     \begin{columns}[T]
%         \begin{column}{0.5\textwidth}
%             \textbf{Step 1: Predict Disease}
%             \begin{itemize}
%                 \item Use a country's exogenous \textbf{demographic composition} (age, gender).
%                 \item Combine with global disease prevalence rates for each demographic group.
%             \end{itemize}
%         \end{column}
%         \begin{column}{0.5\textwidth}
%             \textbf{Step 2: Predict Demand}
%             \begin{itemize}
%                 \item This creates a \textbf{“Predicted Disease Burden” (PDB)} for each country and disease.
%                 \item PDB serves as a plausibly \textbf{exogenous demand shifter} for drugs treating that disease.
%             \end{itemize}
%         \end{column}
%     \end{columns}

%     \begin{alertblock}{The Logic}
%         Demographics $\rightarrow$ Predicted Disease $\rightarrow$ Exogenous Home Demand for Drugs
%     \end{alertblock}

%     This allows for the first clean test of the home-market effect's causal claim.
% \end{frame}

% % --- THEORETICAL FRAMEWORK ---
% \begin{frame}{Theoretical Framework: Weak vs. Strong Effects}
%     The authors build a model to derive two precise, testable hypotheses. Bilateral sales from exporter \textit{i} to importer \textit{j} are modeled as:

%     \begin{center}
%         $ \ln(x_{ij}^n) = \text{Fixed Effects} + \beta_M \ln(\text{Demand}_j^n) + \beta_x \ln(\text{Demand}_i^n) + \varepsilon_{ij}^n $
%     \end{center}

%     \begin{enumerate}
%         \item \textbf{Weak Home-Market Effect: $ \beta_x > 0 $}
%               \begin{itemize}
%                   \item A larger home demand in the exporting country (\textit{i}) increases its foreign sales.
%                   \item \textbf{Crucial Insight:} This effect is only possible with \textbf{economies of scale} (a downward-sloping supply curve). Without them, higher home demand would raise prices and \textit{reduce} exports.
%               \end{itemize}

%         \item \textbf{Strong Home-Market Effect: $ \beta_x > \beta_M $}
%               \begin{itemize}
%                   \item The boost to exports from home demand is larger than the boost to imports from foreign demand.
%                   \item This implies the country becomes a \textbf{net exporter}.
%                   \item Requires \textit{sufficiently strong} economies of scale.
%               \end{itemize}
%     \end{enumerate}
% \end{frame}

% % --- DATA ---
% \begin{frame}{Data Sources}
%     The analysis links two main datasets for a 2012 cross-section:

%     \begin{block}{Bilateral Sales Data}
%         \textbf{IMS MIDAS Dataset}
%         \begin{itemize}
%             \item Comprehensive data on pharmaceutical sales for 56 countries.
%             \item Covers $>$20,000 molecules from $~$2,650 firms.
%             \item Authors match firms to headquarter countries to get origin-destination sales flow.
%         \end{itemize}
%     \end{block}

%     \begin{block}{Demand Shifter Data}
%         \begin{itemize}
%             \item \textbf{WHO Global Burden of Disease (GBD):} Disease incidence by demographic group (age/gender).
%             \item \textbf{U.S. Census Bureau:} Population data for each demographic group in each country.
%         \end{itemize}
%         $\rightarrow$ Together, these are used to construct the \textbf{Predicted Disease Burden (PDB)}.
%     \end{block}
% \end{frame}

% % --- BASELINE RESULTS ---
% \begin{frame}{Baseline Results (Table III)}
%     The regression estimates confirm the theory's predictions.

%     \begin{center}
%         \textbf{Dependent Variable: Log(Bilateral Sales)}
%         \begin{tabular}{lc}
%             \hline
%             \textbf{Variable}                 & \textbf{Coefficient (Std. Err.)} \\
%             \hline
%             Log(PDB, destination) ($\beta_M$) & 0.545 (0.107)                    \\
%             Log(PDB, origin) ($\beta_x$)      & \textbf{0.928 (0.123)}           \\
%             \hline
%             Observations                      & 19,150                           \\
%             \hline
%         \end{tabular}
%     \end{center}

%     \begin{alertblock}{Test 1: Weak Home-Market Effect ($\beta_x > 0$)}
%         The coefficient on origin PDB is 0.928 and highly significant. \\ \textbf{Result: The weak HME is strongly supported.}
%     \end{alertblock}

%     \begin{exampleblock}{Test 2: Strong Home-Market Effect ($\beta_x > \beta_M$)}
%         An F-test rejects the null hypothesis that $\beta_x \le \beta_M$ (p-value = 0.018). \\ \textbf{Result: The strong HME is also supported.}
%     \end{exampleblock}

%     These results are robust to dozens of sensitivity checks.
% \end{frame}

% % --- STRUCTURAL RESULTS ---
% \begin{frame}{Mechanism: Disentangling Supply \& Demand}
%     Is the effect driven by economies of scale (as theory predicts) or something else (e.g., inelastic demand)? The authors use an IV strategy to estimate the structural elasticities.

%     \begin{columns}[T]
%         \begin{column}{0.5\textwidth}
%             \textbf{Demand Elasticity ($ \varepsilon^x $)}
%             \begin{itemize}
%                 \item Estimated using trade distance and price data.
%                 \item Result: $ \varepsilon^x \approx 6.2 $
%                 \item \textbf{Conclusion:} Demand is highly elastic. This is \textit{not} driving the result.
%             \end{itemize}
%         \end{column}
%         \begin{column}{0.5\textwidth}
%             \textbf{Supply Elasticity ($ \varepsilon^s $)}
%             \begin{itemize}
%                 \item Estimated using PDB as an instrument for total sales.
%                 \item Result: $ \varepsilon^s \approx -7.8 $
%                 \item \textbf{Conclusion:} The supply curve is strongly \textbf{downward-sloping}.
%             \end{itemize}
%         \end{column}
%     \end{columns}

%     \begin{alertblock}{The Verdict}
%         The paper provides direct evidence that the home-market effect is driven by substantial \textbf{industry-level economies of scale}.
%     \end{alertblock}
% \end{frame}

% % --- CONCLUSION ---
% \begin{frame}{Conclusion}
%     \begin{itemize}
%         \item This paper provides the most convincing empirical evidence to date in favor of the \textbf{home-market effect}.

%         \item Its novel identification strategy, using demographic structure to predict exogenous demand, solves a long-standing challenge in the field.

%         \item The findings confirm the central mechanism of \textbf{New Trade Theory}: a large home market fosters economies of scale, which in turn drives export performance.

%         \item In short: \textbf{The more we (are predicted to) die at home, the more we sell abroad.}
%     \end{itemize}
% \end{frame}

% % --- CRITICAL ANALYSIS ---
% \begin{frame}{Critical Analysis \& Discussion}
%     This is an outstanding paper, but there are areas for future research.

%     \begin{block}{Generalizability}
%         The findings are from a single, unique industry (pharmaceuticals). Would the effect be as strong in sectors with lower fixed costs or different regulatory structures?
%     \end{block}

%     \begin{block}{Exports vs. FDI}
%         The sales data combine exports and sales by foreign affiliates. The underlying economies of scale may differ (production vs. R\&D). Future work could try to disentangle these channels.
%     \end{block}

%     \begin{block}{Static Analysis}
%         The cross-sectional design captures the effect at one point in time. A panel analysis, though difficult, could reveal the dynamic evolution of the home-market effect as demographics and industries change.
%     \end{block}
% \end{frame}

% %------------------------------------------------

% \begin{frame}{Motivation: The Home-Market Effect Puzzle}
%     \begin{block}{A Core Idea in ``New Trade Theory''}
%         The Home-Market Effect (HME), hypothesized by Linder (1961) and formalized by Krugman (1980), posits that countries with large domestic demand for a product tend to become net exporters of it.
%     \end{block}

%     \begin{alertblock}{A Major Empirical Challenge}
%         Testing this is notoriously difficult due to endogeneity.
%         \begin{itemize}
%             \item Standard demand proxies (e.g., national expenditure) are equilibrium outcomes.
%             \item A positive supply shock can increase both domestic expenditure and exports, creating a spurious correlation.
%             \item \textbf{Key Question:} How can we isolate an exogenous shock to home demand?
%         \end{itemize}
%     \end{alertblock}
% \end{frame}

% %------------------------------------------------

% \begin{frame}{This Paper's Contribution: A Novel Identification Strategy}
%     \begin{block}{The Core Idea}
%         Use a country's exogenous demographic structure as a predictor for its demand for specific pharmaceuticals.
%         \begin{itemize}
%             \item This is a spatial analogue to the time-series strategy of Acemoglu \& Linn (2004).
%             \item It creates a plausibly exogenous demand shifter: the \textbf{Predicted Disease Burden (PDB)}.
%         \end{itemize}
%     \end{block}

%     \begin{exampleblock}{The PDB Instrument}
%         For each country $i$ and disease $n$, the PDB is constructed as:
%         $$ (PDB)_{i}^{n}=\sum_{a,g}\left[\text{pop}_{iag}\times\left(\frac{\sum_{k\ne i}\text{burden}_{kag}^{n}}{\sum_{k\ne i}\text{pop}_{kag}}\right)\right] $$
%         \begin{itemize}
%             \item $\text{pop}_{iag}$: Population of age-gender group $(a,g)$ in country $i$.
%             \item The ratio is the average disease burden for group $(a,g)$ in the \textit{rest of the world} ($k \neq i$).
%             \item This "leave-one-out" construction enhances exogeneity.
%         \end{itemize}
%     \end{exampleblock}
% \end{frame}

% %------------------------------------------------

% \begin{frame}{Sources of Variation for the PDB Instrument}
%     \begin{columns}[c]
%         \column{0.5\textwidth}
%         \textbf{Variation in Demographics Across Countries}
%         \begin{figure}
%             \includegraphics[width=\textwidth]{example-image-a}
%             \caption{Share of population under age 60 varies dramatically (e.g., Japan vs. UAE). Based on Figure IV in Costinot et al. (2019).}
%         \end{figure}

%         \column{0.5\textwidth}
%         \textbf{Variation in Disease Profile Across Demographics}
%         \begin{figure}
%             \includegraphics[width=\textwidth]{example-image-b}
%             \caption{Share of global disease burden borne by population under 60 varies by disease (e.g., Alzheimer's vs. Whooping Cough). Based on Figure V.}
%         \end{figure}
%     \end{columns}
% \end{frame}

% %------------------------------------------------

% \begin{frame}{The Theoretical Mechanism: Why Scale Economies are Key}
%     \begin{columns}[c]
%         \column{0.33\textwidth}
%         \textbf{No HME (Neoclassical)}
%         \begin{figure}
%             \includegraphics[width=\textwidth]{example-image-c}
%             \caption{Higher home demand $\rightarrow$ Higher Price $\rightarrow$ Lower Exports. Based on Figure I.}
%         \end{figure}

%         \column{0.33\textwidth}
%         \textbf{Weak HME}
%         \begin{figure}
%             \includegraphics[width=\textwidth]{example-image-a}
%             \caption{With scale economies, higher demand $\rightarrow$ Lower Price $\rightarrow$ Higher Exports ($\beta_X > 0$). Based on Figure II.}
%         \end{figure}

%         \column{0.33\textwidth}
%         \textbf{Strong HME}
%         \begin{figure}
%             \includegraphics[width=\textwidth]{example-image-b}
%             \caption{With strong scale economies, Exports rise more than Imports ($\beta_X > \beta_M$). Based on Figure III.}
%         \end{figure}
%     \end{columns}
%     \begin{alertblock}{}
%         A home-market effect requires a downward-sloping industry supply curve, i.e., increasing returns to scale.
%     \end{alertblock}
% \end{frame}

% %------------------------------------------------

% \begin{frame}{Empirical Strategy: Baseline Specification}
%     The paper tests the HME by estimating a gravity-style equation:
%     \begin{block}{Equation (16)}
%         $$ \ln x_{ij}^{n}=\delta_{ij}+\delta^{n}+\tilde{\beta}_{M}\ln(PDB)_{j}^{n}+\tilde{\beta}_{X}\ln(PDB)_{i}^{n}+\tilde{\epsilon}_{ij}^{n} $$
%     \end{block}
%     \begin{itemize}
%         \item $x_{ij}^{n}$: Bilateral sales from origin $i$ to destination $j$ for disease $n$.
%         \item $(PDB)_{i}^{n}$: Predicted Disease Burden in the origin (exporter).
%         \item $(PDB)_{j}^{n}$: Predicted Disease Burden in the destination (importer).
%         \item $\delta_{ij}$: Origin-Destination fixed effects (absorbs distance, etc.).
%         \item $\delta^{n}$: Disease fixed effects (absorbs global disease size).
%     \end{itemize}
%     \begin{exampleblock}{Hypothesis Tests}
%         \begin{itemize}
%             \item \textbf{Weak HME}: Test if $\tilde{\beta}_X > 0$.
%             \item \textbf{Strong HME}: Test if $\tilde{\beta}_X > \tilde{\beta}_M$.
%         \end{itemize}
%     \end{exampleblock}
% \end{frame}

% % %------------------------------------------------

% \begin{frame}{Robustness of the Main Finding}
%     The main result is robust to a wide array of alternative explanations and specifications.
%     \begin{itemize}
%         \item \textbf{Controlling for Confounders (Table IV):} Results hold after controlling for interactions between disease characteristics and country characteristics like GDP per capita.
%         \item \textbf{Supply-Side Stories (Table V):}
%               \begin{itemize}
%                   \item The effect remains when controlling for US NIH subsidies.
%                   \item The weak HME is present even when looking only at \textbf{generic drugs}, where R\&D-related scale economies should be weaker.
%               \end{itemize}
%         \item \textbf{Spatial Correlation of Demand (Table VI):} Results are not driven by demand in neighboring countries.
%         \item \textbf{Pricing-to-Market (Table VII):} The weak HME holds within the EU, where parallel trade limits price discrimination.
%         \item \textbf{Zero Trade Flows (Table VIII):} Results are robust to using PPML estimation, which includes observations with zero sales.
%     \end{itemize}
% \end{frame}

% %------------------------------------------------

% \begin{frame}{Structural Results: Quantifying Economies of Scale}
%     The paper goes beyond the reduced-form test to estimate the structural supply elasticity ($\epsilon^s$).
%     \begin{block}{IV Strategy}
%         \begin{itemize}
%             \item \textbf{Goal:} Estimate the elasticity of the industry supply curve.
%             \item \textbf{Problem:} Total sales (scale) is endogenous to supply shocks.
%             \item \textbf{Solution:} Instrument for a country's total sales in a disease category ($\ln r_i^n$) with the exogenous demand shifter ($\ln(PDB)_i^n$).
%         \end{itemize}
%     \end{block}

%     \begin{exampleblock}{Key Structural Finding (Table X)}
%         The IV estimation yields a supply elasticity of:
%         $$\epsilon^s = -7.833$$
%         \begin{itemize}
%             \item The negative sign provides direct evidence of a \textbf{downward-sloping supply curve}, confirming the existence of significant industry-level economies of scale.
%             \item This magnitude is about 25\% smaller than the benchmark prediction from a standard Krugman (1980) model.
%         \end{itemize}
%     \end{exampleblock}
% \end{frame}

% %------------------------------------------------

% \begin{frame}{Conclusion \& Critical Assessment}
%     \begin{block}{Summary of Contributions}
%         \begin{itemize}
%             \item Provides a simple, powerful, and credible test of the home-market effect.
%             \item Solves a major identification problem using a novel demographically-driven instrument (PDB).
%             \item Finds strong evidence for both weak and strong HME in the global pharmaceutical industry.
%             \item Quantifies the underlying economies of scale, finding $\epsilon^s = -7.833$.
%         \end{itemize}
%     \end{block}

%     \begin{alertblock}{Points for Discussion \& Future Research}
%         \begin{itemize}
%             \item \textbf{External Validity:} The pharmaceutical industry is an ideal setting for HME. Would these results generalize to industries with weaker scale economies?
%             \item \textbf{FDI vs. Exports:} The data combine exports and sales from local affiliates. Is this a "production HME" or a "headquarters HME" (driven by R\&D/marketing)?
%             \item \textbf{The "Black Box" of Scale:} The test confirms the existence of scale economies but not their source (e.g., R\&D spillovers, love-of-variety). The welfare implications depend on the source.
%         \end{itemize}
%     \end{alertblock}
% \end{frame}

% %------------------------------------------------

% \begin{frame}
%     \begin{center}
%         \Huge Thank You
%     \end{center}
% \end{frame}

% \end{document}

% \documentclass{beamer}
% \usetheme{Boadilla}
% \usecolortheme{dolphin}
% \usefonttheme{professionalfonts}
% \usepackage[utf8]{inputenc}
% \usepackage{amsmath}

% \title[The More We Die, The More We Sell?]{The More We Die, The More We Sell? A Simple Test of the Home-Market Effect}
% \author{Arnaud Costinot, Dave Donaldson, Margaret Kyle, Heidi Williams}
% \institute{The Quarterly Journal of Economics (2019)}
% \date{\today}

% \begin{document}

% % --- TITLE SLIDE ---
% \begin{frame}
%     \titlepage
%     \begin{center}
%         \vspace{1cm}
%         \small{Presentation and Critical Analysis by: An International Trade Researcher}
%     \end{center}
% \end{frame}

% % --- MOTIVATION ---
% \begin{frame}{Motivation \& Research Question}
%     \textbf{A Foundational Idea in International Trade:}
%     \begin{itemize}
%         \item The \textbf{Home-Market Effect} (Linder, 1961; Krugman, 1980): Countries with large domestic demand for a product tend to become net exporters of that product.
%         \item This is a core prediction of \textbf{New Trade Theory}, which emphasizes increasing returns to scale.
%     \end{itemize}
%     \textbf{The Key Empirical Challenge:}
%     \begin{itemize}
%         \item \textbf{Endogeneity.} How to isolate the \textit{causal} effect of demand?
%         \item Previous tests used expenditure shares, but expenditure is an equilibrium outcome affected by both supply and demand. This leads to biased and inconclusive results.
%     \end{itemize}
%     \textbf{This Paper's Research Question:}
%     \begin{block}{Question}
%         Can we find a clean, exogenous source of variation in home demand to provide a definitive test of the home-market effect?
%     \end{block}
% \end{frame}

% % --- CONTRIBUTION & STRATEGY ---
% \begin{frame}{Contribution: A Novel Identification Strategy}
%     The paper's core innovation is its empirical strategy, applied to the global pharmaceutical industry.

%     \begin{columns}[T]
%         \begin{column}{0.5\textwidth}
%             \textbf{Step 1: Predict Disease}
%             \begin{itemize}
%                 \item Use a country's exogenous \textbf{demographic composition} (age, gender).
%                 \item Combine with global disease prevalence rates for each demographic group.
%             \end{itemize}
%         \end{column}
%         \begin{column}{0.5\textwidth}
%             \textbf{Step 2: Predict Demand}
%             \begin{itemize}
%                 \item This creates a \textbf{“Predicted Disease Burden” (PDB)} for each country and disease.
%                 \item PDB serves as a plausibly \textbf{exogenous demand shifter} for drugs treating that disease.
%             \end{itemize}
%         \end{column}
%     \end{columns}

%     \begin{alertblock}{The Logic}
%         Demographics $\rightarrow$ Predicted Disease $\rightarrow$ Exogenous Home Demand for Drugs
%     \end{alertblock}

%     This allows for the first clean test of the home-market effect's causal claim.
% \end{frame}

% % --- THEORETICAL FRAMEWORK ---
% \begin{frame}{Theoretical Framework: Weak vs. Strong Effects}
%     The authors build a model to derive two precise, testable hypotheses. Bilateral sales from exporter \textit{i} to importer \textit{j} are modeled as:

%     \begin{center}
%         $ \ln(x_{ij}^n) = \text{Fixed Effects} + \beta_M \ln(\text{Demand}_j^n) + \beta_x \ln(\text{Demand}_i^n) + \varepsilon_{ij}^n $
%     \end{center}

%     \begin{enumerate}
%         \item \textbf{Weak Home-Market Effect: $ \beta_x > 0 $}
%               \begin{itemize}
%                   \item A larger home demand in the exporting country (\textit{i}) increases its foreign sales.
%                   \item \textbf{Crucial Insight:} This effect is only possible with \textbf{economies of scale} (a downward-sloping supply curve). Without them, higher home demand would raise prices and \textit{reduce} exports.
%               \end{itemize}

%         \item \textbf{Strong Home-Market Effect: $ \beta_x > \beta_M $}
%               \begin{itemize}
%                   \item The boost to exports from home demand is larger than the boost to imports from foreign demand.
%                   \item This implies the country becomes a \textbf{net exporter}.
%                   \item Requires \textit{sufficiently strong} economies of scale.
%               \end{itemize}
%     \end{enumerate}
% \end{frame}

% % --- DATA ---
% \begin{frame}{Data Sources}
%     The analysis links two main datasets for a 2012 cross-section:

%     \begin{block}{Bilateral Sales Data}
%         \textbf{IMS MIDAS Dataset}
%         \begin{itemize}
%             \item Comprehensive data on pharmaceutical sales for 56 countries.
%             \item Covers $>$20,000 molecules from $~$2,650 firms.
%             \item Authors match firms to headquarter countries to get origin-destination sales flow.
%         \end{itemize}
%     \end{block}

%     \begin{block}{Demand Shifter Data}
%         \begin{itemize}
%             \item \textbf{WHO Global Burden of Disease (GBD):} Disease incidence by demographic group (age/gender).
%             \item \textbf{U.S. Census Bureau:} Population data for each demographic group in each country.
%         \end{itemize}
%         $\rightarrow$ Together, these are used to construct the \textbf{Predicted Disease Burden (PDB)}.
%     \end{block}
% \end{frame}

% % --- BASELINE RESULTS ---
% \begin{frame}{Baseline Results (Table III)}
%     The regression estimates confirm the theory's predictions.

%     \begin{center}
%         \textbf{Dependent Variable: Log(Bilateral Sales)}
%         \begin{tabular}{lc}
%             \hline
%             \textbf{Variable}                 & \textbf{Coefficient (Std. Err.)} \\
%             \hline
%             Log(PDB, destination) ($\beta_M$) & 0.545 (0.107)                    \\
%             Log(PDB, origin) ($\beta_x$)      & \textbf{0.928 (0.123)}           \\
%             \hline
%             Observations                      & 19,150                           \\
%             \hline
%         \end{tabular}
%     \end{center}

%     \begin{alertblock}{Test 1: Weak Home-Market Effect ($\beta_x > 0$)}
%         The coefficient on origin PDB is 0.928 and highly significant. \\ \textbf{Result: The weak HME is strongly supported.}
%     \end{alertblock}

%     \begin{exampleblock}{Test 2: Strong Home-Market Effect ($\beta_x > \beta_M$)}
%         An F-test rejects the null hypothesis that $\beta_x \le \beta_M$ (p-value = 0.018). \\ \textbf{Result: The strong HME is also supported.}
%     \end{exampleblock}

%     These results are robust to dozens of sensitivity checks.
% \end{frame}

% % --- STRUCTURAL RESULTS ---
% \begin{frame}{Mechanism: Disentangling Supply \& Demand}
%     Is the effect driven by economies of scale (as theory predicts) or something else (e.g., inelastic demand)? The authors use an IV strategy to estimate the structural elasticities.

%     \begin{columns}[T]
%         \begin{column}{0.5\textwidth}
%             \textbf{Demand Elasticity ($ \varepsilon^x $)}
%             \begin{itemize}
%                 \item Estimated using trade distance and price data.
%                 \item Result: $ \varepsilon^x \approx 6.2 $
%                 \item \textbf{Conclusion:} Demand is highly elastic. This is \textit{not} driving the result.
%             \end{itemize}
%         \end{column}
%         \begin{column}{0.5\textwidth}
%             \textbf{Supply Elasticity ($ \varepsilon^s $)}
%             \begin{itemize}
%                 \item Estimated using PDB as an instrument for total sales.
%                 \item Result: $ \varepsilon^s \approx -7.8 $
%                 \item \textbf{Conclusion:} The supply curve is strongly \textbf{downward-sloping}.
%             \end{itemize}
%         \end{column}
%     \end{columns}

%     \begin{alertblock}{The Verdict}
%         The paper provides direct evidence that the home-market effect is driven by substantial \textbf{industry-level economies of scale}.
%     \end{alertblock}
% \end{frame}

% % --- CONCLUSION ---
% \begin{frame}{Conclusion}
%     \begin{itemize}
%         \item This paper provides the most convincing empirical evidence to date in favor of the \textbf{home-market effect}.

%         \item Its novel identification strategy, using demographic structure to predict exogenous demand, solves a long-standing challenge in the field.

%         \item The findings confirm the central mechanism of \textbf{New Trade Theory}: a large home market fosters economies of scale, which in turn drives export performance.

%         \item In short: \textbf{The more we (are predicted to) die at home, the more we sell abroad.}
%     \end{itemize}
% \end{frame}

% % --- CRITICAL ANALYSIS ---
% \begin{frame}{Critical Analysis \& Discussion}
%     This is an outstanding paper, but there are areas for future research.

%     \begin{block}{Generalizability}
%         The findings are from a single, unique industry (pharmaceuticals). Would the effect be as strong in sectors with lower fixed costs or different regulatory structures?
%     \end{block}

%     \begin{block}{Exports vs. FDI}
%         The sales data combine exports and sales by foreign affiliates. The underlying economies of scale may differ (production vs. R\&D). Future work could try to disentangle these channels.
%     \end{block}

%     \begin{block}{Static Analysis}
%         The cross-sectional design captures the effect at one point in time. A panel analysis, though difficult, could reveal the dynamic evolution of the home-market effect as demographics and industries change.
%     \end{block}
% \end{frame}

% %------------------------------------------------

% \begin{frame}{Motivation: The Home-Market Effect Puzzle}
%     \begin{block}{A Core Idea in ``New Trade Theory''}
%         The Home-Market Effect (HME), hypothesized by Linder (1961) and formalized by Krugman (1980), posits that countries with large domestic demand for a product tend to become net exporters of it.
%     \end{block}

%     \begin{alertblock}{A Major Empirical Challenge}
%         Testing this is notoriously difficult due to endogeneity.
%         \begin{itemize}
%             \item Standard demand proxies (e.g., national expenditure) are equilibrium outcomes.
%             \item A positive supply shock can increase both domestic expenditure and exports, creating a spurious correlation.
%             \item \textbf{Key Question:} How can we isolate an exogenous shock to home demand?
%         \end{itemize}
%     \end{alertblock}
% \end{frame}

% %------------------------------------------------

% \begin{frame}{This Paper's Contribution: A Novel Identification Strategy}
%     \begin{block}{The Core Idea}
%         Use a country's exogenous demographic structure as a predictor for its demand for specific pharmaceuticals.
%         \begin{itemize}
%             \item This is a spatial analogue to the time-series strategy of Acemoglu \& Linn (2004).
%             \item It creates a plausibly exogenous demand shifter: the \textbf{Predicted Disease Burden (PDB)}.
%         \end{itemize}
%     \end{block}

%     \begin{exampleblock}{The PDB Instrument}
%         For each country $i$ and disease $n$, the PDB is constructed as:
%         $$ (PDB)_{i}^{n}=\sum_{a,g}\left[\text{pop}_{iag}\times\left(\frac{\sum_{k\ne i}\text{burden}_{kag}^{n}}{\sum_{k\ne i}\text{pop}_{kag}}\right)\right] $$
%         \begin{itemize}
%             \item $\text{pop}_{iag}$: Population of age-gender group $(a,g)$ in country $i$.
%             \item The ratio is the average disease burden for group $(a,g)$ in the \textit{rest of the world} ($k \neq i$).
%             \item This "leave-one-out" construction enhances exogeneity.
%         \end{itemize}
%     \end{exampleblock}
% \end{frame}

% %------------------------------------------------

% \begin{frame}{Sources of Variation for the PDB Instrument}
%     \begin{columns}[c]
%         \column{0.5\textwidth}
%         \textbf{Variation in Demographics Across Countries}
%         \begin{figure}
%             \includegraphics[width=\textwidth]{example-image-a}
%             \caption{Share of population under age 60 varies dramatically (e.g., Japan vs. UAE). Based on Figure IV in Costinot et al. (2019).}
%         \end{figure}

%         \column{0.5\textwidth}
%         \textbf{Variation in Disease Profile Across Demographics}
%         \begin{figure}
%             \includegraphics[width=\textwidth]{example-image-b}
%             \caption{Share of global disease burden borne by population under 60 varies by disease (e.g., Alzheimer's vs. Whooping Cough). Based on Figure V.}
%         \end{figure}
%     \end{columns}
% \end{frame}

% %------------------------------------------------

% \begin{frame}{The Theoretical Mechanism: Why Scale Economies are Key}
%     \begin{columns}[c]
%         \column{0.33\textwidth}
%         \textbf{No HME (Neoclassical)}
%         \begin{figure}
%             \includegraphics[width=\textwidth]{example-image-c}
%             \caption{Higher home demand $\rightarrow$ Higher Price $\rightarrow$ Lower Exports. Based on Figure I.}
%         \end{figure}

%         \column{0.33\textwidth}
%         \textbf{Weak HME}
%         \begin{figure}
%             \includegraphics[width=\textwidth]{example-image-a}
%             \caption{With scale economies, higher demand $\rightarrow$ Lower Price $\rightarrow$ Higher Exports ($\beta_X > 0$). Based on Figure II.}
%         \end{figure}

%         \column{0.33\textwidth}
%         \textbf{Strong HME}
%         \begin{figure}
%             \includegraphics[width=\textwidth]{example-image-b}
%             \caption{With strong scale economies, Exports rise more than Imports ($\beta_X > \beta_M$). Based on Figure III.}
%         \end{figure}
%     \end{columns}
%     \begin{alertblock}{}
%         A home-market effect requires a downward-sloping industry supply curve, i.e., increasing returns to scale.
%     \end{alertblock}
% \end{frame}

% %------------------------------------------------

% \begin{frame}{Empirical Strategy: Baseline Specification}
%     The paper tests the HME by estimating a gravity-style equation:
%     \begin{block}{Equation (16)}
%         $$ \ln x_{ij}^{n}=\delta_{ij}+\delta^{n}+\tilde{\beta}_{M}\ln(PDB)_{j}^{n}+\tilde{\beta}_{X}\ln(PDB)_{i}^{n}+\tilde{\epsilon}_{ij}^{n} $$
%     \end{block}
%     \begin{itemize}
%         \item $x_{ij}^{n}$: Bilateral sales from origin $i$ to destination $j$ for disease $n$.
%         \item $(PDB)_{i}^{n}$: Predicted Disease Burden in the origin (exporter).
%         \item $(PDB)_{j}^{n}$: Predicted Disease Burden in the destination (importer).
%         \item $\delta_{ij}$: Origin-Destination fixed effects (absorbs distance, etc.).
%         \item $\delta^{n}$: Disease fixed effects (absorbs global disease size).
%     \end{itemize}
%     \begin{exampleblock}{Hypothesis Tests}
%         \begin{itemize}
%             \item \textbf{Weak HME}: Test if $\tilde{\beta}_X > 0$.
%             \item \textbf{Strong HME}: Test if $\tilde{\beta}_X > \tilde{\beta}_M$.
%         \end{itemize}
%     \end{exampleblock}
% \end{frame}

% % %------------------------------------------------

% \begin{frame}{Robustness of the Main Finding}
%     The main result is robust to a wide array of alternative explanations and specifications.
%     \begin{itemize}
%         \item \textbf{Controlling for Confounders (Table IV):} Results hold after controlling for interactions between disease characteristics and country characteristics like GDP per capita.
%         \item \textbf{Supply-Side Stories (Table V):}
%               \begin{itemize}
%                   \item The effect remains when controlling for US NIH subsidies.
%                   \item The weak HME is present even when looking only at \textbf{generic drugs}, where R\&D-related scale economies should be weaker.
%               \end{itemize}
%         \item \textbf{Spatial Correlation of Demand (Table VI):} Results are not driven by demand in neighboring countries.
%         \item \textbf{Pricing-to-Market (Table VII):} The weak HME holds within the EU, where parallel trade limits price discrimination.
%         \item \textbf{Zero Trade Flows (Table VIII):} Results are robust to using PPML estimation, which includes observations with zero sales.
%     \end{itemize}
% \end{frame}

% %------------------------------------------------

% \begin{frame}{Structural Results: Quantifying Economies of Scale}
%     The paper goes beyond the reduced-form test to estimate the structural supply elasticity ($\epsilon^s$).
%     \begin{block}{IV Strategy}
%         \begin{itemize}
%             \item \textbf{Goal:} Estimate the elasticity of the industry supply curve.
%             \item \textbf{Problem:} Total sales (scale) is endogenous to supply shocks.
%             \item \textbf{Solution:} Instrument for a country's total sales in a disease category ($\ln r_i^n$) with the exogenous demand shifter ($\ln(PDB)_i^n$).
%         \end{itemize}
%     \end{block}

%     \begin{exampleblock}{Key Structural Finding (Table X)}
%         The IV estimation yields a supply elasticity of:
%         $$\epsilon^s = -7.833$$
%         \begin{itemize}
%             \item The negative sign provides direct evidence of a \textbf{downward-sloping supply curve}, confirming the existence of significant industry-level economies of scale.
%             \item This magnitude is about 25\% smaller than the benchmark prediction from a standard Krugman (1980) model.
%         \end{itemize}
%     \end{exampleblock}
% \end{frame}

% %------------------------------------------------

% \begin{frame}{Conclusion \& Critical Assessment}
%     \begin{block}{Summary of Contributions}
%         \begin{itemize}
%             \item Provides a simple, powerful, and credible test of the home-market effect.
%             \item Solves a major identification problem using a novel demographically-driven instrument (PDB).
%             \item Finds strong evidence for both weak and strong HME in the global pharmaceutical industry.
%             \item Quantifies the underlying economies of scale, finding $\epsilon^s = -7.833$.
%         \end{itemize}
%     \end{block}

%     \begin{alertblock}{Points for Discussion \& Future Research}
%         \begin{itemize}
%             \item \textbf{External Validity:} The pharmaceutical industry is an ideal setting for HME. Would these results generalize to industries with weaker scale economies?
%             \item \textbf{FDI vs. Exports:} The data combine exports and sales from local affiliates. Is this a "production HME" or a "headquarters HME" (driven by R\&D/marketing)?
%             \item \textbf{The "Black Box" of Scale:} The test confirms the existence of scale economies but not their source (e.g., R\&D spillovers, love-of-variety). The welfare implications depend on the source.
%         \end{itemize}
%     \end{alertblock}
% \end{frame}

% %------------------------------------------------

% \begin{frame}
%     \begin{center}
%         \Huge Thank You
%     \end{center}
% \end{frame}

% \end{document}

\documentclass[11pt]{beamer}
\usetheme{Boadilla}
\usecolortheme{dolphin}
\usefonttheme{professionalfonts}
\usepackage[utf8]{inputenc}
\usepackage{amsmath, booktabs}
\usepackage{graphicx}
\usepackage{hyperref}

\title[The More We Die, The More We Sell?]{The More We Die, The More We Sell? \\ A Simple Test of the Home-Market Effect}
\author{Arnaud Costinot, Dave Donaldson, Margaret Kyle, Heidi Williams}
\institute{The Quarterly Journal of Economics (2019)}
\date{\today}

\begin{document}

% --- TITLE SLIDE ---
\begin{frame}
    \titlepage
    \begin{center}
        \vspace{0.5cm}
        \small{Presented by: Jingle Fu}
    \end{center}
\end{frame}

% --- INTRODUCTION ---
\begin{frame}{Introduction: The Home-Market Effect Puzzle}
    The Home-Market Effect (HME) stands as one of the central predictions of New Trade Theory:

    \begin{itemize}
        \item Countries tend to export products for which they have large domestic markets (Linder, 1961; Krugman, 1980)
        \item This arises from economies of scale: larger home markets enable firms to achieve lower average costs
        \item The theoretical intuition is compelling, but empirical verification has proven remarkably difficult
    \end{itemize}


    \textbf{The Fundamental Identification Challenge:}

    Traditional demand measures (expenditure shares) are equilibrium outcomes simultaneously determined by both supply and demand forces. A positive correlation between home demand and exports could reflect:

    \begin{itemize}
        \item True causal effect of demand on exports (the HME)
        \item Reverse causality: productive advantages lead to both domestic sales and exports
        \item Omitted variables affecting both supply and demand
    \end{itemize}
\end{frame}

% --- RESEARCH QUESTION ---
\begin{frame}{Research Question and Contribution}
    \textbf{Core Research Question:}

    Can we find a credible source of exogenous variation in home demand to provide a definitive causal test of the home-market effect?

    \textbf{This Paper's Contribution:}

    \begin{itemize}
        \item Develops a novel identification strategy using demographic variation to predict disease burdens
        \item Provides the first credible test of the causal mechanism behind the HME
        \item Quantifies the role of economies of scale in driving the effect
        \item Distinguishes between "weak" and "strong" versions of the HME
    \end{itemize}

    The paper focuses on the global pharmaceutical industry, where the link between demographics, disease prevalence, and drug demand is particularly clear.
\end{frame}

% --- THEORETICAL FRAMEWORK ---
\begin{frame}{Theoretical Framework: A General Approach}
    The authors develop a flexible theoretical model that nests various market structures:

    \begin{itemize}
        \item \textbf{Demand}: Nested structure across diseases and countries of origin
        \item \textbf{Supply}: Industry-level supply curve $s_i^n = \eta_i^n s(p_i^n)$ with elasticity $\varepsilon^s$
        \item \textbf{Key Insight}: The HME requires $\varepsilon^s < 0$ (economies of scale)
        \item \textbf{Market Structures}: Model accommodates perfect competition, monopolistic competition, endogenous innovation, and regulated markets
    \end{itemize}

    Through log-linearization around a symmetric equilibrium, the model yields a testable equation for bilateral sales:
    \[
        \ln(x_{ij}^n) = \delta_{ij} + \delta^n + \beta_M \ln(\theta_j^n) + \beta_X \ln(\theta_i^n) + \varepsilon_{ij}^n
    \]
    where $x_{ij}^n$ represents sales from country $i$ to country $j$ for drugs treating disease $n$.
\end{frame}

% --- HYPOTHESES ---
\begin{frame}{Testable Hypotheses}
    The theoretical framework generates two precise, testable predictions:

    \textbf{Hypothesis 1: Weak Home-Market Effect}

    \begin{itemize}
        \item Prediction: $\beta_X > 0$
        \item Interpretation: Higher home demand increases a country's exports
        \item Economic intuition: With economies of scale, larger home markets reduce average costs, making firms more competitive abroad
        \item Crucial implication: $\beta_X > 0$ \emph{only} occurs when $\varepsilon^s < 0$ (economies of scale present)
    \end{itemize}

    \textbf{Hypothesis 2: Strong Home-Market Effect}

    \begin{itemize}
        \item Prediction: $\beta_X > \beta_M$
        \item Interpretation: The export boost from home demand exceeds the import boost from foreign demand
        \item Economic intuition: Country becomes a net exporter in the product category
        \item Requires sufficiently strong economies of scale
    \end{itemize}
\end{frame}

% --- IDENTIFICATION STRATEGY ---
\begin{frame}{Identification Strategy: The Core Innovation}
    The paper's most important contribution is solving the endogeneity problem through a novel instrument:

    \textbf{Predicted Disease Burden (PDB)}
    \[
        (PDB)_{i}^{n} = \sum_{\text{age }a,\ \text{gender }g} \left[\text{pop}_{iag} \times \left(\frac{\sum_{k \neq i} \text{burden}_{kag}^{n}}{\sum_{k \neq i} \text{pop}_{kag}}\right)\right]
    \]
    measures the average disease burden per capita from disease $n$ for gender $g$ and age group $a$.
    \textbf{Construction Details:}

    \begin{itemize}
        \item Uses country $i$'s demographic composition (age and gender distribution)
        \item Combines with global disease prevalence patterns for each demographic group
        \item "Leave-one-out" construction: excludes country $i$ itself when calculating global disease rates
        \item Creates a plausibly exogenous predictor of disease-specific drug demand
    \end{itemize}

    % \textbf{Identifying Assumption}: PDB is uncorrelated with supply-side capabilities $\eta_i^n$ after controlling for fixed effects.
\end{frame}

% --- IDENTIFICATION JUSTIFICATION ---
\begin{frame}{Why This Identification Strategy Works}

    \textbf{Exogeneity Justification:}

    \begin{itemize}
        \item Demographic composition is predetermined and slow-moving
        \item Global disease patterns are largely exogenous to any single country's economic conditions
        \item "Leave-one-out" construction prevents mechanical correlation with country-specific factors
        \item Instrument relevance: demographics strongly predict disease burdens, which drive drug demand
    \end{itemize}

\end{frame}

\begin{frame}{Sources of Variation for the PDB Instrument}
    \begin{columns}[c]
        \column{0.5\textwidth}
        \textbf{Variation in Demographics Across Countries}
        \begin{figure}
            \includegraphics[width=\textwidth]{image-a.jpeg}
            \caption{Share of population under age 60 varies dramatically.
                Based on Figure IV in Costinot et al. (2019).}
        \end{figure}

        \column{0.5\textwidth}
        \textbf{Variation in Disease Profile Across Demographics}
        \begin{figure}
            \includegraphics[width=\textwidth]{image-b.jpeg}
            \caption{\footnotesize {The labeled global burden of disease (GBD) codes correspond to the following diseases:
                    U087: Alzheimer's disease and other dementia;
                    U078: other neoplasms;
                    U089: multiple sclerosis;
                    U086: alcohol use disorders;
                    and U012: whooping cough.}}
        \end{figure}
    \end{columns}
\end{frame}

% --- DATA SOURCES ---
\begin{frame}{Data Sources and Measurement}

    \textbf{Pharmaceutical Sales Data (IMS MIDAS, 2012)}

    \begin{itemize}
        \item Comprehensive global pharmaceutical sales data
        \item 56 countries, covering over 20,000 molecules
        \item Bilateral sales flows between country pairs
        \item Origin country defined as firm headquarters location
        \item Drugs classified by ATC codes, mapped to specific diseases
        \item Important limitation: combines both exports and sales by foreign affiliates
    \end{itemize}

    \textbf{Disease Burden and Demographic Data}

    \begin{itemize}
        \item WHO Global Burden of Disease: DALYs by disease, age, gender, and country
        \item U.S. Census Bureau International Database: Population by age and gender for each country
        \item Hand-coded mapping from ATC drug classifications to WHO disease categories
    \end{itemize}
\end{frame}

\begin{frame}{Baseline Model}
    \begin{block}{Equation (15) \& (16)}
        $$ \ln \theta _i^n = \gamma \ln(PDB)_{j}^{n} + \gamma_i^n $$
        $$ \ln x_{ij}^{n}=\delta_{ij}+\delta^{n}+\tilde{\beta}_{M}\ln(PDB)_{j}^{n}+\tilde{\beta}_{X}\ln(PDB)_{i}^{n}+\tilde{\epsilon}_{ij}^{n} $$
    \end{block}
    \begin{itemize}
        \item $x_{ij}^{n}$: Bilateral sales from origin $i$ to destination $j$ for disease $n$.
        \item $(PDB)_{i}^{n}$: Predicted Disease Burden in the origin (exporter).
        \item $(PDB)_{j}^{n}$: Predicted Disease Burden in the destination (importer).
        \item $\delta_{ij}$: Origin-Destination fixed effects (absorbs distance, etc.).
        \item $\delta^{n}$: Disease fixed effects (absorbs global disease size).
    \end{itemize}
\end{frame}

% --- BASELINE RESULTS ---
\begin{frame}{Baseline Results: Evidence for Home-Market Effect}

    \begin{center}
        \small
        \begin{tabular}{lcc}
            \toprule
            \textbf{Variable}                                  & \textbf{Coefficient}                            & \textbf{(Std. Err.)} \\
            \midrule
            $\ln(\text{PDB, destination})$ ($\tilde{\beta}_M$) & 0.545                                           & (0.107)              \\
            $\ln(\text{PDB, origin})$ ($\tilde{\beta}_X$)      & 0.928                                           & (0.123)              \\
            \midrule
            Observations                                       & \multicolumn{2}{c}{19,150}                                             \\
            Fixed Effects                                      & \multicolumn{2}{c}{Origin×Destination, Disease}                        \\
            \bottomrule
        \end{tabular}
    \end{center}

    \textbf{Hypothesis Test 1: Weak Home-Market Effect}

    \begin{itemize}
        \item Null hypothesis: $\tilde{\beta}_X \leq 0$
        \item Result: $\tilde{\beta}_X = 0.928$, p-value $< 0.001$
        \item Interpretation: Strong evidence that home demand increases exports
        \item Economic significance: 10\% increase in home PDB associated with 9.3\% increase in exports
    \end{itemize}

    \textbf{Hypothesis Test 2: Strong Home-Market Effect}

    \begin{itemize}
        \item Null hypothesis: $\tilde{\beta}_X \leq \tilde{\beta}_M$
        \item Result: F-test p-value = 0.018
        \item Interpretation: Evidence that countries become net exporters
    \end{itemize}
\end{frame}

% --- ECONOMIC INTERPRETATION ---
\begin{frame}{Economic Interpretation of Baseline Results}

    \textbf{What Do These Coefficients Mean?}

    The positive and significant $\tilde{\beta}_X$ provides the credible causal evidence for the home-market effect. This means:

    \begin{itemize}
        \item Countries systematically export more of drugs for which they have higher \emph{exogenous} home demand
        \item This pattern cannot be explained by supply-side advantages alone
        \item The effect is economically substantial and statistically robust
    \end{itemize}

    \textbf{Why is This Evidence Causal?}

    \begin{itemize}
        \item PDB provides exogenous variation in demand uncorrelated with supply conditions
        \item Extensive fixed effects structure controls for confounding factors
        \item The identification strategy breaks the fundamental endogeneity problem that plagued previous tests
    \end{itemize}

    \textbf{Policy Implication}: Countries may develop export advantages in industries where they have large domestic markets, supporting strategic trade policy arguments.
\end{frame}

% --- ROBUSTNESS CHECKS ---
\begin{frame}{Robustness Checks: Addressing Alternative Explanations}

    % The paper conducts extensive robustness checks to rule out competing explanations:

    \textbf{Supply-Side Confounds}

    \begin{itemize}
        \item Controls for GDP per capita interacted with disease characteristics
        \item Restricts sample to generic drugs (where R\&D advantages matter less)
        \item Focuses on poorer countries
              % (less likely to have advanced pharmaceutical sectors)
              % \item Result: Weak HME remains strong and significant
    \end{itemize}

    \begin{figure}[h!]
        \centering
        \includegraphics[width=0.6\textwidth]{Tab4.png}
    \end{figure}
\end{frame}

\begin{frame}
    \begin{figure}[h!]
        \centering
        \includegraphics[width=0.8\textwidth]{Tab5.png}
    \end{figure}
\end{frame}

\begin{frame}

    \textbf{Spatial Correlation Concerns}

    \begin{itemize}
        \item Controls for disease burden in neighboring countries
        \item Restricts to country pairs that are geographically distant
        \item Result: No evidence that spatial correlation drives the results
    \end{itemize}

    \begin{figure}[h!]
        \centering
        \includegraphics[width=0.8\textwidth]{Tab6.png}
    \end{figure}
\end{frame}

\begin{frame}

    \textbf{Market Structure and Regulation}

    \begin{itemize}
        \item Examines EU markets separately (less price discrimination)
        \item Controls for US NIH research funding
        \item Uses PPML to account for zero trade flows
        \item Result: HME patterns persist across specifications
    \end{itemize}

    \begin{figure}[h!]
        \centering
        \includegraphics[width=0.7\textwidth]{Tab7.png}
    \end{figure}
\end{frame}

% --- MECHANISM ANALYSIS ---
\begin{frame}{Mechanism Analysis: Disentangling Scale Economies}

    A key question: Is the HME driven by economies of scale (as theory predicts) or could alternative mechanisms explain the results?

    \textbf{Estimating Demand Elasticity}

    \begin{itemize}
        \item Method: Uses gravity equation combined with micro-level price data
        \item Estimates how trade costs affect both quantities and prices
        \item Result: $\varepsilon^x \approx 6.22$ (demand is highly elastic)
        \item Implication: Inelastic demand cannot explain the HME findings
    \end{itemize}

    \textbf{Estimating Supply Elasticity}

    \begin{itemize}
        \item Method: Uses PDB as instrument for total production scale
        \item Estimates relationship between scale and producer prices
        \item Result: $\varepsilon^s \approx -7.83$ (strong economies of scale)
        \item Implication: Supply curve is downward sloping, confirming scale economies
    \end{itemize}

    \textbf{Conclusion}: The HME is indeed driven by substantial industry-level economies of scale, exactly as predicted by theory.
\end{frame}

% --- STRUCTURAL RESULTS ---
\begin{frame}{Structural Results: Quantitative Assessment}

    The estimated supply elasticity of $\varepsilon^s \approx -7.83$ provides several important insights:

    \textbf{Magnitude of Scale Economies}

    \begin{itemize}
        \item The negative sign confirms downward-sloping supply (increasing returns)
        \item The magnitude indicates substantial scale economies in pharmaceuticals
        \item Comparison: About 25\% smaller than Krugman's (1980) benchmark prediction ($\varepsilon^s = -\varepsilon^x$)
        \item Suggests real-world scale economies are large but not as extreme as in simplest models
    \end{itemize}

    \textbf{Welfare Implications}

    \begin{itemize}
        \item Scale economies create potential gains from trade through lower average costs
        \item Home-market effects can shape comparative advantage and specialization patterns
        \item Trade policy that expands market access can yield efficiency benefits
    \end{itemize}

    % \textbf{Industry Characteristics}: The findings align with known features of pharmaceuticals:
    % high fixed R\&D costs, significant scale economies in production and distribution.
\end{frame}

% --- CRITICAL ASSESSMENT ---
\begin{frame}{Critical Assessment: Strengths of the Paper}

    \textbf{Methodological Innovation}

    \begin{itemize}
        \item Solves the fundamental endogeneity problem that plagued previous HME tests
        \item PDB instrument is plausible, and well-justified
        \item Theoretical framework is general and accommodates multiple market structures
    \end{itemize}

    \textbf{Empirical Rigor}

    \begin{itemize}
        \item Comprehensive robustness checks address most plausible alternatives
              % \item Goes beyond reduced-form to estimate structural parameters
        \item Clear distinction between weak and strong HME provides nuanced test
    \end{itemize}

    % \textbf{Substantive Contribution}

    % \begin{itemize}
    %     \item Provides the most credible evidence to date for the home-market effect
    %     \item Quantifies the role of scale economies in driving international trade patterns
    %     \item Demonstrates the continued relevance of New Trade Theory insights
    % \end{itemize}

    This paper likely represents the definitive empirical test of the home-market effect in the pharmaceutical industry.
\end{frame}

% --- LIMITATIONS ---
\begin{frame}{Critical Assessment: Limitations and Questions}

    \textbf{External Validity Concerns}

    \begin{itemize}
        \item Single-industry focus (pharmaceuticals) limits generalizability
        \item Pharmaceuticals have unusual characteristics: high regulation, patents, inelastic demand in some cases
        \item Would similar patterns hold in industries with weaker scale economies?
    \end{itemize}

    \textbf{Measurement Issues}

    \begin{itemize}
        \item Sales data combine exports and foreign affiliate sales
        \item Cannot distinguish between "production HME" and "headquarters HME"
        \item Where exactly do the scale economies operate: production, R\&D, or marketing?
    \end{itemize}

    \textbf{Strong HME Evidence}

    \begin{itemize}
        \item Evidence for strong HME ($\beta_X > \beta_M$) is less robust than for weak HME
        \item Statistical significance disappears in some specifications
        \item Suggests the strong version may be context-dependent
    \end{itemize}

    % \textbf{Black Box of Scale Economies}: While the paper confirms scale economies exist,
    % it doesn't identify their precise micro-foundations.
\end{frame}

% --- POLICY IMPLICATIONS ---
\begin{frame}{Policy Implications and Extensions}

    \textbf{Trade Policy Implications}

    \begin{itemize}
        \item Supports the theoretical possibility of ``import protection as export promotion''
        \item Industrial policy may be effective in sectors with significant scale economies
        \item However: Welfare effects depend on market structure and policy details
        \item Strategic trade policy arguments gain some empirical support
    \end{itemize}
\end{frame}

\begin{frame}

    \textbf{Industrial Organization}

    \begin{itemize}
        \item Scale economies appear to operate at the industry level rather than firm level
        \item Has implications for antitrust policy in innovation-intensive industries
        \item Suggests potential tradeoffs between competition and efficiency
    \end{itemize}

    \textbf{Development Economics}

    \begin{itemize}
        \item Countries may develop comparative advantage based on domestic market characteristics
        \item Demographic transitions could shape future export specialization patterns
              % \item Has implications for long-run development strategies
    \end{itemize}
\end{frame}

% % --- RESEARCH AGENDA ---
% \begin{frame}{Future Research Agenda}

%     \textbf{Generalizability Tests}

%     \begin{itemize}
%         \item Apply similar identification strategies to other industries
%         \item Test whether HME patterns vary with industry characteristics
%         \item Examine how scale economy sources affect the strength of HME
%     \end{itemize}

%     \textbf{Micro-Foundations}

%     \begin{itemize}
%         \item Disentangle different sources of scale economies: R\&D, production, marketing
%         \item Examine firm-level dynamics behind industry-level patterns
%         \item Use richer data on production location vs. headquarters location
%     \end{itemize}

%     \textbf{Dynamic Extensions}

%     \begin{itemize}
%         \item Panel analysis of how HME evolves over time
%         \item Study how demographic changes affect future comparative advantage
%         \item Examine path dependence in export specialization patterns
%     \end{itemize}

%     \textbf{Methodological Extensions}

%     \begin{itemize}
%         \item Develop instruments for other industries and contexts
%         \item Incorporate general equilibrium effects more explicitly
%         \item Bridge to quantitative trade models for counterfactual analysis
%     \end{itemize}
% \end{frame}

% --- CONCLUSION ---
% \begin{frame}{Conclusion}

%     \textbf{Summary of Contributions}

%     \begin{itemize}
%         \item Provides the first credible causal evidence for the home-market effect
%         \item Develops an innovative identification strategy that solves the fundamental endogeneity problem
%         \item Quantifies the role of scale economies in driving the effect
%         \item Demonstrates the continued relevance of New Trade Theory for understanding real-world trade patterns
%     \end{itemize}

%     \textbf{Broader Significance}

%     \begin{itemize}
%         \item Shows how clever empirical strategies can resolve long-standing theoretical puzzles
%         \item Demonstrates the value of industry-specific studies for testing general theories
%         \item Provides a template for testing other economic theories with identification challenges
%     \end{itemize}

%     % \textbf{Final Assessment}: This paper represents a methodological breakthrough in international trade empirics and provides compelling evidence for one of the field's central theoretical predictions.
% \end{frame}

\end{document}