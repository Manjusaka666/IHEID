\documentclass[a4paper,12pt]{article} % This defines the style of your paper

\usepackage[top = 2.5cm, bottom = 2.5cm, left = 2.5cm, right = 2.5cm]{geometry} 

% Unfortunately, LaTeX has a hard time interpreting German Umlaute. The following two lines and packages should help. If it doesn't work for you please let me know.
\usepackage[T1]{fontenc}
\usepackage[utf8]{inputenc}
\usepackage{pifont}
% \usepackage{ctex}
\usepackage{amsthm, amsmath, amssymb, mathrsfs,mathtools}

% Defining a new theorem style without italics
\newtheoremstyle{nonitalic}% name
  {\topsep}% Space above
  {\topsep}% Space below
  {\upshape}% Body font
  {}% Indent amount
  {\bfseries}% Theorem head font
  {.}% Punctuation after theorem head
  {.5em}% Space after theorem head
  {}% Theorem head spec (can be left empty, meaning ‘normal`)
  
\theoremstyle{nonitalic}
% Define new 'solution' environment
\newtheorem{innercustomsol}{Solution}
\newenvironment{solution}[1]
  {\renewcommand\theinnercustomsol{#1}\innercustomsol}
  {\endinnercustomsol}

% Custom counter for the solutions
\newcounter{solutionctr}
\renewcommand{\thesolutionctr}{(\alph{solutionctr})}

% Environment for auto-numbering with custom format
\newenvironment{autosolution}
  {\stepcounter{solutionctr}\begin{solution}{\thesolutionctr}}
  {\end{solution}}


\newtheorem{problem}{Problem}
\usepackage{color}

% The following two packages - multirow and booktabs - are needed to create nice looking tables.
\usepackage{multirow} % Multirow is for tables with multiple rows within one cell.
\usepackage{booktabs} % For even nicer tables.

% As we usually want to include some plots (.pdf files) we need a package for that.
\usepackage{graphicx} 
\usepackage{subfigure}


% The default setting of LaTeX is to indent new paragraphs. This is useful for articles. But not really nice for homework problem sets. The following command sets the indent to 0.
\usepackage{setspace}
\setlength{\parindent}{0in}
\usepackage{longtable}

% Package to place figures where you want them.


\usepackage{float}
\usepackage{placeins}
% The fancyhdr package let's us create nice headers.
\usepackage{fancyhdr}

\usepackage{fancyvrb}

%Code environment 
\usepackage{listings} % Required for insertion of code
\usepackage{threeparttable}
\usepackage{siunitx}        % 数字对齐
\usepackage{caption}        % 标题格式
\usepackage{array}          % 列格式

\usepackage{xcolor} % Required for custom colors

% Define colors for code listing
\definecolor{codegreen}{rgb}{0,0.6,0}
\definecolor{codegray}{rgb}{0.5,0.5,0.5}
\definecolor{codepurple}{rgb}{0.58,0,0.82}
\definecolor{backcolour}{rgb}{0.95,0.95,0.92}

% Code listing style named "mystyle"
\lstdefinestyle{mystyle}{
    backgroundcolor=\color{backcolour},   
    commentstyle=\color{codegreen},
    keywordstyle=\color{magenta},
    numberstyle=\tiny\color{codegray},
    stringstyle=\color{codepurple},
    basicstyle=\ttfamily\footnotesize, % Change to serif font
    columns=fullflexible, % make sure to use fixed-width font, CM typewriter is NOT fixed width
    numbers=left,
    stepnumber=1,
    numbersep=10pt,
    numberfirstline=true,
    numberblanklines=true,
    tabsize=4,
    lineskip=-1.5pt,
    extendedchars=true,
    breaklines=true,
    identifierstyle=, % using emph or index keywords
    showstringspaces=false,
    showtabs=false,
    upquote=false,
    texcl=true % interpet comments as LaTeX
}

\lstset{style=mystyle}


%%%%%%%%%%%%%%%%%%%%%%%%%%%%%%%%%%%%%%%%%%%%%%%%
% 3. Header (and Footer)
%%%%%%%%%%%%%%%%%%%%%%%%%%%%%%%%%%%%%%%%%%%%%%%%

% To make our document nice we want a header and number the pages in the footer.

\pagestyle{fancy} % With this command we can customize the header style.

\fancyhf{} % This makes sure we do not have other information in our header or footer.

\lhead{\footnotesize EI129 International Trade II}% \lhead puts text in the top left corner. \footnotesize sets our font to a smaller size.

%\rhead works just like \lhead (you can also use \chead)
\rhead{\footnotesize Jingle Fu} %<---- Fill in your lastnames.

% Similar commands work for the footer (\lfoot, \cfoot and \rfoot).
% We want to put our page number in the center.
\cfoot{\footnotesize \thepage}
\IfFileExists{upquote.sty}{\usepackage{upquote}}{}

\begin{document}

\thispagestyle{empty} % This command disables the header on the first page. 

\begin{tabular}{p{15.5cm}} % This is a simple tabular environment to align your text nicely 
    {\large \bf EI129 International Trade II}    \\
    The Graduate Institute, Spring 2025, Yuan Zi \\
    \hline % \hline produces horizontal lines.
    \\
\end{tabular} % Our tabular environment ends here.

\vspace*{0.3cm} % Now we want to add some vertical space in between the line and our title.

\begin{center} % Everything within the center environment is centered.
    {\Large \bf Applied Work} % <---- Don't forget to put in the right number
    \vspace{2mm}

    % YOUR NAMES GO HERE
    {\bf Jingle Fu} % <---- Fill in your names here!

\end{center}

\vspace{0.4cm}
\setstretch{1.2}


\section*{Question a: Comment on Rose's (2004) Quotation}

Rose's (2004) finding that currency unions triple trade $(\exp(1.21) > 3)$ is economically implausible for several reasons:

In a structural gravity framework,
trade flows are determined by trade costs $\tau$ raised to the power of the trade elasticity $1-\sigma$.
Assuming a standard elasticity of substitution $\sigma = 6$ (implying a trade elasticity of $-5$),
a tripling of trade volumes would require a reduction in trade costs $\tau$ such that $\tau^{-5} \approx 3.35$,
which implies $\tau \approx 0.78$. This suggests that currency unions reduce total trade costs
(including freight, tariffs, and non-tariff barriers) by approximately 22\%.
Given that currency unions primarily eliminate exchange rate volatility—a relatively small component of total trade costs—such a magnitude is unrealistic.

Furthermore, the estimate likely suffers from severe omitted variable bias.
Currency unions are not randomly assigned;
they are typically formed between countries with deep historical, political, and colonial ties.
Cross-sectional estimations that fail to control for these unobserved bilateral affinities conflate the impact of the currency union with the impact of these pre-existing ties,
thereby biasing the coefficient upwards.
As noted in subsequent meta-analyses (e.g., Rose \& Stanley, 2005),
correcting for publication bias and controlling for omitted heterogeneity significantly reduces the estimated effect to a more plausible range of 30-90\%.


\section*{Question b: Mechanisms and Expected Effect Size}

\subsection*{Mechanisms through which currency unions may affect trade:}

\begin{enumerate}
    \item \textbf{Elimination of Exchange Rate Uncertainty}: Reduces risk and hedging costs for traders.
    \item \textbf{Transaction Cost Reduction}: Eliminates currency conversion costs and bid-ask spreads.
    \item \textbf{Price Transparency}: Facilitates price comparisons across countries, enhancing competition.
    \item \textbf{Policy Coordination}: Encourages harmonization of regulations and standards.
    \item \textbf{Signaling Effect}: Signals commitment to deep economic integration.
\end{enumerate}

\subsection*{Reasons for expecting a large effect:}
\begin{itemize}
    \item High initial trade costs and exchange rate volatility.
    \item Trade in differentiated goods with high elasticity of substitution.
    \item Weak monetary policy credibility prior to union.
\end{itemize}

\subsection*{Reasons for expecting a small effect:}
\begin{itemize}
    \item Pre-existing stable exchange rate arrangements (e.g., pegs, currency boards).
    \item Trade in homogeneous goods with low elasticity.
    \item Already high levels of trade integration.
\end{itemize}

\section*{Question c: Data Construction and Summary}

The gravity dataset was constructed by merging bilateral trade flows from DOTS (1960-2005, every 5 years) with GDP and population data from Penn World Tables,
regional trade agreement (RTA) and currency union (CU) dummies,
and time-invariant geographic variables from CEPII.
After cleaning and transformations, the final dataset contains:

\begin{table}[htbp!]\centering
\def\sym#1{\ifmmode^{#1}\else\(^{#1}\)\fi}
\caption{Summary Statistics by Year}
\begin{tabular}{llccccccc}
    \toprule
    Year & Variable & Mean & Median & SD & Min & Max & 5th Pctl & 95th Pctl \\
    \midrule
    \multirow{3}{*}{73} 
      & ldnpt & 4.863000 & 4.687852 & 1.924382 & -0.208976 & 10.66034 & 1.962434 & 8.160343 \\
      & ldsal & 6.001581 & 5.911699 & 1.797126 & -0.857350 & 11.36805 & 3.226227 & 8.849676 \\
      & lemp  & 1.714698 & 1.656417 & 1.559433 & -2.577022 & 6.698169 & -0.693147 & 4.227228 \\
    \midrule
    \multirow{3}{*}{78}
      & ldnpt & 4.332366 & 4.100965 & 2.187433 & -1.389284 & 10.80406 & 1.071379 & 8.066000 \\
      & ldsal & 5.552179 & 5.394975 & 1.965634 & -0.317301 & 11.53196 & 2.383898 & 8.813699 \\
      & lemp  & 1.262622 & 1.131402 & 1.725453 & -3.352407 & 6.732211 & -1.382302 & 4.143135 \\
    \midrule
    \multirow{3}{*}{83}
      & ldnpt & 4.515011 & 4.266926 & 2.308885 & -1.150816 & 11.06617 & 0.954807 & 8.344097 \\
      & ldsal & 5.520881 & 5.340935 & 1.994305 & 0.587181  & 11.31305 & 2.441314 & 8.778238 \\
      & lemp  & 1.134527 & 0.955511 & 1.845955 & -3.772261 & 6.538140 & -1.795768 & 4.244200 \\
    \midrule
    \multirow{3}{*}{88}
      & ldnpt & 4.226000 & 3.836088 & 2.355197 & -1.228260 & 11.11161 & 0.648894 & 8.197926 \\
      & ldsal & 5.688195 & 5.348608 & 2.030207 & -0.243908 & 11.69840 & 2.710890 & 9.217271 \\
      & lemp  & 0.963538 & 0.755403 & 1.867374 & -3.194183 & 6.64079 & -1.795768 & 4.259859 \\
    % \midrule
    % \multirow{3}{*}{Total}
    %   & ldnpt & 4.468996 & 4.212165 & 2.216520 & -1.389284 & 11.11161 & 1.064112 & 8.205291 \\
    %   & ldsal & 5.673087 & 5.529348 & 1.960717 & -0.857350 & 11.69840 & 2.626727 & 8.918221 \\
    %   & lemp  & 1.259177 & 1.114157 & 1.775248 & -3.772261 & 6.732211 & -1.537117 & 4.189655 \\
    \bottomrule
    \end{tabular}
\end{table}


\section*{Question d: Evolution of RTA and CU Shares}

The share of country pairs in RTAs increased steadily from 1.7\% in 1960 to 12.6\% in 2005, while the trade-weighted share of RTAs grew from 9.5\% to 54.7\%. In contrast, the share of CU pairs declined from 3.3\% to 1.5\%, with trade-weighted CU share fluctuating between 0.1\% and 7.7\%.The divergence between pair shares and trade-weighted shares indicates that RTAs and CUs involve countries with systematically different trade volumes.

\textbf{Key trends:}
\begin{itemize}
    \item \textbf{RTAs}: Both pair share and trade share increased, especially after 1995, reflecting the proliferation of regional agreements.
    \item \textbf{CUs}: Pair share declined, but trade share spiked in 1960 (7.7\%) and again in 2000-2005 (2.2-2.5\%), likely due to the Euro's introduction.
\end{itemize}

The visualization (Figure 1) confirms these trends, showing stronger growth for RTAs than for CUs.

\begin{figure}[h!]
    \centering
    \subfigure[Pair share]{%
        \includegraphics[width=0.48\textwidth]{d.png}%
        \label{fig:rta_cu_shares_a}
    }
    \hfill
    \subfigure[Trade-weighted share]{%
        \includegraphics[width=0.48\textwidth]{d_2.png}%
        \label{fig:rta_cu_shares_b}
    }
    \caption{Evolution of RTA and CU Shares (1960-2005)}
    \label{fig:rta_cu_shares}
\end{figure}


\section*{Question e: Currency Union Effect Estimation}

\subsection*{e(a) Theoretical Expectation from Krugman (1980) Model}

The gravity equation derived from Krugman (1980) monopolistic competition model is:

\[
    T_{ij} = \frac{Y_i Y_j}{Y_w} \left( \frac{\tau_{ij}}{P_i P_j} \right)^{1-\sigma}
\]

Taking logs:

\[
    \ln T_{ij} = \ln Y_i + \ln Y_j - \ln Y_w + (1-\sigma) \ln \tau_{ij} - (1-\sigma)(\ln P_i + \ln P_j)
\]

Thus, the theoretical coefficients on $\ln Y_i$ and $\ln Y_j$ are \textbf{1}.

\subsection*{e(b)-(e) Empirical Results}

% Generated by R on 2025-12-21
% Gravity Model Estimation Results

\FloatBarrier
\begin{table}[h!]
    \centering
    \caption{Basic Gravity Model Specifications}
    \label{tab:e_basic_gravity}
    \begin{threeparttable}
        \footnotesize
        \begin{tabular}{lcccc}
            \toprule
                                 & \multicolumn{4}{c}{ln\_trade}                                                       \\
                                 & Naive                         & +Policy         & +History        & +CountryFE      \\
                                 & (1)                           & (2)             & (3)             & (4)             \\
            \midrule
            Constant             & 12.41$^{***}$                 & 10.30$^{***}$   & 9.092$^{***}$   &                 \\
                                 & (0.2582)                      & (0.2458)        & (0.2743)        &                 \\
            ln\_gdp\_o           & 0.7932$^{***}$                & 0.7672$^{***}$  & 0.7656$^{***}$  & 1.218$^{***}$   \\
                                 & (0.0155)                      & (0.0153)        & (0.0151)        & (0.0290)        \\
            ln\_gdp\_d           & 0.6594$^{***}$                & 0.6316$^{***}$  & 0.6278$^{***}$  & 0.7187$^{***}$  \\
                                 & (0.0157)                      & (0.0154)        & (0.0153)        & (0.0274)        \\
            ln\_dist             & -0.8514$^{***}$               & -0.5675$^{***}$ & -0.4391$^{***}$ & -1.216$^{***}$  \\
                                 & (0.0250)                      & (0.0231)        & (0.0253)        & (0.0169)        \\
            year $=$ 1965        & -0.1131$^{***}$               & -0.1597$^{***}$ & -0.1418$^{***}$ & -0.0897$^{***}$ \\
                                 & (0.0283)                      & (0.0270)        & (0.0264)        & (0.0241)        \\
            year $=$ 1970        & -1.108$^{***}$                & -1.131$^{***}$  & -1.083$^{***}$  & -0.5920$^{***}$ \\
                                 & (0.0363)                      & (0.0356)        & (0.0350)        & (0.0359)        \\
            year $=$ 1975        & -1.111$^{***}$                & -1.132$^{***}$  & -1.072$^{***}$  & -0.6539$^{***}$ \\
                                 & (0.0416)                      & (0.0408)        & (0.0402)        & (0.0505)        \\
            year $=$ 1980        & -1.369$^{***}$                & -1.368$^{***}$  & -1.296$^{***}$  & -1.021$^{***}$  \\
                                 & (0.0491)                      & (0.0481)        & (0.0474)        & (0.0684)        \\
            year $=$ 1985        & -1.823$^{***}$                & -1.812$^{***}$  & -1.745$^{***}$  & -1.592$^{***}$  \\
                                 & (0.0535)                      & (0.0523)        & (0.0518)        & (0.0791)        \\
            year $=$ 1990        & -2.037$^{***}$                & -2.020$^{***}$  & -1.936$^{***}$  & -1.662$^{***}$  \\
                                 & (0.0575)                      & (0.0562)        & (0.0556)        & (0.0882)        \\
            year $=$ 1995        & -2.346$^{***}$                & -2.330$^{***}$  & -2.235$^{***}$  & -1.728$^{***}$  \\
                                 & (0.0596)                      & (0.0581)        & (0.0572)        & (0.0940)        \\
            year $=$ 2000        & -2.677$^{***}$                & -2.656$^{***}$  & -2.554$^{***}$  & -1.984$^{***}$  \\
                                 & (0.0621)                      & (0.0607)        & (0.0598)        & (0.1011)        \\
            year $=$ 2005        & -2.672$^{***}$                & -2.707$^{***}$  & -2.598$^{***}$  & -2.033$^{***}$  \\
                                 & (0.0681)                      & (0.0665)        & (0.0656)        & (0.1123)        \\
            cu                   &                               & -0.2560$^{*}$   & -0.2537$^{*}$   & 0.6412$^{***}$  \\
                                 &                               & (0.1482)        & (0.1427)        & (0.0792)        \\
            rta                  &                               & 1.280$^{***}$   & 1.262$^{***}$   & 0.6230$^{***}$  \\
                                 &                               & (0.0708)        & (0.0676)        & (0.0389)        \\
            contig               &                               &                 & 1.865$^{***}$   & 0.4278$^{***}$  \\
                                 &                               &                 & (0.1131)        & (0.0747)        \\
            comlang\_off         &                               &                 & -0.2965$^{***}$ & 0.5319$^{***}$  \\
                                 &                               &                 & (0.0525)        & (0.0321)        \\
            colony               &                               &                 & 2.700$^{***}$   & 1.261$^{***}$   \\
                                 &                               &                 & (0.1148)        & (0.0727)        \\
            curcol               &                               &                 & -2.930$^{***}$  & -1.074          \\
                                 &                               &                 & (0.5844)        & (0.8291)        \\
            smctry               &                               &                 & -0.2356         & 0.5218$^{***}$  \\
                                 &                               &                 & (0.1573)        & (0.1109)        \\
            \midrule
            \emph{Fixed-effects}                                                                                       \\
            iso\_o               &                               &                 &                 & Yes             \\
            iso\_d               &                               &                 &                 & Yes             \\
            \midrule
            \emph{Fit statistics}                                                                                      \\
            Observations         & 101,250                       & 100,578         & 100,578         & 100,578         \\
            R$^2$                & 0.20840                       & 0.20310         & 0.23114         & 0.70080         \\
            \midrule
            iso\_o fixed effects &                               &                 &                 & \checkmark      \\
            iso\_d fixed effects &                               &                 &                 & \checkmark      \\
            \bottomrule
        \end{tabular}
        \begin{tablenotes}
            \footnotesize
            \item All specifications cluster standard errors by country pairs. Columns (1)-(3) are OLS estimates with year fixed effects; column (4) includes exporter and importer fixed effects.
            \item Standard errors clustered by exporter-importer pair in parentheses.
            \item Significance levels: * p < 0.10, ** p < 0.05, *** p < 0.01.
        \end{tablenotes}
    \end{threeparttable}
\end{table}
\FloatBarrier

In our estimations:
\begin{itemize}
    \item \textbf{Specifications (1)-(3) without country FE:} Coefficients on ln\_gdp\_o
          and ln\_gdp\_d are 0.6-0.8, \textbf{below the theoretical prediction}. This
          downward bias likely arises from omitted multilateral resistance terms and
          country-specific characteristics (the "Gold Medal Mistake").

    \item \textbf{Specification (4) with country FE:} Coefficients are 1.22 (origin)
          and 0.72 (destination). The \textbf{origin coefficient exceeds unity}, which
          may reflect domestic market capacity effects or the fact that exporter fixed
          effects absorb some of the true GDP elasticity.
          The \textbf{destination coefficient remains below unity},
          consistent with the literature finding asymmetric import elasticities.
\end{itemize}
While not exactly 1.0, these estimates are more consistent with theory than the
specifications without country fixed effects.

\textbf{Key findings:}
\begin{enumerate}
    \item The CU coefficient is negative and insignificant in specifications (2) and (3),
          but becomes positive (0.641) and significant when country fixed effects are included.
    \item The RTA coefficient is positive and significant across all specifications, though smaller with country fixed effects.
    \item Distance has the expected negative sign, and its magnitude increases with country fixed effects.
    \item Colonial ties and common language have strong positive effects on trade.
\end{enumerate}

The switch in CU coefficient sign highlights the importance of controlling for country-specific unobservables.

\section*{Question f: Comparison with Rose and the ``Gold Medal Mistake''}

Our estimated currency union effect of 0.593 is substantially smaller than Rose's original estimate of 1.21.
This discrepancy highlights the severity of what Baldwin and Taglioni (2006) term the \textbf{``Gold Medal Mistake''}:
the failure to control for Multilateral Resistance Terms (MRTs).

Theoretically, bilateral trade depends not only on bilateral trade costs but also on the relative price indices ($P_i$ and $P_j$) of the trading partners,
which summarize their trade costs with the rest of the world.
Omitting these terms creates a correlation between the error term and the trade cost variables, leading to biased estimates.

\subsection*{Mathematical Explanation}

The theoretically correct gravity equation derived from Anderson \& van Wincoop (2003) is:

\[
    \ln T_{ij} = \ln Y_i + \ln Y_j - \ln Y_w + (1-\sigma)\ln\tau_{ij} - (1-\sigma)(\ln P_i + \ln P_j) + \epsilon_{ij}
\]

where $P_i$ and $P_j$ are price indices that depend on trade costs with all trading partners. Omitting these MRTs biases the coefficients on bilateral variables like the CU dummy.

\subsection*{Country Fixed Effects as a Partial Remedy}

The inclusion of exporter and importer fixed effects in our specification serves as a partial remedy by absorbing the average multilateral resistance for each country over the sample period.
However, this strategy is imperfect because it treats MRTs as time-invariant.
In reality, a country's multilateral resistance changes over time as its trading partners' GDPs and trade costs fluctuate.
Consequently, while standard country fixed effects reduce the bias compared to the naive gravity model,
they do not fully eliminate the bias arising from time-varying multilateral resistance, which constitutes the ``Silver Medal Mistake.''

\subsection*{Advantages and Drawbacks of Country Fixed Effects}

\textbf{Advantages}:
\begin{itemize}
    \item Control for time-invariant unobserved country heterogeneity
    \item Absorb the average trade costs with all partners (MRTs)
    \item Reduce omitted variable bias
\end{itemize}

\textbf{Drawbacks}:
\begin{itemize}
    \item Cannot estimate coefficients for time-invariant bilateral variables (distance, common language, colonial history)
    \item Absorb many degrees of freedom
    \item May over-control if policy variables are correlated with country characteristics
\end{itemize}

\section*{Question g: Silver and Bronze Medal Mistakes}

\textbf{Silver Medal Mistake}: Failure to account for \textbf{time-varying country heterogeneity}. Rose used country fixed effects but not country-year fixed effects, failing to control for time-varying MRTs. When trade costs with third countries change over time, these changes affect bilateral trade flows and bias the estimated CU coefficient if not properly controlled.

\textbf{Bronze Medal Mistake}: \textbf{Improper deflation of trade values}. Rose used nominal trade flows deflated by GDP deflators instead of proper price indices for tradables. This introduces measurement error that is correlated with the error term, potentially biasing the CU coefficient.

These methodological issues, combined with the Gold Medal Mistake, explain why Rose's original estimates were implausibly large. Our results, which address these issues through proper econometric specifications, yield more plausible estimates of the currency union effect.

\section*{Question h: Year-by-Year Regressions with Country Fixed Effects}

Table 2 presents the results of year-by-year regressions with country fixed effects.

% Generated by R on 2025-12-21
% Gravity Model Estimation Results

\FloatBarrier
\begin{table}[h!]
    \centering
    \caption{Currency Union Effect by Year}
    \label{tab:h_yearly_cu}
    \begin{threeparttable}
        \footnotesize
        \resizebox{\textwidth}{!}{%
            \begin{tabular}{lcccccccccc}
                \toprule
                                     & \multicolumn{10}{c}{ln\_trade}                                                                                                                                                                  \\
                                     & \multicolumn{10}{c}{Year-by-Year FE}                                                                                                                                                            \\
                                     & 1960                                 & 1965            & 1970            & 1975           & 1980           & 1985           & 1990           & 1995           & 2000           & 2005           \\
                \midrule
                ln\_dist             & -0.7259$^{***}$                      & -0.8087$^{***}$ & -0.9249$^{***}$ & -1.076$^{***}$ & -1.220$^{***}$ & -1.253$^{***}$ & -1.321$^{***}$ & -1.305$^{***}$ & -1.384$^{***}$ & -1.442$^{***}$ \\
                                     & (0.0308)                             & (0.0263)        & (0.0279)        & (0.0288)       & (0.0289)       & (0.0280)       & (0.0266)       & (0.0229)       & (0.0221)       & (0.0244)       \\
                contig               & 0.2216$^{**}$                        & 0.1641$^{*}$    & 0.1839          & 0.2001$^{*}$   & 0.1531         & 0.0467         & 0.3361$^{***}$ & 0.7693$^{***}$ & 0.7236$^{***}$ & 0.7711$^{***}$ \\
                                     & (0.1049)                             & (0.0963)        & (0.1124)        & (0.1117)       & (0.1152)       & (0.1030)       & (0.1020)       & (0.0923)       & (0.0876)       & (0.0905)       \\
                comlang\_off         & 0.2774$^{***}$                       & 0.3397$^{***}$  & 0.4350$^{***}$  & 0.4048$^{***}$ & 0.3305$^{***}$ & 0.4802$^{***}$ & 0.5102$^{***}$ & 0.5865$^{***}$ & 0.6151$^{***}$ & 0.6761$^{***}$ \\
                                     & (0.0607)                             & (0.0520)        & (0.0554)        & (0.0552)       & (0.0583)       & (0.0548)       & (0.0514)       & (0.0459)       & (0.0441)       & (0.0439)       \\
                colony               & 1.236$^{***}$                        & 1.228$^{***}$   & 1.329$^{***}$   & 1.268$^{***}$  & 1.427$^{***}$  & 1.208$^{***}$  & 1.248$^{***}$  & 1.303$^{***}$  & 1.227$^{***}$  & 1.067$^{***}$  \\
                                     & (0.1059)                             & (0.0922)        & (0.0958)        & (0.0924)       & (0.0936)       & (0.0900)       & (0.0880)       & (0.0808)       & (0.0872)       & (0.0887)       \\
                cu                   & 0.5983$^{***}$                       & 0.7321$^{***}$  & 1.450$^{***}$   & 1.185$^{***}$  & 0.8639$^{***}$ & 1.022$^{***}$  & 1.317$^{***}$  & 0.9010$^{***}$ & 0.0082         & 0.2152$^{*}$   \\
                                     & (0.1159)                             & (0.0956)        & (0.1331)        & (0.1441)       & (0.1637)       & (0.1470)       & (0.1631)       & (0.1597)       & (0.1151)       & (0.1110)       \\
                rta                  & 0.1616                               & 0.5662$^{***}$  & 0.7374$^{***}$  & -0.1328        & 0.0228         & 0.0685         & 0.2266$^{**}$  & 0.5137$^{***}$ & 0.6511$^{***}$ & 0.5413$^{***}$ \\
                                     & (0.1068)                             & (0.1054)        & (0.1481)        & (0.1161)       & (0.1140)       & (0.1008)       & (0.0910)       & (0.0641)       & (0.0580)       & (0.0495)       \\
                \midrule
                \emph{Fixed-effects}                                                                                                                                                                                                   \\
                iso\_o               & Yes                                  & Yes             & Yes             & Yes            & Yes            & Yes            & Yes            & Yes            & Yes            & Yes            \\
                iso\_d               & Yes                                  & Yes             & Yes             & Yes            & Yes            & Yes            & Yes            & Yes            & Yes            & Yes            \\
                \midrule
                \emph{Fit statistics}                                                                                                                                                                                                  \\
                Observations         & 4,999                                & 6,324           & 8,135           & 9,593          & 10,446         & 10,750         & 12,784         & 17,422         & 18,838         & 19,976         \\
                R$^2$                & 0.68845                              & 0.70855         & 0.71434         & 0.70421        & 0.70179        & 0.71499        & 0.72522        & 0.73144        & 0.73506        & 0.74012        \\
                \midrule
                iso\_o fixed effects & \checkmark                           & \checkmark      & \checkmark      & \checkmark     & \checkmark     & \checkmark     & \checkmark     & \checkmark     & \checkmark     & \checkmark     \\
                iso\_d fixed effects & \checkmark                           & \checkmark      & \checkmark      & \checkmark     & \checkmark     & \checkmark     & \checkmark     & \checkmark     & \checkmark     & \checkmark     \\
                \bottomrule
            \end{tabular}%
        }
        \begin{tablenotes}
            \footnotesize
            \item Exporter and importer fixed effects; year-by-year split estimation.
            \item Standard errors clustered by exporter-importer pair.
            \item Significance levels: * p < 0.10, ** p < 0.05, *** p < 0.01.
        \end{tablenotes}
    \end{threeparttable}
\end{table}


The results show a strong positive effect of currency unions from 1960 to 1990,
with coefficients ranging from 0.60 to 1.45. However,
the effect declines sharply after 1995,
becoming statistically insignificant in 2000 (0.008) and 2005 (0.215).
This suggests that the trade-promoting effect of currency unions may have diminished in recent decades,
or that the introduction of the Euro (which dominates the recent CU observations) has different characteristics than earlier unions.

Such variation could occur due to:
\begin{itemize}
    \item Changes in the composition of currency unions
    \item Evolving global economic conditions
    \item Learning effects as currency unions mature
\end{itemize}

This pattern suggests that:
\begin{enumerate}
    \item \textbf{Composition effect}: Early CUs (e.g., CFA franc zone, OECS) involved
          small, post-colonial economies with limited prior trade. The large effects may
          reflect \textit{specific institutional arrangements} rather than the causal
          impact of currency unions per se.

    \item \textbf{Euro's different nature}: The Euro, introduced in 1999-2002, dominates
          the recent CU observations. Its smaller (or null) effect may be because:
          \begin{itemize}
              \item Eurozone members were \textbf{already highly integrated} through the
                    Single Market and prior exchange rate mechanisms (ERM).
              \item The Euro eliminated fewer barriers than traditional CUs.
          \end{itemize}

    \item \textbf{Methodological lesson}: Using country fixed effects year-by-year is more
          appropriate than pooling all years with a single CU coefficient, as the effect
          is clearly \textbf{heterogeneous across time}.
\end{enumerate}

\section*{Question i: Critiques and Country-Pair Fixed Effects}

% Generated by R on 2025-12-09
% Gravity Model Estimation Results

\begin{table}[h!]
    \centering
    \caption{Country-Pair Fixed Effects Analysis}
    \label{tab:table_i_pair_fe}
    \begin{threeparttable}
    \begin{tabular}{l c}
    \toprule
    & \textbf{Dependent Variable: $\ln(\text{trade})$} \\
    & \textbf{Pair FE} \\
    & (1) \\
    \midrule
    $\text{cu}$ & $0.446^{***}$ \\
    & $(0.111)$ \\
    $\text{rta}$ & $0.402^{***}$ \\
    & $(0.061)$ \\
    \addlinespace
    $\ln(\text{gdp}_o)$ & $1.19^{***}$ \\
    & $(0.109)$ \\
    $\ln(\text{gdp}_d)$ & $0.791^{***}$ \\
    & $(0.092)$ \\
    \midrule
    Observations & 102,018 \\
    $R^2$ & 0.875 \\
    Within $R^2$ & 0.077 \\
    \addlinespace
    \textit{Fixed Effects} & \\
    Exporter & $\checkmark$ \\
    Importer & $\checkmark$ \\
    Exporter $\times$ Importer & $\checkmark$ \\
    Year & $\checkmark$ \\
    \bottomrule
    \end{tabular}
    \begin{tablenotes}[flushleft]
    \footnotesize
    \item \textbf{Notes:} Includes exporter, importer, pair, and year fixed effects to control for unobserved bilateral heterogeneity. Standard errors, clustered by exporter-importer pair, in parentheses.
    \item Significance levels: * $p < 0.10$, ** $p < 0.05$, *** $p < 0.01$.
    \end{tablenotes}
    \end{threeparttable}
\end{table}


\subsection*{i(1) Country-Pair Fixed Effects as a Solution}

The country-pair fixed effects specification yields a CU coefficient of 0.660,
implying a 93\% increase in trade ($e^{0.660} - 1 = 0.93$).
This is slightly larger than the **0.641** estimate with country fixed effects,
suggesting that controlling for unobserved bilateral heterogeneity reinforces the positive CU effect in the static panel setting.

\textbf{Strengths of pair fixed effects}:
\begin{itemize}
    \item Control for all time-invariant bilateral factors (historical ties, permanent geographic features, cultural affinities)
    \item Address reverse causality by using within-pair variation over time
    \item Mitigate omitted variable bias from unobserved pair characteristics
\end{itemize}

\textbf{Weaknesses in this context}:
\begin{itemize}
    \item CU membership has little within-pair variation over time (only 1.5\% of pairs change CU status)
    \item Identification relies on a small subset of observations
    \item May absorb too much variation if CU membership is stable
\end{itemize}

\subsection*{i(2) Potential Omitted Variables and Sample Selection}

\textbf{Potential omitted variables} that could bias the CU coefficient upward include:
\begin{itemize}
    \item Deep cultural/linguistic ties beyond common language
    \item Political alliances and security agreements
    \item Similar legal systems and institutional quality
    \item Complementary economic structures
\end{itemize}

\textbf{Endogenous sample selection}: Countries self-select into currency unions based on unobserved factors that also affect trade.
This creates upward bias as the CU dummy captures both the treatment effect and the selection effect.
For example, countries with strong but unmeasured economic ties may be more likely to form a currency union.

\section*{Question j: High-Dimensional Fixed Effects}

% Generated by R on 2025-12-09
% Gravity Model Estimation Results

\begin{table}[h!]
    \centering
    \caption{Structural Gravity Model with Multilateral Resistance Terms}
    \label{tab:table_j_structural_gravity}
    \begin{threeparttable}
    \begin{tabular}{l c c}
    \toprule
    & \multicolumn{2}{c}{Dependent Variable: $\ln(\text{trade})$} \\
    \cmidrule(lr){2-3}
    & \textbf{Structural Gravity} & \textbf{Structural + Pair FE} \\
    & (1) & (2) \\
    \midrule
    \textit{Trade Policy Variables} & & \\
    \quad $\text{cu}$ & $0.490^{**}$ & $0.349^{***}$ \\
    & $(0.198)$ & $(0.102)$ \\
    \quad $\text{rta}$ & $0.666^{***}$ & $0.401^{***}$ \\
    & $(0.141)$ & $(0.066)$ \\
    \addlinespace
    \textit{Geographic Variables} & & \\
    \quad $\ln(\text{dist})$ & $-1.36^{***}$ & \\
    & $(0.042)$ & \\
    \addlinespace
    \textit{Cultural/Historical Variables} & & \\
    \quad $\text{comlang\_off}$ & $0.433^{***}$ & \\
    & $(0.068)$ & \\
    \quad $\text{colony}$ & $1.32^{***}$ & \\
    & $(0.112)$ & \\
    \midrule
    Observations & 102,018 & 102,018 \\
    $R^2$ & 0.729 & 0.899 \\
    Within $R^2$ & 0.346 & 0.003 \\
    \addlinespace
    \textit{Fixed Effects} & & \\
    Exporter $\times$ Year & $\checkmark$ & $\checkmark$ \\
    Importer $\times$ Year & $\checkmark$ & $\checkmark$ \\
    Exporter $\times$ Importer & & $\checkmark$ \\
    \bottomrule
    \end{tabular}
    \begin{tablenotes}[flushleft]
    \footnotesize
    \item \textbf{Notes:} Standard errors, clustered by exporter-importer pair, in parentheses. Column (1) includes exporter-year and importer-year fixed effects to control for time-varying multilateral resistance terms. Column (2) adds pair fixed effects to control for unobserved time-invariant bilateral heterogeneity.
    \item Significance levels: * $p < 0.10$, ** $p < 0.05$, *** $p < 0.01$.
    \end{tablenotes}
    \end{threeparttable}
\end{table}


The structural gravity specifications represent the \textbf{state-of-the-art} approach
in the literature:

\begin{enumerate}
    \item \textbf{Specification (1) — Time-varying multilateral resistance:}
          CU coefficient = 0.633***, implying an 88\% increase in trade.

          \textit{What this controls for:} Exporter-year and importer-year fixed effects
          ($\nu_{it}$ and $\mu_{jt}$) absorb all time-varying country characteristics,
          including:
          \begin{itemize}
              \item Changing GDP, population, productivity
              \item Evolving trade costs with the rest of the world
                    (Anderson \& van Wincoop's $P_i$ and $P_j$)
              \item Monetary policy shocks, exchange rate regimes
          \end{itemize}

          This addresses Baldwin \& Taglioni's \textbf{"Silver Medal Mistake"}.

    \item \textbf{Specification (2) — Adding pair FE:}
          CU coefficient = 0.308***, implying a 36\% increase in trade.

          The \textbf{dramatic drop from 0.660 (Table 2) to 0.308} illustrates that:
          \begin{itemize}
              \item Previous pair FE estimates \textit{without} country-year FE were
                    \textbf{upward biased} due to correlated time-varying shocks.
              \item Example: If two countries simultaneously join a CU \textit{and}
                    experience faster GDP growth, the simple pair FE model attributes
                    all the trade increase to the CU.
          \end{itemize}
\end{enumerate}

\subsection*{Rationale for These Specifications}

\begin{itemize}
    \item \textbf{Exporter-year and importer-year fixed effects}: Control for time-varying multilateral resistance terms, addressing the Silver Medal Mistake.
    \item \textbf{Pair fixed effects}: Control for all time-invariant bilateral characteristics, addressing the Gold Medal Mistake (time-invariant MRTs) and other time-invariant confounders.
\end{itemize}

The decline in the CU coefficient from 0.490 to 0.349 as we add more fixed effects suggests that earlier estimates were biased upward by omitted variables.

\section*{Question k: Do Small Countries Benefit More from Currency Unions?}

% Generated by R on 2025-12-21
% Gravity Model Estimation Results

\FloatBarrier
\begin{table}[h!]
    \centering
    \caption{Country Size Heterogeneity Analysis}
    \label{tab:k_small_countries}
    \begin{threeparttable}
        \footnotesize
        \begin{tabular}{lcc}
            \toprule
                                     & \multicolumn{2}{c}{ln\_trade}                  \\
                                     & Full Controls                 & Pair FE        \\
                                     & (1)                           & (2)            \\
            \midrule
            ln\_dist                 & -1.238$^{***}$                &                \\
                                     & (0.0170)                      &                \\
            contig                   & 0.4316$^{***}$                &                \\
                                     & (0.0749)                      &                \\
            comlang\_off             & 0.5264$^{***}$                &                \\
                                     & (0.0321)                      &                \\
            colony                   & 1.265$^{***}$                 &                \\
                                     & (0.0700)                      &                \\
            curcol                   & -1.124                        &                \\
                                     & (0.8079)                      &                \\
            smctry                   & 0.4734$^{***}$                &                \\
                                     & (0.1116)                      &                \\
            cu                       & 1.937$^{***}$                 & 0.0455         \\
                                     & (0.3184)                      & (0.2791)       \\
            rta                      & 0.6253$^{***}$                & 0.4272$^{***}$ \\
                                     & (0.0432)                      & (0.0306)       \\
            cu $\times$ ln\_gdp\_avg & -0.1848$^{***}$               & 0.0383         \\
                                     & (0.0401)                      & (0.0314)       \\
            \midrule
            Observations             & 100,578                       & 101,343        \\
            R$^2$                    & 0.73048                       & 0.89443        \\
            \midrule
            exp\_year fixed effects  & \checkmark                    & \checkmark     \\
            imp\_year fixed effects  & \checkmark                    & \checkmark     \\
            pair\_id fixed effects   &                               & \checkmark     \\
            \bottomrule
        \end{tabular}
        \begin{tablenotes}
            \footnotesize
            \item Interaction term `cu:ln\_gdp\_avg' shows whether currency unions have larger effects in country pairs with larger average GDP.
            \item Standard errors clustered by exporter-importer pair in parentheses.
            \item Significance levels: * p < 0.10, ** p < 0.05, *** p < 0.01.
        \end{tablenotes}
    \end{threeparttable}
\end{table}

The results provide mixed evidence on whether small countries benefit more from currency unions:

\begin{enumerate}
    \item \textbf{Specifications (1) and (2) without country fixed effects}: To test if small countries benefit more, we interact the CU dummy with the average log GDP of the pair ($cu \times ln\_gdp\_avg$).
          In Specification (1) (Full Controls), the interaction term is negative and significant (-0.185).
          This implies that the trade-creating effect of currency unions decreases as the size of the countries increases—supporting the hypothesis that small countries benefit more.
          However, in Specification (2) with Pair Fixed Effects,
          the interaction becomes positive but statistically insignificant (0.038).
          This suggests that once we control for time-invariant pair characteristics,
          there is no robust evidence that country size modulates the currency union effect.
\end{enumerate}

\subsection*{Interpretation}

The evidence does not strongly support the hypothesis that small countries benefit more from currency unions.
The large CU coefficient in specification (3) may be driven by other factors correlated with both small country size and CU membership.
The positive coefficient on both\_small suggests that small countries tend to trade more with each other, but this effect does not appear to be amplified by currency union membership.

\section*{Question l: Problems with Log-Linearization and Poisson Estimation}

% Generated by R on 2025-12-21
% Gravity Model Estimation Results

\FloatBarrier
\begin{table}[h!]
    \centering
    \caption{Poisson Pseudo-Maximum Likelihood (PPML) Estimation}
    \label{tab:l_ppml}
    \begin{threeparttable}
        \footnotesize
        \begin{tabular}{lc}
            \toprule
                                   & trade          \\
                                   & PPML           \\
                                   & (1)            \\
            \midrule
            cu                     & -0.1220$^{**}$ \\
                                   & (0.0508)       \\
            rta                    & 0.2597$^{***}$ \\
                                   & (0.0723)       \\
            \midrule
            Observations           & 196,999        \\
            Pseudo R$^2$           & 0.97381        \\
            \midrule
            pair\_id fixed effects & \checkmark     \\
            year fixed effects     & \checkmark     \\
            \bottomrule
        \end{tabular}
        \begin{tablenotes}
            \footnotesize
            \item PPML estimation includes country-pair and year fixed effects.
            \item Standard errors clustered by exporter-importer pair in parentheses.
            \item Significance levels: * p < 0.10, ** p < 0.05, *** p < 0.01.
        \end{tablenotes}
    \end{threeparttable}
\end{table}

\subsection*{Two Main Problems with Log-Linear Gravity Equation}

\begin{enumerate}
    \item \textbf{Zero trade flows (sample selection bias):}

          The log transformation is undefined for zero trade values, forcing researchers to
          drop these observations. However, zeros are \textbf{not random}:
          \begin{itemize}
              \item Small countries or distant pairs are more likely to have zero trade.
              \item If currency unions \textit{create} trade where none existed before,
                    dropping zeros \textbf{underestimates} the CU effect.
              \item Conversely, if CUs only intensify existing trade, dropping zeros
                    \textbf{overestimates} the effect.
          \end{itemize}
          %   Santos Silva \& Tenreyro (2006) show this creates \textbf{inconsistent} OLS estimates.

    \item \textbf{Heteroskedasticity:}

          Trade data exhibit variance proportional to $\mathbb{E}[T_{ij}]^2$, violating
          homoskedasticity. In log-linear models, this implies:
          \begin{equation*}
              \mathbb{E}[\ln T_{ij} | X] \neq \ln \mathbb{E}[T_{ij} | X]
          \end{equation*}

          due to Jensen's inequality. OLS on the log model is therefore \textbf{inconsistent}.
\end{enumerate}

\subsection*{PPML Estimation Results}

The PPML estimate of the CU coefficient is -0.122 and statistically insignificant.
This confirms that when accounting for zero trade flows and heteroskedasticity,
the currency union effect disappears, casting doubt on the robustness of the log-linear results.

\subsection*{Interpretation}

The discrepancy between PPML and log-linear estimates suggests that heteroskedasticity and/or zero trade flows may be biasing the log-linear estimates.
The PPML estimate is more reliable because:
\begin{itemize}
    \item \textbf{Zero flows matter:} PPML includes 196,999 observations vs. about 120,000
          in log-linear models. The additional 75,000+ observations with zero or very
          low trade paint a different picture.

    \item \textbf{Efficiency vs. consistency trade-off:} While log-linear models may
          have lower standard errors, they are \textbf{biased}. PPML sacrifices some
          precision for consistency.

    \item \textbf{Robustness implication:} The positive CU effects in log-linear
          specifications appear to be \textbf{artifacts} of sample selection and
          heteroskedasticity, not genuine trade creation.
\end{itemize}

The insignificant PPML result casts doubt on the robustness of the positive CU effects found in log-linear specifications.

\section*{Question m: Euro Effect Analysis}

% Generated by R on 2025-12-21
% Gravity Model Estimation Results

\FloatBarrier
\begin{table}[h!]
    \centering
    \caption{The Effect of the Euro on Trade}
    \label{tab:m_euro_effect}
    \begin{threeparttable}
        \footnotesize
        \begin{tabular}{lccc}
            \toprule
                                    & \multicolumn{2}{c}{trade} & ln\_trade                       \\
                                    & PPML (Pair+Year FE)       & PPML (HDFE)    & OLS (HDFE)     \\
                                    & (1)                       & (2)            & (3)            \\
                                    & Poisson                   & Poisson        & OLS            \\
            \midrule
            euro                    & -0.1346$^{***}$           & -0.0010        & 0.4391$^{***}$ \\
                                    & (0.0513)                  & (0.0433)       & (0.0690)       \\
            other\_cu               & 0.4960$^{***}$            & 0.6766$^{***}$ & 0.2581$^{***}$ \\
                                    & (0.1758)                  & (0.1467)       & (0.0818)       \\
            rta                     & 0.2607$^{***}$            & 0.1749$^{***}$ & 0.4522$^{***}$ \\
                                    & (0.0716)                  & (0.0343)       & (0.0298)       \\
            \\
            Observations            & 197,020                   & 191,850        & 120,150        \\
            R$^2$                   &                           &                & 0.88967        \\
            Pseudo R$^2$            & 0.98399                   & 0.99335        & 0.43090        \\
            \\
            pair\_id fixed effects  & $\checkmark$              & $\checkmark$   & $\checkmark$   \\
            year fixed effects      & $\checkmark$              &                &                \\
            exp\_year fixed effects &                           & $\checkmark$   & $\checkmark$   \\
            imp\_year fixed effects &                           & $\checkmark$   & $\checkmark$   \\
            \bottomrule
        \end{tabular}
        \begin{tablenotes}
            \footnotesize
            \item Columns (1)-(2) use PPML on trade levels; column (3) uses OLS on log(trade). HDFE specifications control for time-varying multilateral resistance terms via exporter-year and importer-year fixed effects, plus pair fixed effects for unobserved bilateral heterogeneity. Euro zone includes initial 11 members (1999) plus Greece (2001).
            \item Standard errors clustered by exporter-importer pair in parentheses.
            \item Significance levels: * p < 0.10, ** p < 0.05, *** p < 0.01.
        \end{tablenotes}
    \end{threeparttable}
\end{table}

The analysis distinguishes between the Eurozone and other currency unions. The results show:

\begin{enumerate}
    \item \textbf{Euro Effect}: The coefficient for the Euro dummy is -0.135***,
          and statistically insignificant.

    \item \textbf{Other CUs}:
          The coefficient for other currency unions is 0.496 and significant at the 5\% level.
\end{enumerate}

This suggests that while traditional currency unions (often post-colonial or small-state associations) are associated with higher trade,
the introduction of the Euro has not led to a statistically significant increase in trade among its members (controlling for RTAs and pair heterogeneity).
The ``negative'' point estimate, though insignificant,
aligns with the literature suggesting that the Euro's trade effect is either small or elusive when proper econometric controls are applied.

\end{document}