\section{Outer Measure}
Outer measure can be defined on every set.
\begin{definition}[Outer Measure]
    \label{def:outer_measure}
    \

    $E \subseteq \mathbb{R}^n$ is a set,
    $I$ is a closed interval: $I = \{ x = (x_1, x_2, \cdots, x_n) \in \mathbb{R}^n | a_i \leq x_i \leq b_i, i=1, \cdots, n\}$,
    and $v(I)$ is the volume of the interval $I$, 
    \begin{gather*}
        v(I) = \begin{cases}
            \prod_{i=1}^n (b_i - a_i), &\text{ if } a_i \leq b_i ;\\
            0, &\text{ otherwise} .
        \end{cases}
    \end{gather*}
    For set $E$, consider a \textit{countable} collection of open, nounded intervals that cover $E$,
    $S = \{I_i\}_{i=1}^{\infty}$, in the sense that $E \subseteq \bigcup_{i=1}^{\infty} I_i$.
    For each such collection, consider the sum of the volumes of the intervals in the collection. 
    We define
    \begin{gather}
        \sigma(S) = \sum_{i=1}^{\infty} v(I_i)
    \end{gather}
    The outer measure of $E$, denoted by $m^{*}(E)$, is
    \begin{gather}
        m^{*}(E) = \inf \sigma(S)
    \end{gather}
    the infimum is taken over all countable collections of closed intervals $S$.
\end{definition}

\begin{lemma}\label{lem1}
    \

    If $I$ is a closed interval, then $m^{*}(I) = v(I).$
\end{lemma}
\begin{proof}[Proof of Lemma \ref{lem2}]
    \

    By definition, $I$ covers itself, so $m^{*}(I) \leq v(I)$.
    Given any $\epsilon > 0$,
    $\exists S = \{I_i\}_{i=1}^{\infty}$, a closed interval cover,
    such that $\sigma(S) \leq m^{*}(I) + \epsilon$.
    We need to show that $v(I) \leq \sum_{i=1}^{\infty} v(I_i) = \sigma(S).$
    For each $i$, choose a bigger $I_i^*$,
    such that $I \subseteq int(I_i^*)$ and $v(I_i^*) \leq v(I_i)(1 + \epsilon)$.
    Then we have $I \subseteq \bigcup_{i=1}^{\infty} int(I_i^*)$.
    By compactness of $I$, (The Heine-Borel theorem),
    we can find an integer $N$ such that $I \subseteq \bigcup_{i=1}^{N} int(I_i^*)$,
    hence
    \begin{gather*}
        v(I) \leq \sum_{i=1}^{N} v(I_i^*) \leq (1 + \epsilon) \sum_{i=1}^{N} v(I_i) \leq (1 + \epsilon) \sigma(S).
    \end{gather*}
    So $v(I) \leq \sigma(S)$, if we take infimum over all $S$,
    we have $v(I) \leq m^{*}(I)$.
\end{proof}