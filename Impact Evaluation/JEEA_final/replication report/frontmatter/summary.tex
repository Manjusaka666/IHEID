\chapter*{Summary}
\addcontentsline{toc}{chapter}{Summary}

本报告旨在复现Reinhart和Trebesch (2016)关于主权债务减免及其经济后果的核心研究思路、模型选择逻辑和关键数学推导。
论文通过对两次截然不同的历史时期(1920-1939年发达经济体的官方债务违约和1978-2010年新兴市场的私人外部债权人债务危机)进行量化分析,揭示了债务减免的规模和影响。
核心发现是,只有涉及债务注销(debt write-offs)的债务减免操作才能显著改善债务国的经济状况,而诸如期限延长和利率削减等较温和的债务减免形式通常不会带来经济增长的显著提高或信用评级的改善。

本报告还将探讨在原文经典双重差分(DiD)方法基础上,应用交错双重差分(Staggered DiD)方法进行分析的潜力及其数学框架。